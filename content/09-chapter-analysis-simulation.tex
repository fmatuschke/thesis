\cleardoublepage
\setcounter{chapter}{8}
\chapter{3D-PLI simulation}
\label{cha:simulation_analysis}
%
\section{Introduction}
%
This chapter deals with the simulation of \ac{3D-PLI} with the generated model library.
The first part focuses on the determination of all necessary physical parameters of the tissue and the microscope as well as on the characterization of the simulation parameters.
Then, the simulation of the previously generated nerve fiber models is created and analyzed.
The orientation of the simulations is calculated using the routine algorithms implemented in \ac{fastPLI}.
The focus of the evaluation is on the accuracy of the tilt analysis for different orientations and crossing configurations.
%
%
%
\section{Parameter characterisation}\label{sec:sim_choose_parameters}
%
\subsection{Tissue}\label{sec:tissueProp}
%
\begin{figure}[t]
% 2_simulation/0_parameter/measure_vervet.ipy
\centering
\setlength{\tikzwidth}{0.425\textwidth}
\setlength{\tabcolsep}{0em}
\begin{tabular}{C{0.5\textwidth}C{0.5\textwidth}}
%
\tikzset{external/export next=false}% takes long for the first time
\inputtikz{gfx/data/vervet_transmittance_zoom} &
\inputtikz{gfx/data/vervet_transmittance_hist} \\[-5mm]
%
\subcaptiontab{0.475\textwidth}{Transmittance.} &
\subcaptiontab{0.475\textwidth}{\label{fig:histTrans}Histogram transmittance.} \\[10mm]
%
\tikzset{external/export next=false}
\inputtikz{gfx/data/vervet_retardation_zoom} &
\inputtikz{gfx/data/vervet_retardation_hist} \\[-5mm]
%
\subcaptiontab{0.475\textwidth}{Retardation.} &
\subcaptiontab{0.475\textwidth}{\label{fig:histRet}Histogram retardation.} 
\end{tabular}
\caption{
Transmittance and retardation map of coronal section $\SI{549}{}$ of a vervet monkey.
The absorption coefficient and birefringence is estimated from the measurments in the corpus calosum (CC).
Two homogeneous regions are labeled for this purpose.
The left are include $\SI{1125858}{\pixel}$ and the right area $\SI{1064629}{\pixel}$.
}
\label{fig:brain_ret_trans_zoom}
\end{figure}
%
The absorption coefficient $\absorp{}$ and the birefringence $\dn{}$ have to be estimated from a measured section for the simulation.
To measure the values, it is important to analyze a homogeneous region filled with flat, dense fibers.
In a coronal section, the corpus callosum (CC) is suitable for this purpose. 
It is the main fiber connection between the two cerebral hemispheres.
\par
% 
\Cref{fig:brain_ret_trans_zoom} shows transmittance and retardation maps of the coronal section $\SI{549}{}$ of a Vervet monkey.
Two \ac{ROI} are manually selected in homogeneous and high retardence regions of the corpus calosum.
The left region contains $\SI{1125858}{\pixel}$ and the right \ac{ROI} $\SI{1064629}{\pixel}$.
The entire section and the analysis for a human and a rodent section are available in the Appendix (see \cref{app:brain_rodent,app:brain_vervet,app:brain_human}).
% 
\paragraph{Evaluation}
To calculate the absorption coefficient for the models, one must consider the volume fraction.
For the models $\fiberRadiusMean = \SI{0.5}{\micro\meter}$, the volume fraction is about $\SI{75}{\percent}$.
The average transmittance in the foreground is about $\SI{1000}{\arbitraryunit}$ and in the background about $\SI{4500}{\arbitraryunit}$.\footnote{The relative transmittance can be changed by adjusting the exposure time of the microscope}
This results in an absorption coefficient of about $\mu = \SI{30}{\milli\meter\tothe{-1}}$.
To calculate birefringence, one must also take into account the g-ratio, \ie{}, the myelin thickness.
A g-ratio of $\SI{75}{\percent}$ gives a birefringence of about $\SI{0.008}{}$ for the radial model when using a median value of the retardation of $\SI{0.8}{}$.
%
% 
% 
\subsection{Optical resolution}
%
The optical resolution depends, among other things, on aberration and diffraction (see \cref{sec:opticalResolution}).
They are modeled as described in \cref{sec:ccdOptic}.
Therefore, the model parameters must be determined for the microscopes to be simulated.
\par
%
To measure the optical resolution of the microscope, previous measurements and analyzes are repeated \cite{MenzelMaster}.
For this purpose, the \textit{1951 United States Air Force (USAF) resolution test chart}\footnote{U.S. Air Force MIL-STD-150A standard of 1951} is used.
It consists of several patterns that have three slots with specific spacing and width (see \cref{fig:usaf}).
They are arranged in fields of three vertical and horizontal lines.
The fields are arranged in a spiral that shrinks from group to group by a factor $\SI{0.5}{}$.
To determine the line width, the fields are numerically ordered according to a main group $i$ and a subgroup $j$.
To determine the resolution of the microscope, it is necessary to determine the group from which the slits can still be resolved according to the Rayleigh criterion (see \cref{fig:rayleigh}).
For this purpose, the intensity profiles perpendicular to the three vertical and horizontal slits are analyzed.
% 
\paragraph{Evaluation}
%
\begin{figure}[!t]
\centering
\setlength{\tikzwidth}{0.35\textwidth} % bigger than 0.475 images???
\setlength{\tabcolsep}{0em}
\begin{tabular}{C{0.5\textwidth}C{0.5\textwidth}}
%
\includegraphics[height=\tikzwidth]{gfx/wiki/USAF-1951.pdf} &
\inputtikz{gfx/chap8/usaf_image} \\[-1em]
% 
\subcaptiontab{0.475\textwidth}{\label{fig:usaf}USAF chart from group -2 to 1: \url{https://en.wikipedia.org/wiki/1951_USAF_resolution_test_chart}} &
\subcaptiontab{0.475\textwidth}{\label{fig:usaf_image}Microscopic image with highlighted groups 7-6 to 8-2.} \\[4em]
% 
\inputtikz{gfx/chap8/usaf_line_plots_lr} &
\inputtikz{gfx/chap8/usaf_line_plots_up} \\[-1em]
%
\subcaptiontab{0.475\textwidth}{\label{fig:usaf_lines_lr}Centered line plots horizontal slits.} &
\subcaptiontab{0.475\textwidth}{\label{fig:usaf_lines_ud}Centered line plots vertical slits.}
% 
\end{tabular}
\caption{Group 8-1 of the USAF chart with a line width of $\SI{1.95}{\micro\meter}$ corresponts to the Raileigh criteria. Therefore an optical convolution of $\opticsigma = \SI{0.75}{\pixel}$ will be applied in the simulation.}
\label{fig:USAF}
\end{figure}
% 
\Cref{fig:usaf_image} shows a section of the image taken with the \ac{LMP} microscope.
The highlighted areas show the analyzed groups 7-6 to 8-2.
The area of group 7-6 \raisebox{.25em}{\tikzset{external/export next=false}\tikz \draw[RED,ultra thick,dashed](0,0)--(0.25,0);} has a linewidth of $\SI{2.19}{\micro\meter}$, group 8-1 \raisebox{.25em}{\tikzset{external/export next=false}\tikz \draw[GREEN,ultra thick,dashed](0,0)--(0.25,0);} of $\SI{1.95}{\micro\meter}$ and group 8-2 \raisebox{.25em}{\tikzset{external/export next=false}\tikz \draw[BLUE,ultra thick,dashed](0,0)--(0.25,0);} of $\SI{1.74}{\micro\meter}$.
The intensity line profiles for the vertical and horizontal cases are shown in \cref{fig:usaf_lines_lr,fig:usaf_lines_ud}.
The Rayleigh criterion can be used to determine the resolution in the second group, which is thus in the range of $\SI{1.95}{\micro\meter}$.
This reproduces the measurements in the \cite{MenzelMaster}.
Therefore, for the simulation the convolution parameter is set to $\opticsigma = \SI{0.75}{\pixel}$ (see \cref{sec:opticalResolution}).
%
%
%
\subsection{Sensor gain and signal noise}\label{sec:sensorGain}
%
\begin{figure}[!t]
\centering
%
% \tikzset{external/export=false}
\setlength{\tikzwidth}{0.3\textwidth}
\setlength{\tikzheight}{0.3\textwidth}
% \tikzset{external/force remake}
%
\setlength{\tabcolsep}{0em}
\begin{tabular}{C{0.5\textwidth}C{0.5\textwidth}}
% \tikzset{external/remake next}
\inputtikz{gfx/pli/pli_focus}
&
\inputtikz{gfx/2/PM_000_image}
\\[-1em]
%
% SUBCAPTIONS
\subcaptiontab{0.475\textwidth}{\label{fig:pliFocus}Scheme of unfocused microscopic image.} &
\subcaptiontab{0.475\textwidth}{\label{fig:pliFocusImage}Microscopic image} \\[2em]
%
% \tikzset{external/remake next}
\inputtikz{gfx/data/PM_noise}
&
\inputtikz{gfx/data/theo_noise}
\\[-1em]
%
% SUBCAPTIONS
\subcaptiontab{0.475\textwidth}{\label{fig:parameterModelNoise} Linear regression results in a gain factor of $\opticgain_{\mathit{LMP}} = \SI{0.1175}{}(?)$.} &
\subcaptiontab{0.475\textwidth}{\label{fig:noiseplot}Expected noise range for human, vervet monkey and rodent. The noise correlates linearly with the light intensity.}
\end{tabular}
%
\caption{Camera noise measurements.}
\label{fig:parameterModelGain}
\end{figure}
%
As described in \cref{sec:ccdOptic}, the optical noise is modeled with a Gaussian model.
The gain factor of a \ac{CCD} camera describes the linearity between a measured electrical signal and the output signal.
This can also be used to calculate the noise, since it correlates with $\sqrt{I \cdot \opticgain}$ \cite{Wiese:887678}.
To measure the noise, the gain is measured by taking several images with different pixel intensities.
For this purpose, the sample stage is covered with a fully absorbing cover so that half of the image is dark.
In addition, the focal length is changed so that the light is distributed over the entire image sensor under the cover (see \cref{fig:pliFocusImage}).
By measuring $N=\SI{500}{}$ images, the variance for the different intensity values is determined.
%
\paragraph{Evaluation}
The results are shown in \cref{fig:parameterModelNoise} and show a gain factor of $\opticgain_{\mathit{LMP}} = \SI{0.1175}{}(?)$.
This gain factor is used to model the integer noise with
\begin{align}
f(x) = \floor{\mathrm{normal}(\mu = x, \sigma=\sqrt{\opticgain x})+0.5}
\end{align}
for intensities $I > 0$ and $I \gg \sqrt{\opticgain I}$.
%
%
%
\subsection{\Voxelsize{} \texorpdfstring{\voxelsize{}}{}}
%
The parameter \Voxelsize{} $\voxelsize$ is the most important parameter for simulation accuracy, as it determines how accurately the models are discretized and how many light rays are modeled.
However, the voxel size cannot be arbitrarily small because the number of calculations and memory consumption increase by $O(n_{\mathit{voxel}}^3)$.
Therefore, it is recommended that the voxel size be as large as possible without introducing significant errors due to discretization.
\par
%
To investigate this effect, a simulation with several voxel sizes in the range of $\voxelsize = \SIrange{0.01}{1.3}{\micro\meter}$ is performed.
The smallest voxel size $\SI{0.01}{\micro\meter}$ is used as a reference.
Since this voxel size is so small, the simulated volume is $3 \cdot \SI{1.3}{\micro\meter} \times 3 \cdot \SI{1.3}{\micro\meter} \times \SI{60}{\micro\meter}$ without tilts.
Otherwise, the number of voxels and thus the memory consumption is too high.
During the simulation, the previously determined tissue and noise parameters are used.
The models to be simulated are $(||,\modelInc = \SI{0}{\degree})$, $(||,\modelInc = \SI{90}{\degree})$, $(\!\times\!, \modelInc = \SI{0}{\degree})$, and $(\!\times\!,\modelInc = \SI{90}{\degree})$ with a fraction of $\modelPsi = \SI{0.5}{}$, since these configurations represent the extrema for two fiber populations.
For statistic the simulations are repeated $\SI{25}{}$ times at the positions
\begin{align}
\begin{split}
    x&=\frac{i\,\%\,5}{5} \cdot \SI{60}{\micro\meter} - \SI{30}{\micro\meter} \\
    y&=\frac{\floor{i/5}}{5} \cdot \SI{60}{\micro\meter} - \SI{30}{\micro\meter} 
\end{split}
\end{align}
% 
% 
% 
\paragraph{Evaluation}
%
\begin{figure}[!t]
% 2_simulation/0_parameter/fiber_radii.py
\centering
\includegraphics[width=\textwidth, page=1]{gfx/rc1/voxel_size/voxel_size_plots_data_r05_output_vs_135_0.01_6_25_vervet_r_rc1.pdf}
\caption{The mean difference is constant for smaller voxel sizes and starts to grow significantly only from $\voxelsize=\SI{0.1}{\micro\meter}$.}
\label{fig:voxelsizeNoise}
\end{figure}
% 
The results in \cref{fig:voxelsizeNoise} show that the relative difference to the smallest voxel size increases statically significantly from a value of $\SI{0.1}{\micro\meter}$.
The variance of the relative difference increases with increasing voxel size from this value.
Smaller than $\SI{0.1}{\micro\meter}$, the difference from the reference simulation does not increase.
Therefore, the voxel size $\voxelsize = \SI{0.1}{\micro\meter}$ is a good compromise between runtime and accuracy.
One has to keep in mind that this is only valid for a pixel size of $\SI{1.3}{\micro\meter}$ and fiber radii of $\SI{0.5}{\micro\meter}$.
% 
%
%
\section{Simulation}
%
\subsection{Setup}
\label{sec:simParameterEnv}
% 
%
As described in \cref{chap:Software}, \ac{fastPLI} contains a pipeline implementation of the simulation with automatic analysis.
This implementation is used to compute the simulation of the signal for the flat and tilted measurements and apply the optical noise and tilt analysis to the resulting signal using the \ac{ROFL} algorithm. 
The input for the simulation is the previously created model library (see \cref{sec:genNerveFiberLibrary}).
The models are rotated according to the previously specified discretization (see \cref{sec:modelParamet}).
\par
% 
Three types of simulations are performed. First, a single inclined fiber population.
Since there is only a single fiber population, only the inclination of the model is changed.
In the second step, simulations are performed with two fiber population orientations: a) flat crossing and b) inclined crossing.
The first fiber population is fixed along the x-axis in these models, while the crossing angle of the second fiber population is increased.
The last simulation contains two fiber populations where the inclination of the first fiber population is increased and the second fiber population is rotated around it as the crossing angle increases (see \dummy{}).
\par
% 
\begin{table}[!b]
\caption{Simulation parameters. \todo{check units}}
\centering
% \sisetup{open-bracket={\{}, close-bracket={\}}, list-final-separator={,},list-pair-separator={,}}%
\pgfplotstabletypeset[%
    thesisTableStyle,
    column type=l,
    columns/variable/.style={string type},
    columns/value/.style={string type},
    every head row/.style={before row=\toprule,after row=\midrule},
    every last row/.style={after row=\bottomrule},
    col sep=&,
    row sep=\\,
]
{variable & value\\
%
simpli.voxel\_size & $\SI{0.1}{\micro\meter}$\\
simpli.pixel\_size & $\SI{1.3}{\micro\meter}$\\
simpli.voi & $[[\SI{-35}{\micro\meter}, \SI{-35}{\micro\meter}, \SI{-30}{\micro\meter}], [\SI{35}{\micro\meter}, \SI{35}{\micro\meter}, \SI{30}{\micro\meter}]]$\\
simpli.filter\_rotations & $\SIlist{0;20;40;60;80;100;120;140;160}{\degree}$\\
simpli.interpolate & \texttt{"Slerp"}\\
simpli.wavelength & $\SI{525}{\nano\meter}$\\
simpli.optical\_sigma & $\SI{0.75}{\pixel}$\\
tilt angle & $\SI{3.9}{\degree}$\\
simpli.light\_intensity & $\SI{8000}{\arbitraryunit}$\\
gain & $\SI{0.1175}{}$\\
simpli.noise\_model & \code{lambda x: np.floor(np.random.normal(x,}\\
 & \ \ \ \ \ \code{np.sqrt(gain * x))+0.5).astype(}\\
 & \ \ \ \ \ \ \ \ \ \ \ \ \ \ \ \ \ \ \ \code{np.uint16)}\\
fiber absorption & Vervet: $\SI{30}{\milli\meter\tothe{-1}}$\\
%rodent: $\SI{14}{\milli\meter\tothe{-1}}$, human: $\SI{60}{\milli\meter\tothe{-1}}$\\
fiber birefringence model & 'r'\\
fiber birefringence & 0.008\\
fiber radii scale factor & axon: $\SI{0.75}{}$, myelin: $\SI{1}{}$\\
model inclination step $\Delta\modelInc$ & one f. pop.: $\SI{5}{\degree}$, two f. pop.: $\SI{30}{\degree}$\\
model rotation step $\Delta\modelRot$ & $\SI{15}{\degree}$\\
}
\label{tab:simParameters}
\end{table}
% 
\Cref{tab:simParameters} lists the parameters of the simulation in the notation of \ac{fastPLI}.
With a pixel size of $\SI{1.3}{\micro\meter}$ and a volume of $\SI{65}{\micro\meter} \times \SI{65}{\micro\meter} \times \SI{60}{\micro\meter}$, $\SI{2500}{\pixel}$ are available per simulation for statistical analysis.
\par
% 
To analyze the simulation results the standard \ac{3D-PLI} analysis pipline, containing the foure analysis (see \dummy{}) and the tilt analysis (see \dummy{}), are performed.
From the fourie analysis the modalities transmittance and retardation are shown.
The tilt analysis \dummy{} the orientation of each pixel, \ie{}, direction and inclination, the effective birefringend thickness \trel{} and the \rvalue{} of the fitting model:
\begin{align}
    R = \mean{|I_i-F_i|} \label{eq:rvalue}
\end{align}
where $I$ is the signal intensity and $F$ the models found best solution.
Finally from the orientations the opening angle \openingAngle{} and angular correlation coefficient \accvalue{} between the models orientation and simulation resulting orientation are calculated.
The \accvalue{} is the angular correlation coefficient which measures the similarity between two \acp{ODF}. The \acp{ODF} are calculated by \dummy{}.
The \accvalue{} is then be calculated by:
\begin{align}
    acc = \frac{\sum_i{\mathit{sh}_{0,i} \cdot \mathit{sh}_{1,i}}}{\sqrt{\sum_i{\mathit{sh}_{0,i}^2}} \cdot \sqrt{\sum_i{\mathit{sh}_{1,i}^2}}}
\end{align}
where $\mathit{sh}_i$ is the $i-$th spherical coefficient of the orientations \ac{ODF} \cite{Schilling2018}.
\par
%
To visualize the distribution of the data boxplots are used.
For the distribution of the angular parameters these, however, have to be handled with care, since the values are periodic.
To overcome this problem the values are centered around the circular mean value in the range of $\big[\circmean (\xi)-\SI{90}{\degree}, \circmean(\xi)+\SI{90}{\degree}\big)$, where $\xi$ is the angular parameter.
This contributes to the interpretability of the angular results with variances $\ll \SI{90}{\degree}$.
Additionally for the angular parameters the distribution of the underlying fiber models are plotted for the individual fiber populations \popa{} and \popb{}.
The area corresponds to the $\SI{25}{\percent}$ to $\SI{75}{\percent}$ quantile and will be refered to as \bvariance{} variance.
% 
% 
% 
\subsection{Single inclined fiber population}
\label{sec:resSingleIncl}
%
\Cref{fig:single_fiber_pop_hist} shows the orientation distribution of a single population of fibers with increasing inclination angle $\modelInc{}$.
% 
\begin{figure}[!t]
    \centering
    \includegraphics[page=1]{gfx/rc1/analysis/plots_single_pop_hist_output_cube_2pop_135_rc1_single.pdf}
    \caption{Single inclined fiber population orientation histograms. \todo{alpha, bzw theta etwas größer?}}
    \label{fig:single_fiber_pop_hist}
\end{figure}
% 
The distributions of the orientations of the model segments are shown on the left, those of the tilt analysis on the right.
The results of the tilt analysis show the same mean orientation as the models with a slightly lower variance in comparison.
\par
% 
\Cref{fig:single_fiber_pop_rofl} shows the results of the simulation on the single inclined fiber models.
\par
% 
\begin{figure}[!p]
    \centering
    \includegraphics[]{gfx/rc1/analysis/plots_single_pop_output_cube_2pop_135_rc1_single.pdf}
    \caption{Single inclined fiber population. Results of the simulation analysis with comparison of the fiber models orientations.}
    \label{fig:single_fiber_pop_rofl}
\end{figure}
% 
%Transmittance
The transmittance is constant with increasing inclination angle up to a value of $\modelInc<\SI{75}{\degree}$.
For $\modelInc \ge \SI{75}{\degree}$ the transmittance increases.
The variance of the transmittance increases slightly with increasing inclination angle \modelInc{}.
\par
%
% Retardation
The retardation plot shows the fourie analysis results as well as a theoretical curve that follows $(\cos(\modelInc) + 1) / 2 \cdot \mean\left(\mathrm{ret}(\SI{0}{\degree})\right)$.
The curve is normalized by $\mean\left(\mathrm{ret}(\SI{0}{\degree})\right)$ for comparison purposes.
The retardation of the simulation signal follows the theoretical line.
However, at the intermediate inclination angles, the measured retardation is slightly higher than the theoretical line.
The variance remains constant for all fiber inclination angles \modelInc{} with the exception for very steep fibers, where the distribution is shifted towords values $>0$.
\par
% 
% Direction
The next graph shows the measured direction from the tilt analysis.
For inclination angles $\modelInc<\SI{75}{\degree}$ the variance is about a few degree and remains constant.
For larger values, the variance increases.
\footnote{It should be noted that the boxplot does not take into account that the values are periodic.
Therefore, especially for $\modelInc=\SI{90}{\degree}$, the values from the $\SIrange{-90}{90}{\degree}$ are uniformly distributed.}
In the background the statistic \bvariance{} variance of the fiber models is plotted.
The measured direction are in agreement with the models variance.
\par
%
% Inclination
The inclination shows the results of the tilt analysis.
The median follows the theoretical curve.
For inclinations $\SI{80}{\degree}$ and $\SI{85}{\degree}$, the variance increases.
For $\modelInc = \SI{90}{\degree}$ the variance is similar to the flat results.
The models \bvariance{} variance is constant and with an exception of $\modelInc=\SI{75}{\degree}$ higher than the tilt analysis results.
\par
%
% trel
The mean value of the relative birefringence thickness \trel{} is stable for values up to $\modelInc < \SI{60}{\degree}$.
The variance increases slightly with increasing inglincation angle \modelInc{}.
From there on the \trel{} value decrasisn up to an inclination angle of $\modelInc=\SI{85}{\degree}$ and increases significant for $\modelInc = \SI{90}{\degree}$.
The variance for $\SIlist{85,90}{\degree}$ is significantly increased and contain values as outliers $\gg 1$.
\par
% 
% R
The \rvalue{} remains constant for inclination angles $\modelInc < \SI{80}{\degree}$.
In the case of $\modelInc=\SIrange[]{80}{90}{\degree}$ the values are slightly increased.
Its variance remains stable for all inclinations.
\par
%
% domega
The opening angle \openingAngle{} is constant up to an inclination angle of $\modelInc \le \SI{70}{\degree}$.
For higher inclination angles, both the median and the variance increase significantly.
In the case of $\modelInc = \SI{90}{\degree}$ the median as well as the variance is very low.
In comparison the fiber models opening angle \openingAngle{} show a significant higher median and higher \bvariance{} variance with the exception of the $\modelInc=\SI{85}{\degree}$.
\par
% 
%ACC
The \accvalue{} remains close to $1$ and starts to decrease significantly at an inclination angle of about $\SI{60}{\degree}$.
It reaches a minimum at an inclination angle of $\SI{80}{\degree}$, from where its increases again.
At an inclination angle of $\SI{90}{\degree}$ the maximum \accvalue{} is reached.
%
%
%
\subsection{Crossing flat fiber populations}
\label{sec:resCrossFlat}
%
This simulation focuses on two flat crossing fiber populations, \ie{}, in the $x-y$-plane.
\Cref{fig:flat_03_fiber_pop_hist,fig:flat_05_fiber_pop_hist} shows the distribution of orientations for the model fiber segments and the resulting tilt analysis orientations for $\modelPsi = \SI{30}{\percent}$ and $\modelPsi = \SI{50}{\percent}$.
The additional fiber population fractions are available in \cref{app:flat_fiber_pop_a,app:flat_fiber_pop_b,app:flat_fiber_pop_c,app:flat_fiber_pop_d,app:flat_fiber_pop_e}.
\par
% 
\begin{figure}[!p]
\centering
\includegraphics[page=3]{gfx/rc1/analysis/plots_flat_pop_hist_output_cube_2pop_135_rc1_flat.pdf}
\caption{Crossing flat fiber population: $\modelPsi=\SI{30}{\percent}$.}
\label{fig:flat_03_fiber_pop_hist}
% 
\vspace{2em}
\includegraphics[page=5]{gfx/rc1/analysis/plots_flat_pop_hist_output_cube_2pop_135_rc1_flat.pdf}
\caption{Crossing flat fiber population: $\modelPsi=\SI{50}{\percent}$.}
\label{fig:flat_05_fiber_pop_hist}
\end{figure}
%
The results show that the measured orientation follows the dominant fiber population.
Depending on the fiber population fraction $\modelPsi$, the resulting orientation is closer or further away from one of the fiber populations.
In the case of $\modelPsi = \SI{50}{\percent}$, the measured orientation lies between the two fiber populations, except for a crossing angle of $\modelOmega=\SI{90}{\degree}$, where the measured orientations are randomly distributed around steep orientations.
\par
% 
\Cref{fig:flat_03_fiber_pop_rofl,fig:flat_05_fiber_pop_rofl} show the results of the \ac{3D-PLI} modalities, the tilt analysis, opening angle $\openingAngle$ and \accvalue{} in detail.
\par
% 
\begin{figure}[!p]
\centering
\includegraphics[page=3]{gfx/rc1/analysis/plots_flat_pop_output_cube_2pop_135_rc1_flat.pdf}
\caption{Crossing flat fiber population: $\modelPsi=\SI{30}{\percent}$. Results of the simulation analysis with comparison of the fiber models orientations.}
\label{fig:flat_03_fiber_pop_rofl}
\end{figure}
%
\begin{figure}[!p]
\centering
\includegraphics[page=5]{gfx/rc1/analysis/plots_flat_pop_output_cube_2pop_135_rc1_flat.pdf}
\caption{Crossing flat fiber population: $\modelPsi=\SI{50}{\percent}$. Results of the simulation analysis with comparison of the fiber models orientations.}
\label{fig:flat_05_fiber_pop_rofl}
\end{figure}
% 
% Transmittance
The transmittance increases with increasing crossing angle \modelOmega{}.
However, the rate of increase becomes flatter with increasing crossing angle.
In the case of $\modelPsi = \SI{30}{\percent}$, the transmittance increases about $\SI{9}{\percent}$ wheres the increase for $\modelPsi = \SI{50}{\percent}$ is about $\SI{10}{\percent}$.
\par
% Retardation
The retardation is negatively linearly correlated with increasing crossing angle.
The retardation starts at $\SI{0.8}{}$ and drops in the case of For $\modelPsi=\SI{30}{\percent}$ to about $\SI{0.3}{}$.
In the case of $\modelPsi=\SI{50}{\percent}$ it drops close to $\SI{0}{}$.
The variance of the retardation is slightly increased for fiber population fractions close to $\SI{10}{\percent}$ and $\SI{90}{\percent}$ around $\modelOmega=\SI{45}{\degree}$.
\par
% Direction
The direction analysis for all simulated fiber population fractions \modelPsi{} follows the theoretical \name{circmean} function.
In the case for $\modelPsi<\SI{50}{\percent}$ the median is slighly higher than the theoretical prediction and in the case for $\modelPsi>\SI{50}{\percent}$ slightly lower.
Depending on the fiber population fraction the variance increases with the crossing angle.
In the case of $\modelPsi=\SI{50}{\percent}$ and a crossing angle of $\modelRot = \SI{90}{\degree}$, the directions are uniformly distributed $\SIrange{-90}{90}{\degree}$.
The distribution of the individual nerve fiber populations show the seperation of the orientation with increasing crossing angle \modelOmega{}.
The models \bvariance{} variance for both populations remain constant.
The dominant fiber population has a slightly lower variance than its counterpart.
\par
% Inclination
The inclination median is about $\SI{0}{\degree}$ for all simulated parameters.
The variance increases with increasing crossing angle \modelOmega{} and slightly for fiber population fractions close to $\modelPsi=\SI{50}{\percent}$.
The models \bvariance{} variance of both fiber models increases with increasing crossing angle and is slighly larger for the domitent fiber population.
Compared to the simulation results the \bvariance{} variance is significantly higher.
\par
% Trel
The relative birefringence thickness \trel{} decreases with increasing crossing angle \modelOmega{}.
The shape of the curve correlates negatively with increasing crossing angle.
The \trel{} values start at about $\SI{0.6}{}$ and decrease to $\trel=\SI{0.2}{}$ for $\modelPsi = \SI{30}{\percent}$ and to nearly $\trel=\SI{0}{}$ for $\modelPsi = \SI{50}{\percent}$ depending on the fiber population fraction.
For both fiber population fractions, the variance increases slightly.
For $\modelPsi = \SI{50}{\percent}$ and $\modelOmega = \SI{90}{\degree}$, the variance of \trel{} increases significantly and outliers occur reaching \trel{} values $>\SI{1}{}$.
\par
% rvalue
The \rvalue{} is nearly constant for all crossing angles \modelOmega{} and fiber population fractions \modelPsi{}. The variance show no significant change.
\todo{-> discuss very hard to find "right" solution, maybe with prior.}
\par
% domega
The opening angle \openingAngle{} is similar for all fiber population fractions up to a crossing angle of $\modelOmega=\SI{50}{\degree}$.
For higher crossing angles and fiber population fractions closer to $\modelPsi = \SI{50}{\percent}$, the median and variance of the opening angle increase significantly.
The fiber models median \openingAngle{} is significantly higher then the results from the tilting analysis.
The dominant fiber population has a lower median and \bvariance{} variance than its counterpart.
\par
% acc
The \accvalue{} starts closes to to a value of $\SI{1}{}$ and decreases up to a crossing angle of about $\modelOmega \approx \SI{60}{\degree}$ depending on the fiber population fraction \modelPsi{}.
After the minimum is reached starts to increases significantly and with the exception for a fiber population fraction of $\modelPsi=\SI{50}{\percent}$ reaching a value closer to $\SI{1}{}$ again.
\todo{discuss -> look at direction, it gets closer to dominant fiber pop again.}
% 
% 
%
\subsection{Inclined crossing fibers population}
\label{sec:resInclCross}
% 
This last special model configuration show the results for the first fober population \popa{} along the x-axis and the second fiber population inclined along the x-z-axis.
The seconds fiber population inclination angle $\inc_{\popb}$ will be identical with the crossing angle \modelOmega{} between both fiber populations.
\par
%
The orientation histograms of the inclined models and their simulation results are shown in \cref{fig:inclined_03_fiber_pop_hist,fig:inclined_05_fiber_pop_hist} and the remaining results are available in \cref{app:incl_fiber_pop_hist}.
The orientation of the tilt analysis of the simulation follows the inclination of the models, with the exception for an inclination angle $\inc_{\popb} = \SI{90}{\degree}$.
The orientation is shifted towords the more dominant fiber population.
The fiber models orientation show a higher variance as the tilt analysis results.
\par
%
\begin{figure}[!p]
\centering
\includegraphics[page=3]{gfx/rc1/analysis/plots_inclined_pop_hist_output_cube_2pop_135_rc1_inclined.pdf}
\caption{Population of inclined crossing fibers: $\modelPsi=\SI{30}{\percent}$.
}
\label{fig:inclined_03_fiber_pop_hist}
% 
\vspace{2em}
\includegraphics[page=5]{gfx/rc1/analysis/plots_inclined_pop_hist_output_cube_2pop_135_rc1_inclined.pdf}
\caption{Population of inclined crossing fibers: $\modelPsi=\SI{50}{\percent}$.
}
\label{fig:inclined_05_fiber_pop_hist}
\end{figure}
%
\Cref{fig:inclined_03_fiber_pop_rofl,fig:inclined_05_fiber_pop_rofl} show the results of \ac{3D-PLI} modalities, the tilt analysis, the opening angle and the \accvalue{} in detail for a fiber population of $\SIlist{30,50}{\percent}$.% 
The remaining fiber population fraction are listed in \cref{app:incl_fiber_pop_a,app:incl_fiber_pop_b,app:incl_fiber_pop_c,app:incl_fiber_pop_d,app:incl_fiber_pop_e}.
\par
%
\begin{figure}[!p]
    \centering
    \includegraphics[page=3]{gfx/rc1/analysis/plots_inclined_pop_output_cube_2pop_135_rc1_inclined.pdf}
    \caption{Population of inclined crossing fibers: $\modelPsi=\SI{30}{\percent}$. Results of the simulation analysis with comparison of the fiber models orientations.}
    \label{fig:inclined_03_fiber_pop_rofl}
    \end{figure}
    %
    \begin{figure}[!p]
    \centering
    \includegraphics[page=5]{gfx/rc1/analysis/plots_inclined_pop_output_cube_2pop_135_rc1_inclined.pdf}
    \caption{Population of inclined crossing fibers: $\modelPsi=\SI{50}{\percent}$. Results of the simulation analysis with comparison of the fiber models orientations.}
    \label{fig:inclined_05_fiber_pop_rofl}
\end{figure}
% 
% Transmittance
The transmittance increases with increasing inclination angle $\inc_{\popb}$.
Fiber population fraction close to $\SI{50}{\percent}$ tend to have a higher transmittance value.
The increase of the transmittance slows down for higher inclination angles  $\inc_{\popb}$.
\par
% Retardation
The retardation is negatively linearly correlated with increasing inclination angle $\inc_{\popb}$.
The slope of the negative correlation is higher with increasing fiber population fraction \modelPsi{}.
The variance increases as the inclination angle increases and increasing fiber population fraction.
\par
% Direction
The direction value is centered around $\SI{0}{\degree}$.
The variance of the direction increases with increasing inclination angle $\inc_{\popb}$.
The fiber popilation fraction has no significant effect except for a value of $\modelPsi = \SI{10}{\percent}$ and an inclination angle of $\inc_{\popb}=\SI{90}{\percent}$.
The models \bvariance{} variance of the first (flat) fiber population \popa{} is always smaller than the \bvariance{} variance of the second fiber population \popb{}.
The \bvariance{} variance for \popa{} increases up to an inclination angle of about $\inc_{\popb} = \SI{50}{\degree}$ and remains constant for higher inclination angles, whereas the \bvariance{} variance of the second fiber bundle increases further.
\par
% Inclination
The inclination results have the shape of a convex curve, starting and ending at a value of $\inc\SI{0}{\degree}$.
Depending on the fiber population fraction \modelPsi{} the maximum of the curve is shifted towords an inclination angle of $\SI{90}{\degree}$ for a low \modelPsi{} and towortds $\SI{45}{\degree}$ for a high \modelPsi{}.
The \bvariance{} of the individual nerve fiber populations remain constant for inclination angles $\geq \SI{10}{\degree}$. 
For lower fiber population fractions \modelPsi{} and higher inclination angles $\inc_{\popb}$ the variance of the simulation results is higher than the \bvariance{} of the models.
Otherwise the simulations inclination variance is significantly lower.
\par
% Trel
The relative thickness \trel{} decreases linearly with increasing inclination angle $\inc_{\popb}$.
The \trel{} value start at about $\SI{0.6}{}$ and drops depending on the fiber population fraction \modelPsi{} to $\SI{0.45}{}$ for $\modelPsi=\SI{90}{\percent}$ to almost $\SI{0}{}$ for $\modelPsi=\SI{10}{\percent}$.
In variance of the \trel{} value is increased at about $\inc_{\popb}=\SI{45}{\degree}$ for lower fiber population fractions and towords $\inc_{\popb}=\SI{90}{\degree}$ for high fiber population fractions.
\par
% r-value
The \rvalue{} show no significant effect for low fiber population fraction. 
Higher fiber poulation fractions have an increased \rvalue{}.
With increasing inclination angle $\inc_{\popb}$ the \rvalue{} and its variance increases slightly for lower fiber population fractions and more strongly for higher fiber population fractions..
\par
% domega
The measured opening angle \openingAngle{} mean value as its variance increases slightly with increasing inclination angle.
For lower fiber population fractions and a crosing angle $\inc_{\popb} \geq \SI{80}{\degree}$ the increas is significant more.
The fiber model opening angles are with the median as well as the \bvariance{} variance significantly higher then the simulation results for both fiber population fractions.
The median and \bvariance{} variance of the opening angle of the dominant fiber population is higher as its counterpart.
\par
% acc-value
The \accvalue{} show a decrease with increasing inclination angle to a minimum value and from there the \accvalue{} raises agein..
The position of the minimum depends strongly on the fiber population fraction.
For lower \modelPsi{} the minimum is close to higher inclination angles and for higher \modelPsi{} the minimum moves to an inclination angle of about $\inc_{\popb} \approx \SI{45}{\degree}$.
The value of the \accvalue is closer to zero for median fiber population fractions an closer to one for \modelPsi{} at $\SI{10}{\percent}$ and $\SI{90}{\percent}$.
%
% 
% 
\subsection{Free crossing fiber populations}
\label{sec:resFreeCross}
%
In this section, two nerve fiber populations \popa{} and \popb{} are simulated with sampled orientations on a sphere (see \cref{fig:twomodelpopdesign_a,tab:simParameters})
A polar plot visualization is used to visualize the results from the \ac{3D-PLI} analysis pipeline (see \cref{fig:simModelSphereSample_b}).
The spherical plots in %\cref{fig:sim_ana_trans,fig:sim_ana_ret,fig:sim_ana_dir,sim_ana_incl,sim_ana_trel,sim_ana_rvalue,fig:sim_ana_acc}
are designed so that the thick black circle shows the orientation of the first fiber population \popa{} and the thin dashed circles show the orientation of the second fiber population \popb{}.
At the position of the second fiber population, the resulting average value is visualized.
A spherical k-nearest neighbor interpolator is used to visualize values between the sampled orientations for none angular data \cite{DBLP:journals/corr/abs-1910-00704}.
\par
% 
\begin{figure}[t]
\centering
\setlength{\tikzwidth}{0.40\textwidth}
\subcaptionbox{\label{fig:simModelSphereSample_a}The spheres surface is sampled around the first fiber population axis $\popa$ of inclination $\modelInc$ with a crossing angle $\modelOmega$ and a rotation $\modelRot$.}
[.475\textwidth]{\inputtikz{gfx/model/sphere_models_c}}
\hfill
\subcaptionbox{\label{fig:simModelSphereSample_b}
    A polar diagram is used to illustrate the scalar results of a fixed orientation of the first fiber population \popa{} and a variable orientation of the fiber population \popb{}.
    The orientation \popa{} is marked with a black thick circle.
    The second orientation is indicated by dashed circles.
    For visualization, the scalar values are interpolated according to a spherical k-nearest neighbor interpolator algorithm.
}
[.475\textwidth]{\inputtikz{gfx/model/sphere_hist_b}}
\caption{Visualization of results for a fixed first fiber population \popa{}.}
\label{fig:simModelSphereSample}
\end{figure}
%
%
%
\paragraph{Transmittance}
% 
The average transmission value is visualized in \cref{fig:sim_ana_trans}.
The first fiber population fraction $\modelPsi=\SI{10}{\percent}$ is almost identical for all inclination angles $\modelInc$.
The transmittance is lowest for models with a crossing angle of $\modelOmega = \SI{0}{\degree}$ and for flat configurations.
\par
% 
\begin{figure}[!p]
\centering
\includegraphics[]{gfx/rc1/analysis/simulation_analysis_hist_0.5_setup_PM_s_Vervet_m_r_trans_mean.pdf}
\caption{Mean transmittance value.}
\label{fig:sim_ana_trans}
\end{figure}
%
For a fiber population fraction of $\modelPsi=\SI{30}{\percent}$, the transmittance increases.
The same behavior holds for $\modelPsi=\SIlist{50,70}{\percent}$.
$\modelPsi=\SI{50}{\percent}$ achieves the highest transmittance for fiber models with crossings.
\par
% 
The average transmittance for the last fiber population fraction $\modelPsi=\SI{90}{\percent}$ is reduced similar to $\modelPsi=\SI{10}{\percent}$.
Fiber configurations with no crossing remain the lowest transmittance for all inclinations $\modelInc$.
%
% 
% 
\paragraph{Retardation}
% 
Retardation results are shown in \cref{fig:sim_ana_ret}.
For a fiber population fraction of $\modelPsi=\SI{10}{\percent}$, the lowest retardation values are present for secondary fiber populations with an inclination of $\SI{90}{\degree}$.
Configurations with a high crossing angle $\modelOmega$ are slightly reduced.
Flat fiber configurations retain a relative high retardation value.
\par
% 
\begin{figure}[!p]
\centering
\includegraphics[]{gfx/rc1/analysis/simulation_analysis_hist_0.5_setup_PM_s_Vervet_m_r_ret_mean.pdf}
\caption{Mean retardation values.}
\label{fig:sim_ana_ret}
\end{figure}
%
The same behavior is also visible for fiber population fraction $\modelPsi=\SIlist{30;50;70}{\percent}$, but the retardation values are significantly reduced, especially for high first fiber population inclinations \modelInc{}.
The reduction in retardation at high crossing angles $\modelOmega$ is also more pronounced, especially for $\modelPsi=\SI{50}{\percent}$.
For fiber population fractions $\modelPsi \geq \SI{50}{\percent}$ and first fiber population inclinations $\modelInc \geq \SI{60}{\degree}$, the retardation is overall very low.
\par
%
For $\modelPsi=\SI{90}{\percent}$, no reduction of retardation is visible for any secondary fiber orientation.
The retardation decreases significantly with increasing inclination $\modelInc$ until it is close to $\SI{0}{}$ for $\modelInc=\SI{90}{\degree}$.
% 
% 
% 
\paragraph{Direction $\dir$ and inclination $\inc$}
%
The results of the circular mean of direction are shown in \cref{fig:sim_ana_dir} and the circular mean of the inclination in \cref{fig:sim_ana_incl}.
An interpolation method is not used for the angular values because the required interpolation on a hemispherical manifold is not easily accomplished.
The k-nearest neighbor interpolator used above would lead to incorrect results.
\par
% 
\begin{figure}[!p]
    \centering
    \includegraphics[]{gfx/rc1/analysis/simulation_analysis_hist_0.5_setup_PM_s_Vervet_m_r_rdir_mean.pdf}
    \caption{Circmean direction $\dir$ from tilt analysis.}
    \label{fig:sim_ana_dir}
\end{figure}
% 
\begin{figure}[!p]
    \centering
    \includegraphics[]{gfx/rc1/analysis/simulation_analysis_hist_0.5_setup_PM_s_Vervet_m_r_rincl_mean.pdf}
    \caption{Circmean inclination $\inc$ from tilt analysis. Because the direction is defined in the range of $[\SI{0}{\degree},\SI{180}{\degree})$ and the inclination $[\SI{-90}{\degree},\SI{90}{\degree})$ the sign of the inclination value flips for the lower quadrants in the polar plots.}
    \label{fig:sim_ana_incl}
\end{figure}
% 
Both angles show a value pattern following the dominant fiber population.
For steep fibers with a $\modelInc = \SI{90}{\degree}$, the orientation results follow the second fiber population orientation, regardless of the fiber population fraction \modelPsi{}.
For fiber population fractions close to $\SI{50}{\percent}$, the resulting orientations behave according to an averaging of the two individual orientations with respect to the fiber population fraction.
From the direction it can be seen that for models with $\modelPsi = \SI{90}{\degree}$ and $\modelInc \leq \SI{60}{\degree}$ as well as $\modelPsi = \SI{70}{\degree}$ and $\modelInc \leq \SI{30}{\degree}$ the second fiber population has no or only a very small influence on the measured direction.
The same is true for inclinations for a $\modelInc = \SI{0}{\degree}$ and $\modelPsi \geq \SI{50}{\percent}$.
% 
% 
% 
\paragraph{Effective birefringence thickness \trel{}}
% 
The results of \trel{} are shown in \cref{fig:sim_ana_trel}.
The \trel{} value for the first fiber population fraction $\modelPsi=\SI{10}{\percent}$ 
is nearly constant for all fiber configurations.
The exceptions are $\modelInc=\SIlist{60;90}{\degree}$ with secondary fiber population inclinations of $\inc_1 = \SI{90}{\degree}$.
\par
%
\begin{figure}[!p]
    \centering
    \tikzset{external/export next=false}
    \begin{tikzpicture}[]  
        \node[inner sep=0pt] (fig) at (0,0)
        {\includegraphics[]{gfx/rc1/analysis/simulation_analysis_hist_0.5_setup_PM_s_Vervet_m_r_rtrel_mean.pdf}};
        \node[inner sep=0pt,fill=white,shift={(-1.8825,0.20)}] at (fig.south east) {\scriptsize$\geq1$}; % fill=red,text=red,inner sep=0pt
    \end{tikzpicture}
    \caption{Mean \trel{} from tilt analysis.}
    \label{fig:sim_ana_trel}
\end{figure}
% 
Areas of reduced \trel{} become visible for $\modelPsi=\SI{30}{\percent}$.
For $\modelInc=\SI{0}{\degree}$ this is located at a secondary fiber inclination of $\inc_1 = \SI{90}{\degree}$.
For $\modelInc=\SIlist[]{30;60}{\degree}$ the reduction of \trel{} is slighly towords the negatice x-axis next to the z-axis.
For $\modelInc=\SI{90}{\degree}$ the \trel{} value remains constant with the exception of both fiber populations orientated along the z-axis, where $\trel \geq 1$.
\par
% 
For a fiber population fraction of $\modelPsi=\SI{50}{\percent}$ a band of reduction in the \trel{} value is visible for a crossing angle of $\modelOmega=\SI{90}{\degree}$.
This \say{band} shaped area of reduced \trel{} values is moved towords the negative x-axis for $\modelInc=\SIlist{30;60}{\degree}$.
Additionally for $\modelInc=\SI{60}{\degree}$ two significant increased \trel{} values $\geq 1$ are visible at $\dir=\SIlist{90;270}{\degree}$ and $\inc=\SI{60}{\degree}$.
For $\modelInc=\SI{90}{\degree}$ the \trel{} value remains constant, but more decreased than before, with the exception of both fiber populations orientated along the z-axis, where the \trel{} value is $\geq 1$.
\par
% 
$\modelPsi=\SI{70}{\percent}$ shows the same behavier than for the $\modelPsi=\SI{50}{\percent}$ case, however, the reduction is not as strong as in the privious case.
The exception is for very steep primary fibers, where the \trel{} values are significant reduced.
Again, the \trel{} value along the z-axis, and its four closes neighbors are $\geq 1$. 
\par
% 
The last line shows the results for the fiber population fraction $\modelPsi=\SI{90}{\percent}$.
All models and configurations remain at a stable \trel{} value.
The exception is the fiber configurations with $\modelInc=\SI{90}{\degree}$, where the \trel{} values along the z-axis with inclinations $\inc_1 \geq \SI{45}{\degree}$ are $\geq \SI{1}{}$
% 
% 
%
\paragraph{R-value}
% 
The \rvalue{} of the tilt analysis fitting model is shown in \cref{fig:sim_ana_rvalue}.
It describes the mean absolute difference between the data and the fitted data from the tilt analysis (see \cref{eq:rvalue}).
\par
% 
\begin{figure}[!p]
\centering
\includegraphics[]{gfx/rc1/analysis/simulation_analysis_hist_0.5_setup_PM_s_Vervet_m_r_R_mean.pdf}
\caption{Mean \rvalue{} from tilt analysis.}
\label{fig:sim_ana_rvalue}
\end{figure}
%
For a fiber population fraction of $\modelPsi=\SI{10}{\percent}$ and a primary inclination angle $\modelInc\leq\SI{30}{\degree}$ the orientation of the second fiber population does not change the \rvalue{} of the tilt analysis.
For $\modelInc=\SI{60}{\degree}$ the \rvalue{} starts increasing for secondary fiber orientations towards the xy-plane, \ie{}, secondary flat fiber orientations.
The \rvalue{} increases even more significant in the case of $\modelInc=\SI{90}{\degree}$ in this area. 
\par
% 
For fiber population fractions $\modelPsi=\SI{30}{\percent}$ and $\modelInc=\SI{0}{\degree}$, the \rvalue{} remains low for flat crossing fiber configurations.
Along the z-axis the \rvalue{} is slightly increased.
For $\modelInc=\SI{30}{\degree}$ an slightly increased bend shape becomes visible close to the y-axis slightly towords the negative x-axis, \ie{}, $\modelOmega \approx \SI{90}{\degree}$.
Inclination angles of $\SIlist{60;90}{\degree}$ show the same behavier as for the $\modelPsi=\SI{10}{\percent}$ case, however the \rvalue does not increases as much.
\par
%
Fiber population fractions $\modelPsi=\SI{50}{\percent}$ and inclinations of $\modelInc=\SI{0}{\degree}$, have an increased \rvalue{} for inclined secondary fiber configurations along the y-axis.
For $\modelInc=\SI{0}{\degree}$, the increased area along the y-axis is curved in the direction of $\SI{180}{\degree}$.
$\modelInc=\SI{60}{\degree}$ leaves a low \rvalue{} along the x-axis and an area around the first fiber population orientation.
The other secondary orientations are slightly increased.
The steep first fiber population inclinations of $\modelInc=\SI{90}{\degree}$ again has an increased \rvalue{} to the flat secondary orientations.
The \rvalue{} relative to the lower first fiber population fractions is further reduced.
\par
%
$\modelPsi=\SI{70}{\percent}$ and $\modelInc=\SI{0}{\degree}$ have only an increased \rvalue{} for the secondary fiber configuration along the z-axis.
The points along the x-axis are slightly more increased than along the y-axis.
For $\modelInc=\SI{30}{\degree}$, the curved shape from the previous result almost disappears.
Only the secondary fiber configurations along the z-axis remain increased.
$\modelInc=\SI{60}{\degree}$ remains a low \rvalue{} along the x-axis.
Only the crossings for flat directions of $\SIlist{110;250}{\degree}$ are increased.
The final inclination angle of $\modelInc=\SI{90}{\degree}$ is again similar to the previous fiber population fraction, but the \rvalue{} is again lower for the increased areas.
\par
% 
The final fiber population fraction $\modelPsi=\SI{90}{\percent}$ has an increased \rvalue{} on the z-axis for inclination angles of $\modelInc=\SIlist{0;30}{\degree}$.
$\modelInc=\SIlist{60;90}{\degree}$ remains a low \rvalue{} for all configurations.
% 
% 
% 
\paragraph{Opening angle \openingAngle{}}
% 
The opening angle \openingAngle{} is shown in \cref{fig:sim_ana_domega}.
\par
% 
\begin{figure}[!p]
\centering
\includegraphics[]{gfx/rc1/analysis/simulation_analysis_hist_0.5_setup_PM_s_Vervet_m_r_domega_mean.pdf}
\caption{Mean opening angle \openingAngle{} from tilt analysis.}
\label{fig:sim_ana_domega}
\end{figure}
% 
Overall the opening angle \openingAngle{} remain relativly low values for most fiber population fractions \modelPsi{} and inclination angles \modelInc{}.
For lower fiber population fractions than $\modelPsi < \SI{50}{\percent}$ the opening angles are reduced when the secondary fiber population is orientatiod along the z-axis.
\par
% 
A fiber population fraction of $\modelPsi = \SI{50}{\percent}$ with an inclination of $\modelInc=\SI{0}{\degree}$ has significant indreased opening angle for flat crossings.
For a primary inclination of $\modelInc=\SI{30}{\degree}$ an increased opening angle is visible for a crossing angle of $\modelOmega=\SI{90}{\degree}$.
$\modelInc=\SI{60}{\degree}$ show an area of increased opening angles close to the z-axis towords the negative x-axis.
$\modelInc=\SI{90}{\degree}$ shows no increased values.
% 
For a fiber population fraction of $\modelPsi=\SI{70}{\percent}$ the inclination angle of $\modelInc=\SI{60}{\degree}$ shows an increase in the opening angle along flat crossings along the y-axis.
$\modelInc=\SI{90}{\degree}$ shows an overall slightly increased value of the opening angle $\openingAngle$.
\par
% 
For the last shown fiber population fraction $\modelPsi=\SI{90}{\degree}$ the opening angle values are very small with the exception for the inclination angle of $\modelInc=\SI{90}{\degree}$ where for secondary fiber orientations towords the x-y-plane the opening angle significantly increases.
% 
% 
% 
\paragraph{acc-value}
% 
\Cref{fig:sim_ana_acc} shows the \acreset{ACC} \ac{ACC} value, \ie{}, how well the coefficients of the \ac{ODF} basis function are matched \cite{Schilling2018}.
\par
%
\begin{figure}[!p]
\centering
\includegraphics[]{gfx/rc1/analysis/simulation_analysis_hist_0.5_setup_PM_s_Vervet_m_r_acc.pdf}
\caption{Mean \accvalue{} between model and tilt analysis orientations.}
\label{fig:sim_ana_acc}
\end{figure}
% 
The fiber population fraction $\modelPsi=\SI{10}{\percent}$ shows no significant decrease in the \accvalue{} for the first fiber model inclination $\modelInc$.
For $\modelPsi=\SI{30}{\percent}$, a reduction of the \accvalue{} for the lower inclined models of the first fiber population is visible.
The reduction is found in the region of the steep secondary fiber populations.
For $\modelInc = \SIlist{30;60}{\degree}$, the reduction shifts in a direction along $\SI{180}{\degree}$.
The effective area of the reduction decreases with increasing \modelInc{} until no area effect is visible for an inclination of $\modelInc=\SI{90}{\degree}$.
However, the \accvalue{} for this last plot is still reduced compared to $\modelPsi=\SI{10}{\percent}$.
For $\modelInc < \SI{90}{\degree}$, a slight increase in the \accvalue{} is visible for flat fiber crossings with a crossing angle of $\modelOmega=\SI{90}{\degree}$.
\par
%
For the equally proportional fiber population $\modelPsi = \SI{50}{\degree}$, the \accvalue{} reaches values close to $\SI{0}{}$ for $\modelInc = \SIlist{0;30}{\degree}$ and a crossing angle of $\modelOmega=\SI{90}{\degree}$.
Inclined secondary fiber populations achieve a higher \accvalue{}.
The $\modelInc = \SI{90}{\degree}$ configuration reaches its maximum \accvalue{} at a crossing angle of $\modelOmega=\SI{0}{\degree}$ and lower stable values for all other configurations.
\par
%
The first to fiber inclinations $\modelInc = \SIlist{0;30}{\degree}$ for $\modelPsi=\SI{70}{\percent}$ show a similar distribution of the \accvalue{}. 
Small reductions are visible.
For $\modelInc=\SI{60}{\degree}$, the \accvalue{} is highest at low crossing angles $\modelOmega$, \ie{}, near the first fiber population.
The \accvalue{} decreases significantly at higher crossing angles.
In the case of $\modelInc=\SI{90}{\degree}$ this is also visible, but the minimum for the \accvalue{} is higher than in the previous case $\modelInc=\SI{60}{\degree}$.
\par
%
The last case $\modelPsi=\SI{90}{\percent}$ shows high values for the \accvalue{} parameter for the first three inclinations $\modelInc=\SIlist{0;30;60}{\percent}$.
The last inclination $\modelInc=\SI{90}{\degree}$ shows a slight decrease of \accvalue{} for higher crossing angles $\modelOmega$.
\par
%
In summary, crossing angles cause a significant reduction in \accvalue{}s.
This effect is clearly visible as a \say{band} of reduced \accvalue{}s, from a low inclination angle and low fiber proportion fraction to a high inclination angle and high fiber proportion fraction.
Fiber population fractions of $\modelPsi \geq \SI{90}{\percent}$ or $\modelPsi \leq \SI{10}{\percent}$ have almost no effect on the \accvalue{}.
%
% 
% 
\section{Discussion}
%
\paragraph{Single inclined fiber population}
\Cref{sec:resSingleIncl} shows the results for the case of a single fiber population with inclined configurations.
The transmittance values show an increase for the last free inclination values.
This is to be expected since the 3D model configurations are aligned in parallel along an axis with some randomness.
When this axis is aligned along the z-axis, \ie{}, $\modelInc=\SI{90}{\degree}$, the light rays, which also travel along the z-axis, statistically strike less tissue, so the transmittance must increase.
Since the density of the tissue is quite high, the effect is only a few percent.
In reality, however, one must take into account that additional effects such as light scattering also have a significant effect on the transmittance \cite{Menzel2021}.
The variance of the transmittance corresponds to the variance of the noise model.
\par
%
The retardation behaves similarly to the theoretical curve for a single retardation signal (see \cref{eq:pli_signal}).
However, the analyzed retardation is slightly higher.
The theoretical signal is calculated by normalizing $\cos(\modelInc)^2$ with the analyzed retardation for the flat case, \ie{}, $\modelInc = \SI{0}{\degree}$.
This could explain the reduced values of the theretical line compared to the simulation results for the mid range inclinations.
\par
%
The direction values have a very small variance, which increases strongly for high inclination values.
Since the direction value is coupled to the inclination angle, this is to be expected: for an $\SI{90}{\degree}$ inclined orientation, there is no direction or retardation.
Therefore, the noise affects the signal, and generates a random phase of the sinus.
The variance of the tilting analysis is expected to be smaller than the actual distribution of the individual nerve fiber segments because the signal results from the overall interaction from the light with all segments in its path.
\par
%
The same behavior can be observed for the inclination curve.
However, here the variance for the last angle is lower.
It may be that this is an effect that for exactly $\modelInc=\SI{90}{\degree}$ the change of the tilted light beam statistically ends in 4 opposite phases in the sinusoidal signal.
This might be more detectable for the tilt analysis than a slightly less inclined fiber configuration where only the amplitude changes.
It is interesting to note that the variance is quite small and constant up to a range of about $\SIrange{70}{75}{\degree}$.
This demonstrates that the slope analysis for a single fiber population is relatively accurate with respect to the dimension used here.
\par
%
The relative thickness \trel{} should theoretically be constant, since the volume fraction remains the same for different inclination angles.
Due to the reduction in retardation, a higher variance is expected with increasing inclination angle.
However, at an inclination angle of about $\modelInc=\SI{65}{\degree}$, the mean value begins to decrease.
For $\modelInc=\SIlist{85;90}{\degree}$, the variance then increases significantly and for $\modelInc=\SI{90}{\degree}$, the mean is $>\SI{1}{}$.
This indicates that the tilt analysis is unable to find a local minimum for \trel{} values $<1$.
This may be one of the following three reasons.
First, there is no local minimum with $\trel<1$.
Second, there is a local minimum, but the optimizer is not able to find it with the initial values.
Finally, there is a local minimum, but the solution found with a \trel{} value $>1$ is a better choice, \ie{}, has a lower $R^2$ value.
Which of these is true cannot be said from this data and remains an interesting investigation for the future.
However, up to this point, the \trel{} value can be used to indicate that these values cannot be trusted, \ie{}, the resulting orientation is not trustworthy.
\par
% 
The \rvalue{} reflects the noise on the signal for all inclination angles and stays constant.
\par
%
The opening angle shows the combined information from the direction results and the inclination results.
This value can possibly serve as additional uncertainty for further calculations like in a tractography.
The median value is expected to be lower than the actual fiber models opening angle for the same reason as for a lower \bvariance{} variance in the direction and inclination angle.
\par
%
The \accvalue{} is for fiber inclinations $\modelInc \leq \SI{75}{\degree}$ very high.
For an inclination angle of about $\modelInc = \SI{80}{\degree}$, a minimum is visible because for higher inclination angles the tilting analysis accuracy increases.
Therefore the \ac{ODF} similary has to also increase again.
\par
% 
The information about orientation, \ie{}, direction and inclination, is also visible in the histogram (\todo{ref}).
However, the statistical limits are not quite as easy to see, \eg median or quantiles.
Nevertheless, it is a good and fast way to give visual feedback to the user, especially because the information about direction and inclination is not decoupled.
\par
% 
In summary, for the case of a single fiber population, the resulting tilt analysis shows good agreement with the individual orientations of the model, except for very steeply inclined configurations $(\gtrsim \SI{70}{\degree})$.
%
% 
% 
\paragraph{Crossing flat fiber population}
\Cref{sec:resCrossFlat} describes the results of the flat crossing configurations.
The transmittance value changes significantly with increasing \modelOmega{}, which is to be expected since non overlapping crossing models require more space.
Therefore, fewer fibers can absorb light and the transmittance value must increase.
This effect is also responsible for the transmittance to reach a maximum at a fiber population fraction of $\modelPsi=\SI{50}{\percent}$, since it is more complicated to find a collision-free state.
\par
%
The retardation decreases almost linearly with increasing crossing angle.
Theoretically, with a crossing angle of $\modelOmega=\SI{90}{\degree}$ and a fiber population fraction of $\modelPsi=\SI{30}{\percent}$, only the remaining $\SI{40}{\percent}$ fibers affect retardation.
However, the measured retardation of about $\SI{0.36}{}$ is larger than the expected $0.8\cdot 0.4 = 0.32$.
An explenation of this could be the noise on the data.
It effects the distribution of the retardation for low amplitudes towords higher values.
This is because the retardation and the individual light intensities cannot be lower than $\SI{0}{}$.
\par
%
The direction for the flat fiber crossings follows the expected theoretical curve of the circular mean.
For the case $\modelPsi=\SI{50}{\percent}$ and $\modelOmega=\SI{90}{\degree}$, the direction values are uniformly distributed since there is no unique direction in the sinosoidal.
This is the effect of retardation cancelation and signal noise.
% These results reproduce the previous results \todo{cite} and show 
For the here investigated distribution of fiber radii at microscopes resolution and pixel size individual fiber population for a intervowen dense fiberbundle cannot be distinguished.
However, the tilt analysis used here uses only a single fiber orientation, or more generally, a single optical axis.
Future, more comprehensive models may be able to reveal the difference.
% \todo{discuss higher faber radii in whats next}
\par
%
The variance of the inclination angle increases with increasing \modelOmega{}, which is to be expected as the retardation decreases, making the model more uncertain.
In the case of a uniformly distributed fiber crossing, the inclination can reach any value as in the case for the direction value.
\par
%
The same effect as for retardation and inclination is also observed for the decreasing value of \trel{}.
Since there is only one optical axis in the tilt analysis model, the \trel{} value must follow the retardation for flat crossing fiber models.
In the case of $\modelPsi=\SI{90}{\degree}$, the tilt analysis also reaches values of $\trel>1$.
As mentioned for the single fiber population, this is an effect of the solution algorithm of the tilt analysis.
It can be used as an indication that the resulting values are not to be trusted and the current model does not fit the data.
\par
% 
The \rvalue{} is constant overall for all crossing angles.
This agrees with theory as long as the tilt analysis is able to find the minimum of the cost function, the $\r-value^2$, because then the remaining difference comes only from the noise.
\par
%
The measured opening angle, \openingAngle{}, does not agree with the underlying distributions of the orientation of the fiber segments.
However, this is to be expected since in a single voxel, \ie{}, has a volume of $\SI{1.3}{\micro\meter} \times \SI{1.3}{\micro\meter} \times \SI{60}{\micro\meter}$, there are many fiber segments with which the light, \ie{}, all light rays, incident on the same ccd pixel can interact.
This means that an average signal is measured and analyzed.
Average value for an aperture angle means, of course, a decrease of its value.
The fact that the measured aperture angles increase with increasing crossing angle is also to be expected.
The main effect of an increasing crossing angle is that the amplitude of the sinusoidal signal, i.e. the retardation, decreases.
Therefore, the estimation from the tilt analysis is more difficult and the result error increases.
% 
\par
The \accvalue{} value has a local minima that depends on the crossing angle \modelOmega{}.
The minima is higher for \modelPsi{} near $0$ or $1$.
The curve is explained by the fact that with the currently used tilt analysis only the main fiber orientation, more precisely the circular mean, can be determined from the measured signal.
Therefore, the \accvalue{} must shrink.
However, beyond a certain point, the delay is reduced to such an extent that the uncertainty increases to such an extent that the measured orientations become more and more random.
Therefore, the \ac{ODF} of the measured signal becomes more and more spherical, which after a certain point, \ie{}, the minima, is closer to the \ac{ODF} of the model than a \ac{ODF} that predicts only one main orientation, \ie{}, a cigar shape.
This also explains why the position of the minimum is closer to $\SI{90}{\degree}$ for fiber population proportions closer to $\SI{50}{\percent}$, because the intersection of the \ac{ODF} of fiber segments is more pronounced.
% 
\paragraph{inclined crossing}
Each of the eight parameters presented for the inclined crossing fiber population can be explained by exactly the same arguments as for the previous flat fiber crossing, except for the inclination parameter.
Here, the inclination follows the circle mean only for smaller crossing angles and reaches a maximum value as a function of the fiber population fraction \modelPsi{} until it drops to $\SI{0}{\degree}$ for $\modelOmega = \SI{90}{\degree}$.
This behavior can be explained by the fact that as the inclination increases, the light rays are no longer affected by the birefringence of the optical axis of the fibers.
Therefore, the only remaining effect comes from the flat fiber population.
% 
\paragraph{Free crossing fiber population}
The free crossing fiber population results can be interepreted as an summary of the previous results.
\par
% 
The transmittance shows the expected behavier, that the lowest values are reached for non crossing fiber population.
Here, the fiber density is the highest and therefore as well the aborption of the light intensity.
For orientations with increasing crossing angles, transmittance increases as the fibers require more space to be collision free.
This effect is further increased for fiber population fractions closer to $\SI{50}{\percent}$.
\par
% 
The retardation value that the highes values are reached for flat non crossing fiber populations.
Here no extinction effect of the \ac{3D-PLI} signal is present.
The retardation value shows a map of which orientations from two fiber population fractions the tilting analysis will lead to a result with high certanty.
This, however, does not mean, that the resulting orienation is the orientation of the underlying fibers, as can seen in the results of the direction and inclination.
As in the previous results of the single flat crossing and oblique crossing fiber population(s), the analyzed angle corresponds to the circular value for flat fibers and the inclination is strongly influenced by the inclination angle.
\par
% 
From the results of \trel{}, the statement that \trel{} values greater than $>1$ indicate an untrustworthy result is still true.
However, the number of fiber configurations where this is the case is relatively small.
In combination with retardation, though, both low \trel{} values and low retardation values indicate considerably more untrustworthy results.
This results in the distrubtion of the \accvalue{}.
The reduced \accvalue{} are the fiber configurations, one has to focus on improving the tilt analsis in the future.
The \accvalue{} can potentially be a good tool to use as a cost function in machine learning techniques, for example.
%
%
%
\section{Speedup}
\label{sec:simSpeedup}
%
This section shows the results of speedup measurements for discretized tissue volume generation (see \cref{sec:dv_generator}) and light matter interaction simulation (see \cref{sec:simulation}) for the implemented parallelization with \ac{OpenMP} and \ac{MPI}.
Simulations were performed using a single compute node with \ac{CPU} architecture \texttt{2x Intel(R) Xeon(R) CPU E5-4657L v2}.
To measure the speedup, each algorithm is executed $N=10$ times.
To calculate the speedup value, the average measured time for $n_\mathit{cpu}=1$ was then divided by the measured time for each value $n_\mathit{cpu}$.
The volume used is the $\modelPsi=\SI{0}{\percent} / \modelInc=\SI{0}{\degree}$ from the \cref{sec:simParameterEnv} parameterization.
% 
% 
% 
\subsection{Results}
% 
\begin{figure}[!t]
\centering
\subcaptionbox{\label{fig:speedTissueMP}
\ac{OpenMP} speedup discrete tissue generation.
}[.475\textwidth]{\includegraphics{gfx/rc1/speed/boxplot_generation_output_generation_mp_v_0.1.csv.pdf}}
\hfill
\subcaptionbox{\label{fig:speedTissueMPI}
\ac{MPI} speedup discrete tissue generation.
}[.475\textwidth]{\includegraphics{gfx/rc1/speed/boxplot_generation_output_generation_mpi_v_0.1.csv.pdf}}
\caption{Discrete tissue generation speedup.}
\end{figure}
%
\Cref{fig:speedTissueMP} shows the speedup results for the parallelization of \ac{OpenMP} of the discrete tissue generation algorithm.
The speedup increases linearly from $\SI{1}{\core}$ to $\SI{8}{\cores}$ up to a speedup of about $\SI{3}{}$.
For $\SIlist{16;24;32;40;48}{\cores}$ the speedup is increased further, but is still in the range of $\SIrange{4}{5}{}$.
\par
% 
\Cref{fig:speedTissueMPI} shows the speedup for the \ac{MPI} implementation.
Here, the speedup up to $\SI{8}{\cores}$ is again linear and gets about $\SI{6.8}{}$.
The speedup from $\SI{16}{\cores}$ to $\SI{48}{\cores}$ is also linear, but the inclination decreases.
For $\SI{48}{\cores}$, a speedup of about $\SI{28}{}$ is obtained.
The variance is significantly lower compared to the \ac{OpenMP} implementation.
\par
% 
\begin{figure}[!t]
\centering
\subcaptionbox{\label{fig:speedSimMP}
\ac{OpenMP} speedup simulation for 5 tilt direction.
}[\textwidth]{\includegraphics{gfx/rc1/speed/boxplot_simulation_output_simulation_mp_v_0.1.csv.pdf}}
\\
\subcaptionbox{\label{fig:speedSimMPI}
\ac{MPI} speedup simulation including all five tilt direction.
}[\textwidth]{\includegraphics{gfx/rc1/speed/boxplot_simulation_output_simulation_mpi_v_0.1.csv.pdf}}
\caption{\ac{3D-PLI}}
\end{figure}
%
The results of the speedup measurements for the light matter interaction are then presented.
All tilt directions (flat, east, north, west, and south) are simulated.
\Cref{fig:speedSimMP} shows the speedup for the \ac{OpenMP} implementation.
The results show ideal speedup up to $\SI{8}{\cores}$.
As the \ac{CPU} count is further increased up to $\SI{48}{\cores}$, the speedup is slightly lower than the ideal line with a speedup of $\SI{42}{}$ for $\SI{48}{\cores}$.
No significant difference in acceleration for different tilt directions is apparent.
\par
% 
\Cref{fig:speedSimMPI} shows the speedup for the \ac{MPI} implementation.
Again, the speedup is ideal up to $\SI{8}{\cores}$.
For \ac{CPU} numbers in the range of $\SIrange{16}{48}{\cores}$, the speedup is again lower than the ideal line with a speedup of about $\SI{35}{}$ for $\SI{48}{\cores}$.
Compared to the previous \ac{OpenMP} implementation, the speedup is further reduced.
A significant difference in the measurement is visible for different tilt directions.
The flat measurement has the highest speedup.
%
%
%
\subsection{Discussion}
%
The speedup for the discrete tissue generator in the case of \ac{OpenMP} is appropriate up to $\SI{4}{\cores}$ with a speedup of slightly above $\num{2}$.
Above $\SI{4}{\cores}$, especially $\geq \SI{16}{\cores}$ is not recommended to use.
This behavior is most likely due to the fact that the cpus read from the same memory address.
However, since the different writing instructions do not use the same memory address and there are no race conditions, a higher speedup was expected.
\par
% 
The speedup for the \ac{MPI} implementation, on the other hand, is almost optimal up to $\SI{8}{\cores}$.
Above that, the speedup is reduced compared to the ideal case, but still quite good.
Since here the \acp{CPU} (cores) run independently, no communication is required.
However, on a single-node system, as in this case, the allocation of memory is a race condition and slows down the algorithm.
This is not expected on a multi-node system.
An additional reduction in computation time results from the fact that all cores must traverse all fiber coordinates.
Pre-filtering could shorten this process.
For example, a global \ac{AABB} can be computed for each fiber.
\par
% 
The speedups for the \ac{3D-PLI} simulation are quite similar and very high for both implementations.
The \ac{OpenMP} implementation is slightly better, which is to be expected since the parallelization is done along a \code{std::vector} of Stokes vectors.
The \ac{MPI} implementation on the other hand still needs to communicate with each other in case of transferring the Stokes vectors.
Also, the discretized volumes are already present in memory, so memory allocation does not slow down the runtime compared to the volume discretizer.
For the \ac{MPI} implementation, the communication effect is also visible for flat and tilted simulations.
In the case of a flat simulation, no communication and thus no barriers are required, so the speedup is higher.
A difference in north-south and east-west tilt could be explained for different number of cores by the fact that the volume has to be divided into sub-volumes.
This is done within the algorithm so that the surface area is minimized.
The splitting process produces sub-volumes with a tendency to have longitudinal quartiles along the x-axis.
This may explain why the north-south tilt has a slightly higher speedup.
\par
%
In summary, the speedup for the implementation of \ac{MPI} is very good.
The user should prefer this implementation over the \ac{OpenMP} library.