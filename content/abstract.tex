\pdfbookmark[0]{Abstract}{Abstract}
\addchap*{Abstract}
%
In the Fiber Architecture group of the Institute of Neuroscience and Medicine, Structural and Functional Organization of the Brain (INM-1), 3D Polarized Light Imaging (3D-PLI) microscopy is used to measure the orientation of nerve fibers in unstained brain sections.
Interpretation of the measurement can be challenging for certain regions, for example where fibers cross or are oriented perpendicular to the sectioning plane.
To understand the behavior of the measured signal of such structures without further external influences, such as non-ideal optics, simulations are used where each parameter is known.
In order to perform simulations, virtual tissue models are needed and a virtual 3D-PLI microscope, being capable of simulating the influence of the tissue on the light.
\\
% 
In order to design realistic models of dense nerve fiber tissue, it must be ensured that individual nerve fibers do not overlap.
This is especially difficult to design in advance for interwoven structures, as is occurs in nerve fiber crossings.
Therefore, a nerve fiber modeling specialized algorithm was designed in this thesis.
The algorithm will check a given volume for overlaps of single nerve fibers, and then slowly push them apart at the affected locations.
Thus, a collision-free tissue model is created over time.
The pre-existing simulation algorithm of the 3D PLI microscope was completely redesigned as part of this work.
The algorithm is now able to run in parallel on multiple CPU cores as well as computational clusters.
Thus, a large number of simulations can be performed, allowing for greater statistics in the analyses.
These two algorithms were published in the software package \textit{fiber architecture simulation toolbox of 3D-PLI} (fastPLI).
\\
% 
Finally, in this thesis, nerve fiber models consisting of two nerve fiber populations, \ie{} two densely packed crossing nerve fiber bundles, were created and subsequently simulated.
The results show, that the orientation of the nerve fiber population, which has a higher proportion in the volume, can be determined.
With the current resolution of the microscopes used, it is not possible to determine both fiber poplation orientations individuel.
The measured orientation seems to follow the circular mean as a function on the proportional volume fraction of the nerve fiber populations, taking into account the decrease of the measured signal due to the increasing tilt angle.
In summary, the development of the algorithm for modeling nerve fibers together with the simulation in a toolbox has proven to be a suitable tool to be able to investigate questions quickly through simulations.
% 
% 
% 
{
\let\cleardoublepage\clearpage
\pdfbookmark[0]{Zusammenfassung}{Zusammenfassung}
\addchap*{Zusammenfassung}
}
% 
In der Faserbahnarchitektur Gruppe des Instituts für Neurowissenschaften und Medizin für strukturelle und funktionelle Organisation des Gehirns (INM-1) wird die 3D Bildgebung mit polarisiertem Licht (3D-PLI) zur Messung der Orientierung von Nervenfasern in ungefärbten Hirnschnitten eingesetzt.
Die Interpretation der Messung kann in bestimmten Regionen schwierig sein, zum Beispiel dort, wo sich Fasern kreuzen oder senkrecht zur Schnittebene ausgerichtet sind.
Um das Verhalten des gemessenen Signals solcher Strukturen ohne weitere äußere Einflüsse, wie z. B. eine nicht ideale Optik, zu verstehen, werden Simulationen verwendet, bei denen jeder Parameter bekannt ist.
Zur Durchführung von Simulationen werden virtuelle Gewebemodelle und ein virtuelles 3D-PLI-Mikroskop benötigt, mit dem der Einfluss des Gewebes auf das Licht simuliert werden kann.
\\
% 
Um realistische Modelle von dichtem Nervenfasergewebe zu entwerfen, muss sichergestellt werden, dass sich die einzelnen Nervenfasern nicht überlappen.
Dies ist insbesondere bei verflochtenen Strukturen, wie sie bei Nervenfaserkreuzungen vorkommen, schwer zu entwerfen.
Daher wurde in dieser Arbeit ein auf die Modellierung von Nervenfasern spezialisierter Algorithmus entwickelt.
Der Algorithmus prüft ein gegebenes Volumen auf Überschneidungen einzelner Nervenfasern und schiebt diese dann an den betroffenen Stellen langsam auseinander.
So entsteht mit der Zeit ein kollisionsfreies Gewebemodell.
Der bereits existierende Simulationsalgorithmus des 3D-PLI-Mikroskops wurde im Rahmen dieser Arbeit komplett neu entwickelt.
Der Algorithmus ist nun in der Lage, parallel auf mehreren CPU-Kernen sowie mit Rechenclustern zu arbeiten.
Dadurch kann eine große Anzahl von Simulationen durchgeführt werden, was eine größere Statistik in den Analysen ermöglicht.
Die hier erarbeiteten Algorithmen wurden in dem Softwarepaket \textit{fiber architecture simulation toolbox of 3D-PLI} (fastPLI) veröffentlicht.
\\
% 
Zuletzt wurden in dieser Arbeit Nervenfasermodelle, die aus zwei Nervenfaserpopulationen, also zwei dicht gepackten, sich kreuzenden Nervenfaserbündeln, bestehen, erstellt und anschließend simuliert.
Die Ergebnisse zeigen, dass die Orientierung derjenigen Nervenfaserpopulation, die einen höheren Anteil am Volumen hat, bestimmt werden kann.
Mit der derzeitigen Auflösung der verwendeten Mikroskope ist es nicht möglich, beide Orientierungen der Faserpopulationen individuell zu bestimmen.
Die gemessene Orientierung scheint dem zirkulären Mittelwert als Funktion des proportionalen Volumenanteils der Nervenfaserpopulationen zu folgen, wobei die Abnahme des gemessenen Signals aufgrund des zunehmenden Neigungswinkels berücksichtigt wird.
Zusammenfassend hat sich die Entwicklung des Algorithmus zur Modellierung von Nervenfasern zusammen mit der Simulation in einer Toolbox als ein geeignetes Werkzeug erwiesen, um Fragestellungen durch Simulationen schnell untersuchen zu können.