\pdfbookmark[0]{Abstract}{Abstract}
\addchap*{Abstract}
%
In the Fiber Architecture group of the Institute of Neuroscience and Medicine, Structural and Functional Organization of the Brain (INM-1), 3D Polarized Light Imaging (3D-PLI) microscopy is used to measure the orientation of nerve fibers in unstained brain sections.
However, the measurement of orientation is difficult to interpret in regions where fibers cross or are strongly inclined with respect to the section plane.
To understand the behavior of the measured signal as a function of such structures without further external influences, such as non-ideal optics, simulations are used where each parameter is known.
Thus, in order to perform simulations, two things are essential.
First, virtual tissue models are needed and second, a virtual 3D-PLI microscope must exist that can simulate the influence of the tissue on the light.

In order to design realistic models of dense nerve fiber tissue, care must be taken to ensure that individual nerve fibers do not overlap.
This is especially difficult to design in advance for interwoven structures, such as those that can occur in nerve fiber crossings.
Therefore, in this work a nerve fiber modeling specialized algorithm was designed.
The algorithm will check a given volume for overlaps of single nerve fibers, and then slowly push them apart at the affected locations.
Thus, a collision-free tissue model is created over time without having to waste too much space.
As another focus of this work, the existing simulation algorithm of the 3D-PLI microscope was redesigned.
The algorithm is now able to exploit parallel on multiple CPU cores as well as computational clusters.
Thus, a large number of simulations are possible, allowing for greater statistics in the analyses.
These two algorithms were published in the software package \textit{fiber architecture simulation toolbox of 3D-PLI} (fastPLI). 

Finally, in this thesis, nerve fiber models consisting of two nerve fiber populations, two densely packed crossing nerve fiber bundles, were created and subsequently simulated.
It turned out that especially the orientation of the nerve fiber population, which has a higher proportion in the volume, can be determined.
With the current resolution of the microscopes used, it is not possible to determine both orientations.
The measured orientation seems to follow the circular mean as a function on the proportional volume fraction of the nerve fiber populations, taking into account the decrease of the measured signal due to the increasing tilt angle.
% 
% 
% 
\pdfbookmark[0]{Zusammenfassung}{Zusammenfassung}
\addchap*{Zusammenfassung}
% 
In der Faserbahnarchitektur Gruppe des Instituts für Neurowissenschaften und Medizin für strukturelle und funktionelle Organisation des Gehirns (INM-1) wird die 3D Bildgebung mit polarisiertem Licht (3D-PLI) zur Messung der Orientierung von Nervenfasern in ungefärbten Hirnschnitten eingesetzt.
Die Messung der Orientierung ist allerdings in Regionen wo sich Faser kreuzen oder stark zur Schnittrichtung inkliniert sind schwierig zu interpretieren.
Um das Verhalten des Messsignals in Abhängigkeit von solchen Strukturen ohne weitere äußere Einflüsse, wie z.B. einer nicht idealen Optik, verstehen zu können, werden Simulationen verwendet, bei denen jeder Parameter bekannt ist.
Um Simulationen durchführen zu können, sind somit zwei dinke von entscheidender Bedeutung.
Erstens, es werden virtuelle Gewebemodelle benötigt und zweitens, ein virtuelles 3D-PLI Mikroskop muss existieren, welches den Einfluss des Gewebes auf das Licht simulieren kann. 

Um realistischere Modelle von dichtem Nervenfasergewebe entwerfen zu können, muss jedoch darauf geachtet werden, dass sich die einzelnen Nervenfasern nicht überschneiden.
Dies ist gerade bei verwobenen Strukturen, wie sie in Nervenfaserkreuzungen auftreten können, schwierig im Vorfeld zu Design.
Daher wurde in dieser Arbeit ein auf die Modellierung von Nervenfasern spezialisierter Algorithmus entwickelt.
Der Algorithmus überprüft ein gegebenes Volumen auf Überschneidungen einzelner Nervenfasern und schiebt sie dann an den betroffenen Stellen langsam auseinander.
Somit entsteht über die Zeit ein kollisionsfreies Gewebemodell, ohne zu viel Platz verschwenden zu müssen.
Als weiterer Fokus dieser Arbeit wurde der bestehende Simulationsalgorithmus des 3D-PLI Mikrosops überarbeitet.
Der Algorithmus ist nun in der Lage parallele auf mehreren CPU Kernen als auch Rechenclustern auszunutzen.
Somit sind eine Vielzahl an Simulationen möglich, die für eine grössere Statistik bei den Analysen erlauben.
Diese beiden Algorithmen wurden in dem Softwarepacket \textit{fiber architecture simulation toolbox of 3D-PLI} (fastPLI) veröffentlicht. 

Zuletzt wurde in dieser Arbeit Nervenfasermodelle bestehend aus zwei Nervenfaerpopulationen, zwei dicht gefüllte kreuzende Nervenfaserbündle, erstellt und im Anschluss simuliert.
Es zeigte sich, dass insbesondere die Orientierung der Nervenfaserpopulation, die einen höheren Anteil im Volumen hat, bestimmt werden kann.
Mit der derzeitigen Auflösung der verwendeten Mikroskope ist es nicht möglich, beide Orientierungen zu bestimmen.
Die gemessene Orientierung scheint dem kreisförmigen Mittelwert in Abhängigkeit auf den proportionalen Volumenanteil der Nervenfaserpopulationen zu folgen, wobei die Abnahme des gemessenen Signals aufgrund des zunehmenden Neigungswinkels berücksichtigt wird.