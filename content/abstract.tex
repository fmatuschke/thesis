\pdfbookmark[0]{Abstract}{Abstract}
\addchap*{Abstract}
%
% 
In the fiber web architecture group of the Institute of ... 3D-PLI microscopy is used to determine the orientation of nerve fibers in a thinly sliced brain section.
In order to understand the exact behavior of the microscopy, simulation techniques are used among others.
These allow, for example, the precise influence of objects such as nerve fibers or other cells within a virtual tissue on the light beam to be determined.
However, this method is very expensive for two reasons.
First, the tissue models must be very close to reality.
Among other things, this means that bundles of nerve fibers can be generated virtually, for example, and can intersect like real tissue, forming intertwining structures.
Second, tools are needed that can generate these complex models for virtual volume sizes in an appropriate time and simulate the behavior of light through them.
Up to the point of this work, mainly simple models are used, which can either be described analytically or contain a high degree of symmetries.

The goal of this work was therefore to develop software that allows users to create virtual nerve fiber tissue and simulate the behavior of light in a virtual 3D PLI microscope using relatively simple methods.

To develop more realistic models, the focus was on the so-called white matter, which contains very dense collections of nerve fibers.
The most important criterion was that two virtual nerve fibers must not overlap in volume, i.e. they must not touch each other.
Since it is virtually impossible for a user to create such a model in advance, the strategy was chosen that the user first fills a volume initially with nerve fibers without having to worry about overlaps.
In a next step, the interactive algorithm presented here is applied, which detects collisions between the objects and solves them by slowly pushing the affected areas apart.
In this way, the virtual mesh is freed from all collisions over time or with increasing iteration, and in the end a collision-free virtual mesh exists.

To generate statistically describable models, a mathematical description of two nerve fiber populations was used.
The first population is described only by the inclination.
The second population has a crossing angle to the first and can be rotated around the first.
This makes it possible to keep the number of models relatively small and still generate all possible orientations that can be assumed by two nerve fiber populations.

The next step is to simulate this tissue in the simulation software presented here.
It was crucial to characterize the microscope to be simulated in order to be able to use the necessary parameters, such as resolution or image noise, for the simulation.

The generated nerve fiber models are then simulated in the thesis and the behavior of the geometries of two nerve fiber populations on the results of the 3D-PLI measurements is analyzed.

It was shown that especially the orientation of the nerve fiber population, which has a higher proportion in the volume, can be measured.
With the current resolution of the microscopes used, it is not possible to determine both orientations for densely intertwined junctions.
The measured orientation seems to follow the circular mean with respect to the proportional volume fraction of the nerve fiber populations, whereas the reduction of the measured signal by increasing inclination has to be taken into account.

Another important result of this work is that the complete modeling algorithms and simulation software have been published as a complete open source tool.
Furthermore, all algorithms have been designed and optimized for use with multiple CPU cores, making it possible to perform a large number of volumes and simulations.

Finally, further important steps are discussed, which discuss further optimizations.
In particular, the use of the GPU architecture is discussed, which can decisively reduce runtimes.
% 
% 
% 
\pdfbookmark[0]{Zusammenfassung}{Zusammenfassung}
\addchap*{Zusammenfassung}
% 
In der Arbeitsgruppe Faserbahnarchitektur des Instituts für ... wird mithilfe der 3D-PLI Mikroskopie die Orientierung der Nervenfasern in einem dünn geschnittenen Hirnschnitt bestimmt.
Um das genaue Verhalten der Mikroskopie verstehen zu können, werden unter anderem Simulationstechniken angewandt.
Diese erlauben es z.B. den genauen Einfluss von Objekten wie Nervenfasern oder anderen Zellen innerhalb eines virtuellen Gewebes auf den Lichtstrahl zu bestimmen.
Allerdings ist diese Methode aus zwei Gründen sehr aufwändig.
Erstens, die Gewebemodelle müssen sehr nahe an der Realität sein.
Das heißt unter anderem, dass z.B. Nervenfaserbündel virtuell generiert werden können, die sich wie echtes Gewebe z.B. kreuzen und dabei verflechtende Strukturen eingehen können.
Zweitens werden Tools benötigt, die diese komplexen Modelle für virtuelle Volumengrößen in einer angebrachten Zeit generieren und das Verhalten des Lichtes durch diese simulieren können.
Bis zum Standpunkt dieser Arbeit werden vor allem simple Modelle verwendet, welche sich entweder analytisch beschreiben lassen oder ein hohes maß an Symmetrien enthält.

Das Ziel dieser Arbeit war es daher eine Software zu entwickeln, welche es Nutzern erlaubt, mit relativ simplen Methoden virtuelles Nervenfasergewebe zu erstellen und das Verhalten des Lichtes in einem virtuelles 3D-PLI Microskop simulieren zu können.

Um realistischere Modelle zu entwickeln, wurde sich dabei auf die sogenannte weiße Substanz, welche sehr dichte Ansammlungen von Nervenfasern enthält, fokusiert.
Das entscheidenste Kriterium war dabei, dass sich zwei virtuelle Nerfenfasern im Volumen nicht berschneiden dürfen, also maximal Berühren.
Da es einem Nutzer quasi unmöglich ist im Vorerein ein solches Modell zuerstellen, wurde die Strategie gewählt, dass der Nutzer zuerst ein Volumen initial mit Nervenfasern füllt ohne sich um überschneidungen kümmern zu müssen.
In einem nächsten Schritt wird der hier vorgestellte interativer Algorithmus angewand, welcher kollisionen zwischen den Objekten erkennt und diese daruch lösst, dass er die betroffenden Bereiche langsam auseinanderdrückt.
Dadurch wird mit der Zeit bzw. mit steigender iteration das virtuelle Gewebe von allen Kollisionen befreit und am ende liegt ein kollisionsfreies virtuelles Gewebe vor.

Um statistisch beschreibbare Modelle zu generieren wurde eine mathematische beschreibung von zwei Nervenfaserpopulationen genutzt.
Die erste Population wird dabei nur durch die Inklination beschrieben.
Die zweite Population hat zu der ersten einen kreuzungswinkel und kann um die erste rotiert werden.
Dadurch ist es möglich die Anzahl der zu benötigten Modelle relativ gering zu halten und dennoch alle möglichen Orientierungen, die durch zwei Nervenfaserpopulationen eingenommen werden zu können zu erzeugen.

Als nächster Schritt wird dieses Gewebe in der hier vorgestellten Simulationssoftware simuliert.
Dabei war es entscheident, dass das zu simulierende Microskop charakterisiert wird, um die nötigen Parameter, wie z.B. Auflösung oder Bildrauschen, für die Simulation verwenden zu können.

Die erzeugten Nervenfasermodelle werden in der Arbeit anschließend simuliert und das Verhalten der Geometrien von zwei Nervenfaserpopulationen auf die Resultate der 3D-PLI Messungen analysiert.

Dabei zeigte sich, dass vorallem die Orientierung der Nervenfaserpopulation, die einen höheren Anteil im Volumen besitzt, gemessen werden kann.
Bei der derzeitigen Auflössung der verwendeten Mikroskope ist es dicht verflochtenden Kreuzungen nicht möglich, beide Orientierungen zu bestimmen.
Die gemessene Orientierung scheint dabei den Zirkularen Mittelwert mit respekt zum anteiligen Volumenanteil der Nervenfaserpopulationen zu folgen, wobei die reduktion des gemessenen Signals durch ansteigende Inklination mit berücksichtigt werden muss.

Ein weiteres entscheidendes Ergebniss dieser Arbeit ist es, dass die komplette Modelirungsalgorithmen und Simulations software als ein komplettes Open Source Tool veröffentlich wurde.
Desweiteren wurden alle Algorithmen für die rehnung mit mehreren CPU Kernen ausgelegt und optimiert wodurch es möglich ist, eine großzahl an Volumen und Simulationen durchführen zu konnen.

Zuleletzt werden weitere wichtige Schritte diskutiet, die weitere optimierungen beshreiben.
Hier ist speziell der Einsatz der GPU Architektur diskutiert worden, welche die Laufzeiten entscheident veringern kann.