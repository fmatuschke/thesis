\pdfbookmark[0]{Abstract}{Abstract}
\addchap*{Abstract}
%
In the fiber architecture group of the institute for structural and functional organization of the brain (INM-1), 3D Polarized Light Imaging (3D-PLI) microscopy is used to measure the orientation of nerve fibers in unstained brain sections.
To understand the exact behavior of the microscope technique, simulations are used.
These allow, for example, the influence of nerve fibers within a virtual tissue to be studied.
However, such simulations are very costly for two reasons.
First, nerve fiber bundles must be virtually generated, for example, crossing nerve fibers.
Second, the algorithms for tissue volume generation and simulation must be executable in a reasonable amount of time.
So far, models have been used that can either be described analytically or have a high degree of symmetry.
This allows a fast generation of the models and simulation of the microscope, but has the disadvantage that the models are very simple. 

Therefore, the goal of this thesis was to develop software that allows, using relatively simple methods, to generate virtual nerve fiber tissue and simulate the behavior of light in a virtual 3D-PLI microscope.

To develop more realistic models, the focus was on white matter, which contains very dense collections of nerve fibers.
The most important criterion was that the virtual nerve fibers should not overlap in volume.
An algorithm was developed that accepts any input of nerve fiber pathways.
All nerve fibers are then checked for collisions.
If a collision occurs, the nerve fibers are slightly pushed apart and the algorithm is repeated until no more collisions are detected.
In this way, all collisions are eliminated over time, resulting in a collision-free virtual volume.

For the simulation, it was crucial to characterize the 3D-PLI microscope in order to model the boundary conditions such as resolution or image noise.
The generated nerve fiber models were then simulated and the influence of the geometries of two nerve fiber populations to the results of the 3D-PLI measurements was analyzed.

It was found that, in particular, the orientation of the nerve fiber population, which has a higher proportion in the volume, can be determined.
With the current resolution of the microscopes used, it is not possible to determine both orientations.
The measured orientation appears to follow the circular mean with respect to the proportional volume fraction of the nerve fiber populations, taking into account the decrease in the measured signal due to the increasing inclination angle.

The modeling algorithm and simulation software were released as an open-source tool.
All algorithms have been designed and optimized for use with multiple CPU cores, allowing a large number of volumes and simulations to be performed.
% 
% 
% 
\pdfbookmark[0]{Zusammenfassung}{Zusammenfassung}
\addchap*{Zusammenfassung}
% 
In der Faserbahnarchitektur Gruppe des Instituts für strukturelle und funktionelle Organisation des Gehirns (INM-1) wird die 3D Bildgebung mit polarisiertem Licht (3D-PLI) zur Messung der Orientierung von Nervenfasern in ungefärbten Hirnschnitten eingesetzt.
Um das genaue Verhalten der Mikroskoptechnik zu verstehen, werden Simulationen verwendet.
Damit lässt sich beispielsweise der Einfluss von Nervenfasern innerhalb eines virtuellen Gewebes untersuchen.
Solche Simulationen sind jedoch aus zwei Gründen sehr aufwendig.
Erstens müssen Nervenfaserbündel virtuell erzeugt werden, zum Beispiel sich kreuzende Nervenfasern.
Zweitens müssen die Algorithmen zur Erzeugung des Gewebevolumens und zur Simulation in angemessener Zeit ausführbar sein.
Bislang wurden Modelle verwendet, die entweder analytisch beschrieben werden können oder einen hohen Grand an Symmetrie aufweisen.
Dies ermöglicht eine schnelle Generierung der Modelle und Simulation des Mikroskops, hat aber den Nachteil, dass die Modelle sehr einfach sind. 

Ziel dieser Arbeit war es daher, eine Software zu entwickeln, die es mit relativ einfachen Methoden ermöglicht, virtuelles Nervenfasergewebe zu erzeugen und das Verhalten von Licht in einem virtuellen 3D-PLI-Mikroskop zu simulieren.

Um realistischere Modelle zu entwickeln, lag der Fokus auf der weißen Substanz, die sehr dichte Ansammlungen von Nervenfasern enthält.
Das wichtigste Kriterium war, dass sich die virtuellen Nervenfasern im Volumen nicht überschneiden sollten.
Es wurde ein Algorithmus entwickelt, der einen beliebigen Input von Nervenfaserverläufen akzeptiert.
Alle Nervenfasern werden dann auf Kollisionen überprüft.
Tritt eine Kollision auf, werden die Nervenfasern leicht auseinander geschoben und der Algorithmus wird so lange wiederholt, bis keine Kollisionen mehr festgestellt werden.
Auf diese Weise werden im Laufe der Zeit alle Kollisionen beseitigt, sodass ein kollisionsfreies virtuelles Volumen entsteht.

Für die Simulation war es entscheidend, das 3D-PLI-Mikroskop zu charakterisieren, um Verhalten wie Auflösung oder Bildrauschen zu modellieren.
Die generierten Nervenfasermodelle wurden dann simuliert und der Einfluss der Geometrien von zwei Nervenfaserpopulationen auf die Ergebnisse der 3D-PLI-Messungen analysiert.

Es zeigte sich, dass insbesondere die Orientierung der Nervenfaserpopulation, die einen höheren Anteil im Volumen hat, bestimmt werden kann.
Mit der derzeitigen Auflösung der verwendeten Mikroskope ist es nicht möglich, beide Orientierungen zu bestimmen.
Die gemessene Orientierung scheint dem kreisförmigen Mittelwert in Abhängigkeit auf den proportionalen Volumenanteil der Nervenfaserpopulationen zu folgen, wobei die Abnahme des gemessenen Signals aufgrund des zunehmenden Neigungswinkels berücksichtigt wird.

Die Modellierung und die Simulationssoftware wurden als Open-Source-Tool veröffentlicht.
Alle Algorithmen wurden für die Verwendung mit mehreren CPU-Kernen konzipiert und optimiert, sodass eine große Anzahl von Volumen und Simulationen durchgeführt werden kann.