\setcounter{chapter}{6}
\chapter{Dense \acs{WM} modelling analysis}
\label{cha:model_analysis}
% 
% 
\todo{erstmal nur beschreiben, diskussion dann so wie es seien muss}
\todo{veroffentlichung in intro}
\todo{steffi email}
% 
% 
\section{Introduction}
% 
After describing the functionality of the \pymodule{fastpli.model}, the next task is to use these methods not only to create realistic dense \ac{WM} models, but to keep them as simple as possible due to the complexity of the tissue.
This is due to the fact that \ac{WM} contains individual nerve fibers that typically have a size in the order of about $\SI{1}{\micro\meter}$.
Therefore, in a \SI{60}{\micro\meter} voxel of a \ac{LAP} measured \ac{3D-PLI} can be image within the \ac{WM} several thousand individual nerve fibers are already present.
Considering that each nerve fiber has a unique trajectory through this volume, this means that theoretically an absurd variety of possible configurations exists.
However, due to the nature of the \ac{3D-PLI} structure, previous studies have shown that the result of the signal depends on much less degrees of freedom than each individual nerve fiber has.
\\[\baselineskip]
% 
Therefore, this work, being the first study of its kind, focuses mainly on two randomly interwoven fiber populations for \ac{3D-PLI} simulations.
Another focus is the generation of such models, whereby the computational time effort shall be kept as low as possible.
The analysis refers to the setting of the parameters for the model generation and their influence on the resulting collision-free configurations.
% 
% 
\section{Brain tissue model}
% 
To create universal reusable \ac{WM} tissue models, a practical approach is to reduce the number of parameters to a minimum.
The approach chosen here is to build cubic volumes of up to two fiber populations with individual characterization.
The most important parameter for \ac{3D-PLI} is orientation.
% 
\subsection{single fiber bundle population}
% 
A single fiber bundle population can have any orientation $O(\varphi,\alpha)$ in 3d space.
However, since $O(\varphi,\alpha) = O(\varphi+\pi,-\alpha)$ only the orientations of a hemisphere are required.
Furthermore, the experimental setup does not allow a difference between a fiber bundle population with $\varphi = \varphi_0$ and $\varphi = \varphi_1$ when the initial optical elements are rotated by $\varphi_1-\varphi_0$, unless one considers the effect of the physical expansion of the camera pixels, which is neglected here.
Therefore it is not necessary to create a single fiber population with a direction other than $\varphi = \SI{0}{\degree}$.
However, the inclination must be sampled in the range of $[0,90]\si{\degree}$.
Since the fiber bundle populations can easily be rotate from $\varphi=\SI{0}{\degree}, \alpha=\SI{0}{\degree}$ to $\varphi=\varphi_{\text{target}}, \alpha=\alpha_{\text{target}}$, it is not necessary to create individual fiber bundle configurations.
\\
%
Each fiber will have an individual radii, sampled from a \name{log normal} distribution $\mathrm{LogNorm(
\mu,\sigma)}$ (see \cref{fig:logNormal}).
Its probability desity function looks like this:
\begin{align}
\mathrm{LogNorm}(x,\mu,\sigma)=\frac{1}{\sigma x \sqrt{2\pi}}\exp\left(-\frac{(\ln(x)-\mu)^2}{2 \sigma^2}\right)
\end{align}
% 
Its property allows to multiply by a logarithmic mean value $\fiberRadiusMu=0$ to sample values which are not allowed to have negative values like a length and keeping the desired mean value $\fiberRadiusMean$:
\begin{align}
\fiberRadius = \fiberRadiusMean \cdot \mathrm{sample}\left(\mathrm{LogNorm}(\mu=0,\sigma=0.1)\right)
\end{align}
% 
To initialize the trajectories the previous introduced method for building a cubic volume is be used (see \cref{sec:cubeModelBuilding}) along the fiber bundle population orientation $F = O(\varphi, \alpha)$.
The seeding along the fiber bundle population will be randomly set with.
% 
\begin{align}
\begin{split}
p &= p + \mathrm{sample}\left(\mathrm{Gaus}(\mu=0,\sigma=0.05 \cdot \fiberRadiusMean)\right)\\
r &= \fiberRadius \cdot \mathrm{sample}\left(\mathrm{LogNorm}(\mu=0,\sigma=0.05)\right)
\end{split}
\end{align}
% 
After the initial seeding process, and after applying the model segment length, each \name{sphere} of a fiber will be randomly changed.
The point will be randomly displaced by a normal distribution and the radii with the log normal distribution $(\mu=0,\sigma=0.05)$.
This will produce a pseudo random collision free configuration.
% 
\vspace{5pt}
\hrule
\vspace{6pt}
% 
\paragraph{dual fiber bundle populations}
The same mathematical properties as above are true for any fiber population inside a superposition.
This means, that only a differential angle between the two orientations, so called crossing angle $\modelOmega$, is needed as well as a density change between both populations \modelPsi{} in:
\begin{align}
    F_G = [\modelPsi \cdot F_0] \circ [(1-\modelPsi) \cdot F_1]
\end{align}
% 
Since the direction of the first fiber bundle population can be again neglected, only the direction and inclination of the second fiber population relative to the first is important. Therefore it was decided to build models with \modelOmega{} from ${0,10,...,90}$ and afterwords rotating the fiber orientations around the x axis (\cref{subfig:Test1})
% 
\paragraph{fiber population additional parameters}
density, dispersion, radii, change of radii, ...
% 
\paragraph{rest}
% 
As described in \cref{chap5:ShapeControl} the choice of shape parameters \segLength and \segRadius are quite important. For a realistic fiber radius distribution of around \SI{0.5}{\micro\meter} the time for a \SI{60}{\micro\meter} usable cube (\SI{105}{\micro\meter} sphere) varies from several hours up to \SI{24}{\hour}.
% 
\begin{figure}[!t]
\centering
\def\tikzwidth{0.5*\textwidth}
\subcaptionbox{\label{fig:modelrotcube}A spherical volume with diameter $d$ is used as boundary, so that a cube of side length $1/1.5 \cdot d$ can be cut in any orientation inside the sphere.}[.49\textwidth]{
\inputtikz{gfx/model/sphere_cube}}\hfill
\subcaptionbox{\label{fig:modelinit}Initial orientation for two fiber populations $F_0$ and $F_1$. The angle between both populations is $\Omega$. The remaining degree of freedom are taken into account by rotating the whole volume (see \cref{fig:modelrotcube})}[.49\textwidth]{
\inputtikz{gfx/model/sphere_models}}
\caption[]{Init Models}
% \label{fig:}
\end{figure}
% 
\begin{figure}[!t]
\centering
\def\tikzwidth{0.5*\textwidth}
\subcaptionbox{fixed first fiber population; half rotation of second fiber population}[.49\textwidth]{
\inputtikz{gfx/model/sphere_models_a}}\hfill
\subcaptionbox{\label{subfig:Test1}increasing inclination for first fiber population}[.49\textwidth]{
\inputtikz{gfx/model/sphere_models_b}}
\caption{two population model library}
\label{fig:twomodelpop}
\end{figure}
% 
% 
\subsection{Setup}
% 
The library \say{cube\_2pop} is build with the parameter listed in \cref{tab:cube2pop, tab:cube2popSoftware}.
% 
\begin{table}[!b]
\sisetup{parse-numbers=false,open-bracket={\{}, close-bracket={\}}, list-final-separator={,},list-pair-separator={,}}%
\centering
\pgfplotstabletypeset[%
    thesisTableStyle,
    column type=lcl,
    columns/name/.style={string type},
    columns/variable/.style={string type},
    columns/values/.style={string type},
    every head row/.style={before row=\toprule,after row=\midrule},
    every last row/.style={after row=\bottomrule},
    col sep=&,
    row sep=\\,
    % string type,
]
{name & variable & values\\
mean fiber radius & $\fiberRadiusMean$ & $\SIlist{0.5;1;2;5;10}{\micro\meter}$\\
mean segment length factor & $\segLengthFactor$ & $\SIlist{1;2;4;8}{\micro\meter}$\\
min segment bending radii factor & $\segRadiusFactor$ & $\SIlist{1;2;4;8}{\micro\meter}$\\
fiber bundle distribution value & $\modelPsi$ & $\SIlist{0.1;0.2;...;1.0}{}$ \\
fiber bundle crossing angle & $\modelOmega$ & $\SIlist{0;10;...;90}{\degree}$\\
}
\caption{parameter\_statistic setup and \textcolor{violet}{variables}.}
\label{tab:cube2pop}
\end{table}
% 
\begin{table}[!b]
\centering
\sisetup{open-bracket={\{}, close-bracket={\}}, list-final-separator={,},list-pair-separator={,}}%
\pgfplotstabletypeset[%
    thesisTableStyle,
    column type=l,
    columns/variable/.style={string type},
    columns/value/.style={string type},
    every head row/.style={before row=\toprule,after row=\midrule},
    every last row/.style={after row=\bottomrule},
    col sep=&,
    row sep=\\,
]
{variable & value\\
pre model diameter & $d = \SI{105}{\micro\meter}$\\
mean fiber radius & $\textcolor{violet}{\fiberRadiusMean} = \SIlist{0.5;1;2;5;10}{\micro\meter}$\\
fiber radius distribution & $\fiberRadiusSig = \SI{0.1}{}$, $\fiberRadiusMu = \SI{0}{\micro\meter}$\\
seed.distance & $2 \cdot \textcolor{violet}{\fiberRadiusMean}$\\
seed.size & $2 \cdot \mathit{volume}$\\
bundle distribution & $\textcolor{violet}{\modelPsi}$\\
bundle crossing & $\textcolor{violet}{\modelOmega}$\\
solver.obj\_mean\_length & $\fiberRadiusMean \cdot \textcolor{violet}{\segLengthFactor}$\\
solver.obj\_min\_radius & $\fiberRadiusMean \cdot \textcolor{violet}{\segRadiusFactor}$\\
solver.max\_steps & $\SI{100000}{}$\\
max runtime & $\SI{24}{\hour}$\\
}
\caption{parameter\_statistic setup and \textcolor{violet}{variables}.}
\label{tab:cube2popSoftware}
\end{table}
% 
% 
\begin{figure}[!t]
\centering
\tikzset{external/export=false}
\begin{tikzpicture}[scale=1, trim axis left, trim axis right]
\begin{axis}[height=0.46\textwidth, width=0.75\textwidth,enlargelimits=false, xlabel={$x$}, ylabel={$f(x,\mu,\sigma)$}, title=${f(x,\mu,\sigma)=\frac{1}{\sigma x \sqrt{2\pi}}\exp\left(-\frac{(\ln(x)-\mu)^2}{2 \sigma^2}\right)}$,
legend style={at={(1,1)},anchor=north east},
legend cell align={left},
]
\pgfmathsetmacro{\muValue}{0}
\pgfmathsetmacro{\sigmaValue}{0.05}
\addplot[blue,thick,domain=0.6:1.4, samples=1000]{1/(\sigmaValue*x*sqrt(2*pi))*exp(-(ln(x)-\muValue)^2/(2*\sigmaValue^2))}; \addlegendentry{$\mu=0.0,\sigma=0.05$}
\pgfmathsetmacro{\muValue}{0}
\pgfmathsetmacro{\sigmaValue}{0.1}
\addplot[red,thick,domain=0.6:1.4, samples=1000]{1/(\sigmaValue*x*sqrt(2*pi))*exp(-(ln(x)-\muValue)^2/(2*\sigmaValue^2))}; \addlegendentry{$\mu=0.0,\sigma=0.1$}
\end{axis}
\end{tikzpicture}
\caption[]{\itodo{check values in literatur} Probability density function of a multiplicative \name{log normal} distribution.}
\label{fig:logNormal}
\end{figure}
% 
\subsection{Results}
% 
\todo{STRANGE behaviour in parameter\_statistic for smaller radii. probably because of rnd seeds}
% 
% \begin{figure}[!t]
% \centering
% \includegraphics[width=\textwidth, page=1]{dev/gfx/2/parameter_statistic_box_plot.pdf}
% \caption{Model (\fiberRadius) characteristic for different parameters}
% % \label{fig:}
% \end{figure}
% 
\begin{figure}[p]
\centering
\includegraphics[width=\textwidth, page=1]{dev/gfx/2/parameter_statistic_box_plot_volume.pdf}
\caption[Model characteristics]{Model characteristic for $\fiberRadius = \SI{1}{\micro\meter}$ different parameters. Additional \fiberRadius are listed in Appendix \dummy{}}
% \label{fig:}
\end{figure}

\begin{figure}[p]
\centering
\includegraphics[width=\textwidth, page=1]{dev/gfx/2/parameter_statistic_time_evolve.pdf}
\caption[Time development crossing.]{Time development of the model generation process of crossing fibers populations.}
% \label{fig:}
\end{figure}
% 
\begin{figure}[p]
\centering
\includegraphics[width=\textwidth, page=2]{dev/gfx/2/parameter_statistic_time_evolve.pdf}
\caption[Time development parallel]{Time development of the model generation process of parallel fiber populations.}
% \label{fig:}
\end{figure}
% 
\subsection{Discussion}
% 
The goal is to choice a set of parameters, which allow a) fast, b) collision free and c) a high volume fraction for densely fibers. 
%  
\subsubsection{timeing}
% 
\paragraph{is overlap important?}
% 
analog analysis like voxelsize -> simulation?
% 
\section{CPU Acceleration}
% 
As described in \dummy{} \openmp is used for acceleration.
This means no usage of multiple computer nodes is currently available.
Since the Code traverses an octree, This would not be feasable.
Also more then 8 cores \dummy{}.
% 
\subsection{Setup}
% 
\subsection{Results}
% 
\subsection{Discussion}
% 
\begin{figure}[!t]
\centering
\includegraphics[]{dev/gfx/4/model_solver_cubes.pdf}
\caption{}
% \label{fig:}
\end{figure}
% 
\section{Building models for simulation}
% 
\subsection{Setup}
% 
\subsection{Results}
% 
\begin{figure}[!t]
\centering
% \resizebox{1.0\textwidth}{!}{
\includegraphics[width=\textwidth, page=1]{dev/gfx/2/cube_2pop_orientation_hist.pdf}
% }
\caption{}
% \label{fig:}
\end{figure}
% 
\subsection{Discussion}
