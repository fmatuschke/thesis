\setcounter{chapter}{7}
\chapter{Simulation}
\label{cha:simulation_analysis}
% \minitoc
% 
% 
% 
This chapter contains focuses on the simulation of \ac{3D-PLI}.
The first part focuses on the definition of all necessary parameters.
A part of this is to determined how accurate the simulation has to be, i.e. the \voxelsize{}.
Following this, these parameters are used to simulate the previously generated models.
A focus is placed on the accuracy of tilt analyses for different orientations and crossing configurations.
% 
% 
% 
\section{Experimental parameter characterisation}\label{sec:sim_choose_parameters}
% 
\subsection{optical resolution}
% 
As describe in \cref{sec:opticalResolution} the optical resolution depence on aberration and diffraction, which themselves depends on a wide range of factors.
The optical resolution and resampling is modelled as described in \cref{sec:ccdOptic}. 
% 
\begin{figure}[!t]
\setlength{\tikzwidth}{0.45\textwidth}
\centering
\begin{tabular}{cc}
\includegraphics[width=0.72\tikzwidth]{dev/wiki/USAF-1951.pdf}
&
% \subcaptionbox{\label{fig:usaf_image}microscopic image}{
\inputtikz[]{dev/gfx/chap8/usaf_image}
% }
\\
\begin{minipage}[t]{0.45\textwidth}
\leavevmode\subcaption{\label{fig:usaf}USAF chart from group -2 to 1 \url{https://en.wikipedia.org/wiki/1951_USAF_resolution_test_chart}}
\end{minipage}&
\begin{minipage}[t]{0.45\textwidth}
\leavevmode\subcaption{\label{fig:usaf_image}microscopic image}
\end{minipage}
\\
% \subcaptionbox{\label{fig:usaf_lines_lr}centered line plots lr}{
% \tikzset{external/export next=false}
\inputtikz{dev/gfx/chap8/usaf_line_plots_lr}
% }
&
% \subcaptionbox{\label{fig:usaf_lines_lr}centered line plots ud}{
% \tikzset{external/export next=false}
\inputtikz{dev/gfx/chap8/usaf_line_plots_up}
% }
\\
\begin{minipage}[t]{0.45\textwidth}
\leavevmode\subcaption{\label{fig:usaf_lines_lr}centered line plots lr}
\end{minipage}&
\begin{minipage}[t]{0.45\textwidth}
\leavevmode\subcaption{\label{fig:usaf_lines_ud}centered line plots ud}
\end{minipage}
\end{tabular}
\caption[USAF test chart measurement]{taorad pm. Line width: magenta: $\SI{2.19}{\micro\meter}$, yellow $\SI{1.95}{\micro\meter}$ and cyan $\SI{1.74}{\micro\meter}$. Resolution at about $\SI{1.95}{\micro\meter}$ yielding to an optical convolution of $\opticsigma = \SI{0.75}{\pixel}$.}
\label{fig:USAF}
\end{figure}
%
To measure the optical resolution of the microscop, the measurements and analysis analog to \cite{MenzelMaster} \todo{check ref} is done again.
For this the \textit{1951 \ac{USAF} resolution test chart}\footnote{U.S. Air Force MIL-STD-150A standard of 1951} is aquired.
It consists of several patterns that have three slits with defined distances and widths (see \cref{fig:usaf}.
They are ordered in patches of three vertical and horizontal lines.
The patches are ordered in a spiral like, which shrinks down from pattern to pattern by a factor of $\SI{0.5}{}$.
To identify the line width, the patches are numerical orded by a main group $i$, and a subgroup $j$.
% 
\paragraph{results}
\Cref{fig:usaf_image} shows a section of a captured PM image.
The highlighted areas show analysed groups 7-6 to 8-2.
The first region corresponding to group 7-6 \raisebox{.25em}{\tikzset{external/export next=false}\tikz \draw[RED,ultra thick,dashed](0,0)--(0.25,0);} has a line width of $\SI{2.19}{\micro\meter}$.
The second group 8-1 \raisebox{.25em}{\tikzset{external/export next=false}\tikz \draw[GREEN,ultra thick,dashed](0,0)--(0.25,0);} has a line width of $\SI{1.95}{\micro\meter}$.
The final group 8-2 \raisebox{.25em}{\tikzset{external/export next=false}\tikz \draw[BLUE,ultra thick,dashed](0,0)--(0.25,0);} has a line width of $\SI{1.74}{\micro\meter}$.
The intensity line profiles for the vertical and horizontal case are shown in \cref{fig:usaf_lines_lr,fig:usaf_lines_ud}.
% 
Using the Rayleigh criterion, the resolution can be limited in the range of the second group and thus around $\SI{1.95}{\micro\meter}$.
This reproduces the measurements in \cite{MenzelMaster} \todo{master thesis?}.
As a result, the convolution to be applied (see \cref{sec:opticalResolution}) is set to $\opticsigma = \SI{0.75}{\pixel}$.
% 
% 
% 
\subsection{sensor gain and image noise}\label{sec:sensorGain}
%
\begin{figure}[!t]
\centering
% 
\begin{tabular}{cc}
% 
\multicolumn{2}{c}{
\def\tikzwidth{0.75\textwidth}
\begin{minipage}{\tikzwidth}
\tikzset{external/export next=false}
\inputtikz{gfx/pli/pli_focus}
\end{minipage}
}
\\[-1em]
% 
\multicolumn{2}{l}{
\begin{minipage}[t]{\textwidth}
\leavevmode\subcaption{\label{fig:intensityFocus}Intensity measurement setup. The focus of the microscop is changed so that the image is blured, resulting in a intensity variation at the border of the absorber material.\\[1em]}
\end{minipage}}
\\
% 
% 1. IMAGE
\begin{tikzpicture}[baseline, trim axis left, trim axis right]
\begin{axis}[
width=0.45\textwidth,
height=0.45\textwidth,
point meta min=0,
point meta max=1670,
xmin=0,xmax=1,
ymin=0,ymax=1,
% 
hide axis,
% axis lines=center,
% % yticklabels={,,},
% axis line style={draw=none},
% tick style={draw=none},
% tick label style={gray},
% 
colorbar left,
colormap/gray,
colorbar style={%
    % axis line style={draw=none},
    % tick style={},
    % tick label style={black},
    every y tick label/.append style={font=\small},
}
% 
]
\addplot [forget plot] graphics [
xmin=0,xmax=1,
ymin=0,ymax=1,
] {dev/gfx/2/PM_000_vmin_0_vmax_1670.png};
\end{axis}
\end{tikzpicture}
&
% 
% 2. IMAGE
\begin{tikzpicture}[baseline, trim axis left, trim axis right]
\begin{axis}[%
width=0.45\textwidth,
height=0.45\textwidth,
% ymajorticks=false,
% axis on top,
axis lines=center,
xlabel={x-pos},
ylabel={intensity},
]
\addplot[mark=none, blue] table[x index=0, y index=1] {dev/gfx/2/PM_000.dat};
\end{axis}
\end{tikzpicture}
\\[-1em]
% 
% SUBCAPTIONS
\multicolumn{1}{l}{
\begin{minipage}[t]{0.45\textwidth}
\leavevmode\subcaption{\label{fig:intensityImage}single microscopic image}
\end{minipage}}
&
\multicolumn{1}{l}{
\begin{minipage}[t]{0.45\textwidth}
\leavevmode\subcaption{\label{fig:intensityProfile}intensity profile.}
\end{minipage}}
\end{tabular}
% 
\caption[intensity image]{y-mean intensity profile}
\label{fig:intensityMeasurement}
\end{figure}
% 
% 
% 
\begin{figure}[!t]
% thesis/2_simulation/0_parameter/sensor_gain.py
\centering
% 
\def\tikzwidth{0.45\textwidth}
\begin{tabular}{cc}
\includegraphics[width=0.45\textwidth]{dev/gfx/2/PM_noise.png}
&
\inputtikz[true]{gfx/data/theo_noise}
\\
% SUBCAPTIONS
\multicolumn{1}{l}{
\begin{minipage}[t]{0.45\textwidth}
\leavevmode\subcaption{\label{fig:parameterModelSimGain} Noise analysis PM. $\opticgain_{\mathit{PM}} = \SI{0.1175}{}(?)$. \itodo{replot}}
\end{minipage}}
&
\multicolumn{1}{l}{
\begin{minipage}[t]{0.45\textwidth}
\leavevmode\subcaption{\label{fig:noiseplot}noise \dummy{} \itodo{check values}}
\end{minipage}}
\end{tabular}
% 
\caption[Noise analysis]{intensity variance and noise plot}
\label{fig:parameterModelSimGain}
\end{figure}
% 
As described in \cref{sec:ccdOptic} the optical noise has to be applied by a noise model.
In \cite{Wiese:887678} the gain factor was measured by measuring multiple times an image with a varieties of intensity values.
The same type of measurement and analysis is performed here on the \ac{PM} setup.
\\
To aquire statistik for multiple a greate range of intensity values, the specimen stage is covered with a fully absorbing cover, so that half of the image is dark.
Additionally the focal length is change in such a way, that the light is spread over the hole image sensor depide of the cover (see \cref{fig:intensityImage}).
By measuring $N=\SI{500}{}$ images, the variance for the different intensity values is aquired.
% 
\paragraph{results}
The results are shown in \cref{fig:parameterModelSimGain} and show a gain value of $\opticgain_{\mathit{PM}} = \SI{0.1175}{}(?)$ which is agreement to the hardware specifics \todo{check and ref}.
This gain factor can be used by the model:
\begin{align}
f(x) = \mathrm{round}\left(\mathrm{normal}(\mu = x, \sigma=\sqrt{\opticgain \cdot x})\right)
\end{align}
\todo{only values > 0}
% 
% 
% 
\subsection{Tissue properties}\label{sec:tissueProp}
% 
\begin{figure}[!t]
\centering
\def\tikzwidth{0.75\textwidth}
\begin{tabular}{c}
% 
\inputtikz[true]{gfx/data/vervet_transmittance}
\\[-1em]
\begin{minipage}[t]{\textwidth}
\leavevmode\subcaption{\label{fig:brain_trans}transmittance}
\end{minipage}
\\
\inputtikz[true]{gfx/data/vervet_retardation}
\\[-1em]
\begin{minipage}[t]{\textwidth}
\leavevmode\subcaption{\label{fig:brain_ret}retardation}
\end{minipage}
% 
\end{tabular}
\caption[Vervet monkey coronal section transmittance and retardation]{%
Vervet monkey coronal section $\SI{549}{}$.
a) transmittance, b) retardation.
The absorption coefficient and birefingence strength can be estimated from flat fibers.
Two \acsp{ROI} with \dummy{} pixels are annotated for this purpose.
\itodo{histograms?, why these two rois?}}
\label{fig:brain_ret_trans}
\end{figure}
% 
% 
% 
\begin{figure}[!t]
% 2_simulation/0_parameter/measure_vervet.ipy
\centering
\hspace*{\fill}
\subcaptionbox{zoom transmittance}{\includegraphics[width=0.45\textwidth]{dev/rc1/transmittance_vervet_cc_0_8339.png}}\hfill
\subcaptionbox{hist transmittance}{\includegraphics[width=0.45\textwidth]{dev/rc1/transmittance_vervet_cc_hist.png}}
\hspace*{\fill}
\\
\hspace*{\fill}
\subcaptionbox{zoom retardation}{\includegraphics[width=0.45\textwidth]{dev/rc1/retardation_vervet_cc_0.000_0.996.png}}\hfill
\subcaptionbox{hist retardation}{\includegraphics[width=0.45\textwidth]{dev/rc1/retardation_vervet_cc_hist.png}}
\hspace*{\fill}
\caption[zoom ret and trans]{%
trans left: $1200 \pm 500$,
trans right: $1200 \pm 500$,
bg: $4530 \pm 240$,
fuellgrad $0.75: \mu \approx 30 \pm 10$,
fuellgrad $0.75: \mu \approx 30 \pm 10$,
ret left: $0.83 \pm0.07$,
ret right: $0.80 \pm0.07$
\itodo{new fig-> path as vector; what happens in the center; check lr flip with images above, histogramm?}}
\label{fig:brain_ret_trans_zoom}
\end{figure}
% 
% analog roden:
% trans: $2100 \pm 600$,
% bg: $4000 \pm 50$,
% fuellgrad $0.75: \mu \approx 14 \pm 7$,
% ret: $0.40 \pm 0.11$,
% and human:
% trans: $170 \pm 80$,
% bg: $2700 \pm 110$,
% fuellgrad $0.75: \mu \approx 61 \pm 11$,
% ret: $0.68 \pm 0.11$,
% 
% 
\begin{figure}[!tp]
\centering
\subcaptionbox{\label{fig:simTransValues}transmittance}[.95\textwidth]{
\includegraphics[width=0.95\textwidth]{dev/rc1/bf_rc1_transmittance_PM_Vervet.pdf}} %\hfill
\subcaptionbox{\label{fig:simRetValues}retardation}[.95\textwidth]{
\includegraphics[width=0.95\textwidth]{dev/rc1/bf_rc1_retardation_PM_Vervet.pdf}
}
\caption{simulations for Vervet PM tissue for different absorption coef and birefringence values.}
\label{fig:parameterModelSim}
\end{figure}
% 
The absorption coefficient $\absorp{}$ and the birefringence $\dn{}$ have to be estimated from the tissue to use suitable values within the simulation.
As the literature shows, these values are in the range of \dummy{}\TODO{!!!} $\dn{}$.
However, since the models generated here use stiff fiber segments, the volume density cannot reach the high density, which is commonly present in the \ac{WM}.
To overcome this problem, the absorption and birefringence strengths must be increased to achieve comparable results.
\par
% 
Both properties can be measured by the \ac{3D-PLI} setup.
Since the focus will be the Vervet monkey (since a large range of registered data is available at this moment), a coronal section will be used to analys both properties.
It is important to ensure that a homogeneous region is chosen.
In a coronal section, the corpus callosum is suitable for this purpose. It is the main fiber connection between the two cerebral hemispheres (see \cref{fig:parameterModelSim}).
% 
% \Cref{fig:parameterModelSim} shows a coronal section of the Vervet monkey brain \dummy[section number 549, aber was ist zu sehen, ist es in der mitte?, ...]{}.
% % 
% To get a approximation of the parameters the models with a radius of $r=\SI{0.5}{\micro\meter}$ are used, which is closed to the real tissue.
% 
\Cref{fig:brain_ret_trans} shows the transmittance and retardation maps of the coronal Cervet section, and \ref{fig:brain_ret_trans_zoom} a zommed version.
The retartdation map shows a visible reduction of the retardation in the center of the corpus calosum.
Because of the necesarry homoginity, this center region is therefore neglectet.
This leaves a left and a right region, which in both transmittance and retardation is feasable homoginious (see histograms).
The left region has $\SI{1125858}{}$ number of pixels, the right part $\SI{1064629}{}$
% 
The same measurements are also aquired for the human and roden section (see appendix). \todo{end of this subsection}
% 
% 
% 
\subsubsection{absorption}
% 
A problem with transmission is that scattering cannot be distinguished from actual absorption in this setup.
However, since this property is mathematically the same, namely a reduction in intensity along the tissue, it can still be used, but it must be remembered that it is the outdated "absorption" and not just the loss of energy in the tissue.
Additionally, the scattering is not constant for all fiber configurations and orientations \dummy[see MM]{}.
Therefore, the value measured here is only valid for the same fiber configurations as the region.
However, it should be a good estimate for the magnitude of absorption in the tissue, which, as will be shown later in the simulation, is sufficient.
% 
\paragraph{results}
A relative transmittance of about $\SIrange{20}{30}{\percent}$ is present inside the \ac{ROI} of the corpus callosum \todo{show histogram}. 
The simulation parameter has to be set such as that for the models a similar transmittance value is reached.
This means, that for a $\SI{60}{\micro\meter}$ thick section the absorption coefficient $\absorp{}$ is about $\SI{30}{}$. 
The absorption for the Roden is $\SI{8}{}$, and the human $\SI{65}{}$.
%
\Cref{fig:simTransValues} shows the simulated transmittance values for different fiber radii. \todo{only one is necesarry, since the birefringence not chaning anythong}.
For smaler radii the variance of the transmittance is very small \dummy{} and increases with the radii.
This behavier is epected, since models with larger radii have more space where no tissue is present, and therefore the light can transmitt freely through the \say{section}.
In comparison to the histogram \cref{fig:brain_ret_trans_zoom} the models with larger radii have more in common with the distribution.
However from literature one can neglect, that this kind of fiber radii is present in corpus calosum \dummy{}. 
Considering the neglection of light scattering one can come to the conclusion, that this beahvier was to be expected.
At this point the here developed simulation is not ready to simulate scattered light.
However, since in the analysis the transmittance is not presents (see \dummy{}) and its role only indicates, how large the noise distribution is (which is for this setup quite small), this ... should be ok.
(Further simulation in the future have to show how large the impact of the transmittance is)
%  
\Cref{fig:noiseplot} shows the intensity noise curve for the ...
% 
% 
%
\subsubsection{birefringence}
% 
To get an estimate of the birefringence of \ac{WM} it is essential in this setup configuration, that the measured nerve fibers have no inlcination.
This of couse is impossible to ensure.
However the corpus callosum again is because of its anatomical structure a good approximation in a coronal section for flat fibers.
Literature measured a birefringence of nerve fibers of around \dummy{}\TODO{!!!}. 
% Scince for this simulations the models are rigide and therefore are spatially farer away from each other, this means that the tissue density is lower than in reality.
% Therefore to reach similar retardation values the birefringence value has to be increased.
To analyse the effect of the lower desity inside the models, the birefringence values is simulated at a voxel size of $\SI{0.125}{\micro\meter}$ at different values.
% 
% 
\paragraph{results}
The results in \cref{fig:parameterModelSim} show that to reach a retardation value similar to the experimental setup of around $\SI{0.8}{}$ a birefringence value of $\SI{-0.004}{}$ for a macroscopic and $\SI{0.008}{}$ for the microscopic model should be chosen.
Since the nerve fibers in this region are not perfectly flat and dense, this is a lower esitmation of the birefingence strength.
However the same is true for the fiber models.
% This is around factor 2 \todo{check} higher than literatur.
% However keeping the lower density in mind and the effect of a fixed myelin radii model of $\SI{75}{\percent}$ this is still in agreement.
Looking at the different model nerve fiber radii the variance of the retardation values increase strong.
This is to be expected since the simulated tissue section has a hight of $\SI{60}{\micro\meter}$.
Therefore for a radii of $\SI{10}{\micro\meter}$ the number of tissue voxels has to vary depending on where the fibers are positioned in the volume.
For larger radii this can lead to an reduction of retardation due to the $\sin(...)$ ambiguity.
That is also the reason why the values are non normal distributed between $\si{0}$ and $\si{1}$.
%  
% 
% 
% \subsection{noise} <--- why this section?
% % 
% From \cref{sec:sensorGain,sec:tissueProp} the noise level can be calculated.
% However, when considering the different species, it is important to note that transmittance levels vary.
% Rodents typically have the highest transmittance, monkeys have a lower transmittance, and humans have the lowest transmittance.
% The reason for the rather low transmission of the human sections is probably due to the long post-mortem period.
% \Cref{fig:noiseplot} shows a log-log plot for this purpose.
% To run the simulation, a relative error of \dummy{} is used for a midrange analysis.
% 
% 
% 
\subsection{voxel size \texorpdfstring{\voxels{}}{}}
% 
% 
\begin{figure}[!tp]%p
% 2_simulation/0_parameter/fiber_radii.py
\centering
\includegraphics[width=\textwidth, page=1]{dev/rc1/voxel_size_plots_data_output_vs_135_0.01_6_25_vervet_r_rc1.pdf}
\caption[voxel size model with noise]{1. $(\modelOmega=0, \modelPsi=1, \modelInc=0)$, 2. $(\modelOmega=0, \modelPsi=1, \modelInc=90)$, 3. $(\modelOmega=90, \modelPsi=0.5, \modelInc=0)$, 4. $(\modelOmega=90, \modelPsi=0.5, \modelInc=90)$: the mean difference between the Signal $\mean(\intensity(\voxels_{\mathit{ref} = \SI{0.0125}{\micro\meter}, \fiberRadius})- \intensity(\voxels, \fiberRadius))$. The mean difference is constant for smaller voxel sizes and does not start to grow significantly before $\voxels=\SI{0.125}{\micro\meter}$. \itodo{labels}}
\label{fig:voxelsizeNoise}
\end{figure}
% 
The \voxelsize{} \voxels{} is the main parameter for simulation accuracy, since it determanse how acurate the models will be discreticed.
Smaller values mean more details, more accurate optical axes, more light rays.
However, it also means an increase in memory by $O(n^3)$.
therefore it is recommended to choose the \voxelsize{} so that it is as large as possible without generating a significant error from the discretiziation.
\par
% 
To investigate this effect, a simulation is performed with different types of voxel sizes from the $\SIrange{0.0125}{1.25}{\micro\meter}$.
The smallest voxel size $\SI{0.0125}{\micro\meter}$ is used as ground truth or comparison. 
However, this voxel size is so small that this is only performed for a single tissue section. \todo{which?, how much memory, ...}
The pixel size \pixelsize{} will be set to $\SI{1.25}{\micro\meter}$. \todo{other parameters}
% 
\paragraph{results}
The results  in \cref{fig:voxelsizeNoise} show that the relative difference to the smallest \voxelsize{} increase statically after a value of \SI{0.125}{\micro\meter}.
The variance of the relative difference increases with the radius of the models as well as with the \voxelsize{}.\todo{was sind die vier unterschiedlichen plots?}
% 
In summary:
The results shown in \cref{fig:voxelsizeNoise} indicate that for noise, a voxel size smaller than $\SI{0.125}{\micro\meter}$ does not increase the accuracy with respect to the noise model chosen here.
However, this is only true for a pixel size of $\SI{1.25}{\micro\meter}$.
% 
% 
% 
% \subsection{Fiber radii}
% % 
% \begin{figure}[!t]
% \centering
% \includegraphics[width=1\textwidth]{dev/rc1/cube_2pop_135_rc1_radius_acc_compare_Vervet_PM_r.pdf}
% \caption[sim acc]{ $\acc{}(GT,GT_{0.5})$, $\acc{}(SIM,GT_{0.5})$ and $\acc{}(SIM,SIM_{0.5})$. hypothese: abweichungen in GtSim bei kleinen radien dadurch, dass zwei richtungen nicht sichtbar sind? schaue dir die histogramme und odfs an. \itodo{komplett neu machen}}
% \label{fig:accVervetPMr}
% \end{figure}
% % 
% The last parameter to investigate is the fiber radius $\fiberRadius{}$ of the models.
% The anatomical fiber radii is around $\fiberRadius{} = \SI{0.5}{\micro\meter}$.
% However to allow faster modelling process it is feasible to analyse with which radius a difference is visible in the microscopic setup.
% To investigate this behavior the distribution of the orientation of the fiber segments was analysed, because the models differ in there geometries. \todo{auf signal angucken?}
% % 
% \paragraph{results}
% \Cref{fig:accVervetPMr} shows the ACC(GT0.5 to GT difference between the orientation of the models of different radii (always to $\fiberRadius{} = \SI{0.5}{\micro\meter}$).
% the second shows the ACC(GT0.5 to SIM) and the last $\acc{}(SIM,SIM_{0.5})$.
% % 
% The first plot shows a very close correspondence between the GT of different radii (see y scala).
% For more intermixed fiber populations (\modelPsi{} closer to $\SI{0.5}{\micro\meter}$) as well as crossing angle \modelOmega{} the \acc{} decreases slightly.
% Overall the difference is almost negligible.
% \par
% %
% Plot 2 shows a different result for the comparison between the $GT$ and $SIM$.
% Here only fiber populations \modelPsi{} close to $\SI{0}{}$ or $\SI{1}{}$ have a high agreement for smaller fiber radii \fiberRadius{}.
% The \acc{} decreeses with incrasing radii \fiberRadius{} down to a value of $\approx \SI{0.7}{}$.
% Fiber populations with \modelPsi{} closer to 0.5 have a very low ACC parameter. For the 0.6 case the \acc{} increases slightly for increasing radii up to acc=0.6.
% The crossing angle has a high impact on the acc value. 
% % 
% \par
% Looking at the results from the $acc(GT,GT_{0.5})$ and $\acc{}(SIM,SIM_{0.5})$ of course $\fiberRadius=\SI{0.5}{\micro\meter}$ has a \acc{} of $\SI{0}{}$ since they are identical.
% On the other hand $acc(GT,SIM_{0.5})$ shows for $\fiberRadius=\SI{0.5}{\micro\meter}$ the smallest values and highest variance for \acc{} since the fiber crossing leads to a low retardation and therefore signal lost and the analysis can't measure the in... signal.
% % 
% From the first and last plot show that a fiber radii of \SI{1}{\micro\meter} seems to have a little impact on the model and simulation side.
% After that the simulation comparison \acc{}\todo{check if acc is defined.} value sinks drastically for fiber populations close to %\fiberPsi=\SI{0.5}{}$.
% % 
% The behavior between the ground truth $GT$ and and simulation $SIM$ will be closely examined in the next section.
% % 
% Since the models exists at this point already for $\fiberRadius = \SI{0.5}{\micro\meter}$ these will be used.
% For future modelling the fibre radii however should be decreased (at this resolution) to allow for faster modelling\footnote{It should also be checked, if the \voxelsize{} can be increased then.}.
% 
% 
% 
\section{Repo}
% 
\paragraph{repo models}
% 
0.0:     1.72 +- 0.03, 1.29 +- 0.02, 0.69 +- 0.03, 1.46 +- 0.04, 2.47 +- 0.05,\\
5.0:     1.73 +- 0.03, 1.30 +- 0.02, 0.70 +- 0.03, 1.46 +- 0.04, 2.48 +- 0.05,\\
10.0:    1.74 +- 0.04, 1.31 +- 0.03, 0.69 +- 0.03, 1.46 +- 0.04, 2.49 +- 0.06,\\
15.0:    1.74 +- 0.02, 1.28 +- 0.02, 0.70 +- 0.02, 1.47 +- 0.03, 2.48 +- 0.04,\\
20.0:    1.72 +- 0.04, 1.27 +- 0.02, 0.71 +- 0.02, 1.45 +- 0.04, 2.45 +- 0.07,\\
25.0:    1.71 +- 0.03, 1.24 +- 0.02, 0.71 +- 0.02, 1.44 +- 0.03, 2.42 +- 0.05,\\
30.0:    1.69 +- 0.02, 1.21 +- 0.02, 0.73 +- 0.02, 1.42 +- 0.03, 2.38 +- 0.04,\\
35.0:    1.69 +- 0.03, 1.19 +- 0.02, 0.75 +- 0.03, 1.41 +- 0.03, 2.35 +- 0.04,\\
40.0:    1.68 +- 0.03, 1.15 +- 0.02, 0.80 +- 0.03, 1.41 +- 0.03, 2.32 +- 0.05,\\
45.0:    1.68 +- 0.03, 1.13 +- 0.02, 0.82 +- 0.03, 1.42 +- 0.03, 2.29 +- 0.04,\\
50.0:    1.71 +- 0.04, 1.11 +- 0.02, 0.88 +- 0.04, 1.47 +- 0.05, 2.31 +- 0.05,\\
55.0:    1.76 +- 0.04, 1.11 +- 0.02, 0.94 +- 0.04, 1.54 +- 0.05, 2.36 +- 0.06,\\
60.0:    1.86 +- 0.03, 1.14 +- 0.04, 1.03 +- 0.03, 1.65 +- 0.04, 2.47 +- 0.04,\\
65.0:    2.03 +- 0.04, 1.21 +- 0.03, 1.16 +- 0.04, 1.82 +- 0.05, 2.68 +- 0.06,\\
70.0:    2.37 +- 0.04, 1.39 +- 0.03, 1.38 +- 0.04, 2.15 +- 0.04, 3.10 +- 0.06,\\
75.0:    3.10 +- 0.05, 1.87 +- 0.09, 1.81 +- 0.05, 2.81 +- 0.06, 4.02 +- 0.07,\\
80.0:    5.79 +- 0.17, 4.90 +- 0.47, 3.24 +- 0.10, 4.95 +- 0.13, 7.00 +- 0.15,\\
85.0:    28.45 +- 0.57, 21.32 +- 0.22, 12.45 +- 0.57, 22.11 +- 0.59, 38.52 +- 1.04,\\
90.0:    45.86 +- 0.90, 25.45 +- 0.43, 24.33 +- 1.24, 46.21 +- 1.42, 67.55 +- 1.41,\\
\par
% 
repo angles are up to an inclination of 65 degree stable < 2.5 degree.
the standard diviation stays under 1.5 degree and is therefore considered stable.
% 
\paragraph{repo simulations}
% 
% 
% 
\section{Simulation}
% 
\begin{enumerate}
\item $r=0.5$ or fabric tissue 
\begin{enumerate}
\item eine population:
\begin{enumerate}
    \item verhalten von transmittance, direction und retardierung wie model
    \item verteilung von orientierung SIM zu GT
\end{enumerate}
% 
\item zwei population:
\begin{enumerate}
    \item verhalten von transmittance, direction und retardierung wie model
    \item analyse signal ohne rauschen, zeigt rofl eine abweichung?
    \item verteilung von orientierung SIM zu GT (odf oder histogram?)
    \item groese "dunkler" Bereich
\end{enumerate}
\end{enumerate}
% 
\item $r > 0.5$ oder "fiberbundle":
\begin{enumerate}
\item zwei populationen
\begin{enumerate}
    \item verhalten von seeded fiber bundles zu solid fiber bundles
    \item sichtbarkeit von zwei orientierungen
\end{enumerate}
\end{enumerate}
% 
\item misc
\begin{enumerate}
    \item speed test
\end{enumerate}
% 
\item future discussions
\begin{enumerate}
    \item n-fiber populations
    \item fiber bundles of different radii
    \item fiber bundles with volume compressing solving steps
\end{enumerate}
\end{enumerate}
% 
Verhalten von transmittance von steilheit?
% 
% 
% 
\subsection{Parameters and execution}
% 
\begin{table}[!b]
\centering
% \sisetup{open-bracket={\{}, close-bracket={\}}, list-final-separator={,},list-pair-separator={,}}%
\pgfplotstabletypeset[%
    thesisTableStyle,
    column type=l,
    columns/variable/.style={string type},
    columns/value/.style={string type},
    every head row/.style={before row=\toprule,after row=\midrule},
    every last row/.style={after row=\bottomrule},
    col sep=&,
    row sep=\\,
]
{variable & value\\
% 
simpli.voxel\_size & $\SI{0.1}{\micro\meter}$\\
simpli.pixel\_size & $\SI{1.3}{\micro\meter}$\\
simpli.voi & $[\SI{-35}{\micro\meter}, \SI{-35}{\micro\meter}, \SI{-30}{\micro\meter}], [\SI{30}{\micro\meter}, \SI{35}{\micro\meter}, \SI{35}{\micro\meter}]$\\
simpli.filter\_rotations & $\SIlist{0;20;40;60;80;100;120;160}{\degree}$\\
simpli.interpolate & \texttt{"Slerp"}\\
simpli.wavelength & $\SI{525}{\nano\meter}$\\
simpli.optical\_sigma & $\SI{0.75}{\pixel}$\\
tilt & $\SI{3.9}{\degree}$\\
simpli.light\_intensity & $\SI{8000}{}$\\
gain & \SI{0.1175}{}\\
% simpli.noise\_model & \texttt{lambda x:np.round(np.random.normal(x,np.sqrt(gain * x))).astype(np.uint16)}\\
simpli.noise\_model & \texttt{lambda x:np.round(np.random.normal(}\\
 & \ \ \ \ \ \texttt{x,np.sqrt(gain * x))).astype(np.uint16)}\\
fiber absorption & ('Roden', 8), ('Vervet', 30), ('Human', 65)\\
fiber model & 'r'\\
fiber birefringence & 0.008 for 'r'\\
fiber radii & (0.75, 1)\\
}
\caption{simpli parameters. \itodo{absorption ist volumenweit, nicht nur in faser} \ITODO{correct my values!!!}}
\label{tab:simParameters}
\end{table}
% 
% 
\Cref{tab:simParameters} lists the parameters of the simulation inside the \fastpli{} variable notation.
The values are chosen due to the discussion of \cref{sec:sim_choose_parameters}.
% 
This configuration results in 
% 
\subsubsection{Execution}
% 
As described in \cref{Software} \fastpli{} contains a pipeline implementation. \dummy{}.
This implementation is used to calculate the discretized volumes (from generator), simulate the signal, performing the optical noise and inclination analysis of the resulting signal with \rofl{}.\footnote{Source code is available in appendix \dummy{}.}
\\
% 
The simulation were performed with a single compute node (\texttt{2x Intel(R) Xeon(R) CPU E5-4657L v2}).
% 
% 
% 
\subsection{single fiber population}
\itodo{mehr inclination winkel fuer diesen Fall? -> JA}
% 
\begin{figure}[!tp]
\centering
\newlength{\width}
\setlength{\width}{0.4\textwidth}
\begin{tabular}{cc}
    \includegraphics[width=\width]{dev/rc1/single/cube_2pop_135_rc1_single_plots_single_pop_hist_0.0.pdf} & \includegraphics[width=\width]{dev/rc1/single/cube_2pop_135_rc1_single_plots_single_pop_hist_50.0.pdf} \\
    \includegraphics[width=\width]{dev/rc1/single/cube_2pop_135_rc1_single_plots_single_pop_hist_10.0.pdf} & \includegraphics[width=\width]{dev/rc1/single/cube_2pop_135_rc1_single_plots_single_pop_hist_60.0.pdf} \\
    \includegraphics[width=\width]{dev/rc1/single/cube_2pop_135_rc1_single_plots_single_pop_hist_20.0.pdf} & \includegraphics[width=\width]{dev/rc1/single/cube_2pop_135_rc1_single_plots_single_pop_hist_70.0.pdf} \\
    \includegraphics[width=\width]{dev/rc1/single/cube_2pop_135_rc1_single_plots_single_pop_hist_30.0.pdf} & \includegraphics[width=\width]{dev/rc1/single/cube_2pop_135_rc1_single_plots_single_pop_hist_80.0.pdf} \\
    \includegraphics[width=\width]{dev/rc1/single/cube_2pop_135_rc1_single_plots_single_pop_hist_40.0.pdf} & \includegraphics[width=\width]{dev/rc1/single/cube_2pop_135_rc1_single_plots_single_pop_hist_90.0.pdf}
\end{tabular}
% 
\caption[sim]{left: 2d log histogramm orientation from rofl analysis of simulation, right: 2d log histogramm of orientation of model segemnts.}
\label{fig:single_fiber_pop_hist}
\end{figure}
% 
First only a single fiber population will be analysed.
Since for a single population the direction is neglectable, only the inclination parameter $\modelInc{}$ remains.
% 
\par
\Cref{fig:single_fiber_pop_hist} shows the orientation distribution of a single fiber population for dirrecent inclinations $\modelInc{}$.
On the left side are the distributions of the model segments orientations, on the right side the ones of the inclination analysis of the simulated images.
The plot are logithimic and weighet by the surface area of the 2d bin.
% For $\modelInc= \SI{90}{\degree}$
The comparison between both distributions show a agreement of the inclination analysis with the models orientations.
The variance however is for the inclination analysis a little bit higher than for the models orientations.\todo{messen}
Since these are log plots, the \say{visible variance} is almost neglectable.
% 
% 
% 
\begin{figure}[!p]
\centering
\setlength{\width}{0.45\textwidth}
\subcaptionbox{\label{fig:single_fiber_pop_epa_trans}transmittance}[\width]{
\includegraphics[width=\width]{dev/rc1/single/cube_2pop_135_rc1_single_plots_single_pop_epa_trans.pdf}}\hfill
\subcaptionbox{\label{fig:single_fiber_pop_epa_dir}direction}[\width]{
\includegraphics[width=\width]{dev/rc1/single/cube_2pop_135_rc1_single_plots_single_pop_epa_dir.pdf}}
\\[2em]
\subcaptionbox{\label{fig:single_fiber_pop_epa_ret}retardation}[\width]{
\includegraphics[width=\width]{dev/rc1/single/cube_2pop_135_rc1_single_plots_single_pop_epa_ret.pdf}}\hfill
\subcaptionbox{\label{fig:single_fiber_pop_rofl_dir}rofl direction}[\width]{
\includegraphics[width=\width]{dev/rc1/single/cube_2pop_135_rc1_single_plots_single_pop_rofl_dir.pdf}}
\\[2em]
\subcaptionbox{\label{fig:single_fiber_pop_rofl_inc}rofl inclination: it is remaped circular around the circmean value}[\width]{
\includegraphics[width=\width]{dev/rc1/single/cube_2pop_135_rc1_single_plots_single_pop_rofl_inc.pdf}}\hfill
\subcaptionbox{\label{fig:single_fiber_pop_rofl_trel}rofl trel}[\width]{
\includegraphics[width=\width]{dev/rc1/single/cube_2pop_135_rc1_single_plots_single_pop_rofl_trel.pdf}}
\caption[]{\dummy[single population rofl analysis; colors left to right: 0, 30, 60, 90 degree inclination; line: theoretical curve]{}}
\label{fig:single_fiber_pop_rofl}
\end{figure}
% 
\par
\Cref{fig:fig:single_fiber_pop_rofl} show the different modalities after analysing the raw measurements for all inclinations.
The transmittance is for inclinations $\modelInc<\SI{90}{\degree}$ almost identical.\footnote{not as in the experiment!}
For $\modelInc= \SI{90}{\degree}$ however the transmittance increases, \ie{} the absorption decreses.
Tje dorectopm va;ies varoamce os wotj \dummy{} very small.
Only for high inclination values its increases.
One must take note, that because the boxplot does not take the periodic signal into account, the values are for the $\modelInc= \SI{90}{\degree}$ actually along the hole range.
The retardation follows for all inclinaten the theoretical model (blue curve).
Its variance stays almost constant with about \dummy{}.
The direction from the inclination analysis is indistinquistable from the fourie analysis.
The inclination follows the models inclination.
Its variance increases for high values significant.
Trel is up to \dummy{} relativ const.
For higher inclination values however its values exceeds 1 which leads to the conclusion, that the model could not fit the data anymore apropiatly.
% 
% 
% 
\subsection{two fiber populations}
% 
\subsubsection{flat crossing fiber configurations}
% 
\begin{figure}[!p]
\centering
\setlength{\width}{0.4\textwidth}
\begin{tabular}{cc}
    \includegraphics[width=\width]{dev/rc1/flat/cube_2pop_135_rc1_flat_plots_flat_pop_hist_omega_0.0_psi_0.3.pdf} &
    \includegraphics[width=\width]{dev/rc1/flat/cube_2pop_135_rc1_flat_plots_flat_pop_hist_omega_50.0_psi_0.3.pdf} \\
    \includegraphics[width=\width]{dev/rc1/flat/cube_2pop_135_rc1_flat_plots_flat_pop_hist_omega_10.0_psi_0.3.pdf} & \includegraphics[width=\width]{dev/rc1/flat/cube_2pop_135_rc1_flat_plots_flat_pop_hist_omega_60.0_psi_0.3.pdf} \\
    \includegraphics[width=\width]{dev/rc1/flat/cube_2pop_135_rc1_flat_plots_flat_pop_hist_omega_20.0_psi_0.3.pdf} & \includegraphics[width=\width]{dev/rc1/flat/cube_2pop_135_rc1_flat_plots_flat_pop_hist_omega_70.0_psi_0.3.pdf} \\
    \includegraphics[width=\width]{dev/rc1/flat/cube_2pop_135_rc1_flat_plots_flat_pop_hist_omega_30.0_psi_0.3.pdf} & \includegraphics[width=\width]{dev/rc1/flat/cube_2pop_135_rc1_flat_plots_flat_pop_hist_omega_80.0_psi_0.3.pdf} \\
    \includegraphics[width=\width]{dev/rc1/flat/cube_2pop_135_rc1_flat_plots_flat_pop_hist_omega_40.0_psi_0.3.pdf} & \includegraphics[width=\width]{dev/rc1/flat/cube_2pop_135_rc1_flat_plots_flat_pop_hist_omega_90.0_psi_0.3.pdf}
\end{tabular}
% 
\caption[sim]{psi=0.3; left: 2d log histogramm orientation from rofl analysis of simulation, right: 2d log histogramm of orientation of model segemnts.}
\label{fig:flat_03_fiber_pop_hist}
\end{figure}
% 
\begin{figure}[!p]
\centering
\setlength{\width}{0.45\textwidth}
\subcaptionbox{\label{fig:flat_03_fiber_pop_epa_trans}transmittance}[\width]{
\includegraphics[width=\width]{dev/rc1/flat/cube_2pop_135_rc1_flat_plots_flat_pop_psi_0.3_epa_trans.pdf}}\hfill
\subcaptionbox{\label{fig:flat_03_fiber_pop_epa_dir}direction}[\width]{
\includegraphics[width=\width]{dev/rc1/flat/cube_2pop_135_rc1_flat_plots_flat_pop_psi_0.3_epa_dir.pdf}}
\\[2em]
\subcaptionbox{\label{fig:flat_03_fiber_pop_epa_ret}retardation}[\width]{
\includegraphics[width=\width]{dev/rc1/flat/cube_2pop_135_rc1_flat_plots_flat_pop_psi_0.3_epa_ret.pdf}}\hfill
\subcaptionbox{\label{fig:flat_03_fiber_pop_rofl_dir}rofl direction}[\width]{
\includegraphics[width=\width]{dev/rc1/flat/cube_2pop_135_rc1_flat_plots_flat_pop_psi_0.3_rofl_dir.pdf}}
\\[2em]
\subcaptionbox{\label{fig:flat_03_fiber_pop_rofl_inc}rofl inclination: it is remaped circular around the circmean value}[\width]{
\includegraphics[width=\width]{dev/rc1/flat/cube_2pop_135_rc1_flat_plots_flat_pop_psi_0.3_rofl_inc.pdf}}\hfill
\subcaptionbox{\label{fig:flat_03_fiber_pop_rofl_trel}rofl trel}[\width]{
\includegraphics[width=\width]{dev/rc1/flat/cube_2pop_135_rc1_flat_plots_flat_pop_psi_0.3_rofl_trel.pdf}}
\caption[]{\dummy[flat population psi=0.3 rofl analysis; colors left to right: 0, 30, 60, 90 degree inclination; line: theoretical curve]{}}
\label{fig:flat_03_fiber_pop_rofl}
\end{figure}
% 
\begin{figure}[!p]
\centering
\setlength{\width}{0.4\textwidth}
\begin{tabular}{cc}
    \includegraphics[width=\width]{dev/rc1/flat/cube_2pop_135_rc1_flat_plots_flat_pop_hist_omega_0.0_psi_0.5.pdf} &
    \includegraphics[width=\width]{dev/rc1/flat/cube_2pop_135_rc1_flat_plots_flat_pop_hist_omega_50.0_psi_0.5.pdf} \\
    \includegraphics[width=\width]{dev/rc1/flat/cube_2pop_135_rc1_flat_plots_flat_pop_hist_omega_10.0_psi_0.5.pdf} & \includegraphics[width=\width]{dev/rc1/flat/cube_2pop_135_rc1_flat_plots_flat_pop_hist_omega_60.0_psi_0.5.pdf} \\
    \includegraphics[width=\width]{dev/rc1/flat/cube_2pop_135_rc1_flat_plots_flat_pop_hist_omega_20.0_psi_0.5.pdf} & \includegraphics[width=\width]{dev/rc1/flat/cube_2pop_135_rc1_flat_plots_flat_pop_hist_omega_70.0_psi_0.5.pdf} \\
    \includegraphics[width=\width]{dev/rc1/flat/cube_2pop_135_rc1_flat_plots_flat_pop_hist_omega_30.0_psi_0.5.pdf} & \includegraphics[width=\width]{dev/rc1/flat/cube_2pop_135_rc1_flat_plots_flat_pop_hist_omega_80.0_psi_0.5.pdf} \\
    \includegraphics[width=\width]{dev/rc1/flat/cube_2pop_135_rc1_flat_plots_flat_pop_hist_omega_40.0_psi_0.5.pdf} & \includegraphics[width=\width]{dev/rc1/flat/cube_2pop_135_rc1_flat_plots_flat_pop_hist_omega_90.0_psi_0.5.pdf}
\end{tabular}
% 
\caption[sim]{psi=0.5; left: 2d log histogramm orientation from rofl analysis of simulation, right: 2d log histogramm of orientation of model segemnts.}
\label{fig:flat_05_fiber_pop_hist}
\end{figure}
% 
\begin{figure}[!p]
\centering
\setlength{\width}{0.45\textwidth}
\subcaptionbox{\label{fig:flat_05_fiber_pop_epa_trans}transmittance}[\width]{
\includegraphics[width=\width]{dev/rc1/flat/cube_2pop_135_rc1_flat_plots_flat_pop_psi_0.5_epa_trans.pdf}}\hfill
\subcaptionbox{\label{fig:flat_05_fiber_pop_epa_dir}direction}[\width]{
\includegraphics[width=\width]{dev/rc1/flat/cube_2pop_135_rc1_flat_plots_flat_pop_psi_0.5_epa_dir.pdf}}
\\[2em]
\subcaptionbox{\label{fig:flat_05_fiber_pop_epa_ret}retardation}[\width]{
\includegraphics[width=\width]{dev/rc1/flat/cube_2pop_135_rc1_flat_plots_flat_pop_psi_0.5_epa_ret.pdf}}\hfill
\subcaptionbox{\label{fig:flat_05_fiber_pop_rofl_dir}rofl direction}[\width]{
\includegraphics[width=\width]{dev/rc1/flat/cube_2pop_135_rc1_flat_plots_flat_pop_psi_0.5_rofl_dir.pdf}}
\\[2em]
\subcaptionbox{\label{fig:flat_05_fiber_pop_rofl_inc}rofl inclination: it is remaped circular around the circmean value}[\width]{
\includegraphics[width=\width]{dev/rc1/flat/cube_2pop_135_rc1_flat_plots_flat_pop_psi_0.5_rofl_inc.pdf}}\hfill
\subcaptionbox{\label{fig:flat_05_fiber_pop_rofl_trel}rofl trel}[\width]{
\includegraphics[width=\width]{dev/rc1/flat/cube_2pop_135_rc1_flat_plots_flat_pop_psi_0.5_rofl_trel.pdf}}
\caption[]{\dummy[flat population psi=0.5 rofl analysis; colors left to right: 0, 30, 60, 90 degree inclination; line: theoretical curve]{}}
\label{fig:flat_05_fiber_pop_rofl}
\end{figure}
% 
\Cref{fig:flat_crossing_hist} shows the histogram distribution for two fiber popilations with a inclination of $\SI{0}{\degree}$.
Lookiing at the distribtiuon the cariance is slightly increases in comparison to the single fiber population case.
The intensity of the log plot follows the models population density \modelPsi{} \todo{check}.
The inclination of the inclination analysis in the mean with $\SI{0}{\degree}$.
The variance is \dummy{}.
The resulting direction of the analysis is between the models direction distribution:
\begin{align}
    \varphi = \omega_0 \varphi_0 + \omega_1 \varphi_1
\end{align}
\itodo{check this formular. also theoretical!}
% 
% 
% 
\subsection{inclined fibers}
% 
\begin{figure}[!p]
\centering
\includegraphics[width=\textwidth, ]{dev/rc1/analysis/simulation_analysis_hist_0.5_setup_PM_s_Vervet_m_r_acc.pdf} 
\caption[sim]{acc 0 - 1 \dummy{}}
% \label{fig:sim_fyjsrg}
\end{figure}
\begin{figure}[!p]
\centering
\includegraphics[width=\textwidth, ]{dev/rc1/analysis/simulation_analysis_hist_0.5_setup_PM_s_Vervet_m_r_ret_mean.pdf} 
\caption[sim]{ret mean 0.014-0.798 \dummy{}}
% \label{fig:sim_fyjsrg}
\end{figure}
\begin{figure}[!p]
\centering
\includegraphics[width=\textwidth, ]{dev/rc1/analysis/simulation_analysis_hist_0.5_setup_PM_s_Vervet_m_r_trans_mean.pdf} 
\caption[sim]{trans mean 99-1113 \dummy{}}
% \label{fig:sim_fyjsrg}
\end{figure}
\begin{figure}[!p]
\centering
\includegraphics[width=\textwidth, ]{dev/rc1/analysis/simulation_analysis_hist_0.5_setup_PM_s_Vervet_m_r_trel_mean.pdf} 
\caption[sim]{trel mean 0-0.55 \dummy{}}
% \label{fig:sim_fyjsrg}
\end{figure}
% 
\subsection{filled fiber populations}
\subsection{misc}
\section{summary/discussion}
% 
% 
% 
% % \subsection{expected results}
% % 
% % 
% % 
% \subsection{Results}
% % 
% \itodo{other results?}
% \begin{itemize}
% \item noise(omega, gamma, ...)
% \item second orientation visible?
% \item absolute birefringence, transmittance, absorption as a f(omega)
% \item trel? abnahme der doppelbrechnung in flachen faser, bzw gegenüber f0?
% \end{itemize}
% % 
% % 
% % 
% \subsection{Retardation comparison}
% % 
% \begin{figure}[!t]
% \centering
% \inputtikz{dev/gfx/sim_cube/data_rofl}
% % \includegraphics[width=0.75\textwidth]{example-image} 
% \caption[data vs rofl]{data vs rofl}
% \label{fig:dataVsRofl}
% \end{figure}
% \itodo{show sinus plot of single pixel}
% % 
% % \begin{figure}[!tp]
% % \centering
% % \includegraphics[width=\textwidth]{dev/gfx/sim_cube/simulation_retardation_PM_Vervet_r.pdf}
% % % 
% % \begin{tikzpicture}
% %  \begin{axis}[
% %  scale only axis, width=0pt, height=0pt, hide axis,
% %  tick label style={/pgf/number format/.cd, fixed},
% %  colorbar,colormap/viridis high res,
% %  point meta min=0,
% %  point meta max=1,
% %   colorbar horizontal,
% %   colorbar style={width=0.75\textwidth},
% %   ]%
% %   {};
% %  \end{axis}
% % \end{tikzpicture}
% % % 
% % \caption[simulation results retardation]{0.5, PM, Vervet, r, retardation  \dummy{}}
% % \label{fig:sim_retardation_05_PM_Vervet_r}
% % \end{figure} 
% % 
% % 
% % 
% \subsection{Orientation comparison}
% % 
% \begin{figure}[!tp]
% \centering
% \includegraphics[width=\textwidth, trim = 115 50 115 0, clip]{dev/gfx/sim_cube/tissue_overlay_PM_Vervet_r_0.60_60.00_10.00_30.00_0.00_.pdf} 
% \caption[sim]{\dummy{}}
% \label{fig:sim_fyjsrg}
% \end{figure}
% % 
% % 
% % 
% % 
% \begin{figure}[!tp]
% \centering
% \includegraphics[height=0.8\textheight]{dev/gfx/sim_cube/test_r_0.5_species_Vervet_microscope_PM_model_r_inc_0.00_rot_0.00.pdf} 
% \caption[sim]{\dummy{}}
% \label{fig:sim_matrix_0}
% \end{figure}
% % 
% % 
% % 
% % 
% \begin{figure}[!tp]
% \centering
% \includegraphics[height=0.8\textheight]{dev/gfx/sim_cube/test_r_0.5_species_Vervet_microscope_PM_model_r_inc_90.00_rot_0.00.pdf} 
% \caption[sim]{\dummy{}}
% \label{fig:sim_matrix_90}
% \end{figure}
% % 
% % 
% % 
% % \begin{figure}[!t]
% % \begin{tikzpicture}[trim axis left, baseline]
% % \begin{axis}[%
% %     axis equal,
% %     axis lines = center,
% %     axis line style={opacity=0.0},
% %     ticks=none,
% %     view/h=145,
% %     scale uniformly strategy=units only,
% %     clip=false, % hide axis,
% %     width=10cm,
% %     height=10cm,
% %     xmin=-1,xmax=1,
% %     ymin=-1,ymax=1,
% %     zmin=-1,zmax=1,
% %     point meta min=0, point meta max=1,
% %     colormap/viridis,
% % ]
% % \addplot3[%
% %     surf,
% %     shader=interp,
% %     z buffer=sort,
% %     mesh/color input=explicit,
% %     ]
% %  table[x=x,y=y,z=z,
% %         meta=rgb,
% %         ] {dev/gfx/sim_cube/cube_2pop_psi_0.60_omega_60.00_r_10.00_v0_120_.solved_vs_0.1250_inc_30.00_rot_0.00_PM_Vervet_r.gt.dat};
% % \end{axis}
% % \end{tikzpicture}
% % \begin{tikzpicture}[trim axis left, baseline]
% % \begin{axis}[%
% %     axis equal,
% %     axis lines = center,
% %     axis line style={opacity=0.0},
% %     ticks=none,
% %     view/h=145,
% %     scale uniformly strategy=units only,
% %     clip=false, % hide axis,
% %     width=10cm,
% %     height=10cm,
% %     xmin=-1,xmax=1,
% %     ymin=-1,ymax=1,
% %     zmin=-1,zmax=1,
% %     point meta min=0, point meta max=1,
% %     colormap/viridis,
% % ]
% % \addplot3[%
% %     surf,
% %     shader=interp,
% %     z buffer=sort,
% %     mesh/color input=explicit,
% %     ]
% %  table[x=x,y=y,z=z,
% %         meta=rgb,
% %         ] {dev/gfx/sim_cube/cube_2pop_psi_0.60_omega_60.00_r_10.00_v0_120_.solved_vs_0.1250_inc_30.00_rot_0.00_PM_Vervet_r.rofl.dat};
% % \end{axis}
% % \end{tikzpicture}
% % % 
% % \caption[odf]{\dummy{}}
% % \label{fig:sim_fyjsrg}
% % \end{figure}
% % 
% % 
% % 
% % 
% \begin{figure}[!tp]
% \centering
% \includegraphics[width=\textwidth]{dev/gfx/sim_cube/simulation_analysis_hist_0.5_setup_PM_s_Vervet_m_r_acc.pdf}
% % 
% \begin{tikzpicture}
%  \begin{axis}[
%  scale only axis, width=0pt, height=0pt, hide axis,
%  tick label style={/pgf/number format/.cd, fixed},
%  colorbar,colormap/viridis high res,
%  point meta min=0,
%  point meta max=1,
%   colorbar horizontal,
%   colorbar style={width=0.75\textwidth,},
%   ]%
%   {};
%  \end{axis}
% \end{tikzpicture}
% % 
% \caption[simulation results acc]{0.5, PM, Vervet, r, acc-distance}
% \label{fig:hist_0.5_setup_PM_s_Vervet_m_r_acc}
% \end{figure}
% % 
% % 
% \begin{figure}[!tp]
% \centering
% \includegraphics[width=\textwidth]{dev/gfx/sim_cube/simulation_analysis_hist_0.5_setup_PM_s_Vervet_m_r_trans_mean.pdf}
% % 
% \begin{tikzpicture}
%  \begin{axis}[
%  scale only axis, width=0pt, height=0pt, hide axis,
%  tick label style={/pgf/number format/.cd, fixed},
%  colorbar,colormap/viridis high res,
%  point meta min=1000.24,
%  point meta max=1109.23,
%   colorbar horizontal,
%   colorbar style={width=0.75\textwidth,},
%   ]%
%   {};
%  \end{axis}
% \end{tikzpicture}
% % 
% \caption[simulation results transmittance]{0.5, PM, Vervet, r, transmittance}
% \label{fig:hist_0.5_setup_PM_s_Vervet_m_r_trans_mean}
% \end{figure}
% % 
% % 
% \begin{figure}[!tp]
% \centering
% \includegraphics[width=\textwidth]{dev/gfx/sim_cube/simulation_analysis_hist_0.5_setup_PM_s_Vervet_m_r_ret_mean.pdf}
% % 
% \begin{tikzpicture}
%  \begin{axis}[
%  scale only axis, width=0pt, height=0pt, hide axis,
%  tick label style={/pgf/number format/.cd, fixed},
%  colorbar,colormap/viridis high res,
%  point meta min=0.016,
%  point meta max=0.798,
%   colorbar horizontal,
%   colorbar style={width=0.75\textwidth,},
%   ]%
%   {};
%  \end{axis}
% \end{tikzpicture}
% % 
% \caption[simulation results ret]{0.5, PM, Vervet, r, retardation}
% \label{fig:hist_0.5_setup_PM_s_Vervet_m_r_ret_mean}
% \end{figure}
% %
% % 
% \begin{figure}[!tp]
% \centering
% \includegraphics[width=\textwidth]{dev/gfx/sim_cube/simulation_analysis_hist_0.5_setup_PM_s_Vervet_m_r_trel_mean.pdf}
% % 
% \begin{tikzpicture}
%  \begin{axis}[
%  scale only axis, width=0pt, height=0pt, hide axis,
%  tick label style={/pgf/number format/.cd, fixed},
%  colorbar,colormap/viridis high res,
%  point meta min=0,
%  point meta max=0.55,
%   colorbar horizontal,
%   colorbar style={width=0.75\textwidth,},
%   ]%
%   {};
%  \end{axis}
% \end{tikzpicture}
% % 
% \caption[simulation results trel]{0.5, PM, Vervet, r, trel}
% \label{fig:hist_0.5_setup_PM_s_Vervet_m_r_trel_mean}
% \end{figure}
% % 
% % 
% \begin{figure}[!tp]
% \centering
% \includegraphics[width=\textwidth]{dev/gfx/sim_cube/test_PM_Vervet_r_0.50_90.00_0.50_0.00_0.00_.pdf}
% \caption[simulation analysis]{\dummy{}}
% \label{fig:PM_Vervet_r_0.50_90.00_0.50_0.00_0.00}
% \end{figure}
% % 
% % 
% \begin{figure}[!tp]
% \centering
% \includegraphics[width=\textwidth]{dev/gfx/sim_cube/test_PM_Vervet_r_0.50_90.00_0.50_0.00_90.00_.pdf}
% \caption[simulation analysis]{\dummy{}}
% \label{fig:PM_Vervet_r_0.50_90.00_0.50_0.00_90.00}
% \end{figure}
% % 
% % 
% % 
% % \subsubsection{LAP}
% % % 
% % -> discussion
% % 
% % 
% % 
% \section{Computational speed}
% % 
% % 
% % 
\begin{figure}[!p]
\centering
\includegraphics[width=1\textwidth]{dev/rc1/cube_2pop_135_rc1_radius_acc_compare_Vervet_PM_r.pdf}
\caption[sim acc]{ $\acc{}(GT,GT_{0.5})$, $\acc{}(SIM,GT_{0.5})$ and $\acc{}(SIM,SIM_{0.5})$. hypothese: abweichungen in GtSim bei kleinen radien dadurch, dass zwei richtungen nicht sichtbar sind? schaue dir die histogramme und odfs an. \itodo{komplett neu machen}}
\label{fig:accVervetPMr}
\end{figure}
% 
