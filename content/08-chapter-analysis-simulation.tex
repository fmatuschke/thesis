\setcounter{chapter}{7}
\chapter{Simulation}
\label{cha:simulation_analysis}
% \minitoc
% 
% 
\section{Introduction}
% 
This chapter addresses the simulation of \ac{3D-PLI}.
The first part focuses on the determination of all necessary physical parameters of the tissue and the microscope as well as on the characterization of the simulation parameters.
The latter focuses on the \voxelsize{}, which significantly determines the accuracy.
Subsequently, these parameters are used to simulate the previously created models.
The simulations are evaluated using the routine algorithms implemented in \ac{fastPLI}. The focus of the evaluations of the results is on the accuracy of the inclination analyses for different orientations and intersection configurations.
% 
% 
% 
\section{Parameter characterisation}\label{sec:sim_choose_parameters}
% 
\subsection{Optical resolution}
% 
As described in \cref{sec:opticalResolution}, optical resolution depends on aberration and diffraction, which in turn depend on a variety of factors.
Optical resolution and resampling are modeled as described in \cref{sec:ccdOptic}. 
% 
\begin{figure}[!t]
\setlength{\tikzwidth}{0.35\textwidth}
\centering
\begin{tabular}{cc}
\includegraphics[width=\tikzwidth]{dev/wiki/USAF-1951.pdf}
&
% \tikzset{external/remake next=true}
\inputtikz{dev/gfx/chap8/usaf_image}
\\
% 
\multicolumn{1}{l}{
\begin{minipage}[t]{0.465\textwidth}
\leavevmode\subcaption{\label{fig:usaf}USAF chart from group -2 to 1: \url{https://en.wikipedia.org/wiki/1951_USAF_resolution_test_chart}}
\end{minipage}}&
\multicolumn{1}{l}{
\begin{minipage}[t]{0.465\textwidth}
\leavevmode\subcaption{\label{fig:usaf_image}microscopic image}
\end{minipage}}
\\[2em]
% 
% \tikzset{external/remake next=true}
\inputtikz{dev/gfx/chap8/usaf_line_plots_lr}
&
% \tikzset{external/remake next=true}
\inputtikz{dev/gfx/chap8/usaf_line_plots_up}
\\
% 
\multicolumn{1}{l}{
\begin{minipage}[t]{0.45\textwidth}
\leavevmode\subcaption{\label{fig:usaf_lines_lr}centered line plots lr}
\end{minipage}}&
\multicolumn{1}{l}{
\begin{minipage}[t]{0.45\textwidth}
\leavevmode\subcaption{\label{fig:usaf_lines_ud}centered line plots ud}
\end{minipage}}
\end{tabular}
\caption[USAF test chart measurement]{\acs{LMP} data. Line width: magenta: $\SI{2.19}{\micro\meter}$, yellow $\SI{1.95}{\micro\meter}$ and cyan $\SI{1.74}{\micro\meter}$. Resolution at about $\SI{1.95}{\micro\meter}$ yielding to an optical convolution of $\opticsigma = \SI{0.75}{\pixel}$.}
\label{fig:USAF}
\end{figure}
%
In order to measure the optical resolution of the microscope, the measurements and analysis are performed again in the same way as \cite{MenzelMaster}.
For this purpose, the \textit{1951 United States Air Force (USAF) resolution test chart}\footnote{U.S. Air Force MIL-STD-150A standard of 1951} is acquired.
It consists of several patterns that have three slots with defined spacing and width (see \cref{fig:usaf}).
They are arranged in fields of three vertical and horizontal lines.
The fields are arranged in a spiral, which shrinks by a factor of $\SI{0.5}{}$ from pattern to pattern.
To determine the line width, the fields are numerically ordered according to a main group $i$ and a subgroup $j$.
To determine the resolution of the microscope, the group corresponding to the imaging resolution system, \ie{} Rayleigh criteria (see \cref{fig:rayleigh}), must be identified.
This is done by analyzing the intensity profiles along the perpendicular line of the three slits.
The variance of the measured intensity is measured by analyzing all parallel lines in the slits.\footnote{Multiple images should be measured. These are performed in \cite{MenzelMaster}. This measurement ensures that the results are reproducible}.
% 
\paragraph{Results}
\Cref{fig:usaf_image} shows a section of a captured \ac{LMP} setup image.
The highlighted areas show the analyzed groups 7-6 to 8-2.
The first area of group 7-6 \raisebox{.25em}{\tikzset{external/export next=false}\tikz \draw[RED,ultra thick,dashed](0,0)--(0.25,0);} has a line width of $\SI{2.19}{\micro\meter}$.
The second group 8-1 \raisebox{.25em}{\tikzset{external/export next=false}\tikz \draw[GREEN,ultra thick,dashed](0,0)--(0.25,0);} has a line width of $\SI{1.95}{\micro\meter}$.
The last group 8-2 \raisebox{.25em}{\tikzset{external/export next=false}\tikz \draw[BLUE,ultra thick,dashed](0,0)--(0.25,0);} has a line width of $\SI{1.74}{\micro\meter}$.
The intensity line profiles for the vertical and horizontal cases are shown in \cref{fig:usaf_lines_lr,fig:usaf_lines_ud}.
% 
Using the Rayleigh criterion, the resolution can be limited in the range of the second group and thus by $\SI{1.95}{\micro\meter}$.
This reproduces the measurements in the \cite{MenzelMaster}.
Therefore, the convolution to be applied (see \cref{sec:opticalResolution}) is set to $\opticsigma = \SI{0.75}{\pixel}$.
% 
% 
% 
\subsection{sensor gain and image noise}\label{sec:sensorGain}
%
\begin{figure}[!t]
\centering
% 
\tikzset{external/export=false}
\setlength{\tikzwidth}{0.3\textwidth}
% 
\begin{tabular}{cc}
\resizebox{!}{\tikzwidth}{
\inputtikz{gfx/pli/pli_focus}}
&
\inputtikz{dev/gfx/2/PM_000_image}
\\[-1em]
% 
% SUBCAPTIONS
\multicolumn{1}{l}{
\begin{minipage}[t]{0.47\textwidth}
\leavevmode\subcaption{\label{fig:pliFocus}unfocused microscopic image}
\end{minipage}}
&
\multicolumn{1}{l}{
\begin{minipage}[t]{0.47\textwidth}
\leavevmode\subcaption{\label{fig:pliFocusImage}image}
\end{minipage}}
\\[2em]
% 
\inputtikz{gfx/data/PM_noise}
&
\inputtikz{gfx/data/theo_noise}
\\[-1em]
% 
% SUBCAPTIONS
\multicolumn{1}{l}{
\begin{minipage}[t]{0.47\textwidth}
\leavevmode\subcaption{\label{fig:parameterModelNoise} Linear regression results in a gain factor of $\opticgain_{\mathit{LMP}} = \SI{0.1175}{}(?)$.}
\end{minipage}}
&
\multicolumn{1}{l}{
\begin{minipage}[t]{0.47\textwidth}
\leavevmode\subcaption{\label{fig:noiseplot}Expected noise range for different species \dummy[check values]{}}
\end{minipage}}
\end{tabular}
% 
\caption[Noise analysis]{Intensity noise measurements and results}
\label{fig:parameterModelGain}
\end{figure}
% 
As described in \cref{sec:ccdOptic}, optical noise must be modeled by a noise model.
In \cite{Wiese:887678}, the gain factor $\opticgain$ of the microscopic setup was measured by multiple measurements of an image with a variety of intensity values. 
The gain factor describes the linearity between a measured signal and the corresponding noise.
The same type of measurement and analysis is performed here with the \ac{LMP} setup.
\\
To obtain statistics for a large number of intensity values, the sample stage is covered with a fully absorbing cover so that half of the image is dark.
In addition, the focal length is changed so that light is distributed over the entire image sensor below the cover (see \cref{fig:intensityImage}).
By measuring $N=\SI{500}{}$ images, the variance for the different intensity values is determined.
% 
\paragraph{results}
The results are shown in \cref{fig:parameterModelNoise} and show a gain value of $\opticgain_{\mathit{LMP}} = \SI{0.1175}{}(?)$ which is agreement to the hardware specifics \todo{check and ref}.
This gain factor can be used by the model
\begin{align}
f(x) = \floor{\mathrm{normal}(\mu = x, \sigma=\sqrt{\opticgain x})+0.5}
\end{align}
for intensities $I > 0$, $I \gg \sqrt{\opticgain I}$ and integer values.
% 
% 
% 
\subsection{Tissue properties}\label{sec:tissueProp}
% 
\begin{figure}[!t]
\centering
\setlength{\tikzwidth}{0.75\textwidth}
\begin{tabular}{c}
% 
\inputtikz{gfx/data/vervet_transmittance}
\\[-2em]
\begin{minipage}[t]{0.75\textwidth}
\leavevmode\subcaption{\label{fig:brain_trans}transmittance}
\end{minipage}
\\[1.5em]
\inputtikz{gfx/data/vervet_retardation}
\\[-2em]
\begin{minipage}[t]{0.75\textwidth}
\leavevmode\subcaption{\label{fig:brain_ret}retardation}
\end{minipage}
% 
\end{tabular}
\caption[Vervet monkey coronal section transmittance and retardation]{%
Transmittance and retardation map of a Vervet monkey coronal section $\SI{549}{}$.
The absorption coefficient and birefingence strength can be estimated from flat fibers.
Two \acsp{ROI} with \dummy{} pixels are annotated for this purpose.
\itodo{histograms?, why these two rois?}}
\label{fig:brain_ret_trans}
\end{figure}
% 
% 
% 
\begin{figure}[!t]
% 2_simulation/0_parameter/measure_vervet.ipy
\centering
\setlength{\tikzwidth}{0.425\textwidth}
\begin{tabular}{cc}
% 
\inputtikz{gfx/data/vervet_transmittance_zoom} &
% \tikzset{external/export next=false}
\inputtikz{gfx/data/vervet_transmittance_hist} \\[-5mm]
% 
\multicolumn{1}{l}{
\begin{minipage}[t]{0.45\textwidth}
\leavevmode\subcaption{zoom transmittance}
\end{minipage}} &
\multicolumn{1}{l}{
\begin{minipage}[t]{0.45\textwidth}
\leavevmode\subcaption{hist transmittance}
\end{minipage}} \\[10mm]
% 
\inputtikz{gfx/data/vervet_retardation_zoom} &
% \tikzset{external/export next=false}
\inputtikz{gfx/data/vervet_retardation_hist} \\[-5mm]
% 
\multicolumn{1}{l}{
\begin{minipage}[t]{0.45\textwidth}
\leavevmode\subcaption{zoom retardation}
\end{minipage}} &
\multicolumn{1}{l}{
\begin{minipage}[t]{0.45\textwidth}
\leavevmode\subcaption{hist retardation}
\end{minipage}} \\
% 
\end{tabular}
\caption[zoom ret and trans]{%
trans left: $1200 \pm 500$,
trans right: $1200 \pm 500$,
bg: $4530 \pm 240$,
fuellgrad $0.75: \mu \approx 30 \pm 10$,
fuellgrad $0.75: \mu \approx 30 \pm 10$,
ret left: $0.83 \pm 0.07$,
ret right: $0.80 \pm 0.07$
\itodo{what happens in the center?}}
\label{fig:brain_ret_trans_zoom}
\end{figure}
% 
% analog roden:
% trans: $2100 \pm 600$,
% bg: $4000 \pm 50$,
% fuellgrad $0.75: \mu \approx 14 \pm 7$,
% ret: $0.40 \pm 0.11$,
% and human:
% trans: $170 \pm 80$,
% bg: $2700 \pm 110$,
% fuellgrad $0.75: \mu \approx 61 \pm 11$,
% ret: $0.68 \pm 0.11$,
% 
% 
\begin{figure}[!t]
\centering
% \begin{tabular}{c}
\subcaptionbox{\label{fig:simTransValues}transmittance}[0.95\textwidth]{
\includegraphics[page=2]{dev/rc1/tissue/birefringence_output_bf_rc1.pdf}}\\[1em]
\subcaptionbox{\label{fig:simRetValues}retardation}[0.95\textwidth]{
\includegraphics[page=5]{dev/rc1/tissue/birefringence_output_bf_rc1.pdf}}
% \end{tabular}
\caption{simulations for Vervet LMP tissue for different absorption coef and birefringence values. \itodo{exp messungen mit einzeichnen}}
\label{fig:parameterModelSim}
\end{figure}
% 
The absorption coefficient $\absorp{}$ and the birefringence $\dn{}$ have to be estimated from the tissue to use appropriate values in the simulation.
As the literature shows, these values are in the range of $\dn{}$ \dummy{}.
Since the models created here use stiff fiber segments the volume density cannot reach the high density usually present in \ac{WM}.
To overcome this problem, the absorption and birefringence strengths must be increased to obtain comparable results.
\par
%
For a measurement of the properties it is important to ensure that a homogeneous region is chosen.
In coronal section, the corpus callosum is suitable for this purpose. It is the main fiber connection between the two cerebral hemispheres (see \cref{fig:parameterModelSim}). In a central section, the nerve fibers generally lie flat in section
% 
\ref{fig:brain_ret_trans} shows transmission and retardation maps of the vervet monkey coronal section, and \ref{fig:brain_ret_trans_zoom} shows a zoomed version.
The retardation map shows a visible reduction of retardation in the center of the corpus callosum.
Therefore, because of the need for homogeneity, this central region is neglected.
A left and a right region remain, which may be homoginous in both transmission and retardation (see histograms).
The left region has $\SI{1125858}{}$ number of pixels, the right part $\SI{1064629}{}$
% 
The same measurements are made for the human and rodent in the corpus callosum (see Appendix). \todo{Appendix}
% 
% 
% 
\subsubsection{absorption}
% 
When the light passes through the tissue, it is absorbed in the matter on the one hand and scattered on the other.
Both effects can be modeled mathematically in this linear optical simulation as a decrease in intensity along the path.
A drawback of this simulation is that the scattering is not constant for all fiber configurations and orientations \dummy[see MM]{}.
Therefore, the measured absorption coefficient is only valid for the same fiber configurations as in the region.
Absorption only changes the transmittance and hence the relative noise, not the delay or direction of the signal. 
Since the noise is quite low in this microscopic configuration, it should have little effect on the results. 
\par
% 
To measure absorption, the transmittance of the coronal slice in the \ac{ROI} of the corpus callosum is measured against the background value.
The absorption coefficient is then calculated by applying \dummy{}.
This values are then increased according to the volume fraction, of the models and then simulated to ensure that the same behavior can be modelled.
% 
\paragraph{Results}
A relative transmittance of about $\SIrange{20}{30}{\percent}$ is present within the \ac{ROI} of the corpus callosum. 
The simulation parameter must be set to achieve a similar transmission value for the models.
This means that for a $\SI{60}{\micro\meter}$ thick section, the absorption coefficient $\absorp{}$ is about $\SI{30}{\milli\meter\tothe{-1}}$. 
The absorption for the roden is about $\SI{8}{\milli\meter\tothe{-1}}$, and that of the human is about $\SI{65}{\milli\meter\tothe{-1}}$.
\\
%
\Cref{fig:simTransValues} shows the simulated transmittance values for different fiber radii. 
For smaler radii the variance of the transmittance is very small \dummy{} and increases with the radii.
This behavier is epected, since models with larger radii have more space where no tissue is present, and therefore the light can travel freely through the \say{section}.
In comparison to the histogram \cref{fig:brain_ret_trans_zoom} the models with larger radii have more in common with the distribution.
However from literature one can neglect, that this kind of fiber radii is present in corpus calosum \dummy{}.
Nevertheless the tissue can have more complicated configurations and effects, which leads to a larger viarity of absorption, \eg{} cells, blood vessels and so on.
Also at this point the here developed simulation is not ready to simulate scattered light, which is to be expected to increase the range of the absorption and therefore resulting a a larger variance of transmittance.
% 
% 
%
\subsubsection{birefringence}
% 
The birefringence cannot simply be measured like the transmittance.
Due to the radial optical axis of the myelin, the relativ obsorving birefringence is smaller than the one of the myelin.
However the simulation up to a point has the same geometrical constrains.
By simulating difference birefringence strength and comparing them to the retardation of the tissue one can find a proper value.
The simulation will use a flat single fiber population and the same parameters as the later simulation model library (see \dummy{}).
% 
\paragraph{Results}
The results in \cref{fig:parameterModelSim} show that to reach a retardation value similar to the experimental setup of around $\SI{0.8}{}$ a birefringence value of $\SI{0.008}{}$ for the microscopic model should be chosen.
Looking at the different model nerve fiber radii the variance of the retardation values increase strong, similar to the transmittance.
This is to be expected since the simulated tissue section has a hight of $\SI{60}{\micro\meter}$.
Therefore for a radii of $\SI{10}{\micro\meter}$ the number of tissue voxels has to vary depending on where the fibers are positioned in the volume.
For larger radii this leads to the problem, that the retardation cannot be over 1 and therefore the signal amplitude reduces again.
% 
% 
% \subsection{noise} <--- why this section?
% % 
% From \cref{sec:sensorGain,sec:tissueProp} the noise level can be calculated.
% However, when considering the different species, it is important to note that transmittance levels vary.
% Rodents typically have the highest transmittance, monkeys have a lower transmittance, and humans have the lowest transmittance.
% The reason for the rather low transmission of the human sections is probably due to the long post-mortem period.
% \Cref{fig:noiseplot} shows a log-log plot for this purpose.
% To run the simulation, a relative error of \dummy{} is used for a midrange analysis.
% 
% 
% 
\subsection{voxel size \texorpdfstring{\voxels{}}{}}
% 
% 
\begin{figure}[!t]%p
% 2_simulation/0_parameter/fiber_radii.py
\centering
\includegraphics[width=\textwidth, page=1]{dev/rc1/voxel_size/voxel_size_plots_data_r05_output_vs_135_0.01_6_25_vervet_r_rc1.pdf}
\caption[voxel size model with noise]{The mean difference is constant for smaller voxel sizes and does not start to grow significantly before $\voxels=\SI{0.1}{\micro\meter}$.}
\label{fig:voxelsizeNoise}
\end{figure}
% 
The \voxelsize{} paramter $\voxels$ is the most important parameter for simulation accuracy, as it determines how accurately the models are discretized.
Smaller values mean more details, more accurate optical axes and more light rays.
However, this also increases the number of calculations and memory by $O(n_{\mathit{voxel}}^3)$.
Therefore, it is recommended to choose the voxel size as large as possible without introducing a significant error due to the discretization.
\par
% 
To investigate this effect, a simulation is performed with different types of voxel sizes from the $\SIrange{0.01}{1.3}{\micro\meter}$.
The smallest voxel size $\SI{0.01}{\micro\meter}$ is used as ground truth. 
Since this voxel size is so small, only a volume of $3 \cdot \SI{1.3}{\micro\meter} \cdot \SI{1.3}{\micro\meter} 3 \cdot \SI{1.3}{\micro\meter} \times \SI{60}{\micro\meter}$ is used without tilting.
Otherwise, if tilting is used, the volume would have to be increased so that the light beam still travels through the tissue.
The other parameters are set as in \dummy{}.
The models to be simulated are $(||,\modelInc = \SI{0}{\degree})$, $(||,\modelInc = \SI{90}{\degree})$, $(\times,\modelInc = \SI{0}{\degree})$, and $(\times,\modelInc = \SI{90}{\degree})$ with a anteil of $\modelPsi = \SI{0.5}{}$, since these configurations represent the extrema for two fiber populations.
To improve the statistics, the simulation is repeated in the volume on an aquidistant $xy$ grid of $\num{5}$ by $\num{5}$ points within the $\SI{65}{\micro\meter}$ by $\SI{65}{\micro\meter}$ model.
% 
\paragraph{Results}
The results in \cref{fig:voxelsizeNoise} show that the relative difference to the smallest \voxelsize{} increases statically significantly above a value of \SI{0.1}{\micro\meter}.
The variance of the relative difference increases with increasing \voxelsize{} from this value.
% 
The results shown in \cref{fig:voxelsizeNoise} suggest that with noise present and a voxel size smaller than $\SI{0.1}{\micro\meter}$, accuracy does not increase with respect to the noise model chosen here.
However, this is only true for a pixel size of $\SI{1.3}{\micro\meter}$ and fiber radii of $\SI{0.5}{\micro\meter}$.
%  
% 
% 
\section{Simulation}
% 
\subsection{Parameter and environment}
\label{sec:simParameterEnv}
%  
\begin{table}[!b]
\caption{Simulation parameters}
\centering
% \sisetup{open-bracket={\{}, close-bracket={\}}, list-final-separator={,},list-pair-separator={,}}%
\pgfplotstabletypeset[%
    thesisTableStyle,
    column type=l,
    columns/variable/.style={string type},
    columns/value/.style={string type},
    every head row/.style={before row=\toprule,after row=\midrule},
    every last row/.style={after row=\bottomrule},
    col sep=&,
    row sep=\\,
]
{variable & value\\
% 
simpli.voxel\_size & $\SI{0.1}{\micro\meter}$\\
simpli.pixel\_size & $\SI{1.3}{\micro\meter}$\\
simpli.voi & $[[\SI{-35}{\micro\meter}, \SI{-35}{\micro\meter}, \SI{-30}{\micro\meter}], [\SI{30}{\micro\meter}, \SI{35}{\micro\meter}, \SI{35}{\micro\meter}]]$\\
simpli.filter\_rotations & $\SIlist{0;20;40;60;80;100;120;140;160}{\degree}$\\
simpli.interpolate & \texttt{"Slerp"}\\
simpli.wavelength & $\SI{525}{\nano\meter}$\\
simpli.optical\_sigma & $\SI{0.75}{\pixel}$\\
tilt angle & $\SI{3.9}{\degree}$\\
simpli.light\_intensity & $\SI{8000}{}$\\
gain & \SI{0.1175}{}\\
simpli.noise\_model & \code{lambda x: np.floor(np.random.normal(x,}\\
 & \ \ \ \ \ \code{np.sqrt(gain * x))+0.5).astype(np.uint16)}\\
fiber absorption & Roden: $\SI{14}{\milli\meter\tothe{-1}}$, Vervet: $\SI{30}{\milli\meter\tothe{-1}}$, Human: $\SI{60}{\milli\meter\tothe{-1}}$\\
fiber model & 'r'\\
fiber birefringence & 0.008\\
fiber radii & axon: \SI{0.75}{}, myelin: \SI{1}{}\\
model delta inclinations & single: $\SI{5}{\degree}$, crossings: $\SI{30}{\degree}$\\
model delta rotation & $\SI{15}{\degree}$\\
}
\label{tab:simParameters}
\end{table}
% 
% 
%
As described in \cref{software}, \fastpli{} contains a pipeline implementation of the simulation with automatic analysis.
This implementation is used to compute the discretized volumes, simulate the signal, and apply the optical noise and tilt analysis to the resulting signal using the algorithm \rofl{}. \footnote{source code is available in the appendix \dummy{}}.
This pipeline is used to simulate and analyze the \ac{3D-PLI} signal for the model library with different orientations of two crossing fiber populations.
\Cref{tab:simParameters} lists the parameters of the simulation in the variable notation \fastpli{}.
With a pixel size of \SI{1.3}{\micro\meter}, $\SI{2500}{\pixel}$ are available per simulation for statistical analysis.
The models are tilted in steps of $\Delta \modelInc = \SI{5}{\degree}$ for a single fiber population and by $\Delta \modelInc = \SI{30}{\degree}$ for intersecting fiber populations. The latter are then also rotated around the first fiber population (see \cref{fig:twomodelpopdesign}).
Depending on the crossing angle $\modelOmega$, the step size is adjusted so that the spatial distance between two resulting rotations is close to $\Delta \modelRot = \SI{15}{\degree}$.
This ensures that the sphere is sampled reasonably equidistantly with consideration of the computing time.
% The simulation were performed with a single compute node (\texttt{2x Intel(R) Xeon(R) CPU E5-4657L v2}).
% 
% 
% 
\subsection{Single inclined fiber population}
\label{sec:resSingleIncl}
% 
% \begin{figure}[!tp]
% \centering
% \newlength{\width}
% \setlength{\width}{0.42\textwidth}
% \subcaptionbox{\label{fig:single_fiber_pop_hist}Histogram}[\textwidth]{
% \begin{tabular}{c|c}
%     \includegraphics[width=\width]{dev/rc1/single/cube_2pop_135_rc1_single_plots_single_pop_hist_0.0.pdf}&
%     \includegraphics[width=\width]{dev/rc1/single/cube_2pop_135_rc1_single_plots_single_pop_hist_30.0.pdf}\\ 
%     \includegraphics[width=\width]{dev/rc1/single/cube_2pop_135_rc1_single_plots_single_pop_hist_60.0.pdf}&
%     \includegraphics[width=\width]{dev/rc1/single/cube_2pop_135_rc1_single_plots_single_pop_hist_90.0.pdf}
% \end{tabular}}
% % 
% \subcaptionbox{\label{fig:single_fiber_pop_rofl}Analysis}[\textwidth]{
% \includegraphics[]{dev/rc1/analysis/plots_single_pop_output_cube_2pop_135_rc1_single.pdf}}
% % 
% \caption[sim]{left: 2d log histogramm orientation from rofl analysis of simulation, right: 2d log histogramm of orientation of model segemnts. \itodo{rfc plots}}
% % \label{fig:single_fiber_pop_hist}
% \end{figure}
% 
\begin{figure}[!t]
\centering
\newlength{\width}
\setlength{\width}{0.45\textwidth}
\begin{tabular}{c|c}
    \includegraphics[width=\width]{dev/rc1/single/cube_2pop_135_rc1_single_plots_single_pop_hist_0.0.pdf}&
    \includegraphics[width=\width]{dev/rc1/single/cube_2pop_135_rc1_single_plots_single_pop_hist_30.0.pdf}\\ 
    \includegraphics[width=\width]{dev/rc1/single/cube_2pop_135_rc1_single_plots_single_pop_hist_60.0.pdf}&
    \includegraphics[width=\width]{dev/rc1/single/cube_2pop_135_rc1_single_plots_single_pop_hist_90.0.pdf}
\end{tabular}
% 
\caption[sim]{Single: left: 2d log histogramm orientation from rofl analysis of simulation, right: 2d log histogramm of orientation of model segemnts. \itodo{rfc plots}}
\label{fig:single_fiber_pop_hist}
\end{figure}
% 
\begin{figure}[!p]
\centering
\includegraphics[]{dev/rc1/analysis/plots_single_pop_output_cube_2pop_135_rc1_single.pdf}
\caption[]{single population rofl analysis}
\label{fig:single_fiber_pop_rofl}
\end{figure}
% 
First, a single population of fibers is analyzed.
Since the direction is negligible for a single population, only the inclination parameter \modelInc{} remains.
\par
% 
\Cref{fig:single_fiber_pop_hist, fig:single_fiber_pop_hist_app} shows the orientation distribution of a single fiber population for different inclinations $\modelInc{}$.
The distributions of the orientations of the model segments are shown on the left side, and those of the inclination analysis of the simulated images are shown on the right side.
The distributions are logarithmic and weighted by the area of the 2d binning.
The comparison between both distributions shows an agreement of the inclination analysis with the model orientations for all inclinations.
\par
% 
% 
% 
% \begin{figure}[!p]
% \centering
% \includegraphics[]{dev/rc1/analysis/plots_single_pop_output_cube_2pop_135_rc1_single.pdf}
% \caption[]{single population rofl analysis}
% \label{fig:single_fiber_pop_rofl}
% \end{figure}
% 
%Transmittance
\Cref{fig:single_fiber_pop_rofl} show the different modalities after analyzing the raw non-tilted measurements.
The transmittance increases slightly with tilt angle for inclinations $\modelInc<\SI{75}{\degree}$.
For $\modelInc \ge \SI{75}{\deg}$ the transmittance increases significantly, \ie{} the absorption decreases.
The variance of transmittance is within a few percent for all inclinations and also increases slightly with inclination angle.
\\
% 
% Retardation
The retardation plot shows the retardation of the non-tilted simulations and the theoretical curve that follows $(\cos(\modelInc) + 1) / 2 \cdot \mean(\mathit{ret}(\SI{0}{\degree})$.
The retardation follows the theoretical line.
However, in the middle slope angles, the measured retardation is higher than the theoretical line.
Otherwise, the variance along all fiber inclinations is equal to \modelInc{}.
\\
% 
% Direction
The next graph shows the measured direction.
For inclination angles smaller than $\SI{75}{\degree}$ the variance remains quite small.
For larger values, the variance increases.
It should be remembered that the boxplot does not take into account that the values are periodic.
Therefore, especially for $\modelInc=\SI{90}{\degree}$, the values from the $\SIrange{-90}{90}{\degree}$ are uniformly distributed.
\\
% 
% Inclination
The inclination is drawn as a line, as well as the set angle of inclination.
The angles are shifted by the $\modelInc$ so that the permitted distribution values are in the range $[\modelInc-\SI{90}{\degree}, \modelInc+\SI{90}{\degree})$.
The median follows the theoretical curve.
The $25-75$ values are \dummy{} \todo{accuracy analysis}.
Again, for values greater than $\SI{75}{\degree}$, the variance increases significantly, except for $\modelInc = \SI{90}{\degree}$ it is similar to the first values.\todo{Why?}
\\
% 
% trel
The median relative birefringence thickness \trel{} is stable for values up to $\modelInc < \SI{60}{\degree}$, but the variance increases slightly.
For larger values, the median decreases while the variance increases.
Some outliers reach values $>\SI{1}{}$.
For inclination angles $\modelInc \ge \SI{85}{\degree}$, the quartile value $\SI{75}{\percent}$ increases by $\SI{1}{}$ and the median for $\SI{90}{\degree}$ reaches about $\SI{2}{}$.
\\
% 
% domega
The opening angle \modelDOmega{} is stable with a median of \dummy{} up to a tilt angle of $\le \SI{70}{\degree}$.
For higher inclination angles, both the median and the variance increase significantly in analogy to the inclination, except in the last case.
% 
% 
% 
\subsection{Crossing flat fiber populations}
\label{sec:resCrossFlat}
% 
\begin{figure}[!t]
\centering
\setlength{\width}{0.45\textwidth}
\begin{tabular}{c|c}
    \includegraphics[width=\width]{dev/rc1/flat/cube_2pop_135_rc1_flat_plots_flat_pop_hist_omega_0.0_psi_0.3.pdf} &
    \includegraphics[width=\width]{dev/rc1/flat/cube_2pop_135_rc1_flat_plots_flat_pop_hist_omega_30.0_psi_0.3.pdf} \\ \includegraphics[width=\width]{dev/rc1/flat/cube_2pop_135_rc1_flat_plots_flat_pop_hist_omega_60.0_psi_0.3.pdf} & \includegraphics[width=\width]{dev/rc1/flat/cube_2pop_135_rc1_flat_plots_flat_pop_hist_omega_90.0_psi_0.3.pdf}
\end{tabular}
\caption[sim]{Flat: psi=0.3; left: 2d log histogramm orientation from rofl analysis of simulation, right: 2d log histogram of orientation of model segemnts. \itodo{legende}}
\label{fig:flat_03_fiber_pop_hist}
\end{figure}
% 
\begin{figure}[!p]
\centering
\includegraphics[page=3]{dev/rc1/analysis/plots_flat_pop_output_cube_2pop_135_rc1_flat.pdf}
\caption[]{flat population psi=0.3 rofl analysis}
\label{fig:flat_03_fiber_pop_rofl}
\end{figure}
% 
\begin{figure}[!t]
\centering
\setlength{\width}{0.4\textwidth}
\begin{tabular}{c|c}
    \includegraphics[width=\width]{dev/rc1/flat/cube_2pop_135_rc1_flat_plots_flat_pop_hist_omega_0.0_psi_0.5.pdf} &
    \includegraphics[width=\width]{dev/rc1/flat/cube_2pop_135_rc1_flat_plots_flat_pop_hist_omega_30.0_psi_0.5.pdf} \\
    \includegraphics[width=\width]{dev/rc1/flat/cube_2pop_135_rc1_flat_plots_flat_pop_hist_omega_60.0_psi_0.5.pdf} &
    \includegraphics[width=\width]{dev/rc1/flat/cube_2pop_135_rc1_flat_plots_flat_pop_hist_omega_90.0_psi_0.5.pdf}
\end{tabular}
\caption[sim]{Flat: psi=0.5; left: 2d log histogramm orientation from rofl analysis of simulation, right: 2d log histogramm of orientation of model segemnts. \itodo{legende}}
\label{fig:flat_05_fiber_pop_hist}
\end{figure}
% 
\begin{figure}[!p]
\centering
\includegraphics[page=5]{dev/rc1/analysis/plots_flat_pop_output_cube_2pop_135_rc1_flat.pdf}
\caption[]{flat population psi=0.5}
\label{fig:flat_05_fiber_pop_rofl}
\end{figure}
% 
The next focus is on two flat crossing fiber populations.
Here the results for the case $\modelPsi = \SI{30}{\percent}$ and $\modelPsi = \SI{50}{\percent}$ are examine.
The other fiber population proportion are available in \dummy{}. %\cref{}
\\
% 
\Cref{fig:flat_03_fiber_pop_hist} show the distribution of the orientation for the simulation and the model fiber segments.
The resulting fiber orientation seems to follow only the second fiber orientation, which has a density of $\SI{70}{\percent}$.
\\
% Transmittance
In \cref{fig:flat_03_fiber_pop_rofl} the transmittance increases for small crossing angles $\modelOmega$ with increasing $\modelOmega$ and then the \dummy{} flattens.
The variance does not change with \dummy[value]{}.
\\
% Retardation
The retardation seems to follow a negative linear \dummy{} with increasing crossing angle.
The variance increases with the crossing angle.
\\
% Direction
The direction is plotted in \dummy{} with the direction of the fiber populations values.
The data follows the second fiber bundle, however is significantly lower in the middle section, even if slightly with a delta of \dummy[value]{}.
\\
% Inclination
The inclination is with its median for all crossing angles at about $\SI{0}{\degree}$.
The variance significantly increases with increasing crossing angle up to \dummy[3]{} times the lowest value.
\\
% Trel
The relative thickness is decreases with increasing crossing angle.
The curve is identical with the retardation value.
\\
% domega
The opening angle is low for the first half of the crossing angles, but slightly increasing.
The variance for this range also increases slightly.
There exist several outliers with values $> \SI{5}{\degree}$.
From $\modelOmega \ge \SI{50}{\degree}$ the value increases with a stronger slope.
The median rises up to about a value of $\SI{5}{\degree}$.
% 
% 
\subsection{Inclined crossing fibers population}
\label{sec:resInclCross}
% 
\begin{figure}[!t]
\centering
\setlength{\width}{0.45\textwidth}
\begin{tabular}{c|c}
    \includegraphics[width=\width]{dev/rc1/inclined/cube_2pop_135_rc1_inclined_plots_inclined_pop_hist_omega_0.0_psi_0.3.pdf} &
    \includegraphics[width=\width]{dev/rc1/inclined/cube_2pop_135_rc1_inclined_plots_inclined_pop_hist_omega_30.0_psi_0.3.pdf} \\ \includegraphics[width=\width]{dev/rc1/inclined/cube_2pop_135_rc1_inclined_plots_inclined_pop_hist_omega_60.0_psi_0.3.pdf} & \includegraphics[width=\width]{dev/rc1/inclined/cube_2pop_135_rc1_inclined_plots_inclined_pop_hist_omega_90.0_psi_0.3.pdf}
\end{tabular}
% 
\caption[sim]{Inclined: psi=0.3; left: 2d log histogramm orientation from rofl analysis of simulation, right: 2d log histogram of orientation of model segments. \itodo{legende}}
\label{fig:inclined_03_fiber_pop_hist}
\end{figure}
% 
\begin{figure}[!p]
\centering
\includegraphics[page=3]{dev/rc1/analysis/plots_inclined_pop_output_cube_2pop_135_rc1_inclined.pdf}
\caption[]{Inclined population psi=0.3 rofl analysis}
\label{fig:inclined_03_fiber_pop_rofl}
\end{figure}
% 
\begin{figure}[!t]
\centering
\setlength{\width}{0.4\textwidth}
\begin{tabular}{c|c}
    \includegraphics[width=\width]{dev/rc1/inclined/cube_2pop_135_rc1_inclined_plots_inclined_pop_hist_omega_0.0_psi_0.5.pdf} &
    \includegraphics[width=\width]{dev/rc1/inclined/cube_2pop_135_rc1_inclined_plots_inclined_pop_hist_omega_30.0_psi_0.5.pdf} \\
    \includegraphics[width=\width]{dev/rc1/inclined/cube_2pop_135_rc1_inclined_plots_inclined_pop_hist_omega_60.0_psi_0.5.pdf} &
    \includegraphics[width=\width]{dev/rc1/inclined/cube_2pop_135_rc1_inclined_plots_inclined_pop_hist_omega_90.0_psi_0.5.pdf}
\end{tabular}
\caption[sim]{Inclined: psi=0.5; left: 2d log histogramm orientation from rofl analysis of simulation, right: 2d log histogramm of orientation of model segemnts. \itodo{legende}}
\label{fig:inclined_05_fiber_pop_hist}
\end{figure}
% 
\begin{figure}[!p]
\centering
\includegraphics[page=5]{dev/rc1/analysis/plots_inclined_pop_output_cube_2pop_135_rc1_inclined.pdf}
\caption[]{Inclined population psi=0.5}
\label{fig:inclined_05_fiber_pop_rofl}
\end{figure}
% 
% 
The last focused few is on two crossing fiber bundles with a rotating angle of $\modelRot = \SI{90}{\degree}$, \ie{} the first fiber population along the $x$-axis and the second population in the $xz$-axis.
As before the two parameters $\modelPsi = \SI{30}{\percent}$ and $\modelPsi = \SI{50}{\percent}$  will be discussed.
The other fiber proportions are listed in the \dummy[appendix]{}.
\\
% 
The histogram data for both cases are show for the measuremnt only one visible orientation.
\dummy{}.
\\
% Transmittance
The transmittance is for both cases almost identical.
It grows with increasing crossing angle, but the slope decreases for larger $\modelOmega$.
The median in both cases grows to a value of $\SI{110}{\percent}$ in contrast to the flat case.
The variance stays similar for all values.
\\
% Retardation
The retardation in both cases decreases almost linear with increasing crossing angle $\modelOmega$.
In the case of $\modelPsi = \SI{30}{\percent}$ the final value is about $\SI{0.18}{}$ in contrast to $\modelPsi = \SI{50}{\percent}$ with a final value of $\SI{0.3}{}$.
The varce increases for larger crossing angles significantly.
\\
% Direction
The direction value is for both cases again very similar.
The median always is around \SI{0}{\degree}.
The quartle grows from about $\SI{0.5}{\degree}$ to $\SI{2}{\degree}$ with increasing crossing angle.
\\
% Inclination
In the case of $\modelPsi = \SI{30}{\percent}$ the inclination is ab to about $\modelOmega =\SI{60}{\degree}$ almost linear with increasing crossing angle.
In the case of $\modelPsi = \SI{50}{\percent}$ the tip point is about $\modelOmega = \SI{60}{\degree}$
The variance is increasing for both parameters with increasing crossing angle.
The main difference of both parameters is, that in the case of $\modelPsi = \SI{30}{\percent}$ the maximum reached inclination is significant higher than in the case of $\modelPsi = \SI{50}{\percent}$.
However in both cases the inlination is always lower then the inclination value of the second fiber population.
\\
% Trel
The relative effective thickness \trel{} decreases in both cases.
In the case of $\modelPsi = \SI{50}{\percent}$ the correlation is almost linear and the variance increases slightly with increasing crossing angle.
In the case of $\modelPsi = \SI{30}{\percent}$ a small hill is visible around $\modelOmega = \SI{50}{\degree}$ where the value is slighly increases as well as the variance.
\\
% domega
The opening angle as well as its variance seems to exponentially increase with increasing crossing angle in the case of $\modelPsi = \SI{30}{\percent}$.
In the case of $\modelPsi = \SI{50}{\percent}$ the median as well as the variance also increases, however for the largest crossing values the increase stopped at a median of about $\SI{6}{\degree}$.
% 
\subsection{Free crossing fiber populations}
\label{sec:resFreeCross}
% 
The following data shows the result of the tilting analysis without the orientation angles, since the mean value cant easyli be calculated for a single angle. \footref{orientations lie on a manifold of a hemispherical with symmetries as well as the angles are depending on each other.}
The data of the previous chapter had the advantage, that the results of the angles are close enough to apply a remapping around the expected value.
\\
% 
The spherical plots in \cref{fig:sim_ana_acc,fig:sim_ana_ret,fig:sim_ana_trans,fig:sim_ana_trel} are designed that the thick black circle shows the orientation of the first fiber population, and the thin dashed circles the orientation of the second fiber population. 
At the position of the second fiber population the resulting mean value is inserted.
% 
% 
% 
\begin{figure}[!p]
\centering
\includegraphics[]{dev/rc1/analysis/simulation_analysis_hist_0.5_setup_PM_s_Vervet_m_r_acc.pdf}
\caption[Simulation acc]{Mean \acc{} between model and tilt analysis orientations.}
\label{fig:sim_ana_acc}
\end{figure}
% 
\paragraph{\acc{}}
\Cref{fig:sim_ana_acc} shows the \acc{} value, \ie{} how good the coefficient of the odf base function are in alignment.
\dummy[acc of psi = 1]{}.
\\
% 
The $\modelPsi=\SI{10}{\percent}$ show for all $\modelInc$ no significant reduction of the \acc{} value.
\\
% 
The for $\modelPsi=\SI{30}{\percent}$ the most secondary inclined values have a significant reduction of the \acc{} value.
This sightly less visible for $\modelInc = \SI{30}{\degree}$ case and even less for the $\modelInc = \SI{60}{\degree}$.
The $\modelInc = \SI{90}{\degree}$ is less than in the $\modelPsi=\SI{10}{\percent}$ case, but still constant for all secondary orientations.
\\
% 
For the equal proportional fiber populations $\modelInc = \SI{50}{\degree}$ the \acc{} value reaches its lowest values for flat inclined first population $\modelInc = \SIlist{0;30}{\degree}$.
In the case of $\modelInc = \SI{0}{\degree}$ the \acc{} value is lowest with a crossing angle of $\modelOmega = \SI{90}{\degree}$.
This crossing angle for the slightly inclined has a higher \acc{} value \dummy[why?, -> show odf]{}.
Otherwise the distribution is similar with respect of the 3d orientation symmetry.
The most first inclined angle $\modelInc=\SI{90}{\degree}$ shows a \acc{} value of about $\SI{0.5}{}$ for all orientations with an increase to $\SI{1}{}$ for the $\modelOmega = \SI{0}{\degree}$ ,\ie{} a single fiber population along the z-axis.
\\
% 
The $\modelPsi=\SI{70}{\percent}$ data shows for the first two inclination angles $\modelInc = \SIlist{0;30}{\degree}$ the same behavier with respect to a inclined rotation.
For the $\modelInc=\SI{60}{\degree}$ case the \acc{} value is only high around the first fiber population.
In the case of $\modelInc=\SI{90}{\degree}$ this is also visible, however the smallest value are not as low as in the previous case.
\\
% 
The last case $\modelPsi=\SI{90}{\percent}$ show for the first three inclination angles $\modelInc=\SIlist{0;30;60}{\percent}$ a good agreement of the models orientation coefficients with the signal analyesed ones.
The last inclination angle $\modelInc=\SI{90}{\degree}$ show a reduction of the \acc{} value for the highest crossing angles $\modelOmega$.
% 
% 
% 
\begin{figure}[!p]
\centering
\includegraphics[]{dev/rc1/analysis/simulation_analysis_hist_0.5_setup_PM_s_Vervet_m_r_ret_mean.pdf}
\caption[Simulation retardation]{Mean retardation.}
\label{fig:sim_ana_ret}
\end{figure}
% 
\paragraph{Retardation}
The results of the retardation in \cref{fig:sim_ret} show in the case of the smallest first fiber population proportion $\modelPsi=\SI{10}{\percent}$ the smallest retardation for an inclination angle of $\alpha=\SI{90}{\degree}$ for the second fiber population and an increase towards the flat case of $\alpha=\SI{0}{\degree}$.
\\
% 
This is also visible for the second case of $\modelPsi=\SI{30}{\percent}$ and $\modelPsi=\SI{60}{\percent}$, however the non crossing fiber populations have a significant higher retardation for the first two $\modelInc = \SIlist{0;30}{\degree}$.
For higher $\modelInc$ the values decrease overall and a crossing reduction is not visible anymore.
\\
In the case of $\modelPsi=\SI{60}{\percent}$ the retardation is very similar to the previous $\modelPsi=\SI{30}{\percent}$ case, however the value is significantly reduces.
% 
\\
The $\modelPsi=\SI{70}{\percent}$ case shoe for the first two inclination angles $\modelInc=\SIlist{0;30}{\degree}$ a rise in the retardation again in contrast to the previous values.
A reduction of the retardation along the highest crossing angles is still visible, however relatively reduced in comparison to the previous values.
The last two inclination angles $\modelInc=\SIlist{60;90}{\degree}$ are significantly lower.
\\
%
The last case of $\modelPsi=\SI{90}{\percent}$ a reduction of the retardation with increasing inclination angle $\modelInc$.
The second fiber population orientation has almost no effect on the results.
% 
% 
% 
\begin{figure}[!p]
\centering
\includegraphics[]{dev/rc1/analysis/simulation_analysis_hist_0.5_setup_PM_s_Vervet_m_r_trans_mean.pdf}
\caption[Simulation transmittance]{Mean transmittance value.}
\label{fig:sim_ana_trans}
\end{figure}
% 
\paragraph{Transmittance}
The mean transmittance value is shown in \cref{fig:sim_ana_trans}.
The first fiber population proportion of $\modelPsi=\SI{10}{\percent}$ is almost identical for all inclination angles $\modelInc$.
The main difference is that the transmittance is lowest for models with an crossing angle of $\modelOmega = \SI{0}{\degree}$.
Otherweise the transmittance value stays stable relatively low.
\\
% 
This changes a bit for the next fiber population proportion of $\modelPsi=\SI{30}{\percent}$.
Here the transmittance also is lowest for no crossing angle, however the raise in the value is more significant than in the case before.
\\
The next case of $\modelPsi=\SI{50}{\percent}$ the effect is even stronger.
Still the transmitance is lowest at no crossing angle.
\\
The last two cases $\modelPsi=\SI{70}{\percent}$ and $\modelPsi=\SI{90}{\percent}$ are the identical with the cases $\modelPsi=\SI{30}{\percent}$ and $\modelPsi=\SI{10}{\percent}$.
% 
% 
% 
\begin{figure}[!p]
\centering
\includegraphics[]{dev/rc1/analysis/simulation_analysis_hist_0.5_setup_PM_s_Vervet_m_r_rtrel_mean.pdf} 
\caption[Simulation \trel{}]{Mean \trel{} from tilt analysis.}
\label{fig:sim_ana_trel}
\end{figure}
% 
\paragraph{\trel}
The \trel{} results are shown in \cref{fig:sim_ana_trel}.
All the values correlate almost perfectly with the retardataion.
The only significant difference is, where the \trel{} value exceeds obviously the value 1.
One has to bare in mind, that the upper limit if the colorbar is reduced to 1, so that the distribution is more visible.
In the cases of a $\trel > 1$, the values of the retardation are the lowest.
Here the interpolation of the spheres background values should not be taken too seriously.
% 
% 
% 
\begin{figure}[!p]
\centering
\includegraphics[]{dev/rc1/analysis/simulation_analysis_hist_0.5_setup_PM_s_Vervet_m_r_R.pdf} 
\caption[Simulation \rvalue]{Mean \rvalue{} from tilt analysis.}
\label{fig:sim_ana_trel}
\end{figure}
% 
\paragraph{\rvalue}
The last value is the \rvalue{}.
It describes the mean absolute difference between the data and the fitted data from the inclination analysis.
Its values is overall in the range of $\SIrange{1.2}{1.8}{\percent}$.
For $\modelInc = \SI{0}{\degree}$ the \rvalue{} rises in the center of the polar plot for higher \modelPsi{} values.
The $\modelInc = \SI{30}{\degree}$ show towards the $\modelPsi = \SI{50}{\percent}$ a curved pattern with slightly higher \rvalue{}.
For higher $\modelPsi$ the pattern changes and only the centered region is increased.
For $\modelInc = \SI{60}{\degree}$ an increase of the \rvalue{} is visible towards the outer regions.
For increasing \modelInc{} the pattern changes as well and is up to $\modelPsi=\SI{70}{\percent}$ increased visible towards the upper and lower part of the polar plot.
Finally for the The $\modelInc = \SI{90}{\degree}$ case the \rvalue{} are low again.
\\
A significant difference has the $\modelPsi = \SI{10}{\percent} / \modelInc = \SI{90}{\degree}$.
There the values reach for the crossing angles towords $\modelOmega = \SI{90}{\degree}$ a higher value of $\approx \SI{2.2}{\percent}$.
The \rvalue{} for higher first fiber population proportions $\modelPsi$ at this inclination angle $\modelInc$ has a similar pattern, but gets lower and lower for higher $\modelPsi$.
% 
% 
% 
% 
\subsection{Speedup}
\label{sec:simSpeedup}
% \tikzset{external/force remake=true}
% \tikzset{external/export=false}
% 
% MPI
\begin{figure}[!t]
\centering
% 
\begin{lrbox}{\newtable}
\pgfplotstabletypeset[%
    thesisTableStyle,
    column type=l,
    fixed, zerofill, precision=2,
    dec sep align,
    font={\scriptsize},
    columns/cpus/.style={precision=0},
    % columns/value/.style={string type},
]
{cpus mean std
1	1.00006472390261	0.008436755542444
2	1.92269431841134	0.013489117814521
3	2.96577613197067	0.026032692317656
4	3.7772210840161	0.0683451109519
5	4.69310330928019	0.053307326948835
6	5.59568645873564	0.08209457262756
7	6.46527803633867	0.078755238845452
8	6.77988547696095	0.070937664466063
16	13.2706748124183	0.227421225827763
24	17.2977004181403	0.356822695290338
32	21.6900417343156	0.489742257920109
40	25.4634907689925	0.543172079794108
48	27.9281802113992	0.714407306509344
}
\end{lrbox}
% 
\begin{tabular}{cc}
\begin{minipage}{0.6\textwidth}
\includegraphics[page=2]{dev/rc1/speed/boxplot_generation_output_generation_mpi_v_0.1.csv.pdf}
\end{minipage}
&
\begin{minipage}{0.25\textwidth}
\usebox{\newtable}
\end{minipage}
\end{tabular}
\caption[]{MPI speedup discrete tissue generation Mean time 1 cpu: 29.9 +- 0.2}
\label{fig:speedTissueMPI}
\end{figure}
% 
\begin{figure}[!t]
\centering
\begin{lrbox}{\newtable}
\pgfplotstabletypeset[%
    thesisTableStyle,
    column type=l,
    fixed, zerofill, precision=2,
    dec sep align,
    font={\scriptsize},
    columns/cpus/.style={precision=0},
    % columns/value/.style={string type},
]
{cpus mean std
1	1.00028560723869	0.016748523109448
2	1.89554114187235	0.061890791019146
3	2.93217274657456	0.137044706938191
4	3.80796194148021	0.151338179581496
5	4.56797840123376	0.294743030055688
6	5.46464911530043	0.244860209863609
7	6.30306494349277	0.558808919534342
8	7.22964159473554	0.370613450811933
16	13.6057964309241	0.869296236931105
24	19.9354473052936	0.76053214292035
32	25.4688807363332	1.33524999091616
40	30.3887245939769	2.98878482975798
48	34.7093030892855	3.59570044944364
}
\end{lrbox}
% 
\begin{tabular}{cc}
\begin{minipage}{0.6\textwidth}
\includegraphics[page=2]{dev/rc1/speed/boxplot_simulation_output_simulation_mpi_v_0.1.csv.pdf}
\end{minipage}
&
\begin{minipage}{0.25\textwidth}
\usebox{\newtable}
\end{minipage}
\end{tabular}
\caption[]{MPI Speedup simulation for 5 tilt direction. Mean time 1 cpu, centered: 80 +- 2}
\label{fig:speedSimMPI}
\end{figure}
% 
% 
% OPENMP
\begin{figure}[!t]
\centering
% 
\begin{lrbox}{\newtable}
\pgfplotstabletypeset[%
    thesisTableStyle,
    column type=l,
    fixed, zerofill, precision=2,
    dec sep align,
    font={\scriptsize},
    columns/cpus/.style={precision=0},
    % columns/value/.style={string type},
]
{cpus mean std
1	1.00003318870801	0.006043762665602
2	1.34745571509197	0.112126679306381
3	1.85706837136119	0.166855662031889
4	2.07043347980059	0.203474087892324
5	2.41719826573049	0.208307641301249
6	2.67381650473413	0.144163042046179
7	2.81024536424621	0.413899849287807
8	3.05678368352906	0.238570289478912
16	3.98524630106596	0.175099042780921
24	4.64005306796557	0.118496100940545
32	4.90099710152266	0.112188961625372
40	4.89783195844876	0.528173540658736
48	5.22901104444621	0.053589607722899

}
\end{lrbox}
% 
\begin{tabular}{cc}
\begin{minipage}{0.6\textwidth}
\includegraphics[page=2]{dev/rc1/speed/boxplot_generation_output_generation_mp_v_0.1.csv.pdf}
\end{minipage}
&
\begin{minipage}{0.25\textwidth}
\usebox{\newtable}
\end{minipage}
\end{tabular}
\caption[]{OpenMP speedup discrete tissue generation. Mean time 1 cpu: 29.9 +- 0.2}
\label{fig:speedTissueMP}
\end{figure}
% 
\begin{figure}[!t]
\centering
% 
\begin{lrbox}{\newtable}
\pgfplotstabletypeset[%
    thesisTableStyle,
    column type=l,
    fixed, zerofill, precision=2,
    dec sep align,
    font={\scriptsize},
    columns/cpus/.style={precision=0},
    % columns/value/.style={string type},
]
{cpus mean std
1	1.00026069793854	0.015929466416549
2	1.95547220178774	0.065680726218777
3	2.97138856408045	0.080728713321266
4	3.90097732250136	0.154778304333818
5	4.89979040078024	0.193465065920687
6	5.81536649890996	0.125590793795792
7	6.71229347977858	0.143352506525363
8	7.57827799547624	0.369746017663724
16	14.6032840374081	0.31614387537932
24	21.6206614940094	1.91501760949786
32	28.8271369113821	1.12327525256459
40	35.6305514268905	1.51516739583375
48	42.1637558599282	1.30733435281682

}
\end{lrbox}
% 
\begin{tabular}{cc}
\begin{minipage}{0.6\textwidth}
\includegraphics[page=2]{dev/rc1/speed/boxplot_simulation_output_simulation_mp_v_0.1.csv.pdf}
\end{minipage}
&
\begin{minipage}{0.25\textwidth}
\usebox{\newtable}
\end{minipage}
\end{tabular}
\caption[]{OpenMP Speedup simulation for 5 tilt direction. Mean time 1 cpu, centered: 74.9 +- 0.2}
\label{fig:speedSimMP}
\end{figure}
% 
Four speedup measurements are shown in \cref{fig:speedTissueMPI,fig:speedSimMPI} for the \mpi{} parallelisation and \cref{fig:speedTissueMP,fig:speedSimMP} for the \openmp{} parallelisation.
To measure the speedup each algorithm was performed $N=10$ times.
To calculate the speedup value the mean measured time for $n_\mathit{cpu}=1$ was then divided by the measured time for the $n_\mathit{cpu}$ individual values.
The used volume is the $\modelPsi=\SI{0}{\percent} / \modelInc=\SI{0}{\degree}$ from the \cref{sec:simParameterEnv} parameterization.
\\
% 
The \cref{fig:speedTissueMPI} shows the parallelization of the discrete tissue generation (see \cref{sec:dv_generator}).
The speedup for $\SI{8}{\cpus}$ about $\SI{6.8}{}$ and for $\SI{48}{\cpus}$ about $\SI{30}{}$ for this architecture.
For the simulation \cref{fig:speedSimMPI} the behavier is similar with a speedup for $\SI{8}{\cpus}$ about $\SI{7.2}{}$ and for $\SI{48}{\cpus}$ about $\SI{35}{}$.
Looking at the plot the speedup values for the different tilting angles are shown.
The centered $(C)$ position has a significant higher speedup then the tilts.
\par
% 
The \openmp{} speedup is for the tissue generation significantly lower than the in \mpi{} case.
The speedup reaches a value of $\SI{3}{}$ for $\SI{8}{\cpus}$ and $\SI{5.23}{}$ for $\SI{48}{\cpus}$.
The speedup is almost linear increases for all tested number of cpus.
\\
% 
In the case for the simulation speedup the values similar to the \mpi{} case.
Up to $\SI{8}{\cpus}$ the speedup is almost identical and ideal.
However for larger number of CPUs the speedup is higher with $\SI{42}{}$ for $\SI{48}{\cpus}$
% 
% 
\section{Discussion}
% 
\subsection{Single inclined fiber population}
\Cref{sec:resSingleIncl} shown the results for the single fiber population case with inclined configurations.
The transmittance values show a rise for the last frew inclination values.
This is expected since the 3d model configurations are orientated parallel, with some randomness, along one axis.
When this axis is orientated along the z-axis, \ie{} $\modelInc=\SI{90}{\degree}$, the light rays, which also travels along the z-axis, will statistically hit less tissue and therefore the transmittance has to rise.
Since the density of the tissue is quite high, the effect is only about a few percent.
In reality however one has to bare in mind, that \cite{Menzel2021}\dummy[check ref]{} could show that the change in the transmittance is much more complicated.
One of the major effects is the significant change of the transmittance with the inclination.
The impact on the here presented results is an increase/decrease of the noise according to \cref{fig:parameterModelGain}.
Since the change of the noise is however linear with the square root of the intensity, the effect is rather small.
\\
% 
The retardation follows almost the theoretical curve for a single retardation signal (see \dummy[pli formel]{}).
The signal however is for the middle values higher than the theoretical expected values.
One has to keep in mind, that the theoretical signal is normed by the maximum value of the retardataion, \ie{} the value for $\modelInc = \SI{0}{\degree}$.
This value however is not correct in that sence, that the inclination angle of the underlying fibers is not perfectly $\SI{0}{\degree}$. The opening angle of the models is around \dummy{}.
Therefore the normalization is underestimated and therefore also the theoretical curve.
\dummy{}.
% 
The direction values have a very small variance, which is largely increasing for the high inclined values.
Since the direction value is coupled to the inclination angle, \ie{} there is no xy-direction for a $\SI{90}{\degree}$ inclined orientation, this is as expected.
\\
% 
The same behavior is visible in the inclination plot.
Here however the variance for the last angle is smaller.
One has to remember, that the change of the signal, which the tilting analysis is fitting, is for the non tilted measurement as for the four tilted measurements.
One possibility is therefore, that because here the fibers are at is most extreme position, the change of the signal is higher, than for a slightly less inclined case.
\\
% 
The \trel{} value follows also the theoretical behavior up to the last value.
For \trel{} smaller one the algorithm is capable of explaining the signal.
For larger \trel{} values however, the optimization fitting algorithm try's to find non physically solutions, which are mathematically better.
This behavior could be constrained by a maximum \trel{} value of one in the optimizer, however this way it is more clear, that one cannot trust the resulting orientation.
This is true for the experimental tissue measurements, since there are a lot more effects than in this simulation.
For this simulation however, the inclination results suggest, that the algorithm still could be possible to find a feasible \trel{} value, which would be very small.
Since \trel{} is the effective thickness of the parallel optical axis model, for inclined fiber configurations this value should theoretical be $\SI{0}{}$ for $\modelInc=\SI{90}{\degree}$.
\\
% 
The opening angle shows the combined information of the direction results with the inclination results as expected.
The inclination values $\modelInc <= \SI{75}{\degree}$ suggest a orientation distribution of \dummy{} with some outliers.
This value can maybe help as an additional uncertainty for further computations like in a tractography.
However this would be a lower limit of the uncertainty, since effects light scattering are not included in this simulation.
\\
% 
The case of a single fiber population suggest a good agreement of the resulting tilting analysis with the models individual orientations except for very steep inclined configurations.
% 
\subsection{Crossing flat fiber population}
\cref{sec:resCrossFlat} presents the results of the flat crossing configurations.
the transmittance value changes significantly with increasing \modelOmega{} which is as epected since crossing models need more space.
therefore in a close volume less fibers, \ie{} less tissue, can absorb light and therefore the transmittance value has to increase.
This also explains the slow down of the value for larger crossing angles, since here the layered structure already sop ports similar crossing angles.
Otherwise the fibers would need to traverse more complicated though the volume, which would be less effective.
% 
The retardation follows almost a linear decrease for the presented fiber portion populations \modelPsi{}.
Since the retardence effect of is getting canceled by a $\SI{90}{\degree}$ retarder, this effect is as expected.
The last value for the retardation is with the theoretical remaining $\approx0.8\cdot \SI{40}{\percent}$ in agreement.
Its values however are slightly, but significant, lower for the mid range \trel{} values.
In the case of the $\modelPsi{} = \SI{50}{\percent}$ they follow the midline between both directions up to the $\modelOmega=\SI{90}{\degree}$ where the retardation effectively $\SI{0}{}$ and therefore no direction can be identified.
\\
% 
The effect of a changed direction comes from \dummy{}. 
Since the optical axis is a orientation, and no vector, the signal gets canceled with a \SI{90}{\degree} case.
Lower values therefore have to drive the resulting phase of the sinusoidal signal into the direction of the lower represented fiber bundle, which effect has to be maximal with $\modelOmega=\SI{45}{\degree}$. 
\\
% 
The variance of the inclination angle increases with increasing \modelOmega{} which is to be expected, since the amplitude, \ie{} the retardation decreases and the model is therefore more uncertain.
In the case of an equal distributed fiber crossing the inclination can reach every value, since there is no effective retardation anymore.
\\
%
This is also visible in the decreasing \trel{} value.
The decrease is linear as expected \dummy[really?]{} since the signal gets \dummy{}.
For the equal distributed crossing the inclination analysis reaches again values of $\trel>1$.
\\
% 
The effects discussed above are reflected in the opening angle \modelDOmega{}, which has to increases with increasing \modelOmega{}.
% 
\subsection{Inclined crossing fiber population}
\Cref{sec:resInclCross} shows the results of the inclined crossing, \ie{} $\modelRot=\SI{90}{\degree}$.
The results are, with the exception of the inclination, very similar to the previous flat crossing fiber population case.
This is as expected, since inclined fibers do not contribute to a retardation.
The inclination in the case $\modelPsi=\SI{30}{\percent}$ rises first, since the main fiber population, \ie{} the second, contribution to the retardence is more dominant until it finally the median falls back to $\SI{0}{\degree}$.
In the case of $\modelPsi=\SI{50}{\percent}$ the tipping point is earlier and less due to the less contributing effects of the second nerve fiber bundle.
Otherwise the rest parameters are not behaving differently from the single fiber population case with respect to a reduced signal and its impact.
% 
\subsection{Free crossing fiber population}
\Cref{sec:resFreeCross} shown the results of the \acc{} and other parameters for the free crossing fiber population.
The \acc{} Parameter, which is a measure for the compliance of the \ac{ODF} coefficent, is a key parameter to highlight problematic fiber population orientations.
However the value itself is not straightforward to be interpret.
\\
From the \Cref{fig:sim_ana_acc} there is a area visible from low \modelPsi{} and low \modelInc{} to higher low \modelPsi{} and higher \modelInc{} where the \acc{} is significantly low, with larger areas close to $\SI{0}{}$.
This areas correspond to the already known problematic areas in crossing and inclined areas. 
This results however give a better overview, how strong he indication is with respect to he \acc{} value.
The \acc{} value can maybe be used as a weight for \dummy[not ...]{}.
A importance result is the correlation between the \acc{} value and the \trel{} value.
Since the \trel{} value is analysed it can therefor be used as an indicator for an underlying crossing or inclined region.
This would then also be a indicator for further data processing \eg{} an tractography.
One has to bare in mind, these results are only valid for dense fiber populations, \ie{} dense white matter.
% 
Th retardation is also a indicator, however not as reliable, since its value is also very low for a single inclined fiber population, which still can be correctly determined, with a high variance, with the tilting analyse (see \cref{fig:single_fiber_pop_rofl}).
The \trel{} value is there also reduces, but not as strong as in the crossing fiber populations.
This effect comes from the simple feature, that a inclined single fiber population rises its retardation value when it is tilted, whereas a flat crossing population does not due to the signal cancellation.
\\
% 
The relatively low \rvalue{} is in agreement to the noise of \SI{14,3}{\percent} (see \cref{fig:noiseplot}).
The risen patterns of the \rvalue{} represent the area of a $\modelOmega \approx \SI{90}{\degree}$, so where a lower retardation is to be expected.
The most interesting pattern is for the $\modelPsi = \SI{90}{\percent}$ case, where the \rvalue{} is maximal for the $\modelInc = \SI{0}{\degree}$ with a $\alpha_1$ \dummy[check variable name]{} angle of $\SI{80}{\degree}$.
This is quite unexpected, since here the dominant nerve fiber population is the first one, \ie{} $\SI{90}{\percent}$.
Nevertheless the small portion of $\SI{10}{\percent}$ seems to be enough for a high inclination case to disturb the fitting algorithm, which is only based on a single fiber population.
However in comparison with the others result at this orientation, the other parameters seem not do be disturbed for their mean values.
% 
% 
% 
\subsection{Speedup} 
% 
The speedup results of the simulation algorithm shown in \cref{fig:speedTissueMPI,fig:speedSimMPI,fig:speedTissueMP,fig:speedSimMP} show a very good speedup with increasing \ac{CPU} number with the exception of the \openmp{} case for tissue generation algorithm.
Since here the \acsp{CPU} have to write in parallel into the same memory, but not the same memory address (see \cref{sec:dvOpti}), the overhead seems to slow the algorithm quite down.
From the results no more than $\SI{4}{\cpus}$ are necessary, otherwise to much ... gets lost.
Since however the \mpi{} implementation is quite good, the algorithm should be speed up over this implementation if needed.
\\
% 
The simulation on the other hand has for the \mpi{} as well as for the \openmp{} a very good speedup up to the $\SI{48}{\cpus}$.
The \openmp{} is here even stronger.
This is the case, since the \openmp{} implementation does not need to communicate with each other, and since the rest memory access are random anyway, the light beams can be ideal parallized.
The same is true for the \mpi{} implementation with the exception of the tilted case. 
Since here the volume is split, the light beams have to be communicated, which takes time.
Therefore the speedup is there reduced.
\\
% 
Overall since the tissue generation algorithm is quite fast (\SI{30}{\second}), in comparison to the simulation (\SI{80}{\second}), which is needed usually 5 times for the 4 additional tilting angles, the letter algorithm is the bottleneck anyway.
therefore the user can use both implementation to significantly improve the speed of the calculations overall.
% 
% 
% 
% \section{Summary}
% % 
% In this chapter \dummy{}.
% \par
% % 
% The simulation is a very fast implemented algorithm with very good parallized capabilities which can simulate a \ac{3D-PLI} signal quite fast for a statistically filled volume of nerve fiber bundles.
% The here presented results for the here used two crossing populations are in agreement with the experimental data as well as the prio simulation, which however had infirior models as well as very little statistics due to to higher runtime.
% For the future this open source software package is capable of simulation a hughe amount of data which makes is usable for many machine learning algorithms where usually the amount on data dictates the \dummy{}.
% \par
% % 
% The discretized models with the paramter sets of \modelInc{},\modelOmega{} and \modelPsi{} were a good choice to investigate such a complex systems with a lot of freedom.