\setcounter{chapter}{7}
\chapter{Simulation}
\label{cha:simulation_analysis}
% \minitoc
% 
% 
\section{Introduction}
% 
This chapter addresses the simulation of \ac{3D-PLI}.
The first part focuses on the determination of all necessary physical parameters of the tissue and the microscope as well as on the characterization of the simulation parameters.
The latter focuses on the \voxelsize{}, which significantly determines the accuracy.
Subsequently, these parameters are used to simulate the previously created models.
The simulations are evaluated using the routine algorithms implemented in \ac{fastPLI}. The focus of the evaluations of the results is on the accuracy of the inclination analyses for different orientations and intersection configurations.
% 
% 
% 
\section{Parameter characterisation}\label{sec:sim_choose_parameters}
% 
\subsection{Optical resolution}
% 
As described in \cref{sec:opticalResolution}, optical resolution depends on aberration and diffraction, which in turn depend on a variety of factors.
Optical resolution and resampling are modeled as described in \cref{sec:ccdOptic}. 
% 
\begin{figure}[!t]
\setlength{\tikzwidth}{0.35\textwidth}
\centering
\begin{tabular}{cc}
\includegraphics[width=\tikzwidth]{dev/wiki/USAF-1951.pdf}
&
% \tikzset{external/remake next=true}
\inputtikz{dev/gfx/chap8/usaf_image}
\\
% 
\multicolumn{1}{l}{
\begin{minipage}[t]{0.465\textwidth}
\leavevmode\subcaption{\label{fig:usaf}USAF chart from group -2 to 1: \url{https://en.wikipedia.org/wiki/1951_USAF_resolution_test_chart}}
\end{minipage}}&
\multicolumn{1}{l}{
\begin{minipage}[t]{0.465\textwidth}
\leavevmode\subcaption{\label{fig:usaf_image}microscopic image}
\end{minipage}}
\\[2em]
% 
% \tikzset{external/remake next=true}
\inputtikz{dev/gfx/chap8/usaf_line_plots_lr}
&
% \tikzset{external/remake next=true}
\inputtikz{dev/gfx/chap8/usaf_line_plots_up}
\\
% 
\multicolumn{1}{l}{
\begin{minipage}[t]{0.45\textwidth}
\leavevmode\subcaption{\label{fig:usaf_lines_lr}centered line plots lr}
\end{minipage}}&
\multicolumn{1}{l}{
\begin{minipage}[t]{0.45\textwidth}
\leavevmode\subcaption{\label{fig:usaf_lines_ud}centered line plots ud}
\end{minipage}}
\end{tabular}
\caption[USAF test chart measurement]{\acs{LMP} data. Line width: magenta: $\SI{2.19}{\micro\meter}$, yellow $\SI{1.95}{\micro\meter}$ and cyan $\SI{1.74}{\micro\meter}$. Resolution at about $\SI{1.95}{\micro\meter}$ yielding to an optical convolution of $\opticsigma = \SI{0.75}{\pixel}$.}
\label{fig:USAF}
\end{figure}
%
In order to measure the optical resolution of the microscope, the measurements and analysis are performed again in the same way as \cite{MenzelMaster}.
For this purpose, the \textit{1951 \ac{USAF} resolution test chart}\footnote{U.S. Air Force MIL-STD-150A standard of 1951} is acquired.
It consists of several patterns that have three slots with defined spacing and width (see \cref{fig:usaf}).
They are arranged in fields of three vertical and horizontal lines.
The fields are arranged in a spiral, which shrinks by a factor of $\SI{0.5}{}$ from pattern to pattern.
To determine the line width, the fields are numerically ordered according to a main group $i$ and a subgroup $j$.
To determine the resolution of the microscope, the group corresponding to the imaging resolution system, \ie{} Rayleigh criteria (see \cref{fig:rayleigh}), must be identified.
This is done by analyzing the intensity profiles along the perpendicular line of the three slits.
The variance of the measured intensity is measured by analyzing all parallel lines in the slits.\footnote{Multiple images should be measured. These are performed in \cite{MenzelMaster}. This measurement ensures that the results are reproducible}.
% 
\paragraph{Results}
\Cref{fig:usaf_image} shows a section of a captured \ac{LMP} setup image.
The highlighted areas show the analyzed groups 7-6 to 8-2.
The first area of group 7-6 \raisebox{.25em}{\tikzset{external/export next=false}\tikz \draw[RED,ultra thick,dashed](0,0)--(0.25,0);} has a line width of $\SI{2.19}{\micro\meter}$.
The second group 8-1 \raisebox{.25em}{\tikzset{external/export next=false}\tikz \draw[GREEN,ultra thick,dashed](0,0)--(0.25,0);} has a line width of $\SI{1.95}{\micro\meter}$.
The last group 8-2 \raisebox{.25em}{\tikzset{external/export next=false}\tikz \draw[BLUE,ultra thick,dashed](0,0)--(0.25,0);} has a line width of $\SI{1.74}{\micro\meter}$.
The intensity line profiles for the vertical and horizontal cases are shown in \cref{fig:usaf_lines_lr,fig:usaf_lines_ud}.
% 
Using the Rayleigh criterion, the resolution can be limited in the range of the second group and thus by $\SI{1.95}{\micro\meter}$.
This reproduces the measurements in the \cite{MenzelMaster}.
Therefore, the convolution to be applied (see \cref{sec:opticalResolution}) is set to $\opticsigma = \SI{0.75}{\pixel}$.
% 
% 
% 
\subsection{sensor gain and image noise}\label{sec:sensorGain}
%
\begin{figure}[!t]
\centering
% 
\begin{tabular}{cc}
% 
\multicolumn{2}{c}{
\setlength{\tikzwidth}{0.75\textwidth}
\begin{minipage}{\tikzwidth}
\tikzset{external/export next=false} % compile error ...
\inputtikz{gfx/pli/pli_focus}
\end{minipage}
}
\\[-1em]
% 
\multicolumn{2}{l}{
\begin{minipage}[t]{0.95\textwidth}
\leavevmode\subcaption{\label{fig:intensityFocus}Intensity measurement setup. The focus of the microscope is changed so that the image is blurred, resulting in a intensity variation at the border of the absorber material.\\[1em]}
\end{minipage}}
\\
% 
% 1. IMAGE
% \hfill
\begin{tikzpicture}[baseline, trim axis left, trim axis right]
\begin{axis}[
width=0.32\textwidth,
height=0.32\textwidth,
point meta min=0,point meta max=1670,
xmin=0,xmax=1,ymin=0,ymax=1,
scale only axis,
hide axis,
colorbar right,colormap/gray,
colorbar style={every y tick label/.append style={font=\small},
                ytick={0,1670}, yticklabels={0,$I_{\mathit{max}}$,}},
]
\addplot[]graphics [
xmin=0,xmax=1,
ymin=0,ymax=1,
] {dev/gfx/2/PM_000_vmin_0_vmax_1670.png};
\end{axis}
\end{tikzpicture}
&
% 
% 2. IMAGE
\begin{tikzpicture}[baseline, trim axis left, trim axis right]
\begin{axis}[%
width=0.32\textwidth,
height=0.32\textwidth,
axis lines=center,
scale only axis,
xlabel={$x / \si{pixel}$},
ylabel={intensity},
xtick={0, 1000, 2000},
% FIXME
% axis on top=true,
% ytick={0},
% yticklabels={0, $I_{\mathit{max}}$},
]
\addplot[mark=none, blue] table[x index=0, y index=1] {dev/gfx/2/PM_000.dat};
\end{axis}
\end{tikzpicture}
% \hfill
\\[-1em]
% 
% SUBCAPTIONS
\multicolumn{1}{l}{
\begin{minipage}[t]{0.45\textwidth}
\leavevmode\subcaption{\label{fig:intensityImage}single microscopic image}
\end{minipage}}
&
\multicolumn{1}{l}{
\begin{minipage}[t]{0.45\textwidth}
\leavevmode\subcaption{\label{fig:intensityProfile}intensity profile. \dummy[fixme]{}}
\end{minipage}}
\end{tabular}
% 
\caption[intensity image]{y-mean intensity profile}
\label{fig:intensityMeasurement}
\end{figure}
% 
% 
% 
\begin{figure}[!t]
% thesis/2_simulation/0_parameter/sensor_gain.py
\centering
% 
\setlength{\tikzwidth}{0.325\textwidth}
\begin{tabular}{cc}
% \tikzset{external/export next=false}
\inputtikz{gfx/data/PM_noise}
&
% \tikzset{external/export next=false}
\inputtikz{gfx/data/theo_noise}
\\[-1em]
% SUBCAPTIONS
\multicolumn{1}{l}{
\begin{minipage}[t]{0.47\textwidth}
\leavevmode\subcaption{\label{fig:parameterModelNoise} Noise analysis PM. $\opticgain_{\mathit{LMP}} = \SI{0.1175}{}(?)$.}
\end{minipage}}
&
\multicolumn{1}{l}{
\begin{minipage}[t]{0.47\textwidth}
\leavevmode\subcaption{\label{fig:noiseplot}noise \dummy[check values]{}}
\end{minipage}}
\end{tabular}
% 
\caption[Noise analysis]{intensity variance and noise plot}
\label{fig:parameterModelGain}
\end{figure}
% 
As described in \cref{sec:ccdOptic}, optical noise must be modeled by a noise model.
In \cite{Wiese:887678}, the gain factor $\opticgain$ of the microscopic setup was measured by multiple measurements of an image with a variety of intensity values. 
The gain factor describes the linearity between a measured signal and the corresponding noise.
The same type of measurement and analysis is performed here with the \ac{LMP} setup.
\\
To obtain statistics for a large number of intensity values, the sample stage is covered with a fully absorbing cover so that half of the image is dark.
In addition, the focal length is changed so that light is distributed over the entire image sensor below the cover (see \cref{fig:intensityImage}).
By measuring $N=\SI{500}{}$ images, the variance for the different intensity values is determined.
% 
\paragraph{results}
The results are shown in \cref{fig:parameterModelNoise} and show a gain value of $\opticgain_{\mathit{LMP}} = \SI{0.1175}{}(?)$ which is agreement to the hardware specifics \todo{check and ref}.
This gain factor can be used by the model
\begin{align}
f(x) = \floor{\mathrm{normal}(\mu = x, \sigma=\sqrt{\opticgain x})+0.5}
\end{align}
for intensities $I > 0$, $I \gg \sqrt{\opticgain I}$ and integer values.
% 
% 
% 
\subsection{Tissue properties}\label{sec:tissueProp}
% 
\begin{figure}[!t]
\centering
\setlength{\tikzwidth}{0.75\textwidth}
\begin{tabular}{c}
% 
\inputtikz{gfx/data/vervet_transmittance}
\\[-2em]
\begin{minipage}[t]{0.75\textwidth}
\leavevmode\subcaption{\label{fig:brain_trans}transmittance}
\end{minipage}
\\[1.5em]
\inputtikz{gfx/data/vervet_retardation}
\\[-2em]
\begin{minipage}[t]{0.75\textwidth}
\leavevmode\subcaption{\label{fig:brain_ret}retardation}
\end{minipage}
% 
\end{tabular}
\caption[Vervet monkey coronal section transmittance and retardation]{%
Transmittance and retardation map of a Vervet monkey coronal section $\SI{549}{}$.
The absorption coefficient and birefingence strength can be estimated from flat fibers.
Two \acsp{ROI} with \dummy{} pixels are annotated for this purpose.
\itodo{histograms?, why these two rois?}}
\label{fig:brain_ret_trans}
\end{figure}
% 
% 
% 
\begin{figure}[!t]
% 2_simulation/0_parameter/measure_vervet.ipy
\centering
\setlength{\tikzwidth}{0.425\textwidth}
\begin{tabular}{cc}
% 
\inputtikz{gfx/data/vervet_transmittance_zoom} &
\inputtikz{gfx/data/vervet_transmittance_hist} \\[-5mm]
% 
\multicolumn{1}{l}{
\begin{minipage}[t]{0.45\textwidth}
\leavevmode\subcaption{zoom transmittance}
\end{minipage}} &
\multicolumn{1}{l}{
\begin{minipage}[t]{0.45\textwidth}
\leavevmode\subcaption{hist transmittance}
\end{minipage}} \\[10mm]
% 
\inputtikz{gfx/data/vervet_retardation_zoom} &
\inputtikz{gfx/data/vervet_retardation_hist} \\[-5mm]
% 
\multicolumn{1}{l}{
\begin{minipage}[t]{0.45\textwidth}
\leavevmode\subcaption{zoom retardation}
\end{minipage}} &
\multicolumn{1}{l}{
\begin{minipage}[t]{0.45\textwidth}
\leavevmode\subcaption{hist retardation}
\end{minipage}} \\
% 
\end{tabular}
\caption[zoom ret and trans]{%
trans left: $1200 \pm 500$,
trans right: $1200 \pm 500$,
bg: $4530 \pm 240$,
fuellgrad $0.75: \mu \approx 30 \pm 10$,
fuellgrad $0.75: \mu \approx 30 \pm 10$,
ret left: $0.83 \pm 0.07$,
ret right: $0.80 \pm 0.07$
\itodo{what happens in the center?; redesign histogramm; which is left and right?}}
\label{fig:brain_ret_trans_zoom}
\end{figure}
% 
% analog roden:
% trans: $2100 \pm 600$,
% bg: $4000 \pm 50$,
% fuellgrad $0.75: \mu \approx 14 \pm 7$,
% ret: $0.40 \pm 0.11$,
% and human:
% trans: $170 \pm 80$,
% bg: $2700 \pm 110$,
% fuellgrad $0.75: \mu \approx 61 \pm 11$,
% ret: $0.68 \pm 0.11$,
% 
% 
\begin{figure}[!t]
\centering
% \begin{tabular}{c}
\subcaptionbox{\label{fig:simTransValues}transmittance}[0.95\textwidth]{
\includegraphics[page=2]{dev/rc1/tissue/birefringence_output_bf_rc1.pdf}}\\[1em]
\subcaptionbox{\label{fig:simRetValues}retardation}[0.95\textwidth]{
\includegraphics[page=5]{dev/rc1/tissue/birefringence_output_bf_rc1.pdf}}
% \end{tabular}
\caption{simulations for Vervet LMP tissue for different absorption coef and birefringence values. \itodo{exp messungen mit einzeichnen}}
\label{fig:parameterModelSim}
\end{figure}
% 
The absorption coefficient $\absorp{}$ and the birefringence $\dn{}$ have to be estimated from the tissue to use appropriate values in the simulation.
As the literature shows, these values are in the range of $\dn{}$ \dummy{}.
Since the models created here use stiff fiber segments the volume density cannot reach the high density usually present in \ac{WM}.
To overcome this problem, the absorption and birefringence strengths must be increased to obtain comparable results.
\par
%
For a measurement of the properties it is important to ensure that a homogeneous region is chosen.
In coronal section, the corpus callosum is suitable for this purpose. It is the main fiber connection between the two cerebral hemispheres (see \cref{fig:parameterModelSim}). In a central section, the nerve fibers generally lie flat in section
% 
\ref{fig:brain_ret_trans} shows transmission and retardation maps of the vervet monkey coronal section, and \ref{fig:brain_ret_trans_zoom} shows a zoomed version.
The retardation map shows a visible reduction of retardation in the center of the corpus callosum.
Therefore, because of the need for homogeneity, this central region is neglected.
A left and a right region remain, which may be homoginous in both transmission and retardation (see histograms).
The left region has $\SI{1125858}{}$ number of pixels, the right part $\SI{1064629}{}$
% 
The same measurements are made for the human and rodent in the corpus callosum (see Appendix). \todo{Appendix}
% 
% 
% 
\subsubsection{absorption}
% 
When the light passes through the tissue, it is absorbed in the matter on the one hand and scattered on the other.
Both effects can be modeled mathematically in this linear optical simulation as a decrease in intensity along the path.
A drawback of this simulation is that the scattering is not constant for all fiber configurations and orientations \dummy[see MM]{}.
Therefore, the measured absorption coefficient is only valid for the same fiber configurations as in the region.
Absorption only changes the transmittance and hence the relative noise, not the delay or direction of the signal. 
Since the noise is quite low in this microscopic configuration, it should have little effect on the results. 
\par
% 
To measure absorption, the transmittance of the coronal slice in the \ac{ROI} of the corpus callosum is measured against the background value.
The absorption coefficient is then calculated by applying \dummy{}.
This values are then increased according to the volume fraction, of the models and then simulated to ensure that the same behavior can be modelled.
% 
\paragraph{Results}
A relative transmittance of about $\SIrange{20}{30}{\percent}$ is present within the \ac{ROI} of the corpus callosum. 
The simulation parameter must be set to achieve a similar transmission value for the models.
This means that for a $\SI{60}{\micro\meter}$ thick section, the absorption coefficient $\absorp{}$ is about $\SI{30}{\milli\meter\tothe{-1}}$. 
The absorption for the roden is about $\SI{8}{\milli\meter\tothe{-1}}$, and that of the human is about $\SI{65}{\milli\meter\tothe{-1}}$.
\\
%
\Cref{fig:simTransValues} shows the simulated transmittance values for different fiber radii. 
For smaler radii the variance of the transmittance is very small \dummy{} and increases with the radii.
This behavier is epected, since models with larger radii have more space where no tissue is present, and therefore the light can travel freely through the \say{section}.
In comparison to the histogram \cref{fig:brain_ret_trans_zoom} the models with larger radii have more in common with the distribution.
However from literature one can neglect, that this kind of fiber radii is present in corpus calosum \dummy{}.
Nevertheless the tissue can have more complicated configurations and effects, which leads to a larger viarity of absorption, \eg{} cells, blood vessels and so on.
Also at this point the here developed simulation is not ready to simulate scattered light, which is to be expected to increase the range of the absorption and therefore resulting a a larger variance of transmittance.
% 
% 
%
\subsubsection{birefringence}
% 
The birefringence cannot simply be measured like the transmittance.
Due to the radial optical axis of the myelin, the relativ obsorving birefringence is smaller than the one of the myelin.
However the simulation up to a point has the same geometrical constrains.
By simulating difference birefringence strength and comparing them to the retardation of the tissue one can find a proper value.
The simulation will use a flat single fiber population and the same parameters as the later simulation model library (see \dummy{}).
% 
\paragraph{Results}
The results in \cref{fig:parameterModelSim} show that to reach a retardation value similar to the experimental setup of around $\SI{0.8}{}$ a birefringence value of $\SI{0.008}{}$ for the microscopic model should be chosen.
Looking at the different model nerve fiber radii the variance of the retardation values increase strong, similar to the transmittance.
This is to be expected since the simulated tissue section has a hight of $\SI{60}{\micro\meter}$.
Therefore for a radii of $\SI{10}{\micro\meter}$ the number of tissue voxels has to vary depending on where the fibers are positioned in the volume.
For larger radii this leads to the problem, that the retardation cannot be over 1 and therefore the signal amplitude reduces again.
% 
% 
% \subsection{noise} <--- why this section?
% % 
% From \cref{sec:sensorGain,sec:tissueProp} the noise level can be calculated.
% However, when considering the different species, it is important to note that transmittance levels vary.
% Rodents typically have the highest transmittance, monkeys have a lower transmittance, and humans have the lowest transmittance.
% The reason for the rather low transmission of the human sections is probably due to the long post-mortem period.
% \Cref{fig:noiseplot} shows a log-log plot for this purpose.
% To run the simulation, a relative error of \dummy{} is used for a midrange analysis.
% 
% 
% 
\subsection{voxel size \texorpdfstring{\voxels{}}{}}
% 
% 
\begin{figure}[!t]%p
% 2_simulation/0_parameter/fiber_radii.py
\centering
\includegraphics[width=\textwidth, page=1]{dev/rc1/voxel_size/voxel_size_plots_data_r05_output_vs_135_0.01_6_25_vervet_r_rc1.pdf}
\caption[voxel size model with noise]{The mean difference is constant for smaller voxel sizes and does not start to grow significantly before $\voxels=\SI{0.1}{\micro\meter}$.}
\label{fig:voxelsizeNoise}
\end{figure}
% 
The \voxelsize{} paramter $\voxels$ is the most important parameter for simulation accuracy, as it determines how accurately the models are discretized.
Smaller values mean more details, more accurate optical axes and more light rays.
However, this also increases the number of calculations and memory by $O(n_{\mathit{voxel}}^3)$.
Therefore, it is recommended to choose the voxel size as large as possible without introducing a significant error due to the discretization.
\par
% 
To investigate this effect, a simulation is performed with different types of voxel sizes from the $\SIrange{0.01}{1.3}{\micro\meter}$.
The smallest voxel size $\SI{0.01}{\micro\meter}$ is used as ground truth. 
Since this voxel size is so small, only a volume of $3 \cdot \SI{1.3}{\micro\meter} \cdot \SI{1.3}{\micro\meter} 3 \cdot \SI{1.3}{\micro\meter} \times \SI{60}{\micro\meter}$ is used without tilting.
Otherwise, if tilting is used, the volume would have to be increased so that the light beam still travels through the tissue.
The other parameters are set as in \dummy{}.
The models to be simulated are $(||,\modelInc = \SI{0}{\degree})$, $(||,\modelInc = \SI{90}{\degree})$, $(\times,\modelInc = \SI{0}{\degree})$, and $(\times,\modelInc = \SI{90}{\degree})$ with a anteil of $\modelPsi = \SI{0.5}{}$, since these configurations represent the extrema for two fiber populations.
To improve the statistics, the simulation is repeated in the volume on an aquidistant $xy$ grid of $\num{5}$ by $\num{5}$ points within the $\SI{65}{\micro\meter}$ by $\SI{65}{\micro\meter}$ model.
% 
\paragraph{Results}
The results in \cref{fig:voxelsizeNoise} show that the relative difference to the smallest \voxelsize{} increases statically significantly above a value of \SI{0.1}{\micro\meter}.
The variance of the relative difference increases with increasing \voxelsize{} from this value.
% 
The results shown in \cref{fig:voxelsizeNoise} suggest that with noise present and a voxel size smaller than $\SI{0.1}{\micro\meter}$, accuracy does not increase with respect to the noise model chosen here.
However, this is only true for a pixel size of $\SI{1.3}{\micro\meter}$ and fiber radii of $\SI{0.5}{\micro\meter}$.
%  
% 
% 
\section{Simulation}
% 
\subsection{Parameter and environment}
%  
\begin{table}[!b]
\caption{Simulation parameters}
\centering
% \sisetup{open-bracket={\{}, close-bracket={\}}, list-final-separator={,},list-pair-separator={,}}%
\pgfplotstabletypeset[%
    thesisTableStyle,
    column type=l,
    columns/variable/.style={string type},
    columns/value/.style={string type},
    every head row/.style={before row=\toprule,after row=\midrule},
    every last row/.style={after row=\bottomrule},
    col sep=&,
    row sep=\\,
]
{variable & value\\
% 
simpli.voxel\_size & $\SI{0.1}{\micro\meter}$\\
simpli.pixel\_size & $\SI{1.3}{\micro\meter}$\\
simpli.voi & $[[\SI{-35}{\micro\meter}, \SI{-35}{\micro\meter}, \SI{-30}{\micro\meter}], [\SI{30}{\micro\meter}, \SI{35}{\micro\meter}, \SI{35}{\micro\meter}]]$\\
simpli.filter\_rotations & $\SIlist{0;20;40;60;80;100;120;140;160}{\degree}$\\
simpli.interpolate & \texttt{"Slerp"}\\
simpli.wavelength & $\SI{525}{\nano\meter}$\\
simpli.optical\_sigma & $\SI{0.75}{\pixel}$\\
tilt angle & $\SI{3.9}{\degree}$\\
simpli.light\_intensity & $\SI{8000}{}$\\
gain & \SI{0.1175}{}\\
simpli.noise\_model & \code{lambda x: np.floor(np.random.normal(x,}\\
 & \ \ \ \ \ \code{np.sqrt(gain * x))+0.5).astype(np.uint16)}\\
fiber absorption & Roden: $\SI{14}{\milli\meter\tothe{-1}}$, Vervet: $\SI{30}{\milli\meter\tothe{-1}}$, Human: $\SI{60}{\milli\meter\tothe{-1}}$\\
fiber model & 'r'\\
fiber birefringence & 0.008\\
fiber radii & axon: \SI{0.75}{}, myelin: \SI{1}{}\\
model delta inclinations & single: $\SI{5}{\degree}$, crossings: $\SI{30}{\degree}$\\
model delta rotation & $\SI{15}{\degree}$\\
}
\label{tab:simParameters}
\end{table}
% 
% 
%
As described in \cref{software}, \fastpli{} contains a pipeline implementation.
This implementation is used to compute the discretized volumes, simulate the signal, and apply optical noise and tilt analysis to the resulting signal using the algorithm \rofl{}. \footnote{source code is available in the appendix \dummy{}}.
This pipeline is used to simulate and analyze the \ac{3D-PLI} signal for the model library with different orientations of two crossing fiber populations.
\Cref{tab:simParameters} lists the parameters of the simulation within the variable notation \fastpli{}.
With a pixel size of \SI{1.3}{\micro\meter}, $\SI{2500}{\pixel}$ is available per simulation for statistical analysis.
The models are inclined in steps of $\Delta \modelInc = \SI{5}{\degree}$ for a single fiber population and by $\Delta \modelInc = \SI{30}{\degree}$ for crossing fiber populations. The latter are then also rotated around the first fiber population (see \cref{fig:twomodelpopdesign}). Depending on the crossing angle $\modelOmega$, the step size adjusts so that the spatial distance between two resulting rotations is close to $\Delta \modelRot = \SI{15}{\degree}$.
This ensures that the sphere is sampled reasonably equidistantly.
% The simulation were performed with a single compute node (\texttt{2x Intel(R) Xeon(R) CPU E5-4657L v2}).
% 
% 
% 
\subsection{Single inclined fiber population}
% 
% \begin{figure}[!tp]
% \centering
% \newlength{\width}
% \setlength{\width}{0.42\textwidth}
% \subcaptionbox{\label{fig:single_fiber_pop_hist}Histogram}[\textwidth]{
% \begin{tabular}{c|c}
%     \includegraphics[width=\width]{dev/rc1/single/cube_2pop_135_rc1_single_plots_single_pop_hist_0.0.pdf}&
%     \includegraphics[width=\width]{dev/rc1/single/cube_2pop_135_rc1_single_plots_single_pop_hist_30.0.pdf}\\ 
%     \includegraphics[width=\width]{dev/rc1/single/cube_2pop_135_rc1_single_plots_single_pop_hist_60.0.pdf}&
%     \includegraphics[width=\width]{dev/rc1/single/cube_2pop_135_rc1_single_plots_single_pop_hist_90.0.pdf}
% \end{tabular}}
% % 
% \subcaptionbox{\label{fig:single_fiber_pop_rofl}Analysis}[\textwidth]{
% \includegraphics[]{dev/rc1/analysis/plots_single_pop_output_cube_2pop_135_rc1_single.pdf}}
% % 
% \caption[sim]{left: 2d log histogramm orientation from rofl analysis of simulation, right: 2d log histogramm of orientation of model segemnts. \itodo{rfc plots}}
% % \label{fig:single_fiber_pop_hist}
% \end{figure}
% 
\begin{figure}[!t]
\centering
\newlength{\width}
\setlength{\width}{0.45\textwidth}
\begin{tabular}{c|c}
    \includegraphics[width=\width]{dev/rc1/single/cube_2pop_135_rc1_single_plots_single_pop_hist_0.0.pdf}&
    \includegraphics[width=\width]{dev/rc1/single/cube_2pop_135_rc1_single_plots_single_pop_hist_30.0.pdf}\\ 
    \includegraphics[width=\width]{dev/rc1/single/cube_2pop_135_rc1_single_plots_single_pop_hist_60.0.pdf}&
    \includegraphics[width=\width]{dev/rc1/single/cube_2pop_135_rc1_single_plots_single_pop_hist_90.0.pdf}
\end{tabular}
% 
\caption[sim]{Single: left: 2d log histogramm orientation from rofl analysis of simulation, right: 2d log histogramm of orientation of model segemnts. \itodo{rfc plots}}
\label{fig:single_fiber_pop_hist}
\end{figure}
% 
\begin{figure}[!p]
\centering
\includegraphics[]{dev/rc1/analysis/plots_single_pop_output_cube_2pop_135_rc1_single.pdf}
\caption[]{single population rofl analysis}
\label{fig:single_fiber_pop_rofl}
\end{figure}
% 
To begin with, a single population of fibers is analyzed.
Since for a single population the direction is negligible, only the inclination parameter $\modelInc{}$ remains.
\par
% 
\Cref{fig:single_fiber_pop_hist, fig:single_fiber_pop_hist_app} shows the orientation distribution of a single fiber population for different inclinations $\modelInc{}$.
The distributions of the orientations of the model segments are shown on the left side, and those of the inclination analysis of the simulated images are shown on the right side.
The distributions are logithmic and are weighted by the area of the 2d binning.
The comparison between both distributions shows an agreement of the inclination analysis with the model orientations for all inclinations.
\par
% 
% 
% 
% \begin{figure}[!p]
% \centering
% \includegraphics[]{dev/rc1/analysis/plots_single_pop_output_cube_2pop_135_rc1_single.pdf}
% \caption[]{single population rofl analysis}
% \label{fig:single_fiber_pop_rofl}
% \end{figure}
% 
%Transmittance
\Cref{fig:single_fiber_pop_rofl} show the different modalities after analyzing the raw non-tilted measurements.
The transmittance increases slightly with inclination angle for inclinations $\modelInc<\SI{75}{\degree}$.
For $\modelInc \ge \SI{75}{\degree}$ the transmittance increases significantly, \ie{} the absorption decreases.
The variance of transmittance is within a few percent for all inclinations and also increases slightly with inclination angle.
\\
% 
% Retardation
The retardation plot shows the retardation of the non-tilted simulations and the theoretical curve that follows $(\cos(\modelInc) + 1) / 2 \cdot \mean(ret(0 deg)$.
The retardation follows the theoretical line.
However, in the middle slope angles, the measured retardation is higher than the theoretical line.
Otherwise, the variance along all fiber inclinations is equal to \modelInc{}.
\\
% 
% Direction
The next graph shows the measured direction.
For inclination angles smaller than $\SI{75}{\degree}$ the variance remains quite small.
For larger values, the variance increases.
It must be remembered that the boxplot does not take into account that the values are periodic.
Therefore, especially for the $\modelInc=\SI{90}{\degree}$, the values from the $\SIrange{-90}{90}{\degree}$ are uniformly distributed.
\\
% 
% Inclination
The inclination is drawn as a line as well as the set angle of inclination.
The angles are shifted around the $\modelInc$ so that the admissible distribution values are in the range $[\modelInc-\SI{90}{\degree}, \modelInc+\SI{90}{\degree})$.
The median follows the theoretical curve.
The $25-75$ values are \dummy{} \todo{accuracy analysis}.
Again for values greater than $\SI{75}{\degree}$ the variance increases significantly, except for $\modelInc = \SI{90}{\degree}$ it is similar to the first values.\todo{why?}
\\
% 
% trel
The median relative birefringence thickness \trel{} is stable for values up to $\modelInc < \SI{60}{\degree}$, but the variance increases slightly.
For larger values, the median decreases while the variance increases.
Some outliers reach values $>\SI{1}{}$.
For inclination angles $\ge \SI{85}{\degree}$, the quartial value $\SI{75}{\percent}$ increases by $\SI{1}{}$ and the median for $\SI{90}{\degree}$ reaches about 2.
\\
% 
% domega
The opening angle is stable with a median of \dummy{} up to a tilt angle of $\le \SI{70}{\degree}$.
At higher inclination angles, both the median and the variance increase significantly analogous to the inclination except for the last case.
% 
% 
% 
\subsection{Crossing flat fiber populations}
% 
\begin{figure}[!t]
\centering
\setlength{\width}{0.45\textwidth}
\begin{tabular}{c|c}
    \includegraphics[width=\width]{dev/rc1/flat/cube_2pop_135_rc1_flat_plots_flat_pop_hist_omega_0.0_psi_0.3.pdf} &
    \includegraphics[width=\width]{dev/rc1/flat/cube_2pop_135_rc1_flat_plots_flat_pop_hist_omega_30.0_psi_0.3.pdf} \\ \includegraphics[width=\width]{dev/rc1/flat/cube_2pop_135_rc1_flat_plots_flat_pop_hist_omega_60.0_psi_0.3.pdf} & \includegraphics[width=\width]{dev/rc1/flat/cube_2pop_135_rc1_flat_plots_flat_pop_hist_omega_90.0_psi_0.3.pdf}
\end{tabular}
\caption[sim]{Flat: psi=0.3; left: 2d log histogramm orientation from rofl analysis of simulation, right: 2d log histogram of orientation of model segemnts. \itodo{legende}}
\label{fig:flat_03_fiber_pop_hist}
\end{figure}
% 
\begin{figure}[!p]
\centering
\includegraphics[page=3]{dev/rc1/analysis/plots_flat_pop_output_cube_2pop_135_rc1_flat.pdf}
\caption[]{flat population psi=0.3 rofl analysis}
\label{fig:flat_03_fiber_pop_rofl}
\end{figure}
% 
\begin{figure}[!t]
\centering
\setlength{\width}{0.4\textwidth}
\begin{tabular}{c|c}
    \includegraphics[width=\width]{dev/rc1/flat/cube_2pop_135_rc1_flat_plots_flat_pop_hist_omega_0.0_psi_0.5.pdf} &
    \includegraphics[width=\width]{dev/rc1/flat/cube_2pop_135_rc1_flat_plots_flat_pop_hist_omega_30.0_psi_0.5.pdf} \\
    \includegraphics[width=\width]{dev/rc1/flat/cube_2pop_135_rc1_flat_plots_flat_pop_hist_omega_60.0_psi_0.5.pdf} &
    \includegraphics[width=\width]{dev/rc1/flat/cube_2pop_135_rc1_flat_plots_flat_pop_hist_omega_90.0_psi_0.5.pdf}
\end{tabular}
\caption[sim]{Flat: psi=0.5; left: 2d log histogramm orientation from rofl analysis of simulation, right: 2d log histogramm of orientation of model segemnts. \itodo{legende}}
\label{fig:flat_05_fiber_pop_hist}
\end{figure}
% 
\begin{figure}[!p]
\centering
\includegraphics[page=5]{dev/rc1/analysis/plots_flat_pop_output_cube_2pop_135_rc1_flat.pdf}
\caption[]{flat population psi=0.5}
\label{fig:flat_05_fiber_pop_rofl}
\end{figure}
% 
The next focus is on two flat crossing fiber populations.
Here the results for the case $\modelPsi = \SI{30}{\percent}$ and $\modelPsi = \SI{50}{\percent}$ are examine.
The other fiber population proportion are available in \dummy{}. %\cref{}
\\
% 
\Cref{fig:flat_03_fiber_pop_hist} show the distribution of the orientation for the simulation and the model fiber segments.
The resulting fiber orientation seems to follow only the second fiber orientation, which has a density of $\SI{70}{\percent}$.
\\
% Transmittance
In \cref{fig:flat_03_fiber_pop_rofl} the transmittance increases for small crossing angles $\modelOmega$ with increasing $\modelOmega$ and then the \dummy{} flattens.
The variance does not change with \dummy[value]{}.
\\
% Retardation
The retardation seems to follow a negative linear \dummy{} with increasing crossing angle.
The variance increases with the crossing angle.
\\
% Direction
The direction is plotted in \dummy{} with the direction of the fiber populations values.
The data follows the second fiber bundle, however is significantly lower in the middle section, even if slightly with a delta of \dummy[value]{}.
\\
% Inclination
The inclination is with its median for all crossing angles at about $\SI{0}{\degree}$.
The variance significantly increases with increasing crossing angle up to \dummy[3]{} times the lowest value.
\\
% Trel
The relative thickness is decreases with increasing crossing angle.
The curve is identical with the retardation value.
\\
% domega
The opening angle is low for the first half of the crossing angles, but slightly increasing.
The variance for this range also increases slightly.
There exist several outliers with values $> \SI{5}{\degree}$.
From $\modelOmega \ge \SI{50}{\degree}$ the value increases with a stronger slope.
The median rises up to about a value of $\SI{5}{\degree}$.
% 
% 
\subsection{Inclined crossing fibers population}
% 
\begin{figure}[!t]
\centering
\setlength{\width}{0.45\textwidth}
\begin{tabular}{c|c}
    \includegraphics[width=\width]{dev/rc1/inclined/cube_2pop_135_rc1_inclined_plots_inclined_pop_hist_omega_0.0_psi_0.3.pdf} &
    \includegraphics[width=\width]{dev/rc1/inclined/cube_2pop_135_rc1_inclined_plots_inclined_pop_hist_omega_30.0_psi_0.3.pdf} \\ \includegraphics[width=\width]{dev/rc1/inclined/cube_2pop_135_rc1_inclined_plots_inclined_pop_hist_omega_60.0_psi_0.3.pdf} & \includegraphics[width=\width]{dev/rc1/inclined/cube_2pop_135_rc1_inclined_plots_inclined_pop_hist_omega_90.0_psi_0.3.pdf}
\end{tabular}
% 
\caption[sim]{Inclined: psi=0.3; left: 2d log histogramm orientation from rofl analysis of simulation, right: 2d log histogram of orientation of model segments. \itodo{legende}}
\label{fig:inclined_03_fiber_pop_hist}
\end{figure}
% 
\begin{figure}[!p]
\centering
\includegraphics[page=3]{dev/rc1/analysis/plots_inclined_pop_output_cube_2pop_135_rc1_inclined.pdf}
\caption[]{Inclined population psi=0.3 rofl analysis}
\label{fig:inclined_03_fiber_pop_rofl}
\end{figure}
% 
\begin{figure}[!t]
\centering
\setlength{\width}{0.4\textwidth}
\begin{tabular}{c|c}
    \includegraphics[width=\width]{dev/rc1/inclined/cube_2pop_135_rc1_inclined_plots_inclined_pop_hist_omega_0.0_psi_0.5.pdf} &
    \includegraphics[width=\width]{dev/rc1/inclined/cube_2pop_135_rc1_inclined_plots_inclined_pop_hist_omega_30.0_psi_0.5.pdf} \\
    \includegraphics[width=\width]{dev/rc1/inclined/cube_2pop_135_rc1_inclined_plots_inclined_pop_hist_omega_60.0_psi_0.5.pdf} &
    \includegraphics[width=\width]{dev/rc1/inclined/cube_2pop_135_rc1_inclined_plots_inclined_pop_hist_omega_90.0_psi_0.5.pdf}
\end{tabular}
\caption[sim]{Inclined: psi=0.5; left: 2d log histogramm orientation from rofl analysis of simulation, right: 2d log histogramm of orientation of model segemnts. \itodo{legende}}
\label{fig:inclined_05_fiber_pop_hist}
\end{figure}
% 
\begin{figure}[!p]
\centering
\includegraphics[page=5]{dev/rc1/analysis/plots_inclined_pop_output_cube_2pop_135_rc1_inclined.pdf}
\caption[]{Inclined population psi=0.5}
\label{fig:inclined_05_fiber_pop_rofl}
\end{figure}
% 
% 
The last focused few is on two crossing fiber bundles with a rotating angle of $\modelRot = \SI{90}{\degree}$, \ie{} the first fiber population along the $x$-axis and the second population in the $xz$-axis.
As before the two parameters $\modelPsi = \SI{30}{\percent}$ and $\modelPsi = \SI{50}{\percent}$  will be discussed.
The other fiber proportions are listed in the \dummy[appendix]{}.
\\
% 
The histogram data for both cases are show for the measuremnt only one visible orientation.
\dummy{}.
\\
% Transmittance
The transmittance is for both cases almost identical.
It grows with increasing crossing angle, but the slope decreases for larger $\modelOmega$.
The median in both cases grows to a value of $\SI{110}{\percent}$ in contrast to the flat case.
The variance stays similar for all values.
\\
% Retardation
The retardation in both cases decreases almost linear with increasing crossing angle $\modelOmega$.
In the case of $\modelPsi = \SI{30}{\percent}$ the final value is about $\SI{0.18}{}$ in contrast to $\modelPsi = \SI{50}{\percent}$ with a final value of $\SI{0.3}{}$.
The varce increases for larger crossing angles significantly.
\\
% Direction
The direction value is for both cases again very similar.
The median always is around \SI{0}{\degree}.
The quartle grows from about $\SI{0.5}{\degree}$ to $\SI{2}{\degree}$ with increasing crossing angle.
\\
% Inclination
In the case of $\modelPsi = \SI{30}{\percent}$ the inclination is ab to about $\modelOmega =\SI{60}{\degree}$ almost linear with increasing crossing angle.
In the case of $\modelPsi = \SI{50}{\percent}$ the tip point is about $\modelOmega = \SI{60}{\degree}$
The variance is increasing for both parameters with increasing crossing angle.
The main difference of both parameters is, that in the case of $\modelPsi = \SI{30}{\percent}$ the maximum reached inclination is significant higher than in the case of $\modelPsi = \SI{50}{\percent}$.
However in both cases the inlination is always lower then the inclination value of the second fiber population.
\\
% Trel
The relative effective thickness \trel{} decreases in both cases.
In the case of $\modelPsi = \SI{50}{\percent}$ the correlation is almost linear and the variance increases slightly with increasing crossing angle.
In the case of $\modelPsi = \SI{30}{\percent}$ a small hill is visible around $\modelOmega = \SI{50}{\degree}$ where the value is slighly increases as well as the variance.
\\
% domega
The opening angle as well as its variance seems to exponentially increase with increasing crossing angle in the case of $\modelPsi = \SI{30}{\percent}$.
In the case of $\modelPsi = \SI{50}{\percent}$ the median as well as the variance also increases, however for the largest crossing values the increase stopped at a median of about $\SI{6}{\degree}$.
% 
\subsection{Free crossing fiber populations}
% 
The following data shows the result of the tilting analysis without the orientation angles, since the mean value cant easyli be calculated for a single angle. \footref{orientations lie on a manifold of a hemispherical with symmetries as well as the angles are depending on each other.}
The data of the previous chapter had the advantage, that the results of the angles are close enough to apply a remapping around the expected value.
\\
% 
The spherical plots in \cref{sim_ana_acc,sim_ana_ret,sim_ana_trans,sim_ana_trel} are designed that the thick black circle shows the orientation of the first fiber population, and the thin dashed circles the orientation of the second fiber population. 
At the position of the second fiber population the resulting mean value is inserted.
% 
\begin{figure}[!p]
\centering
\includegraphics[]{dev/rc1/analysis/simulation_analysis_hist_0.5_setup_PM_s_Vervet_m_r_acc.pdf}
\caption[Simulation acc]{Mean \acc{} between model and tilt analysis orientations.}
\label{fig:sim_ana_acc}
\end{figure}
% 
\Cref{fig:sim_ana_acc} shows the \acc{} value, \ie{} how good the coefficient of the odf base function are in alignment.
\dummy[acc of psi = 1]{}.
The $\modelPsi=\SI{10}{\percent}$ show for all $\modelInc$ no significant reduction of the \acc{} value.
The for $\modelPsi=\SI{30}{\percent}$ the most secondary inclined values have a significant reduction of the \acc{} value.
This sightly less visible for $\modelInc = \SI{30}{\degree}$ case and even less for the $\modelInc = \SI{60}{\degree}$.
The $\modelInc = \SI{90}{\degree}$ is less than in the $\modelPsi=\SI{10}{\percent}$ case, but still constant for all secondary orientations.
For the equal proportional fiber populations the \acc{} value reaches its lowest values for flat inclined first population $\modelInc = \SIlist{0,30}{\degree}$.
In the case of $\modelInc = \SI{0}{\degree}$ the \acc{} value is lowest with a crossing angle of $\modelOmega = \SI{90}{\degree}$.
This crossing angle for the slightly inclined has a higher \acc{} value \dummy[why?, -> show odf]{}.
Otherwise the distribution is similar with respect of the 3d orientation symmetry.
% 
% 
% 
\begin{figure}[!p]
\centering
\includegraphics[]{dev/rc1/analysis/simulation_analysis_hist_0.5_setup_PM_s_Vervet_m_r_ret_mean.pdf}
\caption[Simulation retardation]{Mean retardation.}
\label{fig:sim_ana_ret}
\end{figure}
% 
% 
% 
\begin{figure}[!p]
\centering
\includegraphics[]{dev/rc1/analysis/simulation_analysis_hist_0.5_setup_PM_s_Vervet_m_r_trans_mean.pdf}
\caption[Simulation transmittance]{Mean transmittance value.}
\label{fig:sim_trans}
\end{figure}
% 
% 
% % 
% \begin{figure}[!p]
% \centering
% \includegraphics[]{dev/rc1/analysis/simulation_analysis_hist_0.5_setup_PM_s_Vervet_m_r_rdir_mean.pdf} 
% % colorbar
% \tikzset{external/export next=false}
% \begin{tikzpicture}
% \begin{axis}[
%     hide axis,
%     scale only axis,
%     height=0pt, width=0pt,
%     xmin=0, xmax=1, ymin=0, ymax=1,
%     colormap/twilight,
%     colorbar horizontal,
%     point meta min=0,
%     point meta max=180,
%     colorbar style={
%         width=10cm,
%         xticklabel style={tick label style={font=\footnotesize}},
%         xtick={0,20,40,60,80,100,120,140,160,180},
%         xticklabels={$\SI{0}{\degree}$,$\SI{20}{\degree}$,$\SI{40}{\degree}$,$\SI{60}{\degree}$,$\SI{80}{\degree}$,$\SI{100}{\degree}$,$\SI{120}{\degree}$,$\SI{140}{\degree}$,$\SI{160}{\degree}$,$\SI{180}{\degree}$},
%     }]
% \end{axis}
% \end{tikzpicture}
% \caption[Simulation direction]{Circmean direction from tilt analysis.}
% % \label{fig:sim_fyjsrg}
% \end{figure}
% % 
% % 
% % 
% \begin{figure}[!p]
% \centering
% \includegraphics[]{dev/rc1/analysis/simulation_analysis_hist_0.5_setup_PM_s_Vervet_m_r_rincl_mean.pdf} 
% % colorbar
% \tikzset{external/export next=false}
% \begin{tikzpicture}
% \begin{axis}[
%     hide axis,
%     scale only axis,
%     height=0pt, width=0pt,
%     xmin=0, xmax=1, ymin=0, ymax=1,
%     colormap/twilight,
%     colorbar horizontal,
%     point meta min=-90,
%     point meta max=90,
%     colorbar style={
%         width=10cm,
%         xticklabel style={tick label style={font=\footnotesize}},
%     }]
% \end{axis}
% \end{tikzpicture}
% \caption[Simulation inclination]{Circmean inclination from tilt analysis.}
% % \label{fig:sim_fyjsrg}
% \end{figure}
% 
\begin{figure}[!p]
\centering
\includegraphics[]{dev/rc1/analysis/simulation_analysis_hist_0.5_setup_PM_s_Vervet_m_r_rtrel_mean.pdf} 
\caption[Simulation \trel{}]{Mean \trel{} from tilt analysis.}
\label{fig:sim_ana_trel}
\end{figure}
% 
% 
% 
\subsection{speedup}
% 
\begin{figure}[!t]
\centering
\setlength{\tikzwidth}{0.85\textwidth}
\begin{tikzpicture}[baseline, trim axis left, trim axis right]
\begin{axis}[%
table/col sep=comma,
width=\tikzwidth,
height=0.5\tikzwidth,
axis lines=center,
scale only axis,
xlabel={\small cpus},
ylabel={\small speedup},
xticklabel style={font=\footnotesize},
yticklabel style={font=\footnotesize},
y label style={at={(axis description cs:0,.85)},rotate=90,anchor=north},
cycle multiindex* list={mark list*\nextlist Dark2\nextlist},
]
\addplot+[only marks] table[x index=1, y index=3] {dev/rc1/speed/generate_tissue_v_0.1.dat};
\addplot[black, dashed] coordinates {(1,1) (24,24)} node[pos=0.9, sloped, anchor=south] {ideal};
\end{axis}
\end{tikzpicture}
\caption[]{speedup discrete tissue generation. \itodo{einheitlicher style}}
\end{figure}
% 
% 
\begin{figure}[!t]
\centering
\setlength{\tikzwidth}{0.85\textwidth}
\begin{tikzpicture}[baseline, trim axis left, trim axis right]
\begin{axis}[%
table/col sep=comma,
width=\tikzwidth,
height=0.5\tikzwidth,
axis lines=center,
scale only axis,
xlabel={\small cpus},
ylabel={\small speedup},
xticklabel style={font=\footnotesize},
yticklabel style={font=\footnotesize},
cycle multiindex* list={
    mark list*\nextlist
    Dark2\nextlist
},
]
\addplot+[only marks] table[x index=1, y index=8] {dev/rc1/speed/simulation_v_0.1.dat}; \addlegendentry{flat}
\addplot+[only marks] table[x index=1, y index=9] {dev/rc1/speed/simulation_v_0.1.dat}; \addlegendentry{W}
\addplot+[only marks] table[x index=1, y index=10] {dev/rc1/speed/simulation_v_0.1.dat}; \addlegendentry{S}
\addplot+[only marks] table[x index=1, y index=11] {dev/rc1/speed/simulation_v_0.1.dat}; \addlegendentry{E}
\addplot+[only marks] table[x index=1, y index=12] {dev/rc1/speed/simulation_v_0.1.dat}; \addlegendentry{N}
\addplot[black, dashed] coordinates {(1,1) (32,32)} node[pos=0.9, sloped, anchor=south] {ideal};
\end{axis}
\end{tikzpicture}
\caption[]{Speedup simulation for 5 tilt direction. \itodo{einheitlicher style}}
\end{figure}
% 
% 
% 
\subsection{misc}
% 
\begin{figure}[!p]
\centering
\includegraphics[width=0.9\textwidth]{dev/rc1/cube_2pop_135_rc1_radius_acc_compare_Vervet_PM_r.pdf}
\caption[sim acc]{ $\acc{}(GT,GT_{0.5})$, $\acc{}(SIM,GT_{0.5})$ and $\acc{}(SIM,SIM_{0.5})$. hypothese: abweichungen in GtSim bei kleinen radien dadurch, dass zwei richtungen nicht sichtbar sind? schaue dir die histogramme und odfs an. \itodo{komplett neu machen}}
\label{fig:accVervetPMr}
\end{figure}
% 
% 
% 
\section{Discussion}
% 
\subsection{Single inclined fiber population} 
% 
asd