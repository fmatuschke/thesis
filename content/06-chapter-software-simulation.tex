\cleardoublepage
\setcounter{chapter}{5}
\chapter{\acs{3D-PLI} Simulation}
\label{cha:sof:simulation}
%
Simulations of \ac{3D-PLI} have been used to study multiple effects of the microscopic technique involved with brain sections \cite{Dohmen2015,Menzel2015,Menzel2016,Menzel2020,Menzel2021,MenzelMaster,MenzelDissertation}.
The algorithm presented here for designing new collision-free fiber models allowed to simulate the effect of scattered light in  simulations based on a finite-difference time-domain algorithm without superimposed interference signals.
These simulations enabled the understanding of scattering effects due to fiber bundle and crossing configurations as well as the transmission change for tilted fiber configurations \cite{MenzelDissertation,Menzel2020,Menzel2021}.
\par
% 
The current formulation allows reproducing linear optics simulations coherent with the experimental results of \cite{Dohmen2015,Menzel2015,Menzel2016}.
In contrast to \cite{Dohmen2015,Menzel2015,Menzel2016}, whose algorithm is computationally and memory intensive due to the pre-computations for discretized tissue volume, this work proposes using the foundations for more efficient parallel computing within a supercomputer architecture presented in \cite{Lucksch2016}.
The simulation algorithm was redesigned considering computational efficiency and the manufacturing design of the new \ac{LMP3D} microscope. 
Additionally, the linear optics simulation uses the M{\"u}ller-Stokes calculus also to take the polarization effects of filters into account.
\par
%
The \ac{3D-PLI} simulation is divided into two sequential parts: the discrete volume generator and the light matter simulation.
The discrete volume generator discretizes the virtual nerve fiber models onto a Cartesian grid.
Then, the discretized model is used to compute the light matter interaction in the second step.
A parallelization technique with \ac{MPI} allows the volume to be partitioned among different \acp{CPU} or compute nodes.
Due to the tilting approach, the light vector in such a parallelized volume must be able to leave the current volume of a single \ac{CPU} and traverse to the next volume/process.
The computationally intensive algorithms are written in \cpp{} with an additionally user-friendly designed wrapper function in \python{}.
%
% 
% 
\section{Discrete Volume Generator}
\label{sec:dv_generator}
%
\begin{figure}[!t]
\centering
\setlength{\tikzwidth}{0.5\textwidth}
\inputtikz{gfx/simpli/disc_volume}
\caption{Discretized tissue volume with a voxel size \voxelsize. The volume is defined by a \ac{AABB}, which itself is defined by two points $\vec{v}_\mathit{min}$ and $\vec{v}_\mathit{max}$.}
\label{fig:discVol}
\end{figure}
%
The first part of the \ac{3D-PLI} simulation is the calculation of a discretized tissue volume.
It represents a discrete, voxel model of the tissue.
This helps to drastically speed up the light matter interaction of the next step (see \cref{sec:simulation}), at the cost of a large memory requirement.
\par
%
The discretized tissue volume represents a cuboid divided into smaller cubes of equal size, called voxels, \ie{}, 3D pixels (see \cref{fig:discVol}).
Each voxel contains the physical properties of absorption, birefringence and optic axis orientation of the tissue at its current position.
The total volume is bounded by a \ac{VOI} defined by a minimum and maximum value: $\voi = [(x_{\mathit{min}}, y_{\mathit{min}}, z_{\mathit{min}}), (x_{\mathit{max}}, y_{\mathit{max}}, z_{\mathit{max}})]$.
Additionally, the parameter \Voxelsize{} \voxelsize{} is set to a floating point number and defines the edge length of the equilateral voxels.
In case the division is not integer, the number of voxel of the affected axis is rounded up.
%
%
%
\subsection{Nerve fiber layers}
%
As described in \cref{sec:fiberArchitecture}, nerve fibers are axons wrapped by multiple turns of myelin (see \cref{fig:myelinLayer}).
Especially for light wave simulations, the myelin windings are an important feature \cite{MenzelDissertation}.
\par
%
The windings are represented as individual layers (see \cref{fig:fiberLayer}).
Which dramatically simplifies the creation process.
A layer is defined by a factor between $\SI{0}{}$ and $\SI{1}{}$ that scales with the radius of the nerve fiber.
For example, $\SI{0.75}{}$ means that from $0 \leq r < 0.75$ of the radii is interpreted as the first layer (see \cref{fig:fiberLayer}).
\par
%
Each layer requires a number of physical properties in addition to its radius:
%
\begin{itemize}[nosep]
    \item birefringence value: $\dn$
    \item absorption coefficient: $\mu$
    \item optical axsis model: $p=\mathit{parallel}$, $r=\mathit{radial}$, $b=\mathit{background}$
\end{itemize}
%
The properties are specified as a list of tuples within the algorithm (see \cref{alg:fiberbundleprops}).
Subsequently, the discretized volume generator returns the arrays \tissue{}, \opticalaxis{} and \propertylist{} for used in the light matter simulation.
% 
\begin{lstfloat}[!ht]
\lstset{style=python}
\begin{lstlisting}[]
fbs_properties = [[(r, dn, mu, 'p'), (second layer), ...],
                  [(first layer of second bundle), ...],
                  [...]]
\end{lstlisting}
\caption{Definition of the properties of fiber bundles.}
\label{alg:fiberbundleprops}
\end{lstfloat}
% 
% 
% 
\subsection{Discretization of a nerve fiber model}
%
\begin{figure}[!t]
\centering
\setlength{\tikzwidth}{0.3\textwidth}
\subcaptionbox{\label{fig:myelinLayer}Schematic representation of a nerve fiber with axon and myelin sheath}
[\tikzwidth]{\includegraphics[height=\tikzwidth]{gfx/brain/myelin_layers.pdf}\vspace{0mm}}\hfill
\subcaptionbox{\label{fig:fiberLayer}Cross section through a nerve fiber with layered structure defined by $n$ radii}
[\tikzwidth]{\inputtikz{gfx/simpli/fiber_layer}\vspace{-5mm}}\hfill
\subcaptionbox{\label{fig:vectormodel}Cross section of a discretized nerve fiber with resulting optic axis vectors}
[\tikzwidth]{\inputtikz{gfx/simpli/vector_model}\vspace{-5mm}}
\caption{Discretization of nerve fibers with layered structure.}
\label{fig:fiber_discretization}
\end{figure}
%
In order to discretize nerve fiber models, nerve fiber segments are discretized individually.
Since fibers are a chain of consecutive segments, each voxel inside a fiber segment must be labeled as a tissue with the physical properties at the voxel center position $\vec{q}$ (see \cref{fig:fiber_discretization}).
The discretized mesh represents an array, where an element at position $[i,j,k]$ occupies the space from $(i,j,k)$ to $(i+1,j+1,k+1)$ in the unit of $\SI{1}{\voxelsize}$.
To identify all voxels inside a volume, all voxels inside the fiber segments are checked if they are inside the fiber segment.
Therefore, a loop over all voxels is performed.
\par
%
To determine if a voxel is inside the nerve fiber segment,
a computation analogously to the collision between two nerve fiber segments is calculated (see \cref{alg:pseudocodeCollisionDetection}).
The distance vector $\vec{d}$ is used to check whether the voxel is inside the fiber segment, and if so, in which layer of the fiber segment it is located, and to calculate the orientation of the birefringence axis.
\par
%
Two values are stored in the case of a voxel occupied by a fiber segment. 
The first one is an index within an array \name{tissue} which will be used later to retrieve the properties from a list with the same index order.
The second information is the orientation of the optic axis within the current layer, stored in the array \name{optic\_axis}.
The orientation can be parallel to the fiber segment in the case of the macroscopic model or radial in the case of the microscopic model.
A layer can also be marked as \say{background}, allowing the user to specify an area without any birefringence.
\par
%
Two consecutive fiber segments can occupy the same space because they have a common point.
Then, they fill some voxels of the tissue volume simultaneously.
This is an issue because the processed second fiber segment overwrites the values of the first.
This is solved by using an additional array that stores the smallest distance calculated when filling the voxels space.
The values are only overwritten when a new distance is calculated that is smaller than the one already stored.
This also solves the problem that in a radial optic axis model, the optic axis is star-shaped at the end points of each fiber segment inside the fiber.
The first and last point of a fiber, however, are not affected by this.
\par
%
The algorithm for the discretization loop is shown in \cref{alg:fillVolume}.
%
\begin{lstfloat}[!tb]
\lstset{style=python}
\begin{lstlisting}[]
for fiber_segment in fiber_bundle:
    for i,j,k in fiber_segment.aabb().voxels():
        min_dist, min_point = calculate_min_distance((i,j,k), cc)
        if min_dist < cc.radius:
            if min_dist < current_distance[i,j,k]:
                optic_axis[i,j,k,:] = get_axis_orientation(
                                            (i,j,k), min_dist,
                                            min_point)
                tissue[i,j,k] = get_layer_id(min_dist)
                current_distance[i,j,k] = min_dist
\end{lstlisting}
\caption{Pseudocode for filling the discretized volume.}
\label{alg:fillVolume}
\end{lstfloat}
%
%
%
\subsection{\Voxelsize}
%
\begin{figure}[!t]
\centering
% \tikzset{external/export=false}
\setlength{\tikzwidth}{.24\textwidth}
\definecolor{c1}{rgb}{0.25,0.4,0.1}
\definecolor{c2}{rgb}{1.0,0.73,0}
\definecolor{c3}{rgb}{0.98,0.4,0.25}
\definecolor{c4}{rgb}{0.22,0.36,0.59}
%
% \definecolor{c1}{HTML}{440154FF}
% \definecolor{c2}{HTML}{38598CFF}
% \definecolor{c3}{HTML}{1E9B8AFF}
% \definecolor{c4}{HTML}{FDE725FF}
%
\def\xc{0.7}
\def\yc{0.2}
\def\rin{1.5}
\def\rout{3}
%
% 
\newcommand{\fiber}[3]{
	\def\dd{#1}
	\pgfmathsetmacro{\xmin}{int(floor(\xc-\rout))}
	\pgfmathsetmacro{\xmax}{int(ceil(\xc+\rout))}
	\pgfmathsetmacro{\xd}{\xmin+\dd}
	\pgfmathsetmacro{\ymin}{int(floor(\yc-\rout))}
	\pgfmathsetmacro{\ymax}{int(ceil(\yc+\rout))}
	\pgfmathsetmacro{\yd}{\ymin+\dd}
	%
	\pgfmathsetmacro{\rmin}{\rin*\rin*100}
	\pgfmathsetmacro{\rmax}{\rout*\rout*100}
	\foreach \x in {\xmin,\xd,...,\xmax} {
		\foreach \y in {\ymin,\yd,...,\ymax} {
			\pgfmathsetmacro{\d}{int(((\x-\xc+\dd/2)*(\x-\xc+\dd/2)+(\y-\yc+\dd/2)*(\y-\yc+\dd/2))*100)}
			\ifnum\d>\rmin
			\ifnum\d<\rmax
			\path [#3] (\x,\y) rectangle ($ (\x, \y) + (\dd, \dd) $);
			\draw[#2] ($ (\x, \y) + (0, 0) $) -- ($ (\x, \y) + (\dd, 0) $);
			\draw[#2] ($ (\x, \y) + (\dd, 0) $) -- ($ (\x, \y) + (\dd, \dd) $);
			\draw[#2] ($ (\x, \y) + (\dd, \dd) $) -- ($ (\x, \y) + (0, \dd) $);
			\draw[#2] ($ (\x, \y) + (0, \dd) $) -- ($ (\x, \y) + (0, 0) $);
			\fi\fi
		}
	}
%	\foreach \x in {\xmin,\xd,...,\xmax} {
%		\draw[#2] (\x,\ymin) -- (\x,\ymax);
%	}
%	\foreach \y in {\ymin,\yd,...,\ymax} {
%		\draw[#2] (\xmin,\y) -- (\xmax,\y);
%	}
}
%
\subcaptionbox{}[\tikzwidth]{
\resizebox{\tikzwidth}{!}{
\begin{tikzpicture}[]
\path[] (-3.75, -3.5) rectangle (3.75, 3.25);
\begin{scope}[shift={(-\xc,-\yc)}]
\fiber{2}{line width = 0.2mm}{pattern color=c1,pattern=horizontal lines}
\draw[line width = 0.4mm] (\xc,\yc) circle (\rin);
\draw[line width = 0.4mm] (\xc,\yc) circle (\rout);
\end{scope}
\end{tikzpicture}
}}
\hfill
%
\subcaptionbox{}[\tikzwidth]{
\resizebox{\tikzwidth}{!}{
\begin{tikzpicture}[]
\path[] (-3.75, -3.5) rectangle (3.75, 3.25);
\begin{scope}[shift={(-\xc,-\yc)}]
\fiber{1}{line width = 0.1mm}{pattern color=c2,pattern=vertical lines}
\draw[line width = 0.4mm] (\xc,\yc) circle (\rin);
\draw[line width = 0.4mm] (\xc,\yc) circle (\rout);
\end{scope}
\end{tikzpicture}
}}
\hfill
%
\subcaptionbox{}[\tikzwidth]{
\resizebox{\tikzwidth}{!}{
\begin{tikzpicture}[]
\path[] (-3.75, -3.5) rectangle (3.75, 3.25);
\begin{scope}[shift={(-\xc,-\yc)}]
\fiber{0.5}{line width = 0.05mm}{pattern color=c3,pattern=north east lines}
\draw[line width = 0.4mm] (\xc,\yc) circle (\rin);
\draw[line width = 0.4mm] (\xc,\yc) circle (\rout);
\end{scope}
\end{tikzpicture}
}}
\hfill
%
\subcaptionbox{}[\tikzwidth]{
\resizebox{\tikzwidth}{!}{
\begin{tikzpicture}[]
\path[] (-3.75, -3.5) rectangle (3.75, 3.25);
\begin{scope}[shift={(-\xc,-\yc)}]
\fiber{0.25}{line width = 0.025mm}{pattern color=c4,pattern=crosshatch dots}
\draw[line width = 0.4mm] (\xc,\yc) circle (\rin);
\draw[line width = 0.4mm] (\xc,\yc) circle (\rout);
\end{scope}
% \fiber{2}{}{fill, c1, opacity=0.25}
% \fiber{1}{very thin}{fill, c2, opacity=0.25}
% \fiber{0.5}{ultra thin}{fill, c3, opacity=0.25}
% \fiber{0.25}{ultra thin}{fill, c4, opacity=0.25}
%
\end{tikzpicture}
}}
\caption{Discretization error. Cross-section through a single fiber with a myelin layer in the discretized tissue volume. The colored pattern shows the resulting voxels corresponding to the fiber. The smaller the \Voxelsize, the smaller the discretization error.}
\label{fig:vectorfield_disc_error}
\end{figure}
%
The parameter \Voxelsize{} is an essential property of the light matter simulation.
It determines how accurate the volume is represented (see \cref{fig:vectorfield_disc_error}).
Additionally, in the light matter simulation one light ray is cast from each voxel of the bottom plane (see \cref{sec:pathOfLight}). 
Therefore it is also responsible for the sampling of the resulting intensities.
%
%
% 
\subsection{Code optimizations}\label{sec:dvOpti}
%
All arrays are implemented as contiguous c-arrays, accessible externally as \name{NumPy-arrays} without copying the data.
Since these arrays grow with $\mathcal{O}(\frac{1}{\voxelsize}^3)$, the ownership of the data is movable in both the \cpp{} libraries and \python{} code.
The memory order of the arrays is in the $x\text{-}y\text{-}z$ direction, so the largest memory shift is in $x$ and the smallest in $z$.
Thus, when the light moves along the $z$-direction, the information can be read out linearly from the memory, and the acp{CPU} cache prefetcher can be used effectively. 
\par
%
Two methods are used to parallelize the algorithm on the \ac{CPU}.
The first uses \ac{OpenMP} to parallelize the filling of the \ac{AABB} volume of each object.
The second uses \ac{MPI} to allow distribution across multiple \ac{CPU} cores without sharing memory (detailed description in \cref{sec:mpiSim}).
\par
%
Parallelizing the filling process of the voxels of the discretized volume leads to a race condition when multiple threads want to write or read to the same memory address, \ie{}, the same coordinate in the volume.
A solution with a lock would be suboptimal and since many of the voxels do not need to be overwritten, most of the locks would be unnecessary.
To share the work, thread $n$ processes only the memory for the first index $i$ (or volume dimension $x$) if:
%
\begin{align}
\begin{split}
    i \bmod N_{\mathit{Threads}} == \mathit{thread}_{\mathit{id}}
\end{split}
\end{align}
%
\begin{figure}[!t]
\centering
\setlength{\tikzwidth}{0.5\textwidth}
\inputtikz{gfx/simpli/disc_volume_thread}
\caption{Discretization volume parallelization with \ac{OpenMP}. Each thread processes every $n$-th $yz$-section. This ensures both thread safety and a more balanced workload.}
\label{fig:discVolThread}
\end{figure}
% 
Instead of dividing the volume into $n$ subvolumes, each thread loops over every $N_{\mathit{num\_threads}}$ slice of the volume (see \cref{fig:discVolThread}) to distribute the work if the volume is not filled homogeneously with fibers. 
This procedure leads to a thread-safe writable operation.
One disadvantage of this algorithm is that all threads have to check if the \ac{AABB} of all fiber segments is inside the current \ac{VOI}.
% 
% 
%
\section{Light Matter Simulation}
\label{sec:simulation}
%
The light matter simulation algorithm performs the M{\"u}ller-Stokes calculus (see \cref{sec:MuellerStokes}) on the previously calculated discrete volume (see \cref{sec:dv_generator}) for the light rays along their paths.
Since no scattering or refraction effects are considered in this simulation, each light path follows a straight line.
Initially, the light vector is multiplied by the first polarizer of the optical system (see \cref{sec:expSetup}).
The path inside the tissue is discretized into steps.
The interaction between the light ray and the tissue is calculated after each step according to $ \vec{S}' = \prod_i \left(\mat{R}_i \mat{M}_i \mat{R}_i\right) \cdot \vec{S}$ (see \cref{sec:MuellerStokes}).
% 
%
%
\subsection{Light ray path}
\label{sec:pathOfLight}
%
The light matter simulation allows for a tilted light beam.
For this purpose, the \ac{LAP} uses a tilting stage to which the tissue sections are attached (see \cref{fig:tilting_camera_view}).
%
\begin{figure}[!t]
\setlength{\tikzheight}{0.42\textwidth}
\subcaptionbox{camera view}
[.475\textwidth]{\inputtikz{gfx/simpli/tilting_3d_a}}\hfill
\subcaptionbox{perspective view}
[.475\textwidth]{\inputtikz{gfx/simpli/tilting_3d_b}}
\tikzset{external/export=false}
\caption{3D tilting: around $xy$-axis, \raisebox{.25em}{\tikz \draw[red,thick](0,0)--(0.25,0);} top, \raisebox{.25em}{\tikz \draw[green,thick](0,0)--(0.25,0);} middle, \raisebox{.25em}{\tikz \draw[blue,thick](0,0)--(0.25,0);} bottom, \raisebox{.25em}{\tikz \draw[dash pattern=on 1.25pt off 1.25pt,thick](0,0)--(0.25,0);} original, \raisebox{.25em}{\tikz \draw[gray](0,0)--(0.25,0);} axis of rotation.}
\label{fig:tilting_camera_view}
\end{figure}
% 
The \ac{LMP3D}, on the other hand, has a tilted light beam.
This is achieved by a conical light path, from which an aperture is then used to sample the desired light direction \cite{Wiese:887678}.
Both methods can be mathematically represented by the same procedure.
\par
%
An additional effect changing the light path is the refraction at the tissue-air boundary, which is described by Snell's law for isotropic media (see \cref{eq:Snellius}).
Since this only adds a parallel shift, simulation is only necessary when the effects of the resampling process and image registration is investigated.
\par
%
\begin{figure}[!t]
\setlength{\tikzwidth}{0.45\textwidth}
\subcaptionbox{normal}[.475\textwidth]{
\def\tilt{0}
\def\nindex{2.25}
\inputtikz{gfx/simpli/tilting_a}}\hfill
\subcaptionbox{tilted}[.475\textwidth]{
\inputtikz{gfx/simpli/tilting_b}}
\caption{Light ray path for a normal (a) and a tilted (b) case. In the tilted case, the light beam $\vec{l}_1$ is tilted within the tissue and thus experiences an optical shift $\Delta$.}
\label{fig:tilted_side_view}
\end{figure}
%
The initial position of the light beam is calculated by traversing the light path backwards (see \cref{fig:tilted_side_view}).
This has the advantage that each index inside the \ac{CCD} always receives exactly one light beam.

From the \ac{CCD} array, the light path can be shifted back to the top plane of the tissue $\mathfrak{S}_{top}$.
Subsequently, the light beam $\vec{l}_1$ is traced back through the tissue to the bottom plane $\mathfrak{S}_{bottom}$.
The point on the lower tissue plane corresponds to the initial position of the light beam.
This light path change results in a shift $\delta$ along the same direction of the tilting.
In the light matter simulation only light rays are considered, which will go at least partially through the volume.
All remaining \ac{CCD} elements are left with a \code{NaN} value.
\par
%
The tilting of the tissue leads to a distortion of the image (see \cref{fig:tilting_camera_view}).
This distortion can be described by an affine transformation (see \cref{fig::affine_transformation}):
%
\begin{figure}[!t]
\centering
\pgfmathsetmacro{\size}{10}
% 
\subcaptionbox{}[.225\textwidth]{
\resizebox{.225\textwidth}{!}{
\begin{tikzpicture}[]
\pgfmathsetmacro{\dx}{2.25}
\pgfmathsetmacro{\dy}{1.5}
\draw[help lines] (0,0) grid (\size,\size);
\foreach \i in {0,...,\size}{
	\draw [very thick] (\i+\dx, 0+\dy) -- (\i+\dx, \size+\dy);
	\draw [very thick] (0+\dx, \i+\dy) -- (\size+\dx, \i+\dy);
}
\end{tikzpicture}
}}
% 
\subcaptionbox{}[.225\textwidth]{
\resizebox{.225\textwidth}{!}{
\begin{tikzpicture}[]
\pgfmathsetmacro{\sx}{1.5}
\pgfmathsetmacro{\sy}{0.75}
\draw[help lines] (0,0) grid (\size,\size);
\foreach \i in {0,...,\size}{
	\draw [very thick] ({\i*\sx}, 0) -- ({\i*\sx}, {\size*\sy});
	\draw [very thick] (0, {\i*\sy}) -- ({\size*\sx}, {\i*\sy});
}
\end{tikzpicture}
}}
%
\subcaptionbox{}[.225\textwidth]{
\resizebox{.225\textwidth}{!}{
\begin{tikzpicture}[]
\draw[help lines] (0,0) grid (\size,\size);
\begin{scope}[rotate around={30:(0.5*\size,0.5*\size)}]
\foreach \i in {0,...,\size}{
	\draw [very thick] (\i, 0) -- (\i, \size);
	\draw [very thick] (0, \i) -- (\size, \i);
}
\end{scope}
\end{tikzpicture}
}}
% 
\subcaptionbox{}[.225\textwidth]{
\resizebox{.225\textwidth}{!}{
\begin{tikzpicture}[]
\pgfmathsetmacro{\cx}{0.5}
\pgfmathsetmacro{\cy}{0}
\draw[help lines] (0,0) grid (\size,\size);
\foreach \i in {0,...,\size}{
	\draw [very thick] (\i+0*\cx, 0+\i*\cy) -- (\i+\size*\cx, \size+\i*\cy);
	\draw [very thick] (0+\i*\cx, \i+0*\cy) -- (\size+\i*\cx, \i+\size*\cy);
}
\end{tikzpicture}
}}
\caption{Examples of affine transformations.}
\label{fig::affine_transformation}
\end{figure}
%
\begin{align}
f(\vec{x}) = \mat{A} \cdot \vec{x} + \vec{t}
\end{align}
where $\vec{x}$ is the coordinate input, $\mat{A}$ and $\vec{t}$ are the transformation values, and $f(\vec{x})$ is the transformed coordinate.
\par
%
The light matter simulation with the light paths sampling as described above takes this distorted view into account and removes it.
The simulation can also sample the light rays so that distortion occurs.
Thus, the resulting images have to be registered onto each other by an affine transformation which is also available in the algorithm.
% 
% 
% 
\subsection{Tissue voxel interpolation}
%
If the \Stepsize{} of the light ray is not equal to the \Voxelsize{} or if the light path is tilted, the light ray position after a step no longer matches the center of the voxel.
In order to correctly recreate the physical properties, interpolation is required (see \cref{fig:vectorfield_disc}).
%
\begin{figure}[!t]
\centering
\setlength{\tikzwidth}{0.475\textwidth}
% \tikzset{external/force remake=true}
\subcaptionbox{\label{fig:triInterp}Trilinear interpolation}
[\tikzwidth]{
\hfill\inputtikz{gfx/simpli/trilinear_interpolation}\hfill}\hfill
\subcaptionbox{\label{fig:sphInterp}Spherical interpolation}
[\tikzwidth]{
\inputtikz{gfx/simpli/vector_interpolation}}
\caption{Interpolation techniques: Trilinear interpolation can be represented as an axial step interpolation. The difference between linear and spherical interpolation is that linear interpolation has a constant distance $s$ between each point, while spherical interpolation has a constant angle $\varphi$ between two steps.}
\label{fig:vectorfield_disc}
\end{figure}
%
Currently, three interpolation methods are implemented: \name{nearest neighbor}, \name{linear interpolation} and \name{spherical interpolation}.
The voxels considered for interpolation are the adjacent eight neighboring voxels, \ie{}, array indices $(\floor{x\pm0.5},\floor{y\pm0.5},\floor{y\pm0.5})$.
The \name{nearest neighbor} and \name{linear interpolation} are still present from the development phase.
However, they should not be used because they are prone to errors, given that the data represent orientations in the space.
Thus, the use of \name{spherical interpolation} considers the relationship existing in the data components, and it is highly recommended.
%
%
%
\subsection{Simulation of light matter interaction}\label{sec:simLightTissue}
%
After the initial light positions are calculated, each light beam is traversed within the tissue.
To calculate the light-tissue interaction, the change of the Stokes vector is calculated after each light beam step.
The step size is a user defined value with a default value of the \Voxelsize{} $\voxelsize$.
To simulate the light beam hitting the volume's boundary, the light beam is multiplied by the matrix of the remaining optical elements of the microscope and the intensity is stored in the \ac{CCD}-array.
The pseudocode is shown in \cref{alg:simulationLoop}.
%
\begin{lstfloat}[!tb]
\lstset{style=python}
\begin{lstlisting}[]
light_beams = calculate_light_starting_positions()

for light in light_beams:
    light = optical_elements_start * light
    while light.pos in volume:
        properties = get_properties(light.pos)
        light.intensity *= exp(-step_size * properties.absorbtion)
        light = matrix(properties) * light
        light.pos += step_size
   
    light = optical_elements_end * light
    ccd_array[light.ccd_pos] = light.intensity
\end{lstlisting}
\caption{Loop over the light vectors for the light-tissue interaction. Their intensity value is stored inside the \ac{CCD} array.}
\label{alg:simulationLoop}
\end{lstfloat}
%
To resample the array to the final size of the \ac{CCD} sensor and apply the noise model, there are \python{} functions outside the simulation software available.
% 
% 
%
\subsection{Optical system and signal analysis}
\label{sec:ccdOptic}
%
The image sensor, as described in \cref{sec:expSetup}, is a \ac{CCD}-sensor.
The resampling and noise modeling, are implemented according to \cref{sec:opticalResolution}.
These calculations are performed on the \python{} side of the algorithm (\code{fastpli.simulation.Simpli}) and can be executed with the \code{multiprocessing} library of \python{} to use multiple \ac{CPU} cores.
Therefore when using \ac{MPI} (see \cref{sec:mpiSim}), the intensity image must be reduced to a single process.
Since the resulting image is only 2D, there is no need to further speed up the process with \ac{MPI}.
\par
%
To analyze the signal, the same algorithms are implemented as in the \ac{3D-PLI} routine pipeline (see \cref{sec::intSignal,sec::InclAnalysis}).
These include the modalities analysis transmittance, direction and retardation and the tilt analysis performed by \ac{ROFL}.
The tilt analysis is a fork of the existing routine analysis developed in python.
To parallelize the execution, the native \code{multiprocessing} library of \python{} can be used.
% 
% 
%
\section{Speedup Strategies}\label{sec:sim:opt}
%
\subsection{Code design}
% 
The following key optimizations are used in the pipeline:
\begin{itemize}
    \item The order of the tissues stored in memory is along the z-axis (as described in \cref{sec:dvOpti}) so that the light ray \textit{traverses} along the aligned memory.
    \item The for-loop for the light beams (see \cref{alg:simulationLoop}) is paralellized with \ac{OpenMP}.
    These threads are completely separated and there are no race conditions.
    \item Vector and matrix calculations are optimized for their small sizes by the compiler with the help of tools like \name{Compiler Explorer}\footnote{\url{https://godbolt.org/}} and \name{C++ Insight}\footnote{\url{https://cppinsights.io/}}.
\end{itemize}
%
%
% 
\subsection{MPI parallelization}\label{sec:mpiSim}
%
The algorithms for discrete volume generation and light matter simulation can additionally use a parallelization technique.
The computation must be split among multiple physical \acp{CPU} and nodes for large volumes larger than the local memory size.
For this purpose, \acreset{MPI} \ac{MPI} is used.
A method is implemented to automatically partition the volume along the $x$-axis and the $y$-axis into blocks with the minimal surface area along both axis (see \cref{fig:com_halo}).
\par
%
\begin{figure}[!t]
    \centering
    \setlength{\tikzwidth}{0.85\textwidth}
    \inputtikz{gfx/simpli/com_halo}
    \caption{ This example uses six \ac{MPI} ranks to split the entire volume into six sub-volumes. To calculate the value at a given point a eight-neighborhood is necessary. Therefore a \name{halo} area (coloured voxels) with the same information shared by neighboring \ac{MPI} process is necessary.}
    \label{fig:com_halo}
\end{figure}
% 
Each \ac{CPU} can perform the discrete volume generation without the knowledge of each other.
The exception is the light matter simulation for tilted light beams.
Here, when a light ray leaves the local volume, it must be transmitted to the adjacent volume.
\ac{MPI} provides several methods to send information to another \name{rank}.
Since the sub-volumes are split along a Cartesian grid, the Cartesian implementations of \ac{MPI} are used (\eg{} \code{MPI\tu Cart\tu create}).
A problem is the calculation at the edge of the volume, since the light beam needs the information of the surrounding eight voxels for the interpolation.
If this voxel information were to be transmitted to the neighboring \name{rank}, this would mean a large amount of communication, which is much slower compared to the local \ac{CPU} or \ac{RAM} instructions.
To solve this communication problem, the algorithm uses a \name{halo}.
This is a commonly used concept where the boundaries (in this case the volume) are increased by a certain size so that the same information about the shared regions is available everywhere (see \cref{fig:com_halo}).
Such approach reduces the number of necessary communications between different ranks (see \cref{fig:com_halo}).
\par
% 
To further speed up the communication process, all outgoing light beams are first stored locally in a communication buffer, and only after all local light beams have been processed the buffer is passed to the neighbors.
This ensures minimal communication overhead.
The main loop of the light beam algorithm is then restarted on all \ac{MPI} \name{ranks} for the communicated light beams.
This is repeated until no more communication is required.
\par
%
Each \ac{MPI} process can additionally use \ac{OpenMP} to allow multiple cores to benefit from shared memory.
