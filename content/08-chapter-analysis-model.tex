\newpage\null\thispagestyle{empty}\newpage
\clearpage{\thispagestyle{empty}\cleardoublepage}
\part{Software Application and Evaluation}
% \acbarrier
\parttoc
%
%
%
\setcounter{chapter}{7}
\chapter{Dense Nerve Fiber Modeling}
\label{cha:model_analysis}
%
\section{Introduction}
%
In the previous chapter \cref{chap:sof:modeling}, the module \pymodule{fastpli.model} was described for creating non-colliding nerve fiber models.
An important question is how the software parameters affect the resulting models.
\par
%
Three conditions are required for the software to be applicable.
First, it must be possible to create a dense volume.
This, of course, depends on the original fiber configuration.
For example, the density is lower for configurations such as crossings.
The second requirement is the runtime.
The lower the runtime, the more objects and larger volumes the user can create in the same time.
The last requirement is that the original fiber orientations remain intact.
Because of the motion phase in the collision solver algorithm, the orientations will change.
The question is how much they change and whether the results still meet the user's expectations and anatomical constraints.
\par
%
To study the behavior of the software, a general data set with simplistic model parameters is needed.
These parameters should be able to define a fiber configuration in a volume without having to describe each individual fiber.
This description could then be used for both the study of the solver parameters and the subsequent study of the \ac{3D-PLI} simulation.
\par
%
The first part of this chapter is the presentation of the parameterization used to describe the fiber populations.
This is followed by an investigation of the collision solver parameters on the behavior of the resulting collision-free nerve fiber models.
On this basis, a set of software parameters is determined to create a library of nerve fiber models for \ac{3D-PLI} simulations.
In the last part, the orientation distribution of the generated models is investigated.
%
%
%
\section{Designing Fiber Populations}
%
For the \ac{3D-PLI} simulation, a volume of size $\SI{65}{\micro\meter} \times \SI{65}{\micro\meter} \times \SI{60}{\micro\meter}$ is required (see \cref{cha:simulation_analysis}). \footnote{$\SI{65}{\micro\meter}$ is divisible by $\SI{1.3}{\micro\meter}$, the pixel size of the microscope under study}
In this work, up to two fiber populations are studied.
A fiber population is a fiber bundle with a specific orientation, density, and radius distribution.
These parameters of a fiber population have the advantage of being anatomically motivated and describe a nerve fiber volume based on statistical properties.
This is for example also used in \ac{dMRI} simulations \cite{Ginsburger2018,Ginsburger2019,ginsburgerDis2019}.
The number of fiber populations is limited to two to reduce the degrees of freedom within the models.
%
%
%
\subsection{Orientation and proportion}\label{sec:modelParamet}
%
For the study of a single fiber population, only one model is needed, since the model can be rotated to any 3D orientation.
For two fiber populations, the crossing angle $\modelOmega$ between the two fiber population orientations is relevant (see \cref{fig:modelinit}).
By introducing a rotation $\modelRot$ along the first fiber population, the orientation of the second fiber population can be changed (see \cref{fig:modelrota}).
If the first fiber population is tilted by an angle $\modelInc$, an arbitrary fiber configuration with two fiber populations can be described (see \cref{fig:modelrotb}).
\par
%
%
\begin{figure}[!t]
    \centering
    \setlength{\tikzwidth}{0.40\textwidth}
    \subcaptionbox{\label{fig:modelinit}
    Initial orientation for two fiber populations $\popa$ and $\popb$.
    The angle between the two populations is $\modelOmega$.
    The remaining degrees of freedom are accounted for by rotating the entire volume.}
    [\textwidth]{\inputtikz{gfx/model/sphere_models}}
    \\
    %
    \subcaptionbox{\label{fig:modelrota}
    The first fiber population is stationary.
    The second fiber population is described by the opening angle $\modelOmega$ and a rotation $\modelRot$ around the first fiber population.}
    [.475\textwidth]{\inputtikz{gfx/model/sphere_models_a}}\hfill
    \subcaptionbox{\label{fig:modelrotb}
    The inclination of the first fiber population is modified by $\modelInc$.}
    [.475\textwidth]{\inputtikz{gfx/model/sphere_models_b}}
    \caption[]{Parameterization of two mutually relatively oriented fiber populations for the creation of a library.}
    \label{fig:twomodelpopdesign}
    \end{figure}
%
A further parameter is the first population proportion:
\begin{align}
    \modelPsi=\frac{N_0}{N_0+N_1}
\end{align}
This can range between $\SI{0}{}$ and $\SI{1}{}$.
Since only dense \ac{WM} phantoms are generated in this study, a lower overall density of $\SI{1}{}$ is not investigated.
\par
%
To reduce the number of models required, the parameters are sampled.
Ideally, the surface of a sphere is equidistantly sampled.
However, this is impossible.
There are fairly good approximations, but one would still have to create a model for each sampled point.
For simplicity, the intersection angle $\modelOmega$ is sampled instead.
By rotating the model with $\modelRot$ along the first fiber population axis for all sampled crossing angles $\modelOmega$, one can sample the entire sphere.
\par
%
Therefore, the following parameter samples are chosen:
%
\begin{align}
    \begin{split}
        \modelOmega &= \{\SI{10}{\degree}, \SI{20}{\degree}, ..., \SI{90}{\degree}\}\\
        \modelPsi &= \{\SI{10}{\percent}, \SI{20}{\percent}, ..., \SI{90}{\percent}\}
    \end{split}
\end{align}
%
In the case of $\modelOmega = \SI{0}{\degree}, \modelPsi = 0$ or $\modelPsi = 1$, no second unique orientation exists, so the single fiber population model can be applied.
This results in a total number of $\SI{82}{}$ models that are required.
%
%
%
\subsection{Fiber placement}
%
To design the individual fiber configurations for each fiber population, the methods described in \cref{sec:sandbox} are used, \ie{}, the generation of fiber bundles with seed points.
Seed points on a 2D plane are generated with a uniform distribution:
\begin{align}
p_i = (\mathrm{Uniform}(-\frac{1}{2}\mathit{L}, \, \frac{1}{2}\mathit{L}), \mathrm{Uniform}(-\frac{1}{2}\mathit{L}, \, \frac{1}{2}\mathit{L}))
\end{align}
Since the simulation volumes are cubic and the model is rotated, a spherical boundary with diameter $d_{\mathit{sphere}}=\sqrt{3} \cdot \mathit{L}$ is picked, where $\mathit{L}=\SI{65}{\micro\meter}$ is the length of the largest edge length of the cube.
To ensure that objects exist outside the simulation volume that can build up pressure on the interior, $\mathit{L}$ is increased by two fiber diameters.
This also prevents edge effects in the simulation.
The number of seed points is set to a value of
\begin{align}
N_{0,\mathit{seeds}} &= \round \left(\modelPsi \frac{A_{\Box}}{\pi \cdot \fiberRadiusMean}\right)\\
N_{1,\mathit{seeds}} &= \round \left((1-\modelPsi) \frac{A_{\Box}}{\pi \cdot \fiberRadiusMean}\right)
\end{align}
with $A_{\Box}=L^2$ as the area of the seed point plane.
Thus, also with respect to the fiber radius distribution introduced next, there are enough fibers to fill the 2D plane.
\par
%
\begin{figure}[!t]
\centering
% \tikzset{external/export=false}
\begin{tikzpicture}[scale=1, trim axis left, trim axis right]
        \begin{axis}[%
            height=0.46\textwidth, width=0.75\textwidth,
            enlargelimits=false, xlabel={$x$},
            ylabel={$f(x,\mu,\sigma)$},
            title=${f(x,\mu,\sigma)=\frac{1}{\sigma x \sqrt{2\pi}}\exp\left(-\frac{(\ln(x)-\mu)^2}{2 \sigma^2}\right)}$,
            legend style={at={(1,1)},anchor=north east},
            legend cell align={left},
        ]
        \pgfmathsetmacro{\muValue}{0}
        \pgfmathsetmacro{\sigmaValue}{0.05}
        \addplot[BLUE,thick,domain=0.6:1.4, samples=2*42, smooth]{1/(\sigmaValue*x*sqrt(2*pi))*exp(-(ln(x)-\muValue)^2/(2*\sigmaValue^2))}; \addlegendentry{$\mu=0.0,\sigma=0.05$}
        \pgfmathsetmacro{\muValue}{0}
        \pgfmathsetmacro{\sigmaValue}{0.1}
        \addplot[GREEN,thick,domain=0.6:1.4, samples=2*42, smooth]{1/(\sigmaValue*x*sqrt(2*pi))*exp(-(ln(x)-\muValue)^2/(2*\sigmaValue^2))}; \addlegendentry{$\mu=0.0,\sigma=0.1$}
    \end{axis}
\end{tikzpicture}
\caption[]{Probability density function of a multiplicative \name{log normal} distribution.
$\sigma=0.1$ is used for the initial variation of each fiber radius.
$\sigma=0.05$ for the variation along the fiber for each fiber point.}
\label{fig:logNormal}
\end{figure}
%
These seed points are then used to place straight, parallel fibers within a cubic volume with edge length $L$.
To add a random distribution of fiber radii, the target mean fiber radius $\fiberRadiusMean$ is multiplied by a random value of the \name{LogNormal} distribution (see \cref{fig:logNormal}) to ensure that the mean value is preserved:
\begin{align}
\fiberRadius = \fiberRadiusMean \cdot \mathrm{sample}\left(\mathrm{LogNormal}(\mu=0, \, \sigma=0.1)\right)
\end{align}
%
To achieve a random distribution of positions and radii along the fibers, they must initially be divided into fiber segments.
The function \code{Solver.apply\_boundaries} is used for exactly such a case.
It applies the fiber segment length $\segLength$ to the fiber configuration, which in this case divides the fiber along its trajectory into fiber segments of equal length (except for the last segment).
The fiber points thus generated are then randomly shifted in all three dimensions with a normal distribution.
The standard deviation of the normal distribution is set to a factor of the average fiber radii \fiberRadiusMean{}.
In addition, the radius is changed with a random multiplicative factor from a \name{LogNormal} distribution:
%
\begin{align}
\begin{split}
p_i &= p_i + \mathrm{Normal}(\mu=0,\sigma=0.05 \cdot \fiberRadiusMean)\\
r_i &= r_i \cdot \mathrm{LogNormal}(\mu=0,\sigma=0.05)
\end{split}
\end{align}
%
%
%
\section{Software Parameters Characterization}\label{sec:modelSetup}
%
The shape control mechanisms used in the algorithm of the collision solver \code{fastpli.model.solver} (see \cref{chap:sof:modeling}) must be characterized.
These are the mean segment length \segLength{} and the minimum allowed radius of curvature \segRadius{}.
It must be ensured that a set of parameters is identified ensuring that the configurations entered by the user remain the same \ie{}, the orientation distributions remain intact with respect to the fact that the fiber segments must move.
In addition, the achievable fiber density for this parameter set should remain high.
Finally, it must also be ensured that a parameter set can be chosen that has an acceptable runtime.
The runtime should be minimized if possible, since it is usually a limited resource.
\par
%
To investigate the behavior of \segLength{} and \segRadius{}, the two relative factor variables \segLengthFactor{} and \segRadiusFactor{} are defined:
\begin{align}
    \begin{split}
        \segLength &= \segLengthFactor \cdot \fiberRadiusMean\\
        \segRadius &= \segRadiusFactor \cdot \fiberRadiusMean
    \end{split}
\end{align}
%
This allows the investigation of the model characterization independent of the fiber radius.
The parameter ranges selected to study the effects are listed in \cref{tab:parameterSetupChar}.
%
\begin{table}[!b]
\sisetup{parse-numbers=false,open-bracket={\{}, close-bracket={\}}, list-final-separator={,},list-pair-separator={,}}%
\centering
\pgfplotstabletypeset[%
    thesisTableStyle,
    column type=lcl,
    columns/name/.style={string type},
    columns/variable/.style={string type},
    columns/values/.style={string type},
    every head row/.style={before row=\toprule,after row=\midrule},
    every last row/.style={after row=\bottomrule},
    col sep=&,
    row sep=\\,
    % string type,
]
{name & variable & values\\
mean fiber radius & $\fiberRadiusMean$ & $\SIlist{0.5;1;2;5;10}{\micro\meter}$\\
mean segment length factor & $\segLengthFactor$ & $\numlist{1;2;4;8}$\\
min segment bending radius factor & $\segRadiusFactor$ & $\numlist{1;2;4;8}$\\
fiber bundle fraction/crossing & $\modelPsi$/$\modelOmega$ & $\SI{1.0}{}/\SI{0}{\degree}$, $\SI{0.5}{}/\SI{90}{\degree}$\\
}
\caption[]{Parameter sets for the characterization of the software parameters $\segLengthFactor$ and $\segRadiusFactor$.}
\label{tab:parameterSetupChar}
\end{table}
%
To be able to check the results statistically, the generation of the models with different seed points is repeated  $n_{\mathit{repeat}} = \SI{24}{}$ times. \footnote{The chosen \ac{CPU} architecture has $\SI{24}{\cores}$.}
For characterization, the pairs $\modelPsi/\modelOmega$ with $\pfbs = \SI{1.0}{}/\SI{0}{\degree}$ and $\cfbs=\SI{0.5}{}/\SI{90}{\degree}$ are considered since they are edge cases.
The number of steps is limited to $n_{\mathit{max}}=\SI{100000}{}$ to constrain the computation.\footnote{In hindsight, this should have been limited by the runtime.}
A cubic volume of $\SI{60}{\micro\meter} \times \SI{60}{\micro\meter} \times \SI{60}{\micro\meter}$ is picked.
Every $\SI{50}{}$ steps, the fiber model is cut into a $\SI{60}{\micro\meter}+4 \cdot \fiberRadiusMean$ cube to delete unnecessary fibers and reduce the number of objects and therefore the runtime.
To measure the volume fraction, the discretized volume from the simulation is generated by the simulation module.
There, the individual label IDs are counted and divided by the total number of voxels to calculate the volume fraction.
To examine the final orientation of the model, the volume is reduced to $\SI{60}{\micro\meter}$.
The \ac{CPU} architecture used to generate the models is an
Intel(R) Xeon(R) CPU E5-4657L v2 @ $\SI{2.4}{\giga\hertz}$, L1d cache: $\SI{1.5}{\mega\byte}$, L1i cache: $\SI{1.5}{\mega\byte}$, L2 cache: $\SI{12}{\mega\byte}$, L3 cache: $\SI{120}{\mega\byte}$.
%
%
%
\subsection{Results} \label{sec:solverParameterResults}
%
\paragraph{Development over time}
%
\begin{figure}[p]
\centering
Single fiber population \pfbs{}\\[0em]
\includegraphics[page=1]{gfx/rc1/mpara/pre_stats_time_evolve_r05_output_parameter_statistic_rc1.pdf}
\caption[]{Time evolution of the model building process of parallel fiber populations. Error bars indicate $\SI{25}{\percent}$ and $\SI{75}{\percent}$ quantiles.}
\label{fig:timeDevelopmentNone}
\end{figure}
%
\begin{figure}[p]
\centering
Crossing fiber population \cfbs{}\\[0ex]
\includegraphics[page=2]{gfx/rc1/mpara/pre_stats_time_evolve_r05_output_parameter_statistic_rc1.pdf}
\caption[]{Time evolution of the model building process of crossing fiber populations. Error bars indicate $\SI{25}{\percent}$ and $\SI{75}{\percent}$ quantiles.}
\label{fig:timeDevelopmentCross}
\end{figure}
%
The results of the time measurement for the parameter set \cref{tab:parameterSetup} are shown in \cref{fig:timeDevelopmentNone} for the single fiber population \pfbs{} and in \cref{fig:timeDevelopmentCross} for the crossing fiber population \cfbs{}.
They include the number of steps $\# steps$, the number of colliding fiber segments $\# objcol$, the overlap fraction of colliding fiber segments $\# overlapfrac$, the number of objects $\# obj$, and the current step time $\Delta t$.
The overlap fraction of colliding fiber segments is defined as the average of the minimum distance between two colliding fiber segments divided by the sum of their radii.
The results of the additional fiber radii are available in \cref{app:pste1,app:pste2,app:pste3,app:pste4,app:pste5}.
\par
%
% SINGLE
The single fiber populations \pfbs{} show a strong linear correlation between the runtime and the number of steps for all parameters.
The number of steps increases significantly with decreasing \segLengthFactor{}.
A change in the running time when changing the fiber bending radius \segRadiusFactor{} is not visible in the logarithmic plot.
The total number of steps increases slightly with an increase in the fiber segment length factor \segLengthFactor{}.
The total runtime increases significantly with a decrease in \segLengthFactor{}.
All models of all parameter sets were able to solve the collisions in the maximum number of steps.
\par
%
The number of colliding fiber segments in the second plot of \cref{fig:timeDevelopmentNone} shows a strongly increasing decrease for all parameters with increasing runtime.
The total number increases with decreasing \segLengthFactor{}.
The \segRadiusFactor{} also has no significant effect here.
\par
%
The overlap fraction for all parameters is about $\SI{5}{\percent}$ at the beginning and decreases with time to about $\SI{1}{\percent}$.
The decrease is approximately linear after the initial phase.
At the end of the collision solver process, the variance increases significantly.
The curve behavior between the different \segLengthFactor{} is about the same, except of course for an offset in the total runtime.
A difference when changing the \segRadiusFactor{} is again not visible.
\par
%
The number of objects decreases slightly over time.
A significant increase can be seen by a decreasing value of \segLengthFactor{}.
The \segRadiusFactor{} is negligible except in the case of $\segLengthFactor = \SI{1}{}$ where the number of objects is increased.
\par
%
The last diagram shows the average time $\Delta t$ needed for a single step.
The relative length factor \segLengthFactor{} has the biggest influence on the results.
A change with \segRadiusFactor{} is almost not visible.
The step time slowly decreases to about half the time required for the first step.
The variance increases as the step size increases and can reach about half an order of magnitude in the case of \segLengthFactor{}.
\par
%
The total runtime, \ie{}, the last data point in the graph, increases with \segLengthFactor{}.
The other parameters seem to have almost no effect.
The runtime ranges from $\SI{1000}{\second}$ to $\SI{10000}{\second}$.
\par
%
% CROSSING
The crossing fiber population \cfbs{} in \cref{fig:timeDevelopmentCross} shows an equally linear behavior for all parameter sets for the number of steps as in the case of the single fiber population.
The total runtime is significantly increased compared to the models with a single fiber population.
Further the runtime increases with a decrease in \segLengthFactor{} and decreases with a decrease in \segRadiusFactor{}.
In the case of $\segLengthFactor = \SI{1}{}$ and $\segLengthFactor = \SI{2}{}$, the maximum number of steps is reached.
For $\segLengthFactor = \SI{1}{}$ this has happened after $\SI{24}{\hour}$.
\par
%
The number of colliding fiber segments shows the same decreasing behavior in the case of the smallest \segRadiusFactor{} as in the case of the single fiber population.
The key difference is that this behavior changes significantly with an increase in the minimum fiber bending radius factor \segRadiusFactor{}.
The curves appear to split at a critical number of steps and progressively drop linearly towards higher \segRadiusFactor{} and smaller \segLengthFactor{}.
\par
%
The overlap fraction shows a significant difference for the smaller fiber segment length factor \segLengthFactor{} compared to the single fiber population \pfbs{} case.
For high \segLengthFactor{} values, the curve almost follows the same path as in the single fiber population case, ending in the collision free state.
However, for lower values and depending on the fiber segment radius \segRadius{}, the curve continues at about $\SI{1}{\percent}$ overlap.
In the case of $\segLengthFactor = \SI{1}{}$ and $\segRadiusFactor = \SI{8}{}$, it starts to grow slightly again at the end of its lifetime.
\par
%
The number of objects shows the same behavior as in the case of the single fiber population.
There is a slight decrease in the number and a significant increase with a decreasing value of the fiber segment length factor \segLengthFactor{}.
The splitting of the curves for the \segRadiusFactor{} is also visible, but not as prominent as in the case of the number of colliding objects.
In the case of $\segLengthFactor = \SI{1}{}$ and $\segRadiusFactor = \SI{8}{}$, an increase in the number of objects is observed in the last phase of the collision solver algorithm.
\par
%
The time per step decreases for all parameters over the number of steps until it reaches a constant value for each \segLengthFactor{}.
The \segRadiusFactor{} does not seem to play a significant role.
For $\segLengthFactor{} = \SI{8}{}$, the variance increases again at the end of the runtime, as in the case of the single fiber population, but not as much.
In the case of $\segLengthFactor{} = \SI{1}{}$ and $\segRadiusFactor = \SI{8}{}$, one can observe an increase in the step time at the end of the runtime.
\par
%
The total runtime is influenced by both the fiber segment length factor \segLengthFactor{} and the fiber segment radius factor \segRadiusFactor{}.
The splitting behavior visible in the graph of the number of colliding objects shows the significant difference in runtime for the \segRadiusFactor{}.
Increasing \segRadiusFactor{} results in an increase in runtime for this parameter by up to an order of magnitude.
The differences between \segLengthFactor{} are comparable to the single fiber case.
\par
%
The other fiber radii show very similar behavior.
The strongest difference is a very significant decrease in runtime with an increase in \fiberRadiusMean{} (see \cref{app:pste1,app:pste2,app:pste3,app:pste4,app:pste5}).
%
%
%
\paragraph{Volume fraction}
%
\begin{figure}[!t]
\centering
\includegraphics{gfx/rc1/mpara/pre_stats_box_plot_volume_05_output_parameter_statistic_rc1.pdf}
\caption[]{Volume fractions $V_f/V_0$ for parallel \pfbs{} and crossing \cfbs{} fiber populations of different relative fiber segment lengths $\segLengthFactor$ and relative fiber bending radii $\segRadiusFactor$.}
\label{fig:psbp1}
\end{figure}
\Cref{fig:psbp1} shows the resulting volume fraction $V_f/V_0$ for the parameter series.
\par
%
% single fiber
The populations with a single fiber orientation \pfbs{} have volume fractions greater than $\SI{0.74}{}$.
For constant $\segLengthFactor$, there is no significant change with respect to the fiber bending radius factor $\segRadiusFactor$.
In the case of $\segLengthFactor = \SI{1}{}$, there is a small decrease in volume fraction with increasing bend radius.
There is a significant small decrease in the volume fraction when the fiber segment length factor $\segLengthFactor$ is increased.
\par
%
% crossing fiber
In contrast, the volume fraction of the crossing fiber population \cfbs{} is significantly reduced.
For $\segLengthFactor=\SI{1}{}$ and $\segLengthFactor=\SI{2}{}$, the volume fraction decreases sharply between $\segRadiusFactor = \SI{2}{}$ and $\segRadiusFactor = \SI{4}{}$.
The values become more similar for $\segLengthFactor=\SI{2}{}$.
For $\segLengthFactor=\SI{8}{}$ the change with $\segRadiusFactor$ disappears, and the volume fraction reaches only a value around $\SI{0.57}{}$.
\par
%
% radius
For a mean fiber radius $\fiberRadiusMean = \SIlist{1;2}{\micro\meter}$ the behavior is very similar (see \cref{app:appModelVolumeBoxPlot}).
The variance increases and the median decreases significantly for $\fiberRadiusMean \geq \SI{5}{\micro\meter}$.
The effect of the parameter is not changed otherwise.
For mean fiber radii $\fiberRadiusMean \geq \SI{5}{\micro\meter}$, the volume fraction does not decrease significantly.
%
%
%
\subsection{Discussion}
%
For parallel fibers, the collision solving process is always faster than for crossing bundles.
This is visible in the plot of the overlap fraction \cref{fig:timeDevelopmentNone,fig:timeDevelopmentCross}.
The overlap is significantly higher for the crossing fibers than for the single fibers.
Not only does this mean that the fiber segments that need to be moved out of the way are longer, but also that in higher octree levels the segments cannot be separated anymore and therefore exists in multiple leafs.
As a result, the runtime for solving the volume increases significantly.
\par
%
% \segLengthFactor
The segment length factor \segLengthFactor{} is the most important parameter to reduce the runtime for constant mean fiber radii \fiberRadiusMean{} because the number of objects is smaller.
However, as expected, larger \segLengthFactor{} leads to a lower volume fraction, especially for crossing fibers.
For a single fiber population \pfbs{}, it makes sense that the volume fraction does not change, since the direction of motion is radially symmetric along their main orientation axis for all fiber segments to avoid overlap.
\par
%
% \segRadiusFactor{}
The fiber segment bending radius factor \segRadiusFactor{} restricts the bending radius, it is to be expected and visible in the results that the volume fraction is strongly influenced since the volume can no longer be filled optimally.
As before, changing the \segRadiusFactor{} in the case of a single fiber bundle should not lead to any difference.
However, crossing fibers will be strongly affected.
This is especially evident in the number of colliding objects, where the \segRadiusFactor{} splits the data into individual branches over time.
Smaller values of \segRadiusFactor{} allow more curved geometries here.
However, this can lead to unnatural results in terms of anatomical structure.
As a result, a large part of the volume would be filled with fibers or rather fiber segments, but the actual number of fibers would be smaller.
A non-intuitive effect for smaller \segLengthFactor{} and higher \segRadiusFactor{} is visible.
The number of colliding objects increases again, as does the total number of objects.
Therefore, larger \segRadiusFactor{} should be avoided, as this effect also occurs when $\segLengthFactor = 2$ and $\segRadiusFactor \geq 4$.
Since such a large radius factor represents an unnatural stiffness for a nerve fiber, this is not a major concern for this type of models.
\par
%
To narrow down the selection of a parameter set, the crossover fiber population \cfbs{} is the most important data set.
The most essential parameter is the length factor \segLengthFactor{}.
From the results of the volume fraction, it can be concluded that \segLengthFactor{} should be as small as possible.
However, since the difference between $\segLengthFactor=\SI{1}{}$ and $\segLengthFactor=\SI{2}{}$ is small, and a large jump is visible for $\segLengthFactor=\SI{4}{}$, $\segLengthFactor=\SI{2}{}$ can be used to significantly reduce the runtime.
\par
%
The choice of the fiber bending radius factor \segRadiusFactor{} is an anatomical decision.
To counteract excessive deformation, a value of at least $\segRadiusFactor = \SI{2}{}$ should be chosen.
A larger value of \segRadiusFactor{} would significantly increase the runtime and add unnatural stiffness.
\segLengthFactor{} should be as small as possible to achieve the highest accuracy.
However, the runtime would increase too much for very small values.
Therefore, a value of $\segLengthFactor = \SI{2}{}$ was selected, since higher values already have a significant impact on the volume fraction.
\par
%
In summary, the following parameters are selected:
%
\begin{table}[H]
\sisetup{parse-numbers=false,open-bracket={\{}, close-bracket={\}}, list-final-separator={,},list-pair-separator={,}}%
\centering
\caption[]{Selection of parameters for the creation of the \ac{3D-PLI} simulation model.}
\pgfplotstabletypeset[%
    thesisTableStyle,
    column type=lcc,
    columns/name/.style={string type},
    columns/variable/.style={string type},
    columns/value/.style={string type},
    every head row/.style={before row=\toprule,after row=\midrule},
    every last row/.style={after row=\bottomrule},
    col sep=&,
    row sep=\\,
]
{name & variable & value\\
mean fiber radius & $\fiberRadiusMean$ & $\SI{0.5}{\micro\meter}$\\
mean segment length factor & $\segLengthFactor$ & $\num{2}$\\
min segment bending radius factor & $\segRadiusFactor$ & $\num{2}$\\
}
% \vspace{1ex}
\label{tab:parameterSetup}
\end{table}
%
For fiber radii $\fiberRadiusMean > \SI{0.5}{\micro\meter}$, the most important effect is the shortening of the runtime at constant volume.
In addition, boundary effects become apparent due to the limited volume.
For example, the mean volume fraction for each configuration decreases significantly and the variance increases.
This is the effect of being able to place only a few fibers into a $\SI{60}{\micro\meter}$ thick volume.
If they are then randomly arranged and deformed, as in this case, the volume fraction must decrease.
If the volume were infinite, there would be no difference between the results since all parameters are independent of the radii.
Therefore, an increase in the fiber radii should be considered if the volume to be calculated is larger or the runtime is finite.
It should then be verified that the simulation does not cause significant changes in the results.
\par
%
Another way to shorten the runtime is to not solve the model completely.
Looking at the overlap fraction, a value of $<\SI{1}{\percent}$ might be feasible.
It should be noted that this is only the fraction of the fiber segment that still overlaps.
The number of colliding fiber segments decreases steadily for the chosen parameters.
This potentially saves an order of magnitude in runtime.
Here, the influence on the simulation must also be investigated in advance.
%
%
%
\section{Nerve Fiber Model Library for \acs{3D-PLI} Simulations}
\label{sec:genNerveFiberLibrary}
%
With the model parameters selected above (see \cref{sec:modelSetup}), the simulation models are generated.
As described in \cref{sec:modelParamet}, only fibers with $\modelOmega = \SIrange{10}{90}{\degree}$ and $\modelPsi = \SIrange{0.1}{0.9}{}$ are generated with the addition of $\modelOmega=\SI{0}{\degree}, \modelPsi=\SI{1.0}{}$.
The volume for the simulation is scaled up to a sphere with a diameter of $\SI{135}{\micro\meter}$ so that a simulated volume can be generated from the sphere in any orientation.
%
%
%
\subsection{Results}
%
\begin{figure}[!t]
\centering
% \resizebox{1.0\textwidth}{!}{
\includegraphics[width=\textwidth, page=1]{gfx/rc1/model/cube_2pop_orientation_hist2d_output_cube_2pop_135_rc1.pdf}
% }
\caption[]{Density distribution of fiber segment orientation in the simulation models.
The value of the segments is weighted according to the area on a spherical surface and normalized so that the integral over one hemisphere is $\SI{1}{}$.
The dashed white line indicates the orientation of the two fiber populations.}
\label{fig:modelOrientation}
\end{figure}
%
\Cref{fig:modelOrientation} shows the orientation distribution of the fiber segments as a polar histogram.
The rest of the dataset is available in \cref{app:modelOrientation}.
The orientation distributions show that the individual fiber segments statistically follow the orientation of the fiber population.
The density reflects the population fraction $\modelPsi$ of the fiber populations.
\par
%
\begin{figure}[!t]
    \centering
    \includegraphics[]{gfx/rc1/model/cube_2pop_domega_analysis_output_.pdf}
    \caption[]{Direction $\dir$, inclination $\inc$ and opening angle $\openingAngle$ distribution of the model library.}
    \label{fig:modelAngleBoxPlot}
\end{figure}
%
\Cref{fig:modelAngleBoxPlot} shows the distributions of direction $\varphi$, inclination $\modelInc$, and opening angle $\openingAngle$, which is the relative angle to the mean fiber population orientation.
Since the inclination depends on the direction, the opening angle is also analyzed.
The individual measurements of the opening angle distributions are available in \cref{app:modelDistribution}.
\par
%
The direction and inclination angles are around $\SI{0}{\degree}$ for the first fiber population $\popa$ for all parameter combinations.
The variance decreases with increasing population fraction $\modelPsi$ of the second fiber population.
The crossing angle $\modelOmega$ does not seem to have any influence.
The opening angle $\openingAngle$ has its mean value between about $\SI{10}{\degree}$ and $\SI{20}{\degree}$ for all parameter configurations.
The variance decreases with increasing fiber population fraction $\modelPsi$.
The crossing angle $\modelOmega$ does not seem to have any influence.
\par
%
% second fiber population
The mean direction $\dir$ follows the initial crossing angle $\modelOmega$.
The fiber population fraction $\modelPsi$ does not seem to have a significant influence.
The inclination $\modelInc$ is stable around $\SI{0}{\degree}$ for all parameters.
The parameters do not seem to have a significant effect on their distribution.
The opening angle $\openingAngle$ is between $\SI{15}{\degree}$ and $\SI{20}{\degree}$ for the second fiber population.
For smaller crossing angles, the mean values almost do not change.
For large crossing angles, the mean value of the opening angle increases with increasing $\modelPsi$.
\par
%
\begin{figure}[!t]
\centering
\includegraphics[page=1]{gfx/rc1/images/cube_2pop_images_output_cube_2pop_135_rc1.pdf}
\caption[]{3D visualization of the simulation model library \ac{3D-PLI}.
The inner $\SI{10}{\micro\meter} \times \SI{10}{\micro\meter} \times \SI{10}{\micro\meter}$ of the volumes are shown.}
\label{fig:modelImages}
\end{figure}
%
\Cref{fig:modelImages} shows the visualization of the fiber models.
The visualization shows the inner $\SI{10}{\micro\meter} \times \SI{10}{\micro\meter} \times \SI{10}{\micro\meter}$ cube, so more details are visible.
Depending on the population $\modelPsi$, more or less layered distinct structures are visibly parallel to the intersection plane.
\par
%
% Depending on the parallel or crossing fiber populations, the runtime was in the range of $\SIrange{32}{40}{\hour}$ per model with $\SIrange{11000}{17000}{\steps}$.
%
%
%
\subsection{Discussion}
%
These models have the advantage of producing a naturally inspired angular distribution, which should result in a more realistic distribution.
The orientation distribution of the simulation model library is stable at all crossing angles.
However, the distribution variance is lower for the fiber population with higher population fraction and vice versa.
This is plausible because the main fiber bundle exerts more radial pressure on the orientation, resulting in a more stable configuration for cylindrical objects.
This probably has little effect on the simulation results.
The stiff models lead to a volume fraction that depends on the crossing angle.
Whether this is also true for anatomical tissue remains to be explored.
However, since real nerve fibers are not stiff and also not perfectly cylindrical, it can be expected to be higher.
Also, the orientation distribution may differ slightly from real tissue due to the stiffness of the model.
The layered structure can be understood for a randomly introduced pattern.
Real fabric will also have a different pattern, \eg{} interwoven fiber bundles instead of interwoven single fibers.
In linear \ac{3D-PLI} simulations, this has no effect because the retardation matrices for the intensity signal are commutative.
Or more generally, the order of the tissue matrices is commutative for the intensity signal.
In other fields, such as \ac{dMRI}, additional models are probably required in this regard.
%
%
%
\section{Multicore CPU Acceleration}
%
As described in \cref{sec:modelOpt}, \ac{OpenMP} is used to accelerate \code{for} loops in the algorithm.
This means that it is currently not possible to use multiple compute nodes.
\par
%
To investigate the speedup, the runtime for the parameters from \cref{tab:parameterSetup} is measured for parallel and crossing fibers.
The volume used is $\SI{60}{\micro\meter} \times \SI{60}{\micro\meter} \times \SI{60}{\micro\meter}$.
Speedup is calculated for an average of 100 consecutive steps.
To measure whether the speedup changes with time, the measurements are started after a certain number of steps $\Delta_{\mathit{steps}} = \SIlist{0;100;1000;10000}{\steps}$.
The calculation is repeated for each case $n=\SI{24}{}$ times.
For the measurements, the cores were bound to physical CPU cores.
%
%
%
\subsection{Results}
%
\begin{figure}[!t]
\centering
\includegraphics{gfx/rc1/speed/boxplot_output_r_0.5__.pkl_speedup.csv.pdf}
\caption[]{\code{model.Sovler} speedup. Time measurements are performed after $\Delta_{\mathit{steps}}$ for the next $\SI{25}{\steps}$ for parallel \pfbs{} and crossing \cfbs{} fiber configurations.}
\label{fig:solverSpeedup}
\end{figure}
%
\Cref{fig:solverSpeedup} shows the speedup for different starting positions $\Delta_{\mathit{steps}}$ for the parallel \pfbs{} and crossing \cfbs{} fiber bundles.
Up to $\SI{4}{cores}$, the speedup increases linearly to a speedup of $\SI{3}{}$.
From $\SI{4}{cores}$, the speedup increases more slowly until a significant increase to a speedup of about $\SI{5}{}$ is observed for $\SI{8}{cores}$.
\par
%
Overall, no significant change is visible for parallel or crossing fibers.
Also, for different starting points $\Delta_{\mathit{steps}}$, no change in acceleration is visible.
The speedup increases up to $\SI{8}{}$.
%
%
%
\subsection{Discussion}
%
Up to $\SI{4}{cores}$ the speedup is expected with $\SI{3}{}$.
The slower increase and jump for $\SI{8}{cores}$ can be explained by the structure of the \name{octree}.
Since for $\SI{8}{cores}$ the parallel alignment on a \name{octree} (can) be optimal, the speedup increases.
More cores are not feasible.
\par
%
Nevertheless, the data suggests that using a multicore system shortens the runtime considerably.
However, for small volumes or low-fiber objects, one should use only one or two cores and prefer to run multiple models in parallel.
The results also show that an optimized algorithm is needed, especially when the volume or the number of objects increases.
Here, the \ac{GPU} seems to be the hardware of choice with a more advanced algorithm \cite{Karras2012} (see \cref{sec:outlook}).