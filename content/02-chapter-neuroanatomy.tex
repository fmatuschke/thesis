\newpage\null\thispagestyle{empty}\newpage
\clearpage{\thispagestyle{empty}\cleardoublepage}
\part{Theoretical basics}
% \cleanchapterquote{We are part of the universe that has developed a remarkable ability: We can hold an image of the world in our minds. We are matter contemplating itself.}{Sean Carroll}{The Particle at the End of the Universe}
% \acbarrier
\parttoc
% 
% 
% 
\cleardoublepage
\setcounter{chapter}{1}
\chapter{Neuroanatomy}
\label{chap:neuro}
%
% \paragraph{TODO}
% \begin{itemize}
%     \item was für parameter treten wo im Gehirn auf
% \end{itemize}
%
%
\section{Introduction}
%
Neuroanatomy is the study of the structure of the brain.
Its describing regions and the structure of the nervous system in humans and animals.
Techniques such as \ac{dMRI}, fluorescence microscopy, microscopy on stained tissue, autoradiography (to name a few), have been able to study more and more structures from different perspectives in the brain with different resolutions, modalities and contrasts on different species.
% 
\par
%
This chapter gives a general overview of the structure of the brain with its most important regions as well as the nerve fiber architecture.
More in-depth information can be found, for example, in the following literatures \dummy{}.
% 
%
% 
% The human brain is one of the most complex organs with a great diversity of cells, connections, topography and resulting functionalities.
% 
% 
%
\section{Brain Architecture}
%
The mammal brain consists of three main parts: the brainstem, the cerebellum and the cerebrum (see \cref{fig:humanBrain}).
% 
The brainstem is the connection between the different brain areas and the spinal cord located at the bottom of the brain.
It can be further subdivided into the midbrain, the pons and the medulla obiongata.
The cerebellum is located at the lower rear of the brain. Its most important function is motor control.
It is highly folded \todo{species?} and therefore has a particularly large surface area.
The cerebrum is the largest part of the human brain.
As the cerebellum its surface is folded as well \todo{species?}.
The cerebrum is split into a left and right hemisphere.
In addition, the cerebrum can be divided into four parts:
the frontal, parietal, temporal, and occipital lobes (see \cref{fig:brainLobes}).
The frontal lobe is responsible for voluntary movements of specific body parts as well as the human personality.
The parietal lobe's main functionalities are the processsing of the sensory informations.
The primary function of the occipital lobe is signal processing of the visual system.
The temporal lobe contains auditory functions and language perception in addition to visual memories.
Beneath the brain surface there are also other structures such as the basal gangila or the thalamus.
\par
% 
\begin{figure}[!t]
\centering
\subcaptionbox{%
        \label{fig:brainLobes}%
        Sagital view of the human brain with lobes colored in: {\color[RGB]{117,112,179}frontal}, {\color[RGB]{230,171,2}parietal}, {\color[RGB]{27,158,119}temporal} and {\color[RGB]{231,41,138}occipital} lobe.
        The {\color[RGB]{102,166,30}cerebellum} is at the bottom, with the {\color[RGB]{217,95,2}brainstem} next to it.
        Modivied version of \url{https://en.wikipedia.org/wiki/Frontal_lobe}%
    }[.47\textwidth]{\includegraphics[height=0.3\textwidth]{gfx/neuroanatomy/brain_lobes.pdf}}
\hfill
\tikzset{external/export next=false}
\subcaptionbox{%
        \label{fig:coronalStained}%
        \todo{BB 4201} Coronal section stained for cell bodies. The \ac{GM} is dark while the \ac{WM} is bright. The left and right hemisphere is connected via the corpus calosum.}%
    [0.47\textwidth]{\includegraphics[height=0.3\textwidth]{dev/brain/BB_4201.png}}
    % \begin{tikzpicture}[]
    %     \node[inner sep=0pt, anchor = south west] (fig) at (0,0)
    %     {\includegraphics[height=0.3\textwidth]{dev/brain/BB_4201.png}};
    %     \coordinate (WM1) at ($(fig.north west)!0.6!(fig.north east)$);
    %     \coordinate (WM2) at ($(fig.north east)!0.35!(fig.south east)$);
    %     \draw[RED, ultra thick, <-] (WM1 |- WM2) -- ++ (-42:0.75) node[pos=1, anchor=north] {\textbf{WM}};
    %     \coordinate (GM1) at ($(fig.north west)!0.275!(fig.north east)$);
    %     \coordinate (GM2) at ($(fig.north east)!0.125!(fig.south east)$);
    %     \draw[GREEN, ultra thick, <-] (GM1 |- GM2) -- ++ (-65:0.75) node[pos=1, anchor=north] {\textbf{GM}};
    % \end{tikzpicture}
% }
\caption{Illustration of the human brain and a cell body stained coronal section.}
\label{fig:humanBrain}
\end{figure}
%
The cerebellum and cerebrum contain a \ac{GM} stucture at the brain surface.
This structure is filled with neurons.
These cells have the task of processing the information of all signals coming from inside and outside of the brain.
These cells are arranged in cortical layers that have different thicknesses, cell types, and densities specific to a brain area.
These cells have a relatively high density and are not only locally interconnected with each other, but connect also with different brain areas.
%  (see \cref{fig:nerveFiber,fig:cortLayers}).
Therefore, the folding of the brain is particularly important to increase the surface and therefore the number of cells.
In the human brain there are several billions of nerve cells.
There are many types of cells, \eg{} granule or pyramidal cells.
The highly interconnected structure and arrangement of the various cells is the source of its high number of different functionalities.
It is imporatant to investiagte the human brain to gain a better understanding of the brain's function and an improved understanding of pathophysiological processes that may lead to improved medical treatment of brain deseases.
%
%
%
\section{Fiber architecture} \label{sec:fiberArchitecture}
%
\begin{figure}[!t]
\centering
% \subcaptionbox{
%     \label{fig:cortLayers}
%     Cortical layers \todo{wichtig?}
%     }[0.225\textwidth]{\includegraphics[height=4.5cm]{dev/wiki/layers.png}}
% \hfill
\tikzset{external/export next=false}
% \subcaptionbox{
%     \label{fig:nerveCell}
%     Nerve cell structure.
% }[0.75\textwidth]{
\begin{tikzpicture}[every node/.style={font=\small,},]
    \node[inner sep=0pt, anchor = south west] (fig) at (0,0) {\includegraphics[ angle=90,width=0.75\textwidth]{gfx/neuroanatomy/neuron-axon.pdf}}; %height=4.2cm
    \begin{scope}[overlay]
        % \foreach \p in {0,0.1,0.2,0.3,0.4}{
        %     \draw[blue] (rel cs:name=fig,x=\p,y=\p) rectangle (rel cs:name=fig,x=1-\p,y=1-\p);
        % }
        % 
        \node[anchor=south] at (rel cs:name=fig,x=0.72,y=1.025) {Oligodendrocyte};
        \node[anchor=north] at (rel cs:name=fig,x=0.235,y=0.22) {Cell Body};
        \node[anchor=west] at (rel cs:name=fig,x=1,y=0.675) {Axon};
        \node[anchor=north] at (rel cs:name=fig,x=0.7,y=0.4) {Myelin};
        \draw[<-] (rel cs:name=fig,x=0.9,y=0.5) -- ++(-69:0.5) node[pos=1,anchor=north] {Node of Ranvier};
        \draw[<-] (rel cs:name=fig,x=0.295,y=0.53) -- ++(-140:0.75) node[pos=1,anchor=north] {Nucleus};
        \node[anchor=south] at (rel cs:name=fig,x=0.24,y=0.8) (D) {Dendrites};
        \draw[->] (rel cs:name={D},x=0.3,y=0) -- ++(-155:0.6){};
        \draw[->] (rel cs:name={D},x=0.7,y=0) -- ++(-25:0.6){};
    \end{scope}
    \path[] ($(fig.south west)-(0,0)$) rectangle ($(fig.north east)+(1,0.75)$);
\end{tikzpicture}
% }
\caption{Illustration of a neuron with axon and oligodendrocytes. The olegodendrocyte build up a layered lipid structure up, surrounding the axon. The myelin layers are separated along the axon by nodes of Ranvier.}
\label{fig:CortexAndNerveCell}
\end{figure}
%
Nerve cells are connected by nerve fibers.
A typical nerve cell (see \cref{fig:nerveCell}) comprises a cell body, called a soma, that processes incoming information.
The information arrives via dendrites, which are star-shaped branches.
The axon is the cell's only information output.
It travels through the brain often associated with nerve fiber bundles to its destination, where it connects with the axon terminal to other neurons.
\par
%
The axon is surrounded by a myelin sheath, a lipid layered structure deriving from nearby olegodendrides (see \cref{fig:human-brain}).
The myeling electrically insulates the axon and improves the speed of propagation of the electrical action potential along the axon and also its signal strength.
The diamater of the myelin ranges from about $\SI{0.5}{\micro\meter}$ to several $\si{\micro\meter}$ (see \cref{tag:axonDiameter}).
There are many different types of axons.
Some contain a very thick myelin layer, while others have none.
The high density of axons and myelin makes the brainappear white and is therefore called \ac{WM}, whereas the outer cell bodys appear darker and ist called \ac{GM}.
This color difference is clearly visible in a Nissle stained histological sections (see \cref{fig:coronalStained}).
% \par
%
% \ac{WM} has a relativly high density of fibers, for example up to about $\SI{200}{\million}$ in the corpus callosum connecting the left and right brain hemisphere.
% 
% 
% 
\section{WM imaging techniques / State of the art models/imageing/... ?}
% 
\subsection{diffusion MRI}
% 
\ac{dMRI} is used to measure the diffusion of water molecules inside a volume.
Since inside the brain nerve fibers form high density of nerve fiber bundles, the water molecules have another diffusion constant along the nerve fibers orienation in contrast to the perpenticular diretion.
Therefore it is posible to extract the orientation of nerve fibers inside a subvolume of the brain.
Depending on the \ac{MRI}, the sequence and the measurment time, the resolution can be in the order of several hundres of micro meter up to several milli meters.
Compared to other techniques this resolution is rather small. 
However up to this point it is the only existinc technique capable of measuring in vivo.
Therefore it is ecential for diognoses in the medical field.
% 
% 
% 
\subsection{Stained tissue microscopy}
% 
With a microscope it is possible to measure up to a resolution in the order of the wavelength of the light.
For visible light this is in the range of several hundres of nanometer.
This makes it possible to study brain sections under the microscop.
Since nerve fibers are quite small, the brain section also has to be small to be able to see the orientation and path of individual nerve fibers.
However one cannot section the brain arbitrary thin, since it becomes qute unstable.
Typical section thickness for a light microscopy is in the order of several 10th of micrometers.
\par
% 
\begin{figure}[!t]
	\centering
	\tikzset{external/export next=false}
	\begin{tikzpicture}[]  
        \node[inner sep=0pt, anchor = south west] (fig) at (0,0)
           {\includegraphics[width=\textwidth]{gfx/neuroanatomy/NeuralNet-BrainAtlasDotOrg.png}};
        %  
        \coordinate (A) at (fig.west);
        \coordinate (B) at (fig.west);
        %  
        \draw[Orange, ultra thick] (7.65,6.15) ellipse (1 and 0.5);
        \draw[ProcessBlue, ultra thick] (10,4.5) ellipse (2 and 1);
        % \draw[yellow, ultra thick] (fig.north west) -- ++ (13.87303cm,0);
    \end{tikzpicture}
	\caption[Myelin staining of the human thalamus]{Myelin staining of the human thalamus, sagital section. \textcolor{Orange}{nerve fiber bundles}, \textcolor{ProcessBlue}{"neural net"}. \url{http://brainmaps.org/HBP3/h.sapiens/sag/h5thal-myelin/17a}}
	\label{fig:brainMyelinStain}
\end{figure}
% 
To enhance the contrast of the nerve fibers to the background, staining like Nissle is used to darken the myelin (see \cref{fig:brainMyelinStain}).
This allows to follow small nerve fibers down to individual nerves depending on their myelination degree.
Larger nerve fiber bundles are such dark, that mostly no orientation can be exracted.
% 
% 
% 
\subsection{Electron microscop}
% 
Due to the ... principle of quantum mechanic, an electron also behaves light a wave with a wave length of hundreds of femtometer and less, depending on the energy of the electron.
This makes it possible to measure with an electron microscop images of very high resolutions.
However, to be able to measure the electron, one needs a vacuum.
Therefore if one images a tissue sample of the brain, the tissue sample dries out.
This process deforms the tissue and therefore the nerve fiber paths.
Still it is possible to study the small samples of nerve fiber tissue.
% 
% 
% 
% \subsection{3D-PLI}
% % 
% \ac{3D-PLI} is an imaging techique based on the change of polarization of the light due to the birefingence property of myeling (see \cref{sec:expSetup}).
% The from this it is possible to extract the orientation of the nerve fibers (see \cref{fig:brainFOM}).
% % 
% \todo{was sollte in diesm abschnitt stehen? mindestens fom?}
% % 
% % \begin{figure}[!t]
% % 	\centering
% %     \setlength{\tikzwidth}{0.75\textwidth}
% %     \begin{tikzpicture}[baseline, trim axis left, trim axis right]
% %     \begin{axis}[
% %     width=\tikzwidth,
% %     axis equal image,
% %     enlargelimits=false,
% %     scale only axis,
% %     xmin=0,xmax=2069,ymin=0,ymax=1694,
% %     hide axis,
% %     ]
% %     \addplot graphics [
% %     xmin=0,xmax=2069,ymin=0,ymax=1694,
% %     ] {dev/Brain31011_70mu_70ms_s0936_ROFL_FOM_HSVBlack.png};
% %     \addplot [forget plot] graphics [
% %     xmin=0,xmax=300,ymin=1394,ymax=1694,
% %     ] {gfx/pli/color_sphere.png};
% %     \end{axis}
% %     \end{tikzpicture}
% % 	\caption{\ac{FOM} of a coronal human brain section. The color decodes the orientation in 3d. \todo{Brain31011\_s0936}}
% % 	\label{fig:brainFOM}
% % \end{figure}
% % 
% % 
% % 
% \subsubsection{Sectioning}
% %
% \begin{figure}[!t]
% 	\centering
%     \setlength{\tikzwidth}{0.75\textwidth}
% 	\inputtikz{gfx/neuroanatomy/brain_sectioning}
% 	\caption{Illustration of sectioning. a)-b) block with imbedded brain, c) vibrating knife.}
% 	\label{fig:brain_sectioning}
% \end{figure}
% %
% From Miriam thesis:
% Brain removed from skull within \SI{24}{\hour} and immersed in a buffered solution of $\SI{4}{\percent}$ formaldehyde
% $\SI{20}{\percent}$ glycerin solution for several days after \dummy{}.
% $\SI{2}{\percent}$ Dimethyl sulfoxide (DMSO) and $\SI{4}{\percent}$ formaldehyde.
% \par
% % 
% For the cutting process, the tissue, \ie{} the brain, has to be frozen and imbedded into a solid material.
% One commen material is parafine, however this is not possible for \ac{3D-PLI} \todo{why?}.
% For \ac{3D-PLI} it very is important, that the tissue does not build up crystal structures, \eg{} from sugar molecules, because these are also birefringence.
% Therefore the tissue is first \dummy{} into a \dummy{} liquid \dummy{}
% After a time of a few months the tissue will then be frozen into a $\SI{-80}{\degree}$.
% Then the frozen tissue can be fixated with a liquid glue on a cutting plate.
% The glue is \dummy{}
% It is also be used to build up a surrounding fixating material to stabilize the brain in the cutting process.
% Additionally markers are fixed which help in the later registration process, which will align the tissue sections in a 3d volume again \cite{Schober2016,Ali2018,Schmitz2018}.
% \par
% %
% The cutting is done in a microtome (see \cref{fig:brain_sectioning}).
% In there the temperature can be held at about $\SI{-70}{\degree}$ and allows no heating of the tissue.
% The brain is moved against a vibrating very sharp knife.
% This allows for the thin sectioning of about $\SI{60}{\micro\meter}$.
% After the cutting, the tissue is put onto a glass specimen.
% Since the tissue is such thin and filigran, it is not always possible to avoid cracks for example.
% This also will be as best as possible corrected in the registration process.
% The tissue will be \dummy{} with a glycerin \dummy{} and finally siled with another glass.
% To prevent the formation of waves in the tissue, the glass is weighted.
% The tissue sections are then stored into a refrigerator at $\SI{-70}{\degree\celsius}$.
% The tissue can then finally be measured in the \ac{3D-PLI} microscopes (see \cref{sec:expSetup} for further techniqule informations).
%
%
\section{Axon Literature}
\label{sec:axonMicroscopy}
% 
\TODO{umschreiben, die tabelle ist nicht wichtig, die grossenordnung ist fuer diese arbeit entscheident.}
%
Axon diameter \cite{Liewald2014}:
%
\begin{table}[!b]
\centering
\pgfplotstabletypeset[
thesisTableStyle,
font=\footnotesize,
col sep=comma,
columns/Name/.style={string type},
columns/Mean/.style={fixed zerofill},
columns/SD/.style={fixed zerofill},
columns/Median/.style={fixed zerofill},
columns/Max/.style={fixed zerofill},
columns/Min/.style={fixed zerofill},
columns/n/.style={dec sep align},
rowbf={1},rowbf={8},rowbf={19},
rowem={2},rowem={5},
rowem={9},rowem={12},rowem={15},
rowem={20},rowem={23},rowem={26},
]{data/axon_distribution.csv}
\caption{axon diameter distribution of the human brain in $\si{\micro\meter}$ \cite{Liewald2014}.}
\label{tag:axonDiameter}
\end{table}
%
\begin{table}[!b]
\centering
\pgfplotstabletypeset[
thesisTableStyle,
font=\footnotesize,
col sep=semicolon,
columns={article,cite,gratio},
columns/article/.style={string type, column name=name, column type = {l}},
columns/cite/.style={string type, column name=cite, column type = {l}},
columns/gratio/.style={string type, column name=$g_{\mathit{ratio}}$},
]{data/gratio.csv}
\caption{human $g_{\mathit{ratio}}$ from invivo mri studies.}
\label{tab:gratio}
\end{table}
%
axon = 0.5-1.0 diameter (most frequent
thickness of myelin mean = 0.09, median 0.08
-> g-ratio 0.9 (electron microscop, upper boundry)
%
g-ratio
\cite{Cercignani2017} -> 0.65-0.8 mrt, healty male and female different age, different regions\\
\cite{FitzGibbon2013} -> 0.58-0.84 (Retina), electron microscopy
%
\todo{brechungsindex}
%
% \begin{figure}[!t]
% 	\centering
% 	 \includegraphics[clip, trim=0.65cm 6.25cm 0.4cm 0.5cm]{gfx/neuroanatomy/nature_3d_em_optical_nerve.png}
% 	\caption{\dummy{only show a} Three dimensional electron microscopy reveals changing axonal and myelin morphology along normal and partially injured (*, light green) optic nerves. Origin: \cite{Giacci2018} (reative Commons Attribution 4.0 International License).}
% % 	\label{fig:brain_sectioning}
% \end{figure}
%
%
% \par
% \noindent\rule{\textwidth}{2pt}
% \par
%