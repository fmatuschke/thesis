\newpage\null\thispagestyle{empty}\newpage
\clearpage{\thispagestyle{empty}\cleardoublepage}
\part{Appendices}
%
\chapter{Modeling}
\label{app:modelAnalysis}
%
\setlength{\tikzwidth}{\textwidth}
\begin{sidewaysfigure}[!ht]
\centering
\includegraphics[height=\tikzwidth,page=1]{gfx/rc1/mpara/pre_stats_time_evolve_all_output_parameter_statistic_rc1.pdf}
\includegraphics[height=\tikzwidth,page=2]{gfx/rc1/mpara/pre_stats_time_evolve_all_output_parameter_statistic_rc1.pdf}
\label{app:pste1}
\caption{Time evolution of the model building process of parallel and crossing fiber populations with $\fiberRadiusMean=\SI{0.5}{\micro\meter}$. Error bars indicate $\SI{25}{\percent}$ and $\SI{75}{\percent}$ quantiles (see \cref{sec:solverParameterResults}).}
\end{sidewaysfigure}
%
\begin{sidewaysfigure}[!ht]
\centering
% \resizebox{\tikzwidth}{!}{
\includegraphics[height=\tikzwidth,page=3]{gfx/rc1/mpara/pre_stats_time_evolve_all_output_parameter_statistic_rc1.pdf}
\includegraphics[height=\tikzwidth,page=4]{gfx/rc1/mpara/pre_stats_time_evolve_all_output_parameter_statistic_rc1.pdf}
% }
\label{app:pste2}
\caption{Time evolution of the model building process of parallel and crossing fiber populations with $\fiberRadiusMean=\SI{1.0}{\micro\meter}$. Error bars indicate $\SI{25}{\percent}$ and $\SI{75}{\percent}$ quantiles (see \cref{sec:solverParameterResults}).}
\end{sidewaysfigure}
%
\begin{sidewaysfigure}[!ht]
\centering
% \resizebox{\textwidth}{!}{
\includegraphics[height=\tikzwidth,page=5]{gfx/rc1/mpara/pre_stats_time_evolve_all_output_parameter_statistic_rc1.pdf}
\includegraphics[height=\tikzwidth,page=6]{gfx/rc1/mpara/pre_stats_time_evolve_all_output_parameter_statistic_rc1.pdf}
\label{app:pste3}
\caption{Time evolution of the model building process of parallel and crossing fiber populations with $\fiberRadiusMean=\SI{2.0}{\micro\meter}$. Error bars indicate $\SI{25}{\percent}$ and $\SI{75}{\percent}$ quantiles (see \cref{sec:solverParameterResults}).}
\end{sidewaysfigure}
%
\begin{sidewaysfigure}[!ht]
\centering
% \resizebox{\textwidth}{!}{
\includegraphics[height=\tikzwidth,page=7]{gfx/rc1/mpara/pre_stats_time_evolve_all_output_parameter_statistic_rc1.pdf}
\includegraphics[height=\tikzwidth,page=8]{gfx/rc1/mpara/pre_stats_time_evolve_all_output_parameter_statistic_rc1.pdf}
\label{app:pste4}
\caption{Time evolution of the model building process of parallel and crossing fiber populations with $\fiberRadiusMean=\SI{5.0}{\micro\meter}$. Error bars indicate $\SI{25}{\percent}$ and $\SI{75}{\percent}$ quantiles (see \cref{sec:solverParameterResults}).}
\end{sidewaysfigure}
%
\begin{sidewaysfigure}[!ht]
\centering
% \resizebox{\textwidth}{!}{
\includegraphics[height=\tikzwidth,page=9]{gfx/rc1/mpara/pre_stats_time_evolve_all_output_parameter_statistic_rc1.pdf}
\includegraphics[height=\tikzwidth,page=10]{gfx/rc1/mpara/pre_stats_time_evolve_all_output_parameter_statistic_rc1.pdf}
\label{app:pste5}
\caption{Time evolution of the model building process of parallel and crossing fiber populations with $\fiberRadiusMean=\SI{10.0}{\micro\meter}$. Error bars indicate $\SI{25}{\percent}$ and $\SI{75}{\percent}$ quantiles (see \cref{sec:solverParameterResults}).}
\end{sidewaysfigure}
% 
\begin{figure}[!ht]
    \centering
    \includegraphics[width=1\textwidth]{gfx/rc1/mpara/pre_stats_box_plot_volume_all_output_parameter_statistic_rc1.pdf}
    \caption[]{Volume fractions $V_f/V_0$ for parallel \pfbs{} and crossing \cfbs{} fiber populations of different relative fiber segment lengths $\segLengthFactor$ and relative fiber bending radii $\segRadiusFactor$ and multiple mean fiber radii $\fiberRadiusMean$. (see \cref{sec:solverParameterResults}).}
    \label{app:appModelVolumeBoxPlot}
\end{figure}
%
\begin{figure}[!ht]
    \centering
    \includegraphics[width=\textwidth, page=2]{gfx/rc1/model/cube_2pop_orientation_hist2d_output_cube_2pop_135_rc1.pdf}
    \caption[]{Density distribution of fiber segment orientation in the simulation models.
    The value of the segments is weighted according to the area on a spherical surface and normalized so that the integral over one hemisphere is $\SI{1}{}$.
    The dashed white line indicates the orientation of the two fiber populations. (see \cref{sec:08Results}).}
    \label{app:modelOrientation}
\end{figure}
%
\begin{figure}[!ht]
    \centering
    \includegraphics[]{gfx/rc1/model/cube_2pop_domega_analysis_all_output.pdf}
    \caption[]{Direction $\dir$, inclination $\inc$ and opening angle $\openingAngle$ distribution of the model library (see \cref{sec:08Results}).}
    \label{app:modelAngleBoxPlot}
\end{figure}
%
\begin{figure}[!ht]
    \centering
    \resizebox{\textwidth}{!}{
    \includegraphics[page=2]{gfx/rc1/images/cube_2pop_images_output_cube_2pop_135_rc1.pdf}}
    \caption[]{D: Simulation model library. The inner $\SI{10}{\micro\meter} \times \SI{10}{\micro\meter} \times \SI{10}{\micro\meter}$ of the volume is shown (see \cref{sec:08Results}).}
    \label{app:modelImages}
\end{figure}
%
\begin{table}[!ht]
\centering
\caption[]{Orientation statistic of simulation model library (see \cref{sec:08Results}).}
\resizebox{\textwidth}{!}{
\pgfplotstabletypeset[%
    thesisTableStyle,
    columns={omega,psi,pop,incl_mean,incl_std,incl_25,incl_50,incl_75,phi_mean,phi_std,phi_25,phi_50,phi_75,mean,std,25,50,75},
    every head row/.style={
        before row={
            \toprule
        },
        after row={$\si{\degree}$ & $\si{\percent}$ & & $\si{\degree}$ & $\si{\degree}$ & $\si{\degree}$ & $\si{\degree}$ & $\si{\degree}$ & $\si{\degree}$ & $\si{\degree}$ & $\si{\degree}$ & $\si{\degree}$ & $\si{\degree}$ & $\si{\degree}$ & $\si{\degree}$ & $\si{\degree}$ & $\si{\degree}$ & $\si{\degree}$ \\ \midrule},
    },
    columns/omega/.style={column name=$\Omega$},
    columns/psi/.style={column name=$\Psi$, multiply by=100,zerofill,precision=0},
    columns/pop/.style={column name={pop.}},
    columns/mean/.style={column name=$<\openingAngle>$,zerofill,precision=0},
    columns/std/.style={column name=$\sigma(\openingAngle)$,zerofill,precision=0},
    columns/25/.style={column name=$25(\openingAngle)$,zerofill,precision=0},
    columns/50/.style={column name=$50(\openingAngle)$,zerofill,precision=0},
    columns/75/.style={column name=$75(\openingAngle)$,zerofill,precision=0},
    columns/incl_mean/.style={column name=$<\alpha>$,fixed,fixed zerofill,precision=0},
    columns/incl_std/.style={column name=$\sigma(\alpha)$,fixed,fixed zerofill,precision=0},
    columns/incl_25/.style={column name=$25(\alpha)$,fixed,fixed zerofill,precision=0},
    columns/incl_50/.style={column name=$50(\alpha)$,fixed,fixed zerofill,precision=0},
    columns/incl_75/.style={column name=$75(\alpha)$,fixed,fixed zerofill,precision=0},
    columns/phi_mean/.style={column name=$<\varphi>$,fixed,fixed zerofill,precision=0},
    columns/phi_std/.style={column name=$\sigma(\varphi)$,fixed,fixed zerofill,precision=0},
    columns/phi_25/.style={column name=$25(\varphi)$,fixed,fixed zerofill,precision=0},
    columns/phi_50/.style={column name=$50(\varphi)$,fixed,fixed zerofill,precision=0},
    columns/phi_75/.style={column name=$75(\varphi)$,fixed,fixed zerofill,precision=0},
    col sep=comma,
    %
]
{gfx/rc1/model/omegas_ms_2pop.csv}
}
\label{app:modelDistributionTable}
\end{table}
%
%
%
\chapter{Simulation}
\label{app:Simulation}
%
% \fakesection{Tissue}
%
\begin{figure}[!ht]
\centering
\tikzset{external/export=false} % dont ask me why
\setlength{\tikzwidth}{0.425\textwidth}
\setlength{\tabcolsep}{0em}
\begin{tabular}{C{0.5\textwidth}C{0.5\textwidth}}
\inputtikz{gfx/data/rodent_transmittance_zoom} &
\tikzset{external/export next=true}
\inputtikz{gfx/data/rodent_transmittance_hist} \\[-5mm]
\subcaptiontab{0.475\textwidth}{Transmittance.} &
\subcaptiontab{0.475\textwidth}{Transmittance histogram.} \\[10mm]
\inputtikz{gfx/data/rodent_retardation_zoom} &
\tikzset{external/export next=true}
\inputtikz{gfx/data/rodent_retardation_hist} \\[-5mm]
\subcaptiontab{0.475\textwidth}{Retardation.} &
\subcaptiontab{0.475\textwidth}{Retardation histogram.}
\end{tabular}
\caption[]{Rodent coronal brain section.}
\label{app:brain_rodent}
\end{figure}
%
% \hspace{1em}
% %
\begin{figure}[!ht]
\centering
\tikzset{external/export=false} % dont ask me why
\setlength{\tikzwidth}{0.425\textwidth}
\setlength{\tabcolsep}{0em}
\begin{tabular}{C{0.5\textwidth}C{0.5\textwidth}}
\inputtikz{gfx/data/human_transmittance_zoom} &
\tikzset{external/export next=true}
\inputtikz{gfx/data/human_transmittance_hist} \\[-5mm]
\subcaptiontab{0.475\textwidth}{Transmittance.} &
\subcaptiontab{0.475\textwidth}{Transmittance histogram.} \\[10mm]
\inputtikz{gfx/data/human_retardation_zoom} &
\tikzset{external/export next=true}
\inputtikz{gfx/data/human_retardation_hist} \\[-5mm]
\subcaptiontab{0.475\textwidth}{Retardation.} &
\subcaptiontab{0.475\textwidth}{Retardation histogram.}
\end{tabular}
\caption[]{Right hemisphere human coronal brain section.}
\label{app:brain_human}
\end{figure}
%
%
% \fakesection{Voxelsize}
%
\begin{figure}[!p]
% 2_simulation/0_parameter/fiber_radii.py
\centering
\includegraphics[width=\textwidth, page=1]{gfx/rc1/voxel_size/voxel_size_plots_data_output_vs_135_0.01_6_25_vervet_r_rc1.pdf}
\caption[]{The mean difference is constant for smaller voxel sizes and does not start to grow significantly before $\voxelsize=\SI{0.125}{\micro\meter}$.}
\label{app:voxelsizeNoise}
\end{figure}
%
%
%
% %%%%%%%%%%%%%%%%%%%%%%%%%%%%%%%%%%%%%%%%%%%%%%%%%%%%%%%%%%%
% SINGLE
% %%%%%%%%%%%%%%%%%%%%%%%%%%%%%%%%%%%%%%%%%%%%%%%%%%%%%%%%%%%
%
% \fakesection{Single fiber population}
%
\begin{figure}[!p]
\centering
% \begin{minipage}{0.45\textwidth}
\resizebox{\textwidth}{!}{
\rotatebox{90}{
\begin{tabular}{c|c}
    \includegraphics[page=2]{gfx/rc1/analysis/plots_single_pop_hist_output_cube_2pop_135_rc1_single.pdf}
    &
    \includegraphics[page=3]{gfx/rc1/analysis/plots_single_pop_hist_output_cube_2pop_135_rc1_single.pdf}
\end{tabular}
}}
%
\caption[]{left: 2D log histogramm orientation from rofl analysis of simulation, right: 2D log histogramm of orientation of model segemnts. \todo{rfc plots}}
\label{app:single_fiber_pop_hist}
\end{figure}
%
%
% %%%%%%%%%%%%%%%%%%%%%%%%%%%%%%%%%%%%%%%%%%%%%%%%%%%%%%%%%%%
% FLAT
% %%%%%%%%%%%%%%%%%%%%%%%%%%%%%%%%%%%%%%%%%%%%%%%%%%%%%%%%%%%
%
% \fakesection{Crossing fiber population}
%
\begin{figure}[!p]
\centering
\resizebox{\textwidth}{!}{
\begin{tabular}{c|c}
    \includegraphics[page=1]{gfx/rc1/analysis/plots_flat_pop_hist_output_cube_2pop_135_rc1_flat.pdf}&
    \includegraphics[page=2]{gfx/rc1/analysis/plots_flat_pop_hist_output_cube_2pop_135_rc1_flat.pdf} \\
    \includegraphics[page=3]{gfx/rc1/analysis/plots_flat_pop_hist_output_cube_2pop_135_rc1_flat.pdf} &
    \includegraphics[page=4]{gfx/rc1/analysis/plots_flat_pop_hist_output_cube_2pop_135_rc1_flat.pdf} \\
    \includegraphics[page=5]{gfx/rc1/analysis/plots_flat_pop_hist_output_cube_2pop_135_rc1_flat.pdf} &
    \includegraphics[page=6]{gfx/rc1/analysis/plots_flat_pop_hist_output_cube_2pop_135_rc1_flat.pdf} \\
    \includegraphics[page=7]{gfx/rc1/analysis/plots_flat_pop_hist_output_cube_2pop_135_rc1_flat.pdf} &
    \includegraphics[page=8]{gfx/rc1/analysis/plots_flat_pop_hist_output_cube_2pop_135_rc1_flat.pdf} \\
    \includegraphics[page=9]{gfx/rc1/analysis/plots_flat_pop_hist_output_cube_2pop_135_rc1_flat.pdf} &
    % \includegraphics[page=10]{gfx/rc1/analysis/plots_flat_pop_hist_output_cube_2pop_135_rc1_flat.pdf}
\end{tabular}
}
%
\caption[]{psi=0.3; left: 2D log histogramm orientation from rofl analysis of simulation, right: 2D log histogram of orientation of model segemnts. \todo{legende}}
\label{app:flat_fiber_pop_hist}
\end{figure}
%
%
%
\begin{figure}[!p]
\centering
\resizebox{\textwidth}{!}{
\rotatebox{90}{
\begin{tabular}{cc}
\includegraphics[page=1]{gfx/rc1/analysis/plots_flat_pop_output_cube_2pop_135_rc1_flat.pdf} &
\includegraphics[page=2]{gfx/rc1/analysis/plots_flat_pop_output_cube_2pop_135_rc1_flat.pdf}
\end{tabular}}}
\label{app:flat_fiber_pop_a}
\end{figure}
%
\begin{figure}[!p]
\centering
\resizebox{\textwidth}{!}{
\rotatebox{90}{
\begin{tabular}{cc}
\includegraphics[page=3]{gfx/rc1/analysis/plots_flat_pop_output_cube_2pop_135_rc1_flat.pdf} &
\includegraphics[page=4]{gfx/rc1/analysis/plots_flat_pop_output_cube_2pop_135_rc1_flat.pdf}
\end{tabular}}}
\label{app:flat_fiber_pop_b}
\end{figure}
%
\begin{figure}[!p]
\centering
\resizebox{\textwidth}{!}{
\rotatebox{90}{
\begin{tabular}{cc}
\includegraphics[page=5]{gfx/rc1/analysis/plots_flat_pop_output_cube_2pop_135_rc1_flat.pdf} &
\includegraphics[page=6]{gfx/rc1/analysis/plots_flat_pop_output_cube_2pop_135_rc1_flat.pdf}
\end{tabular}}}
\label{app:flat_fiber_pop_c}
\end{figure}
%
\begin{figure}[!p]
\centering
\resizebox{\textwidth}{!}{
\rotatebox{90}{
\begin{tabular}{cc}
\includegraphics[page=7]{gfx/rc1/analysis/plots_flat_pop_output_cube_2pop_135_rc1_flat.pdf} &
\includegraphics[page=8]{gfx/rc1/analysis/plots_flat_pop_output_cube_2pop_135_rc1_flat.pdf}
\end{tabular}}}
\label{app:flat_fiber_pop_d}
\end{figure}
%
\begin{figure}[!p]
\centering
\resizebox{\textwidth}{!}{
\rotatebox{90}{
\begin{tabular}{cc}
\includegraphics[page=9]{gfx/rc1/analysis/plots_flat_pop_output_cube_2pop_135_rc1_flat.pdf} &
\end{tabular}}}
\label{app:flat_fiber_pop_e}
\end{figure}
%
%
% %%%%%%%%%%%%%%%%%%%%%%%%%%%%%%%%%%%%%%%%%%%%%%%%%%%%%%%%%%%
% INCLINED
% %%%%%%%%%%%%%%%%%%%%%%%%%%%%%%%%%%%%%%%%%%%%%%%%%%%%%%%%%%%
%
% \fakesection{Inclined crossing fiber population}%
%
\begin{figure}[!p]
\centering
\resizebox{\textwidth}{!}{
\begin{tabular}{c|c}
    \includegraphics[page=1]{gfx/rc1/analysis/plots_inclined_pop_hist_output_cube_2pop_135_rc1_inclined.pdf} &
    \includegraphics[page=2]{gfx/rc1/analysis/plots_inclined_pop_hist_output_cube_2pop_135_rc1_inclined.pdf} \\
    \includegraphics[page=3]{gfx/rc1/analysis/plots_inclined_pop_hist_output_cube_2pop_135_rc1_inclined.pdf} &
    \includegraphics[page=4]{gfx/rc1/analysis/plots_inclined_pop_hist_output_cube_2pop_135_rc1_inclined.pdf} \\
    \includegraphics[page=5]{gfx/rc1/analysis/plots_inclined_pop_hist_output_cube_2pop_135_rc1_inclined.pdf} &
    \includegraphics[page=6]{gfx/rc1/analysis/plots_inclined_pop_hist_output_cube_2pop_135_rc1_inclined.pdf} \\
    \includegraphics[page=7]{gfx/rc1/analysis/plots_inclined_pop_hist_output_cube_2pop_135_rc1_inclined.pdf} &
    \includegraphics[page=8]{gfx/rc1/analysis/plots_inclined_pop_hist_output_cube_2pop_135_rc1_inclined.pdf} \\
    \includegraphics[page=9]{gfx/rc1/analysis/plots_inclined_pop_hist_output_cube_2pop_135_rc1_inclined.pdf} &
    % \includegraphics[page=10]{gfx/rc1/analysis/plots_inclined_pop_hist_output_cube_2pop_135_rc1_inclined.pdf}
\end{tabular}
}
%
\caption[]{psi=0.3; left: 2D log histogramm orientation from rofl analysis of simulation, right: 2D log histogram of orientation of model segemnts. \todo{legende}}
\label{app:incl_fiber_pop_hist}
\end{figure}
%
%
%
\begin{figure}[!p]
\centering
\resizebox{\textwidth}{!}{
\rotatebox{90}{
\begin{tabular}{cc}
\includegraphics[page=1]{gfx/rc1/analysis/plots_inclined_pop_output_cube_2pop_135_rc1_inclined.pdf} &
\includegraphics[page=2]{gfx/rc1/analysis/plots_inclined_pop_output_cube_2pop_135_rc1_inclined.pdf}
\end{tabular}}}
\label{app:incl_fiber_pop_a}
\end{figure}
%
\begin{figure}[!p]
\centering
\resizebox{\textwidth}{!}{
\rotatebox{90}{
\begin{tabular}{cc}
\includegraphics[page=3]{gfx/rc1/analysis/plots_inclined_pop_output_cube_2pop_135_rc1_inclined.pdf} &
\includegraphics[page=4]{gfx/rc1/analysis/plots_inclined_pop_output_cube_2pop_135_rc1_inclined.pdf}
\end{tabular}}}
\label{app:incl_fiber_pop_b}
\end{figure}
%
\begin{figure}[!p]
\centering
\resizebox{\textwidth}{!}{
\rotatebox{90}{
\begin{tabular}{cc}
\includegraphics[page=5]{gfx/rc1/analysis/plots_inclined_pop_output_cube_2pop_135_rc1_inclined.pdf} &
\includegraphics[page=6]{gfx/rc1/analysis/plots_inclined_pop_output_cube_2pop_135_rc1_inclined.pdf}
\end{tabular}}}
\label{app:incl_fiber_pop_c}
\end{figure}
%
\begin{figure}[!p]
\centering
\resizebox{\textwidth}{!}{
\rotatebox{90}{
\begin{tabular}{cc}
\includegraphics[page=7]{gfx/rc1/analysis/plots_inclined_pop_output_cube_2pop_135_rc1_inclined.pdf} &
\includegraphics[page=8]{gfx/rc1/analysis/plots_inclined_pop_output_cube_2pop_135_rc1_inclined.pdf}
\end{tabular}}}
\label{app:incl_fiber_pop_d}
\end{figure}
%
\begin{figure}[!p]
\centering
\resizebox{\textwidth}{!}{
\rotatebox{90}{
\begin{tabular}{cc}
\includegraphics[page=9]{gfx/rc1/analysis/plots_inclined_pop_output_cube_2pop_135_rc1_inclined.pdf} &
\end{tabular}}}
\label{app:incl_fiber_pop_e}
\end{figure}