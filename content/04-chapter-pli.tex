\cleardoublepage
\setcounter{chapter}{3}
\chapter{3D-PLI}
\label{sec:pli}
%
\section{Introduction}
% 
This chapter gives an overview of \ac{3D-PLI}.
First, the necessary preparation of the brain tissue is described, including its cutting process and the subsequent covering.
The second part describes the \ac{3D-PLI} microscopic technique, which allows measuring the orientation of the optical axis that corresponds to the orientation of the myelinated nerve fibers.
The foundation of the measured \ac{3D-PLI} signal and the optical limits in microscopy are described as a basis for the simulation in the following chapters. 
% 
% 
% 
\section{Brain tissue preparation and sectioning}
% 
After the organism deatch, the brain is removed from the skull within $\SI{24}{\hour}$.
In order to prevent the further degeneration of the brain tissue, it is immersed in a solution of $\SI{4}{\percent}$ formaldehyde and $\SI{20}{\percent}$ glycerin for several days.
Then it is frozen at $\SI{-80}{\celsius}$ as preparation for sectioning.
Then the tissue is fixated with liquid glue on a plate, which allows to place it within a microtome.
The microtome consists out of a chamber cooled at about $\SI{-70}{\celsius}$.
A vibrating knife is used to section the brain into $\SI{60}{\micro\meter}$ sections (see \cref{fig:brain_sectioning}).
After each cut, the sections are manually placed onto a glass specimen, surrounded by glycerin and covered with a thin optical glass (see \cref{fig:brain_section}).
To preserve the sections for the microscopic measurement, they are placed in a freezer at approximately $\SI{-70}{\celsius}$.
The \ac{3D-PLI} measurement takes place at room temperature. \cite{Axer2011}
% 
\begin{figure}[!t]
	\centering
    \setlength{\tikzwidth}{0.475\textwidth}
    \subcaptionbox{%
        \label{fig:brain_sectioning}
        Illustration of the sectioning process. The brain is fixed with glue to stabilize it for the cutting process. A microtome is used to cut the tissue by using a vibrating diamond blade over which the tissue block is moved.
    }[\tikzwidth]{\inputtikz{gfx/neuroanatomy/brain_sectioning}}
    \hfill
    \subcaptionbox{%
        \label{fig:brain_section}
        A section is placed on a glass specimen, to protect it from environmental influences. It is covered with a second, thinner glass plate, which is sealed with nail polish.
    }[\tikzwidth]{\inputtikz{gfx/neuroanatomy/brain_section}}
	\caption{Brain sectioning and sealing illustation.}
\end{figure}
% 
% 
% 
\section{Experimental setup}\label{sec:expSetup}
%
In \ac{INM-1}, there are three microscopic setups based on identic physical principle \cite{Axer2011} (see \cref{fig:pli_setup}).
Polarized light with a wavelength of about $\SI{525}{\nano\meter}$ is irradiated through the tissue section \footnote{The exact wavelength depends on the microscope.}.
A circular polarizer, a $\SI{45}{\degree}$ oriented quarter-wave retarder, and polarizer are placed behind the tissue.
By rotating the polarizer, the change in intensity is measured with a \ac{CCD}-sensor.
\par
% 
The first experimental setup is called \ac{LAP}.
It is used to measure an entire brain section at a resolution of $\SI{60}{\micro\meter}$. \footnote{It is also possible to aquire images with a pixel size of $\approx \SI{40}{\micro\meter}$ and $\approx \SI{20}{\micro\meter}$ by changing the focal length.}
\par
% 
The \ac{LMP} microscope allows measuring a tile of $\num{2048}\times\num{2048}$ pixels with a pixel size of $\SI{1.3}{\micro\meter}$.
By measuring the tissue with multiple overlapping images, the overlapping tiles can be combined into an overall image with a stitching algorithm. 
This setup is not able to change the light path.
The 3D information can be estimated by analyzing the distribution of retardation and transmittance, however, the inclination sign cannot be detected.
\par
% 
\begin{figure}[!t]
    \captionsetup[sub]{position=top}
    \setlength{\tikzwidth}{\textwidth}
	\centering
    \inputtikz{gfx/pli/pli_setup_pm}
	\caption{Illustration of \acs{3D-PLI} setup.}
	\label{fig:pli_setup}
\end{figure}
% 
The third setup is the \ac{LMP3D} microscope \cite{Wiese:887678}.
By using a conical light path and a slit, only light of a particular angle can pass and therefore through the tissue.
By changing the position of the slit, different light paths can be applied with a maximal tilt angle of $\SI{3.9}{\degree}$.
\par
% 
The tilted light beam in the \acs{LAP} and the \acs{LMP3D} are measured at four perpendicular orientations: North, East, South and West.
\par
%
The final image is captured by a camera that uses a \ac{CCD} sensor, consisting of an array of \ac{MOS} capacitors.
Each capacitor stores an electric charge that is released by incident photons using the photoelectric effect.
After a readout process, which also includes electrical amplification, the resulting values can be stored as an image.
Its value, as long as the capacitors are not saturated or the amplification does not exceed its limits, is linearly correlated with the number of photons.
The signal, however, is affected by noise which comes mainly from two parts.
First is the thermal noise that can lead to electrical charges in the \ac{MOS} capacitors. 
Second, the amplification of the signal underlies a noise usually from a non-ideal direct current and non-ideal electric components like transistors, which is needed for the amplification process.
These noise sources combine to produce a poison-like distribution due to the nature of digital positive values produced by the analog-to-digital converter.
A normal distribution can model the intensity values $\gg 0$.
%
%
%
\section{Intensity signal}\label{sec::intSignal}
%
From the Mueller-Stokes matrices (\cref{sec:MuellerStokes}), the intensity signal, which is the first component of the Stokes vector, follows a sinusoidal curve \cite{MenzelMaster,MenzelDissertation}:
%
\begin{align}
\label[pluralequation]{eq:pli_signal}
\centering
\begin{gathered}
I(\rho, \varphi, \alpha, d) =\frac{I_0}{2}\bigl[ 1+ \sin\bigl(2\rho - 2\varphi \bigr)\cdot \sin \bigl( 2\pi\frac{d \dn}{\lambda} \cos^2\left( \alpha \right) \bigr) \bigr] \\
\text{with} \enspace \delta \coloneqq 2\pi\frac{d \dn}{\lambda} \cos^2\left( \alpha \right) \enspace
\text{and} \enspace \trel \coloneqq 4 \frac{d \dn}{\lambda}
\end{gathered}
\end{align}
%
\Cref{eq:pli_signal} describes a sinusoidal signal.
For a discrete, equidistant measurement of the rotation angles $\rho$, one can use the Fourier series with the first three coefficients to describe the signal:
%
\begin{align}
\begin{split}
\rho &= [0, \frac{\pi}{N+1}, \frac{2\pi}{N+1}, ..., \frac{N\pi}{N+1}]\\
a_0 &= \frac{1}{N} \sum_i^N I_i\\
a_1 &= \frac{2}{N} \sum_i^N I_i \cdot \sin(2 \rho_i)\\
b_1 &= \frac{2}{N} \sum_i^N I_i \cdot \cos(2 \rho_i)
\end{split}
\end{align}
%
The current routine measurements take $N=\SI{9}{}$ images.
These are used to calculate the final \ac{3D-PLI} modalities:
%
\begin{align}
\begin{split}
\mathit{transmittance} &\coloneqq 2 \cdot a_0 \vphantom{I_0/2} \\
\mathit{direction} &\coloneqq 0.5 \cdot \atantwo(-b_1 / a_1) \vphantom{\varphi} \\
\mathit{retardation} &\coloneqq \frac{\sqrt{a_1^2 + b_1^2}}{a_0}  \vphantom{\sin(\delta)}
\end{split}
\hspace{-7em}
\begin{split}
& \mathrel{\widehat{=}} I_0/2 \vphantom{2 \cdot a_0}\\
& \mathrel{\widehat{=}} \varphi \vphantom{0.5 \cdot \atantwo(-b_1 / a_1)} \\
& \mathrel{\widehat{=}} |\sin(\delta)| \vphantom{\frac{\sqrt{a_1^2 + b_1^2}}{a_0}}
\end{split}
\end{align}
%
The \textit{transmittance} describes the tissue light transmittance, \ie{}, how much light passes through the tissue.
The \textit{direction} describes the phase of the signal, corresponding to the direction of the macroscopic optical axis and therefore the direction of the nerve fibers.
The \textit{retardation} corresponds to the amplitude of the signal, \ie{}, the strength of the retardation.
%
\begin{figure}[t]
\setlength{\tabcolsep}{0pt}
\setlength{\tikzwidth}{0.43\textwidth}
\tikzset{external/export=false}
\begin{tabular}{cc}
% 
\begin{tikzpicture}[baseline, trim axis left, trim axis right]
\begin{axis}[
width=\tikzwidth,
axis equal image,
enlargelimits=false,
scale only axis,
point meta min=6851,point meta max=13703,
% xmin=0,xmax=947,ymin=0,ymax=727,
hide axis,
colorbar horizontal,colormap/cividis,
colorbar style={%
    every x tick label/.append style={font=\footnotesize},
    xtick={6851,13703},
    xticklabels={0,$I_{\mathit{max}}/2$},
    scaled x ticks=false,
    }
]
\addplot [forget plot] graphics [
xmin=0,xmax=947,ymin=0,ymax=727,
] {gfx/vervet1818/Vervet1818_60mu_70ms_s0550_a00_d000_Transmittance_6851_13703_.png};
\end{axis}
\end{tikzpicture}
&
\begin{tikzpicture}[baseline, trim axis left, trim axis right]
\begin{axis}[
width=\tikzwidth,
axis equal image,
enlargelimits=false,
scale only axis,
point meta min=0,point meta max=180,
% xmin=0,xmax=947,ymin=0,ymax=727,
hide axis,colorbar horizontal,colormap/twilight,
colorbar style={%
    every x tick label/.append style={font=\footnotesize},
    xtick={0,90,180},
    xticklabels={$\SI{0}{\degree}$,$\SI{90}{\degree}$,$\SI{180}{\degree}$},
    }]
\addplot [forget plot] graphics [
xmin=0,xmax=947,ymin=0,ymax=727,
] {gfx/vervet1818/Vervet1818_60mu_70ms_s0550_a00_d000_Direction_0_180_.png};
\end{axis}
\end{tikzpicture}
\\[-1.5ex]
% 
\multicolumn{1}{l}{
\begin{minipage}{0.5\textwidth}
\leavevmode\subcaption{transmittance}
\end{minipage}}
&
\multicolumn{1}{l}{
\begin{minipage}{0.5\textwidth}
\leavevmode\subcaption{direction}
\end{minipage}}
\\[1.5em]
% 
\begin{tikzpicture}[baseline, trim axis left, trim axis right]
\begin{axis}[
width=\tikzwidth,
axis equal image,
enlargelimits=false,
scale only axis,
point meta min=0,point meta max=0.858,
% xmin=0,xmax=947,ymin=0,ymax=727,
hide axis,colorbar horizontal,colormap/cividis,
colorbar style={%
    every x tick label/.append style={font=\footnotesize},}]
\addplot [forget plot] graphics [
xmin=0,xmax=947,ymin=0,ymax=727,
] {gfx/vervet1818/Vervet1818_60mu_70ms_s0550_a00_d000_Retardation_0_0858_.png};
\end{axis}
\end{tikzpicture}
&
\begin{tikzpicture}[baseline, trim axis left, trim axis right]
\begin{axis}[
width=\tikzwidth,
axis equal image,
enlargelimits=false,
scale only axis,
% xmin=0,xmax=947,ymin=0,ymax=727,
hide axis,
]
\addplot graphics [
xmin=0,xmax=947,ymin=0,ymax=727,
] {gfx/vervet1818/Vervet1818_60mu_70ms_s0550_a00_d000_FOM_HSVBlack.png};
\addplot [forget plot] graphics [
xmin=0,xmax=125,ymin=602,ymax=727,
] {gfx/pli/color_sphere.png};
\end{axis}
\end{tikzpicture}
\\
% 
\multicolumn{1}{l}{
\begin{minipage}{0.5\textwidth}
\leavevmode\subcaption{retardation}
\end{minipage}}
&
\multicolumn{1}{l}{
\begin{minipage}{0.5\textwidth}
\leavevmode\subcaption{\label{fig:exampleFom}fiber orientation map (FOM)}
\end{minipage}}
\end{tabular}

\caption[3D-PLI modalities]{3D-PLI modalities for a coronal section of a Vervet monkey brain.}
\end{figure}
%
%
\section{Tilting analysis} \label{sec::InclAnalysis}
%
To be able to analyse the optical axis inclination $\alpha$, one has to distinguish the relative strength of the birefringence from the term $\cos^2(\alpha)$.
For this purpose, a tiltable sample was developed that allows the light signal to be measured through the tissue at a different angle of incidence \cite{Axer2011, Wiese:887678} (see \cref{sec:expSetup}).
This means that the tissue, and therefore the nerve fibers, change their orientation due to the tilt angles $\theta, \varphi$.
In addition, the distance the light travels through the tissue increases by $1/\cos(\theta)$ (see \cref{fig:tilted_side_view}).
\par
%
Depending on the \Pixelsize{}, this light passes through the same volume but with a different orientation.
Therefore, a measurement of multiple light paths can be captured, and the resulting signals can be used to analyse the inclination $\alpha$ and relative birefringence tissue thickness \trel{}.
The angle of incidence of the light changes the path of the light by Snellius law \cref{eq:Snellius}.
This results in a perspective shift that must be corrected by a registration process.
However, this effect is neglected in the simulation, since it only adds a parallel offset.
\par
%
To analyse the tilting signal, an algorithm was developed under the name \ac{ROFL} \cite{Wiese:887678,Schmitz2018}.
The idea is to fit the measured signals of all light paths simultaneously.
However, since the change in the signal is proportional to $\cos(\alpha)$, this means that for steep fibers not only are the changes $\frac{\partial}{\partial \alpha} \cos(\alpha)$ small, but also the amplitude of the signal is very small.
This leads to the problem of increasing uncertainty with increasing inclination angle.
\par
%
Another difficulty is that for a smaller \Pixelsize{} the light path can no longer be neglected.
For the \ac{LMP3D}-system, this means that for an expected tilt angle of about $\SI{3.9}{\degree}$ and a tissue thickness of $\SI{60}{\micro\meter}$, the light path is measured about $\SI{4}{\pixel}$ away from the non-tilting case when the center of rotation is in the middle of the tissue.
Additionally these means, that the light from the different tilt views has effected by other parts inside the tissue.
For homogeneous tissue regions, such as parts in the dense \ac{WM}, this can be neglected.
However, for single fiber paths such as visible in the \ac{GM} or complicated paths like in crossing the effect on th inclination analysis is an open question.
%
%
%
\section{Optical resolution}
\label{sec:opticalResolution}
%
The optical resolution of an imaging system describes the minimum size of an object that can still be resolved.
This property is limited by aberration and diffraction.
Aberration causes blurring of the image, while diffraction can lead to superimposed diffraction patterns.
If diffraction is caused by many small objects in relation to the resolution, this also looks like blurring.
\par
%
Ernst Abbe was one of the first to describe that the resolution correlates with the light wave $\lambda$:
\begin{align}
d=\frac{ \lambda}{2 n \sin \theta} = \frac{\lambda}{2\mathrm{NA}} 
\end{align}
$d$ is the minimum resolvable distance, $n$ the refractive index, $\theta$ the angle of a light spot, which can be combined to the better known numerical aperture $\mathrm{NA}$.
A more common definition is Rayleigh limit given by
\begin{align}
d=1.22\frac{\lambda}{2\mathrm{NA}} 
\end{align}
, where $d$ is the distance between two light spots, where the first light spots maxima is in the seconds light spots first minima (see \cref{fig:rayleigh}).
For the wavelength used in \ac{3D-PLI} with $\lambda = \SI{525}{\nano\meter}$ and a numerical aperture of $\mathrm{NA} \approx \SI{0.15}{}$, which results in a limit of about $\SI{2.1}{\micro\meter}$ (see \cite{MenzelDissertation}).
%
\begin{figure}[!t]
\setlength{\tikzwidth}{0.5\textwidth}
\centering
% \subcaptionbox{}{
% \tikzset{external/force remake=true}
\inputtikz{gfx/pli/rayleigh}
\caption[Raylay criterium]{Rayleigh's criteria. The minima of the one function is in the maxima of the other.}
\label{fig:rayleigh}
\end{figure}
%
To account for optical setup, three things must be applied to a simulated measurement.
%
\paragraph{Blurring}
To account for the blurring or out of focus image, the light rays intensity must be blurred on the \ac{CCD}-array.
This is done via a 2D Gaussian convolution $g$ of the image $f$:
\begin{align}
\begin{split}
    (f * g)(x,y) =& \iint f(\tau,\upsilon) \cdot g(x-\tau, y-\upsilon)d\tau \, d\upsilon\\
    g(x,y) =& \frac{1}{2\pi\sigma^2} \exp(-\frac{x^2+y^2}{2\sigma^2})
\end{split}
\end{align}
%
\paragraph{Sampling}
Since the number and final position of the light rays correspond to the voxels (see \cref{sec:simulation}), all intensities of an image pixel must be combined.
Here, this is done via a mean value scan:
\begin{align}
    \hat{I}(n,m) =\frac{1}{N_x \cdot N_y} \sum_{i=n \cdot N_x}^{(n+1) \cdot N_x-1}\sum_{j=m \cdot N_y}^{(m+1) \cdot N_y-1} I(i,j)
\end{align}
Unlike resizing, this does not interpolate the image.
%
\paragraph{Noise}
The final step is to recreate the noise of the image composition.
To account for this, a noise model must be applied to each image pixel.
\cite{Wiese:887678} showed that this can be modelled via a normal distribution:
%
\begin{align}
    I = \gauss(\sigma = \sqrt{I*gain}, \mu=I)
\end{align}
%
\par
%
All three effects must be characterized for the system being simulated.