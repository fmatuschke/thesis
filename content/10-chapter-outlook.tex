\newpage\null\thispagestyle{empty}\newpage
\clearpage{\thispagestyle{empty}\cleardoublepage}
\part{Next/Finalize/Closing}
% 
% 
% 
\setcounter{chapter}{9}
\chapter{Outlook}
\label{sec:outlook}
% 
\paragraph{Nerve fiber modelling software}
% 
Currently, the collisionless nerve fiber modeling algortihm is capable of generating densly packed fiber models which are suitable nerve fiber models for \ac{3D-PLI} simulations.
However, the algorithm is limited in terms of computational time or more specific, the number of objects in the volume to be solved.
Since the number of individual nerve fiber segments increases with volume, it is not possible to generate large volumes in a reasonable time.
\par
% 
There are optimization options to improve performance.
First, one can improve the sandbox design models so that the initial fiber configurations have less overlap.
This would also mean that for a given volume, the number of objects is reduced.
However, it was found that in the first $\approx \SI{10}{\percent}$ most collisions can be solved and the remaining time is needed for the remaining minimal overlaps \cref{sec:solverParameterResults}.
Therefore, the total runtime would probably not decrease significantly.
\par
% 
A more suitable strategy is to design the algorithm in such a way that the number of calculation steps is reduced.
However, this is a very complicated task.
For example, one could increase the motion that a fiber segment is allowed to perform.
This strategy would probably lead to less densely packed models, since without an external force or attraction between the nerve fibers, they can only move apart. \footnote{With an external force one would introduce oscilations.}
Another possibility is to speed up the runtime per step.
This can be done, for example, by choosing a simpler calculation.
In the \ac{MEDUSA} algorithm this is already been realized by using spheres instead of fiber segments.
The collision of spheres is only be calculated by the euclidian distance with respect to the sum of the spheres radii.
However, the use of spheres is associated with a disadvantage.
The number of objects increases dramatically as many spheres are needed to approximate the surface of a single conical fiber segment.
\par
% 
A current limitation of the algorithm presented here is parallelization on a multi-core system.
For shared memory parallelization, atomic operations must be introduced, which takes a lot of time.
For an algorithm without shared memory, the required data must be exchanged between the individual CPUs. 
This communication introduces an overhead compared to the \ac{CPU} instructions.
However, such an algorithm could work on a multi-node system if the performance does not decrease drastically with increasing number of CPUs involved, as is currently the case.
Therefore, one needs to redesign the algorithm.
Common solutions are to divide the volume into subvolumes (as in the case of \ac{3D-PLI} simulation), where each volume can be computed separately.
However, since the boundaries of such subvolumes contain objects from neighboring volumes, it is necessary to merge the information of neighboring subvolumes.
In a sense, this is already happening in the current implementation of the octree, but the volume is split into subvolumes at each step.
If this can be improved so that only the necessary information needs to be transmitted, performance should increase by quite a bit.
\par
% 
At this point, the most promising optimization would be to use the architecture of the \ac{GPU} as in the \ac{MEDUSA} algorithm.
One type of these algorithms that is particularly appropriate here is based on the \ac{AABB} used here and a z-ordered tree instead of an octree \cite{Karras2012}.
Not only is the collision checking computed in parallel, but also the generation of the z-ordered tree is done in parallel on the \ac{GPU}.
Also the algorithm completly runs on the \ac{GPU} and therefore does not need to communicate with the \ac{CPU} or \ac{RAM} as long as the complete information fits on the \ac{GPU} memory.
This drastically increases the speed of collision checking.
% 
% 
% 
\paragraph{\ac{3D-PLI} software}
% 
As part of this work, two proof-of-concept projects were conducted to develop a parallel GPU implementation for \ac{3D-PLI} simulations.
The first was a seminar project to implement a parallel discrete tissue volume computation algorithm for the \ac{3D-PLI} simulation on the \ac{GPU} architecture \cite{Kobusch:Seminar}.
It was shown that the speedup of the discrete volume computation was very fast, but the large memory requirements of the discretized volume negated this speedup overall, as \acp{GPU} are relatively limited here.
In addition, the need to transfer the data back to \ac{RAM} was too much of an overhead.
Therefore, the second project was to implement a ray tracing algorithm that computes a light matter collision and matrix computation without pre-computing the discretized volume \cite{Kobusch:887783}.
Therefore a light particle only computes the M\"{u}ller-Stokes calculus if it collides fith a fiber object.
This second project includes a proof of concept that implements a simple collision detection algorithm on the \ac{GPU}, a uniaxially aligned collision search algorithm \cite{Karras2012}.
The results showed that the acceleration possibilities on the \ac{GPU} are enormous.
However, due to the rather simple algorithm, the runtime was slightly longer, but still in the same order of magnitude as the \ac{CPU} version.
\par
% 
The next logical step is therefore to use a more sophisticated algorithm.
As in the case of the fiber collision detection algorithm, the same collision detection algorithm and z-ordered tree can be used to detect collisions between light particles and fiber segments \cite{Karras2012}.
This has several advantages.
The most important is that the discretized volume is no longer necessary, which saves a lot of memory.
With such an implementation, a normal computing system should already be fast enough for reasonable simulation sizes.
Additionally the parallel algorithm developed above also allows light rays to change their orientation.
This will allow simulations of light scattering in the future.
% 
% 
% 
\paragraph{Nerve fiber modelling and \ac{3D-PLI} simulation}
% 
To increase the statistical significance of the results, the number of nerve fiber models and simulations should be increased.
Since two models are independent, all models can be generated in parallel using the entire computer architecture.
However, a more appropriate step would be to determine the essential parameters for the models and simulations, thus reducing the number of model generation required.
Modern machine learning algorithms are a suitable tool for this purpose.
\par
% 
Another important task is the study of larger nerve fiber radii.
This would include the larger nerve fiber radii that are anatomically present in the brain.
In addition, larger nerve fiber radii could also be an approximation for a nerve fiber bundle consisting of multiple nerve fibers in the \ac{3D-PLI} simulation.
They should then be simulated with a macroscopic birefringent model \cite{Menzel2015}.
If this is possible, one could significantly reduce the number of objects for model generation, which means that one could have either very small run times or larger volumes.
Further optimization can be achieved by using nerve fiber models that are not fully solved.
If the effect is negligible in the \ac{3D-PLI} simulation, the runtime for the models can be significantly reduced.
\par
% 
In addition, more complex models, \eg{} three fiber populations, should be investigated.
Furthermore, boundary regions consisting of neighboring nerve fiber tracts must also be investigated because they significantly change the signal when the light beam is tilted.
This change is currently not accounted for by tilting analysis.
Here, the use of machine learning can be used to find a new characterization of the \ac{3D-PLI} signals and better identify the underlying fiber structure.
In \ac{dMRI}, nerve fiber models and their simulations, as well as the use of deep learning, have already shown that the underlying fiber structure can be identified from the original signals \cite{ginsburgerDis2019}.
This can potentially be applied to \ac{3D-PLI} in a similar way.
% 
\par
These type of models also can be used in other field of research.
Since all these models focus on non-overlapping tubular structures, they could already be applied to skeletal muscle fiber \cite{Ji2021}.