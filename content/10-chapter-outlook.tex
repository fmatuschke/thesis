\newpage\null\thispagestyle{empty}\newpage
\clearpage{\thispagestyle{empty}\cleardoublepage}
\part{Next/Finalize/Closing}
% 
% 
% 
\setcounter{chapter}{9}
% \chapter{Future guidance - where to go next?}
\chapter{What's next?}
% Recommendations/ Enhancements/ Recommended development/ Future strategy/ Follow-up .../ Future advice/ Further guidance/ Where to go?
\label{sec:outlook}
% 
\paragraph{Nerve fiber modelling software}
% 
The collision-free nerve fiber modelling algorithm can generate densely packed fiber models, which are suitable for \ac{3D-PLI} simulations.
However, for larger volumes, the algorithm is limited in terms of computation time or, more precisely, the number of objects in the volume to be solved.
\par
% 
However, there are optimization options available to improve performance.
First, the models created by the sandbox can be improved so that the initial fiber configurations have less overlap. 
This means that for a given volume, the number of objects and thus the runtime are reduced. 
Additionally it was discovered that most collisions can be solved in the first $\approx \SI{10}{\percent}$ of the runtime, and the remaining time is required for the remaining minimal overlaps (see \cref{sec:solverParameterResults}).
\par
% 
A second strategy is to design the algorithm in such a way that the number of calculation steps is reduced.
For example, one could increase the motion that a fiber segment is allowed to perform.
This strategy would probably lead to less densely packed models, since without an external force or attraction between the nerve fibers, they can only move apart. \footnote{With an external or attracting force one would introduce oscilations.}
% 
\par
Another possibility is to speed up the runtime per step.
This can be done, for example, by choosing a simpler calculation.
In the \ac{MEDUSA} algorithm this is already been realized by using spheres instead of fiber segments.
The collision of spheres is only be calculated by the euclidian distance with respect to the sum of the spheres radii.
However, the use of spheres is associated with a disadvantage.
The number of objects increases dramatically as many spheres are needed to approximate the surface of a single conical fiber segment.
\par
% 
A current limitation of the algorithm is the parallelization on a multicore system.
For shared memory parallelization, atomic operations must be introduced, which takes a lot of time.
For an algorithm without shared memory, the required data must be exchanged between the individual CPUs, which takes also quite some time. 
Therefore, one needs to redesign the algorithm to limit the necessary locks or data exchange.
Common solutions are to divide the volume into sub-volumes (as in the case of \ac{3D-PLI} simulation), where each volume can be computed separately.
The boundaries of such sub-volumes contain then the objects from neighboring volumes (halo).
Therefore only the information of the halo needs to be exchanged.
\par
% 
At this point, the most promising optimization is to use the architecture of the \ac{GPU} as \eg{} in the \ac{MEDUSA} algorithm.
An algorithm exists, which operates on \ac{AABB} of the objects and uses a z-ordered tree instead of an octree \cite{Karras2012}.
This algorithm also has the advantage that not only the collision checking computation is executed in parallel on the \ac{GPU}, but also the generation of the z-ordered tree.
The entire algorithm runs on the \ac{GPU} and therefore does not need to communicate with the \ac{CPU} or \ac{RAM} as long as the complete information fits on the \ac{GPU} memory.
This will drastically increase the speed of the collision checking.
The remaining steps of the modeling algorithm, such as the movement of the fiber segments, can also be easily performed in parallel on the \ac{GPU}.
% 
% 
% 
\paragraph{\ac{3D-PLI} software}
% 
As part of this work, two proof-of-concept projects were conducted to develop a parallel GPU implementation for \ac{3D-PLI} simulations.
The first was a seminar project to implement a parallel discrete tissue volume computation algorithm for the \ac{3D-PLI} simulation on the \ac{GPU} architecture \cite{Kobusch:Seminar}.
It was shown that the speedup of the discrete volume computation was very high, but the large memory requirements of the discretized volume negated this speedup overall, as \acp{GPU} are relatively limited here.
In addition, the need to transfer the data back to \ac{RAM} was too much of an overhead.
Therefore, the second project was to implement a ray tracing algorithm that computes a light matter collision and matrix computation without pre-computing the discretized volume \cite{Kobusch:887783, GPUfastpli}.
A light particle only computes the M\"{u}ller-Stokes calculus only if it collides fifth a fiber object.
This means, that a collision algorithm needs to be used.
As a first implementation a collision detection algorithm on the \ac{GPU}, an uniaxially aligned collision search algorithm \cite{Karras2012} was implemented.
The results showed that the acceleration possibilities on the \ac{GPU} are enormous.
However, due to the rather simple collision checking algorithm, the runtime was slightly longer, but still in the same order of magnitude as the \ac{CPU} version.
As with fiber generation, implementing the z-order algorithm mentioned above \cite{Karras2012} would be a vast improvement and solve the runtime issue.
%
\par
This second work also implemented the algorithm is such a way, that the light rays can have any direction in space.
This allows for the future an implementation of directional change such as it would be necessary for light scattering simulations.
% 
% 
% 
\paragraph{Nerve fiber modelling and \ac{3D-PLI} simulation}
% 
As a first achievement, the number of models was limited to a reasonable number.
The next study can increase the number of nerve fiber models and simulations should be increased.
Since two models are independent, all models can be generated in parallel using the entire computer architecture.
In addition, the essential parameters for the models and simulations should be further investigated so that the number of required models can be reduced.
Modern machine learning algorithms are a suitable tool for this purpose.
\par
% 
Another important task is the study of larger nerve fiber radii.
This would include the larger nerve fiber radii that are anatomically present in the brain.
In addition, larger nerve fiber radii could also be an approximation for a nerve fiber bundle consisting of multiple nerve fibers in the \ac{3D-PLI} simulation.
They should then be simulated with a macroscopic birefringence model \cite{Menzel2015}.
If this is possible, one could significantly reduce the number of objects for model generation, which means that one could have either very small runtimes or larger volumes.
\par
% 
A further study should investigate if nerve fiber models can be used for \ac{3D-PLI} simulation, which are not fully collision free.
Since about $\SI{90}{\percent}$ of the runtime is used on the last small remaining colliding objects, this would reduce the runtime by an order of magnitude.
\par
% 
Another interesting study is to investigate more complex models, \eg{} three fiber populations or boundary regions consisting of neighboring nerve fiber tracts.
This can be very interesting since this could improve the understanding how the signal changes when the light beam is tilted and therefore improve the tilting analysis further.
For more complex models, machine learning can possibly be used to identify the underlying fiber structure within a voxel using the \ac{3D-PLI} signals. 
In \ac{dMRI}, nerve fiber models and their simulations, as well as the use of deep learning, have already shown that the underlying fiber structure can be identified from the original signals \cite{ginsburgerDis2019}.
This can potentially be applied to \ac{3D-PLI} similarly.
\par
% 
By publishing the code as open access software other researcher could already use the fiber modelling software to generate non colliding fiber models and use them as skeletal muscle fiber models \cite{Ji2021}.