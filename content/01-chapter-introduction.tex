\setcounter{chapter}{0}
\chapter{Introduction}
\label{sec:intro}
%
\cleanchapterquote{Inside you is the potential to make yourself better ... and that is what it is to be human.
To make yourself more than you are.}{Jean-Luc Picard}{Captain USS Enterprise NCC-1701-E} %Captain USS Enterprise NCC-1701-E
%
% \begin{itemize}
%    \item Present the problem and the proposed solution
%    \item Presents nature and scope of the problem investigated
%    \item Reviews the pertinent literature to orient the reader
%    \item States the method of the experiment
%    \item State the principle results of the experiment
% \end{itemize}

% \begin{itemize}
%    \item Indicate the field of the work, why this field is important, and what has already been done (with proper citations).
%    \item Indicate a gap, raise a research question, or challenge prior work in this territory.
%    \item Outline the purpose and announce the present research, clearly indicating what is novel and why it is significant.
%    \item Avoid: repeating the abstract; providing unnecessary background information; exaggerating the importance of the work; claiming novelty without a proper literature search. 
% \end{itemize}
% 
Neuroscience is the scientific study of the nervous system.
The creation of a map of the nerve fiber tracts occupies a special position.
In the past, these were only studied postmortem. 
In modern times, diffusion MRI was added and is frequently used clinically.
However, it suffers from a resolution problem that cannot be readily increased in vivo.
Post-mortem studies currently achieve resolutions of a few hundred micrometers for an entire human brain.
Smaller fiber bundles or even individual fibers are lost at this resolution.
Other imaging techniques, such as clearing, make it possible to track individual nerve fibers even in small volumes (a few centimeters).
However, for larger volumes such as the entire human brain, the procedure is very difficult.
A special position is occupied by \ac{3D-PLI}, which allows the measurement of orientation on brain slices by the polarization change of light on myelinated nerve fiber tracts.
\par
% 
In \ac{3D-PLI} currently has challenges are
\begin{itemize}
   \item what is the underlying nerve fiber structure.
   \item what are the possible fiber orientations in crossing or stepp inclined nerve fiber regions
   \item distribution of orientations
\end{itemize}
where simulations could and can provide an answer.
\par
% 
The dissertation provides a novel open-source software package \ac{fastPLI} whose main purpose is to provide a method for densely non-colliding 3D nerve fiber models and to simulate such models in a linear optics using the M\"{u}ller-Stokes calulus of the \ac{3D-PLI} experiment.
Another focus is the use of supercomputing architectures with efficiently developed algorithms to enable simulations of larger models and volumes.
The usefulness will be investigated and limitations will be demonstrated.
% 
% 
%
\section{Global todos}
%
\cite{Angles2019}\\
\cite{Callaghan2019}\\
%
\todo{twoside = true, BCOR=25mm}
\todo{alle axen gleiche schriftgroese, small?}