\cleardoublepage
\setcounter{chapter}{6}
\chapter{\acs{fastPLI}}
\label{chap:Software}
% 
% 
% 
\section{Introduction}\label{sec:fastpliIntro}
%
The previous chapters described the algorithms for creating dense \ac{WM} fiber models (see \cref{chap:sof:modelling}) and for simulating such models in \ac{3D-PLI} (see \cref{cha:sof:simulation}).
Both algorithms are designed to operate autonomously without knowledge of the other.
This can simplify the use of the algorithms for other domains.
For example, the fiber models can be used in \ac{dMRI} as well (\cite{Ginsburger2019,ginsburgerDis2019}).

These algorithms are designed to be fast.
This usually means code design with very few abstractions.
Therefore, an \ac{API} is needed on top of the algorithms, which provide a user interface which is easy to use.
Among other things, this means a high level of abstraction.
In addition, the algorithms should be easily installable to provide easy access.
\par
%
In summary, this means that a software package must be developed that contains the algorithms and helper functions.
For the above reasons, the \python{} programming language was chosen.
It has become increasingly popular over the last decade, especially in data science.
The core algorithms remain in \cpp{} as described in their chapters to ensure efficiency, speed, and parallelization.
%
% 
% 
\section{fastPLI Toolbox}
%
\begin{figure}[!ht]
\centering
\inputtikz{gfx/fastpli/fastpli_pipeline}
\caption{\acs{fastPLI} package structure.}
\label{fig:fastpli}
\end{figure}
%
The here designed \python{} package is called \acreset{fastPLI} \ac{fastPLI}.
Its source code is publicly available and reviewed in the \ac{JOSS} \cite{fastpli,Matuschke2021}.
The software package includes functionalities for the analysis and visualization of the nerve fiber models as well as for the analysis of the simulation analogous to the current routine experimental measurements, \eg{} tilt analysis (see \cref{fig:fastpli}).
%
%
%
\subsection{Documentation}
%
\begin{figure}[!t]
    \centering
    \resizebox{\textwidth}{!}{\fbox{
    \begin{tabular}{c|c}
    \includegraphics[valign=T,trim=0 1300 0 0, clip]{gfx/fastpli/fastpli_wiki.png} &
 	\includegraphics[valign=T,trim=0 0 0 1580, clip]{gfx/fastpli/fastpli_wiki.png} \\
    \end{tabular}
    }}
	\caption{Documentation wiki page of the \name{Github} repository \url{https://github.com/3d-pli/fastpli/wiki}.}
	\label{fig:fastpli_wiki}
\end{figure}
%
All methods are provided with documentation strings (docstrings)
These are automatically displayed by modern editors during programming, for example, to provide direct assistance during programming.
These docstrings are also used for an automatic release of the \ac{API} documentation.
\footnote{\url{https://3d-pli.github.io/fastpli/}}
In addition to the API, there is a wiki page (see \cref{fig:fastpli_wiki}) that describes the main features, which is an essential part of the review process for release in \ac{JOSS} \cite{Matuschke2021}. 
\footnote{review openly accessible at \url{https://github.com/openjournals/joss-reviews/issues/3042}}
The wiki page is structured as a guide that walks through the aspects of designing nerve fiber models, applying the collision solver algorithm, visualizing nerve fibers, an introduction in \ac{3D-PLI}, and finally applying the models in the simulation.
Both executable \python{} scripts and Jupyter notebooks are provided as examples to get users started quickly.
As a general example using all presented methods, a nerve fiber crossing is presented.
This is based on the optic chiasm which is the nerve fiber pathway from the eyes to the occipital lobe.
%
%
% 
\newpage
\subsection{Dependencies}
%
\paragraph{Python:}
\begin{description}
\item[numpy:] Base N-dimensional array package \cite{2019arXiv190710121V}\\
\url{https://numpy.org/}
\item[scipy:] Fundamental library for scientific computing \cite{2019arXiv190710121V}\\
\url{https://www.scipy.org/}
\item[numba:] Acceleration of Python Functions \cite{Lam2015}\\
\url{https://numba.pydata.org/}
\item[mpi4py:] MPI for Python \cite{Dalcn2005, Dalcn2008, Dalcin2011}\\
\url{https://bitbucket.org/mpi4py/mpi4py/src/master/}
\item[h5py:] HDF5 for Python \cite{collette_python_hdf5_2014, hdf5}\\
\url{https://www.h5py.org/}
\end{description}
%
\paragraph{C++:}
\begin{description}
\item[MPI:] Message Passing Interface \cite{message2015mpi}\\
\url{https://www.mpi-forum.org/}
\item[OpenMP:] Open Multi-Processing, API for multi-platform shared memory multiprocessing programming \cite{dagum1998openmp}\\
\url{https://www.openmp.org/}
\item[OpenGL:] Open Graphics Library \cite{khronos}\\
\url{www.opengl.org}
\item[Pybind11:] Seamless operability between C++11 and Python \cite{pybind11}\\ \url{https://github.com/pybind/pybind11}
\end{description}
%
%
At this point, only Linux builds are supported.
However, for current Windows versions ($10\geq$), the \ac{WSL} provides a fully functional Linux kernel within Windows.
This makes it possible to run the same software as on native Linux distributions.
Current macOS versions are not supported, but due to the minimalistic style of the \ac{fastPLI} package, the required changes should be feasible with minimal modifications.
%
%
%
\subsection{Installation}
%
The installation instructions are scripted in a \code{Makefile}.
It first starts a \name{CMake} routine which searches for all the required libraries and programs.
Then the \cpp{} code is compiled and the resulting \name{shared object libraries} are stored in the \python{} routines.
Finally, the provided code \code{setup.py} allows the user to install the compiled package in his environment.
%
\begin{lstfloat}[!ht]
\lstset{style=common}
\begin{lstlisting}
make fastpli
pip3 install .
\end{lstlisting}
\end{lstfloat}
%
%
%
\subsection{Tests, verification \& issue tracking}
%
Each module with its main methods is automatically tested with a \name{Github action} \footnote{\name{Github actions} are commonly used to automatically build, test, and deploy the software and documentation}. after each \name{git push}.
\footnote{uploud of the current \textit{software stage}.}
This action runs the two latest Ubuntu Long Term Support versions (currently 18.04 LTS and 20.04 LTS) and the most commonly used Python3 versions (currently 3.6 and 3.8) to provide a wide range of supported common versions.
In addition, the \name{Github actions} run all test scripts, check tutorial files, check code format and linting for consistency, and publish the latest documentation after a sucessfull tested release.
\par
%
\name{Github} allows to tracking \name{Issues}.
This feature is originally used to document software bugs.
However, it is also used to discuss ideas, new features, and so on.
As part of the open source release, it was also used communicate with the reviewer.
\footnote{\url{https://github.com/openjournals/joss-reviews/issues/3042}}
This allows to track the development process.
%
% 
% 
\section{Modules}
%
A \python{} \name{package} consists of \name{modules} which contain the definitions of functions, classes and so on (see \cref{fig:fastpli}).
In the following the different modules are listed alphabetically.
%
%
%
\subsection{\Code{fastpli.analysis}}
%
This module contains all functionalities to analyze the \ac{3D-PLI} simulations analogous to the routine measurements.
This includes the analysis of the signal to the three image modalities transmission, direction and retardation.
Furthermore, it provides the tilt analysis \ac{ROFL} \cite{Schmitz2018}.
In addition, further helper functions exist that provide methods to convert the direction and tilt results into a \ac{FOM}.
\par
% 
For the analysis of fiber models, the module providing a few simple helper functions.
This for example allow the user to generate a histogramm of the orientations of the fiber segments like the ones shown in this thesis.
%
% 
% 
\subsection{\Code{fastpli.io}}
%
This method provides the read and write routines that allow the user to load and save fiber models (\ie{},\code{fiber\_bundles}) to or from disk.
There are two formats available.
The first is a text file with the extension \code{.dat} (see \cref{alg:dat-file}).
Here, each $(x,y,z,r)$ tuple of a fiber point is stored as a single line in the file.
Two fibers are separated by one blank line, while two fiber bundles are separated by two blank lines.
This data format is provided to allow a very simple format for manipulating, exchanging and \eg{} reading the files into other programs.
\par
%
\begin{lstfloat}[!ht]
\lstset{style=common,morecomment=[l][\color{syntax_green}]{##},}
\begin{lstlisting}
-6.55 -18.93 -64.98 3.75 # x y z r
-5.73 -14.89 -63.37 3.4
-4.42 -13.66 -58.95 3.05
                         # empty line indicates new fiber
-1.96 -10.07 -52.5 2.92
-1.03 -9.4 -48.62 2.93

                         # two empty lines indicates new fiber bundle
3.4 -4.02 -44.76 3.11
6.22 -1.04 -42.45 3.26
\end{lstlisting}
\caption{Exemplary \name{.dat} file format. Comments are not allowed.}\label{alg:dat-file}
\end{lstfloat}
%
The second format uses \ac{HDF5} \cite{hdf5} which uses a binary data format.
\ac{HDF5} allows the data to be stored as \name{datasets} in \name{groups}.
This is analogous to a file in an operating system being stored in folders.
The \ac{HDF5}-\name{groups} are used to store the \code{fiber} in \code{fiber\_bundle} and \code{fiber\_bundles}.
The $(x,y,z,r)$ information of each fiber is then stored as a 2d-array (see \cref{alg:hdf5}).
%
\begin{lstfloat}[!ht]
\lstset{style=common,morecomment=[l][\color{syntax_green}]{##},}
\begin{lstlisting}
GROUP "/" { # fiber_bundles path
  GROUP "0" { # id of fiber_bundle
      DATASET "0" { # id of fiber
         DATATYPE  H5T_IEEE_F64LE
         DATASPACE  SIMPLE { ( 3, 4 ) / ( 3, 4 ) }
         DATA {
         (0,0): -6.55, -18.93, -64.98, 3.75,
         (1,0): -5.73, -14.89, -63.37, 3.4,
         (2,0): -4.42, -13.66, -58.95, 3.05,
         }
      }
      DATASET "1" { # id of fiber
         DATATYPE  H5T_IEEE_F64LE
         DATASPACE  SIMPLE { ( 2, 4 ) / ( 2, 4 ) }
         DATA {
         (0,0): -1.96, -10.07, -52.5, 2.92,
         (1,0): -1.03, -9.4, -48.62, 2.93,
         }
      }
  }
  GROUP "1" { # id of fiber_bundle
      DATASET "0" { # id of fiber
         DATATYPE  H5T_IEEE_F64LE
         DATASPACE  SIMPLE { ( 2, 4 ) / ( 2, 4 ) }
         DATA {
         (0,0): 3.4, -4.02, -44.76, 3.11,
         (1,0): 6.22, -1.04, -42.45, 3.26,
         }
      }
  }
}
\end{lstlisting}
\caption{Example structure of the fiber format in \ac{HDF5}. This output is generated with the official \code{h5dump} tool.}
\label{alg:hdf5}
\end{lstfloat}
%
%
%
\subsection{\Code{fastpli.model.sandbox}}
%
The sandbox module provides all the functions described in \cref{sec:sandbox}.
The module is divided into two submodules: \code{fastpli.sandbox.build} and \code{fastpli.sandbox.seeds}.
\code{fastpli.sandbox.seeds} contains all the methods for populating a 2d-plane, as described in \cref{sec:seeds}.
To populate the fiber from the seeds, the \code{sandbox.build} module provides the methods.
This includes all the described functions from \cref{sec:fillBundle}.
%
%
%
\subsection{\Code{fastpli.model.solver}}
%
The module \code{fastpli.model.solver} contains the compiled solver algorithm, which is explained in detail in \crefrange{sec:Solver}{sec:modelOpt}.
Additionally, the solver algorithm is wrapped in the \code{fastpli.model.solver.Solver} class.
This wrapper class provides a higher level of abstraction (see \cref{sec:fastpliIntro}).
All variables are read and writable by attributes, \eg{} \code{Solver.obj\_mean\_length}.
Each attribute checks if the user input valid and returns an appropriate warning or error message if necessarry.
This class also includes a \code{Solver.get\_dict()} method that returns a \python{} dictionary containing all variables and their values for reproducibility.
It is also possible to store the state of the class with the current state of the \code{fiber\_bundles} as an \ac{HDF5} object.
Finally, this class also provides the possibility to use a simple visualization of the solver process (see \cref{sec:visualization}).
%
%
%
\subsection{\Code{fastpli.objects}}
%
This module provides a wrapper class for \code{fastpli.objects.fibers} and \code{fastpli.objects.layers}.
Essentially, \code{layers} are a \code{list} of \code{layer}, which in turn are a \code{tuple} of the four attributes \code{absorption}, \code{birefringence}, \code{model}, and \code{scale} (see \cref{sec:dv_generator}).
This wrapper class contains attributes that allow the user to access these values by name, rather than by index \code{tuple} \code{[i]}.
This is helpful to reduce user errors.
\par
% 
The same is also provided for the classes \code{fastpli.objects.fibers} which contain \code{fastpli.objects.FiberBundles}, \code{fastpli.objects.FiberBundle} and \code{fastpli.objects.Fiber}.
\code{FiberBundles} are a \code{list} of \code{FiberBundle} which are a list of \code{Fiber}.
The data of a fiber is stored in a \code{numpy.ndarray} which stores the values contiguously in memory.
Manipulation methods are provided for each class, allowing the user to \code{translate}, \code{rotate}, \code{scale}, and \code{cut} the model.
The latter helps especially in the collision solver process to reduce the number of objects if only a certain volume is to be generated, since the collision solver process pushes the fiber objects apart and thus the volume would be increased.
%
%
%
\subsection{\Code{fastpli.simulation}}
% 
Like the \code{fastpli.model.solver.Solver} class, this method provides a wrapper for simulation called \code{fastpli.simulation.Simpli}, which is based on the original algorithm \cite{Dohmen2015,Lucksch2016}.
It contains the two algorithms \code{generator} and \code{simulation} described in \cref{sec:dv_generator,sec:simulation}.
These two algorithms operate separately, but since they share a number of parameters, they coexist within the class.
As in the \code{fastpli.model.solver.Solver} class, all necessary attributes are available and checked for input errors.
Since analysis is usually always performed on the resulting simulations, they are also available in this class and are performed with the same defined parameters as in the simulation.
Methods for saving the variables as \code{dict} or \ac{HDF5} files are available as well.
\par
% 
Typically, as in the experiment the flat and four tilt measurements are simulated.
This means that many parameters of the simulation pipeline does not change.
For this purpose there are \code{pipeline} methods (see \cref{alg:Pipeline}) which provide a high level of abstraction.
Here, all data is automatically analyzed and stored.
%
\begin{lstfloat}[!tb]
\centering
\scalebox{0.75}{
\begin{minipage}{\the\textwidth}
\lstinputlisting[style=python]{code/pipeline.py.tex}
\end{minipage}}
\caption{Simulation pipeline \code{simpli.run\_pipeline}.}
\label{alg:Pipeline}
\end{lstfloat}
%
%
\subsection{\Code{fastpli.tools}}
% 
The last module contains a set of helper functions.
They provide access to the current version as well as to the git hash so that all calculations can be reproduced.
For fiber modeling, rotation matrices are provided to allow the use of linear algebra.
% 
% 
% 

\section{Computational speedup techniques}\label{sec:theorySpeedup}
%
Among other specific techniques described in the previous chapters \cref{chap:sof:modelling,cha:sof:simulation}, two important technique are used to speed up the calculations.
\par
%
The computationally intensive code is written in \cpp{}.
There, the \code{std::vector} has the advantage that the data in memory is linear.
The data must be prepared and sent from the \ac{RAM} to the cache of the \acp{CPU}.
This takes relative to the time for a single \ac{CPU} instruction a very long time.
The main advantage of the cache is that it is very fast, however its capacities are usually in the order of a few $\si{\mega\byte}$ and therefore quite limited.
It is built inside the \ac{CPU}.
Modern \acp{CPU} have a built-in method called \textit{cache prefetching}.
The \ac{CPU} cache prefetcher is a sophisticated directive that requests not only the element at address $i$ in memory, but also the elements next to it ($i+1$ or $i-1$, depending on the algorithm).
Since many algorithms traverse arrays, the next element to be computed is typically the next (or previous) element.
Therefore, the total time required to copy the data from the memory to the cache is reduced.
It can be shown that for linear operations on memory, the cache prefetcher reduces the time so much that it behaves as if the \ac{CPU} had an infinite cache.
\par
%
Another technique is to use modern compilers such as \name{Clang v11}\footnote{\url{https://clang.llvm.org/}} or \name{G++ v10}\footnote{\url{https://gcc.gnu.org/}}.
These have an optimization algorithms built in that optimizes the code to the architecture of the machine, and much more sophisticated methods.
For example, if the number of iterations is known at compile time, a for loop can be \name{unrolled} to speed up the computations since it no longer needs to check if the conditions are met to end of each loop cycle.
To review these optimizations, the time critical code was tested with tools like \name{Compiler Explorer}\footnote{\url{https://godbolt.org/}} and \name{C++ Insight}\footnote{\url{https://cppinsights.io/}}.
%