\setcounter{chapter}{8}
\chapter{Conclusion and outlook}
\label{sec:summary}
% \cleanchapterquote{I would rather have questions that can't be answered than answers that can't be questioned.}{Richard P. Feynman }{}
% 
\ac{fastPLI} is a unique tool to provide user friendly methods which are capable of designing non colliding dense white matter nerve fiber bundles and simulating them inside a virtual \ac{3D-PLI} microscopy.
% \par
% 
\paragraph{\nameref{chap:sof:modelling}}
The tissue modelling software is capable of producing non colliding dense white nerve fiber models.
This library is parallelized for the \ac{CPU}.
\par
% outlook
The algorithm is written for the \ac{CPU}. However this type of algorithm is highly recomended the usage of a \ac{GPU}.
\cite{Karras2012} describes such an algorithm based on \ac{AABB} and a parallelized z-ordered tree instead of an octree.
Since the \-ordered tree can be build in parallel, the collision checking can begin almost immediately.
% 
\paragraph{\nameref{cha:sof:simulation}}
\dummy{}
\par
% outlook
Due to this thesis two projects were completed to start a implemenmtation of a parallel \ac{GPU} implementation for the \ac{3D-PLI} simulation.
The first was a Seminararbeit \cite{Kobusch:Seminar} which targeted the implementation of the light tissue voxel calculations \dummy{}.
It was shown, that the speedup of the matrix calculus was very \dummy{}, however the large need of memory of the discretizied volume neglected this speedup in total.
Therefore the next project \cite{Kobusch:887783} was to implement a raytracing algoithm, which would calculate a light matter collision and performing the matrix calculus on the fly without the need of precalculating the discretised volume.
Due to the time limitations of a Bachelorthesis a simple collision checking algorithm on the \ac{GPU} was implemented, a single axis aligned search algorithm \cite{Karras2012}.
At the end of the project it could be shown that the speedup capabilities are hughe.
However due to the fact of the rather simple algorithm, the runtime was slighly larger, but in the same order of magnitude, then the \ac{CPU} version.
However the algorithm was already implemented in such a way, that the light rays are also not limited to follow all the same parallel path.
This can achieve in a future implementation also effects light scattering light rays.
A probably perfect algorithm for this kind of code was described in \cite{Karras2012}.
As in the case of the collision checking algorithm for the fibers \dummy{}, the same collision checking algorithm and z-orded tree can be used to identify the light fiber segments, \ie{} conical cones, and calculate on the fly with a much smaller step size the light matter calculations.
Next to the speedup, which should be quite large compared to the \ac{CPU} version, the much smaller need of memory is a huge advantage.
With such a implementation it is highly likely that there will be no need for reasonable simulations on multiple nodes on a super computing system.
\\
% 
Further speedups should then also be the implementation of the already developed parallel tilting analysis \dummy{}.
%
% 
% 
\paragraph{\nameref{chap:Software}}
% 
The Software as a open source cannot be underestimated.
The sharebility of knowledge is a, if not the key in science.
Already since a few years the modelling algorithm is used for different kind of science next to the obvious in \ac{3D-PLI} and \ac{dMRI}.
\cite{Ji2021} can use such none colliding fiber models for simulations of \dummy{}.
The software \ac{fastPLI} allows for a easy usage and fast generation of simple as also complicated simulations.
% 
% 
% 
\paragraph{\nameref{cha:model_analysis}}
The results from the modeling characterization could identify a very good arguable working point for the models.
The limiting factor is the runtime and therefore the number of objects in the \ac{VOI}.
For \ac{3D-PLI} simulations a certain remaining overlapping percentage is probably vertretbar and therefore about a order of magnitude in runtime can be saved.
Otherwise with a up to date computing systems a reasonable amount of models can be generated.
However here the number of models were highly limited (considering anatomical and simulation constrains).
A future study has to show, if certain configuration show a unique behavior.
Specially for simulations like scattering light this can be quite important.
% 
% 
% 
\paragraph{\nameref{cha:simulation_analysis}}
The results with the new generated irregular dense white matter models could reproduce the experiments as well as old results.
Additionally a lot more simulations and configurations could be investigated.
\par
% outlook
For the future the usage of machine learning can probably be used to try to find a new characterization of \ac{3D-PLI} signals to better be able to know he underlying fiber structure.
Here \cite{ginsburgerDis2019} could show the first steps already for \ac{dMRI} with similar methods. 