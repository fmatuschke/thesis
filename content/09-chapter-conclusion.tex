\newpage\null\thispagestyle{empty}\newpage
\clearpage{\thispagestyle{empty}\cleardoublepage}
\part{Conclusion}
% 
% 
% 
\setcounter{chapter}{8}
\chapter{Outlook}
\label{sec:outlook}
% 
\paragraph{Nerve fiber modelling software}
% 
Currently, the collisionless nerve fiber modeling alrotihm is capable of generating densly packed fiber models which are suitable nerve fiber models for \ac{3D-PLI} simulations.
However, the algorithm is limited in terms of computational time in the volume that can be generated.
Since the number of individual nerve fiber segments increases with volume by the third power, it is not possible to generate large volumes in a reasonable time relative to the nerve fiber radii.
\par
% 
There are many optimization possibilities to improve the performance.
First, one can already try to design a better initial model for the nerve fibers that takes less time to solve.
However, this limits the user's ability to generate an arbitrary initial configuration. 
In addition, it was also found that in the first $\SI{10}{\percent}$ most of the collisions can be solved and the remaining time is needed for the remaining minimal overlaps.
\par
% 
A more suitable strategy is therefore to try to design the algorithm in such a way that the number of steps required is reduced.
However, this is a very complicated task.
For example, one could increase the motion that a fiber segment is allowed to perform.
This strategy would probably lead to less densely packed models, since without an external force or attraction between the nerve fibers, they can only move apart.
Another possibility is to speed up the runtime per step.
This can be done, for example, by choosing a simpler calculation.
In the \ac{MEDUSA} algorithm this is already been realized by using spheres instead of fiber segments.
The collision of spheres is only be calculated by the euclidian distance with respect to the sum of the radii.
However, the use of spheres is associated with a disadvantage.
The number of objects increases drastically, since many more spheres are needed instead of conical fiber segments.
Otherwise, the surface of the fiber cannot be scanned.
\par
% 
A current limitation of the algorithm presented here is also parallelization on a multi-core (or even multi-node) system.
The problem with such an algorithm is that the system needs to communicate a lot, which takes a relatively long time compared to instructions on a \ac{CPU}.
Therefore, one probably needs to redesign the algorithm and find one that does not require much communication.
For example, one could divide the volume into subvolumes at the very beginning of the algorithm and distribute the tasks among several \acp{CPU}.
At the boundary between the subvolumes, one still needs to exchange the positions of the nerve fiber segments, but one could do this, for example, only after a certain number of steps, thus reducing the number of communications.
However, this may lead to artifacts at the boundaries.
\par
% 
At this point, the most promising optimization would be to use the architecture of the \ac{GPU} as in the \ac{MEDUSA} algorithm.
A collision detection algorithm.
One type of these algorithms that is particularly appropriate here is based on the \ac{AABB} used here and a z-ordered tree instead of an octree.
Not only is the collision checking computed in parallel, but also the preparation of the data into a z-tree is done in parallel on the \ac{GPU}.
There is no longer a need to use the \ac{CPU} between steps or to copy data between the \ac{CPU} and \ac{GPU}.
This drastically increases the speed of collision checking \cite{Karras2012}. 
% 
% 
% 
\paragraph{\ac{3D-PLI} software}
% 
As part of this work, two proof-of-concept projects were conducted to develop a parallel GPU implementation for \ac{3D-PLI} simulations.
The first was a seminar project to implement a parallel discrete tissue volume computation algorithm for the \ac{3D-PLI} simulation on the \ac{GPU} architecture \cite{Kobusch:Seminar}.
It was shown that the speedup of the discrete mesh computation was very fast, but the large memory footprint of the discretized volume negated this speedup overall, as GPUs are relatively limited here.
In addition, the need to transfer the data back to \ac{RAM} was too much of an overhead.
Therefore, the second project was to implement a ray tracing algorithm that computes a light matter collision and matrix computation "on the fly" without pre-computing the discretized volume \cite{Kobusch:887783}.
This second project includes a proof of concept that implements a simple collision detection algorithm on the \ac{GPU}, a uniaxially aligned collision search algorithm \cite{Karras2012}.
The results showed that the acceleration possibilities on the \ac{GPU} are enormous.
However, due to the rather simple algorithm, the runtime was slightly longer, but still in the same order of magnitude as the \ac{CPU} version.
\par
% 
The next logical step is therefore to use a more sophisticated algorithm.
As in the case of the fiber collision detection algorithm, the same collision detection algorithm and z-ordered tree can be used to detect collisions between light particles and fiber segments \cite{Karras2012}.
This has several advantages.
The most important is that the discretized volume is no longer necessary, which saves a lot of memory.
With such an implementation, a normal computing system should already be fast enough for reasonable simulation sizes.
The parallel algorithm developed above has already been implemented to allow light rays to change their orientation.
This will also allow simulations of light scattering in the future.
% 
% 
% 
\paragraph{Nerve fiber modelling and \ac{3D-PLI} simulation}
% 
To increase the statistical power of the results, it is a good idea to increase the number of nerve fiber models.
Since two models are independent, all models can be generated in parallel using the entire computer architecture.
However, a more appropriate step would be to try to determine the essential parameters for the models and simulations, thus reducing the number of model generation required.
Modern machine learning algorithms are a suitable tool for this purpose.
\par
% 
Another important task is to study larger nerve fiber radii.
This would include the larger nerve fiber radii that are anatomically present in the brain.
However, larger nerve fiber radii could also be an approximation for a nerve fiber bundle that consists of multiple nerve fibers in the \ac{3D-PLI} simulation.
If this is possible, one could reduce the number of objects quite substantially, which means that one could either have very small runtimes or larger volumes.
\par
% 
In addition, more complex models, e.g. three fiber populations, should be investigated.
Border regions consisting of Adjacent Nerve Fiber Tracts significantly alter the signal when the light beam is tilted.
This change is not taken into account by the inclination analysis.
Here, the use of machine learning can be used to find a new characterization of the \ac{3D-PLI} signals and better identify the underlying fiber structure.
In computed tomography, nerve fiber models and their simulations, as well as the use of deep learning, have already shown that the underlying fiber structure can be \cite{ginsburgerDis2019} identified from the original signals.
Makine learning models such as those in Deep Learning may be able to exploit effects on signals that are not easily visible analytically.
% 
% 
% 
% -----------------------------------------------------------------
% 
% 
\chapter{Conclusion}
\label{sec:conclusion}
% 
In this work, two algorithms were presented within the software package \ac{fastPLI}.
The first algorithm is capable of designing non-colliding nerve fiber models in a 3d volume.
These models can then be used in the second algorithm, which simulates the interaction of polarized light with the modeled nerve fibers in a virtual \ac{3D-PLI} microscopic setup.
Each algorithm was implemented as an independent \python{} module that can use multicore systems.
% 
% 
% 
\paragraph{Nerve fiber modelling software}
% 
The algorithm for developing non-colliding nerve fiber models takes as input a list of 4d points for each nerve fiber, where the first three values are the $x,y,z$ coordinate in space and the fourth value is the radius of the nerve fiber at that point.
Therefore a nerve fiber is represented with this configuration as a cylindrical tube that can change its radius along the path.
For each fiber segment containing two adjacent points, a test is performed to determine if there is a collision with another fiber segment.
If this is the case, both fiber segments are moved slightly away from each other.
This is done for all fiber segments in the volume as long as a collision is detected.
Both the length of a fiber segment and the bending radius of a nerve fiber are controlled by the user-defined software parameters.
These parameters change the sampling rate and the stiffness of the fiber models.
\par
% 
To speed up model generation, an octree is used in the algorithm to divide the volume into subvolumes.
This octree can be executed in parallel on multiple \acp{CPU}.
A visualization is provided to display the volume, allowing the user to interact with the algorithm after each computation step.
\par
% 
Another project called \ac{MEDUSA} was developed in collaboration with Neurospin at \ac{CEA}, which allows designing non-colliding nerve fiber models with cells such as astrocytes or olegodendrocytes that connect to myelin-enveloped axons.
% 
% 
% 
\paragraph{Simulation software}
% 
The simulation software for \ac{3D-PLI} takes a configuration of nerve fibers as input and simulates the light-matter interaction within a \ac{3D-PLI} setup.
The simulation is divided into two consistent parts.
The first part generates a discretized 3d volume.
This volume is used by the second part, which calculates the resulting light intensity using the M\"{u}ller-Stokes calculation and the simulation of several light vectors through the volume.
The intensity is stored in a 2d array, which can be modeled as an \ac{CCD} array with a specified resolution and noise model.
The simulation is capable of simulating multiple tilted light beam, allowing multiple views of the same sub-volume that the light is traversing.
In addition, the analysis algorithms are implemented into the software package, allowing the nerve fiber orientation to be calculated using the inclination analysis \ac{ROFL}.
\par
% 
The simulation is capable of using multiple cores as well as a system with multiple nodes communicating via \ac{MPI}.
This allows the simulation of a large volume of nerve fibers, which takes up a large amount of memory due to the discretized volume required.
% 
% 
% 
\paragraph{Software package \acs{fastPLI}}
% 
All algorithms are written as modules in \python{} inside the software package \ac{fastPLI}.
The Software package is published as open source software package allowing users to share, ask and further develop the software for a high exchangeability. 
The software was tested by ultiple users and published inside the revied \ac{JOSS}.
% 
% 
% 
\paragraph{Nerve fiber modelling results}
% 
The nerve fiber modelling algorithm controlls the necesarry fiber movement to solve the collisions by the fiber segment length and the fiber bending radius.
Both parameters were characterized against the resulting orientations and computational speed.
To reduce the dimensionality of the possible configurations a set of parameters to describe the model was designed.
This set consit out of 2 relativ angles between the two fiber populations, an inclination angle of the entire model and a population fraction parameter.
This set allows to investigate nerve fiber models up to two nerve fiber populations with any crossing angle in 3d without describing every fiber by itself.
\par
% 
Using this parameters first the characteristics of the modelling parameters was charakterized.
A set of values could be identified which is suitable to generate non colliding nerve fiber models without distoring the initial configuration too much.
Additionally this parameter values are suitable to reduce the runtime in a reasanable amount without loosing any configuration charachteristics or impliing a bias on the resulting models.
To reduce the runtime even further it could be identified that about 1 order of magnitude the processing time can be reduced by not solving the models completly.
The influance of a non collision free model however has to be investigated in an aditionall study.
\par
% 
With the found model software parameters a library of up to two nerve fiber models described by the four model parameters was generated fo the later \ac{3D-PLI} simulation.
The orientation distribution was analysed to be able to use as a comparison for the simulation orientation analysis.
% 
% 
% 
\paragraph{\acs{3D-PLI} simulation results}
% 
Analog to the nerve fiber modelling software first the simulation software parameters were charackerized and measured.
The optical resolution of the used micrsoscope was reproduced with former results as well as the optical noise of the system.
From in \ac{3D-PLI} measured tissue samples the tissues properties were derived so that the simulation with means of its limitations can reproduce the results.
The most important characteristica of the simulations accuracy and speed, the \Voxelsize{} was characterized by using the upper prepaired models.
A lower limit could be identified which has to be used for a given nerve fiber radius.
\par
% 
Using the found software parameters and the prepaird nerve fiber models simulations were performed with a orientation analysis.
The models were spit into four groups, a single, a flat crossing, an inclined crossing and freely crossing nerve fiber populations.
This allowed to concentraid on specific behavier.
\par
% 
In the case of a single nerve fiber population only the inclination of the models was a variable.
With the simulation results it coud be shown, that in the case of a single fiber population the orientation can be correctly identified with an increased uncertanty for very stepp fibers.
Given multiple image pixel statistically the mean value can be correctly measured.
For very steep nerve fibers the relative effective tissue thickness becomes unstable in the inclination analysis.
This behavier can be used to take as an hint how uncertain the results are.
\par
% 
The flat crossing nerve fiber population results for the inclination analysis in a single value, which seems to follows the circular mean value of the individual orientations.
With this behavier mostly the dominant present nerve fiber population is visible, with a slighly systematic error prown to the orientation to the second nerve fiber direction.
With the here choosen nerve fiber radiius distribution and image resolution the involed nerve fiber population orientation could not be resolved.
\par
% 
The inclined crossing nerve fiber population has similar characteristics to the flat crossing case.
The inclination seems also to follor the circular mean value of each population, however due to the fact that inclined nerv fibers result in a smaller change of the polarization of the light, the resulting orientation is distorted to the less inclined fiber population.
Here also no individual orientation of one of the nerve fiber population can be identified from the inclination analysis.
\par
% 
The last studied models were a freely ... nerve fiber populations, so any configuration describable by the four model parameters were possible.
From a \ac{ODF} comparison metric model angles could be identified where the incination analysis of the \ac{3D-PLI} signal is tendet to an error prown and were it is very reliable.
The individual parameters were discussed as well and the underliing behavier could be identified leading to the error prown orientations.
\par
% 
A speedup analysis of the simulation showed that the simulation software with the capabilities of the \ac{MPI} is very parallizable and can utalize a multi node system which makes it possible to simulate large sections without the loss of computation time.
However the tissue generation process is sighly less performend.
However since multiple \ac{3D-PLI} tilted light beam simulation are run on the same discterized tissue this is not an issue.
\par
% 
Many possible further optimizations of the algorithm could be identified.
The overall ... is to increase the computation by utalizing parallel architecuters like an \ac{GPU}.
Already existing algorithms suitable for this kind of calculations, the nerve fiber modelling as well as the \ac{3D-PLI} simulations, could be identified.
\par
% 
Overall this thesis \dummy{}.