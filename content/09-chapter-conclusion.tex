\newpage\null\thispagestyle{empty}\newpage
\clearpage{\thispagestyle{empty}\cleardoublepage}
\part{Conclusion}
% 
% 
% 
\setcounter{chapter}{8}
\chapter{Outlook}
\label{sec:outlook}
% 
\paragraph{Nerve fiber modelling software}
% 
Currently, the collisionless nerve fiber modeling algortihm is capable of generating densly packed fiber models which are suitable nerve fiber models for \ac{3D-PLI} simulations.
However, the algorithm is limited in terms of computational time or more specific, the number of objects in the volume to be solved.
Since the number of individual nerve fiber segments increases with volume, it is not possible to generate large volumes in a reasonable time.
\par
% 
There are optimization options to improve performance.
First, one can improve the sandbox design models so that the initial fiber configurations have less overlap.
This would also mean that for a given volume, the number of objects is reduced.
However, it was found that in the first $\approx \SI{10}{\percent}$ most collisions can be solved and the remaining time is needed for the remaining minimal overlaps \cref{sec:solverParameterResults}.
Therefore, the total runtime would probably not decrease significantly.
\par
% 
A more suitable strategy is to design the algorithm in such a way that the number of calculation steps is reduced.
However, this is a very complicated task.
For example, one could increase the motion that a fiber segment is allowed to perform.
This strategy would probably lead to less densely packed models, since without an external force or attraction between the nerve fibers, they can only move apart. \footnote{With an external force one would introduce oscilations.}
Another possibility is to speed up the runtime per step.
This can be done, for example, by choosing a simpler calculation.
In the \ac{MEDUSA} algorithm this is already been realized by using spheres instead of fiber segments.
The collision of spheres is only be calculated by the euclidian distance with respect to the sum of the spheres radii.
However, the use of spheres is associated with a disadvantage.
The number of objects increases dramatically as many spheres are needed to approximate the surface of a single conical fiber segment.
\par
% 
A current limitation of the algorithm presented here is parallelization on a multi-core system.
For shared memory parallelization, atomic operations must be introduced, which takes a lot of time.
For an algorithm without shared memory, the required data must be exchanged between the individual CPUs. 
This communication introduces an overhead compared to the \ac{CPU} instructions.
However, such an algorithm could work on a multi-node system if the performance does not decrease drastically with increasing number of CPUs involved, as is currently the case.
Therefore, one needs to redesign the algorithm.
Common solutions are to divide the volume into subvolumes (as in the case of \ac{3D-PLI} simulation), where each volume can be computed separately.
However, since the boundaries of such subvolumes contain objects from neighboring volumes, it is necessary to merge the information of neighboring subvolumes.
In a sense, this is already happening in the current implementation of the octree, but the volume is split into subvolumes at each step.
If this can be improved so that only the necessary information needs to be transmitted, performance should increase by quite a bit.
\par
% 
At this point, the most promising optimization would be to use the architecture of the \ac{GPU} as in the \ac{MEDUSA} algorithm.
One type of these algorithms that is particularly appropriate here is based on the \ac{AABB} used here and a z-ordered tree instead of an octree \cite{Karras2012}.
Not only is the collision checking computed in parallel, but also the generation of the z-ordered tree is done in parallel on the \ac{GPU}.
Also the algorithm completly runs on the \ac{GPU} and therefore does not need to communicate with the \ac{CPU} or \ac{RAM} as long as the complete information fits on the \ac{GPU} memory.
This drastically increases the speed of collision checking.
% 
% 
% 
\paragraph{\ac{3D-PLI} software}
% 
As part of this work, two proof-of-concept projects were conducted to develop a parallel GPU implementation for \ac{3D-PLI} simulations.
The first was a seminar project to implement a parallel discrete tissue volume computation algorithm for the \ac{3D-PLI} simulation on the \ac{GPU} architecture \cite{Kobusch:Seminar}.
It was shown that the speedup of the discrete volume computation was very fast, but the large memory requirements of the discretized volume negated this speedup overall, as \acp{GPU} are relatively limited here.
In addition, the need to transfer the data back to \ac{RAM} was too much of an overhead.
Therefore, the second project was to implement a ray tracing algorithm that computes a light matter collision and matrix computation without pre-computing the discretized volume \cite{Kobusch:887783}.
Therefore a light particle only computes the M\"{u}ller-Stokes calculus if it collides fith a fiber object.
This second project includes a proof of concept that implements a simple collision detection algorithm on the \ac{GPU}, a uniaxially aligned collision search algorithm \cite{Karras2012}.
The results showed that the acceleration possibilities on the \ac{GPU} are enormous.
However, due to the rather simple algorithm, the runtime was slightly longer, but still in the same order of magnitude as the \ac{CPU} version.
\par
% 
The next logical step is therefore to use a more sophisticated algorithm.
As in the case of the fiber collision detection algorithm, the same collision detection algorithm and z-ordered tree can be used to detect collisions between light particles and fiber segments \cite{Karras2012}.
This has several advantages.
The most important is that the discretized volume is no longer necessary, which saves a lot of memory.
With such an implementation, a normal computing system should already be fast enough for reasonable simulation sizes.
Additionally the parallel algorithm developed above already allows light rays to change their orientation.
This will allow simulations of light scattering in the future.
% 
% 
% 
\paragraph{Nerve fiber modelling and \ac{3D-PLI} simulation}
% 
To increase the statistical significance of the results, the number of nerve fiber models and simulations should be increased.
Since two models are independent, all models can be generated in parallel using the entire computer architecture.
However, a more appropriate step would be to determine the essential parameters for the models and simulations, thus reducing the number of model generation required.
Modern machine learning algorithms are a suitable tool for this purpose.
\par
% 
Another important task is the study of larger nerve fiber radii.
This would include the larger nerve fiber radii that are anatomically present in the brain.
In addition, larger nerve fiber radii could also be an approximation for a nerve fiber bundle consisting of multiple nerve fibers in the \ac{3D-PLI} simulation.
They should then be simulated with a macroscopic birefringent model \cite{Menzel2015}.
If this is possible, one could significantly reduce the number of objects for model generation, which means that one could have either very small run times or larger volumes.
Further optimization can be achieved by using nerve fiber models that are not fully solved.
If the effect is negligible in the \ac{3D-PLI} simulation, the runtime for the models can be significantly reduced.
\par
% 
In addition, more complex models, \eg{} three fiber populations, should be investigated.
Furthermore, boundary regions consisting of neighboring nerve fiber tracts must also be investigated because they significantly change the signal when the light beam is tilted.
This change is currently not accounted for by tilting analysis.
Here, the use of machine learning can be used to find a new characterization of the \ac{3D-PLI} signals and better identify the underlying fiber structure.
In \ac{dMRI}, nerve fiber models and their simulations, as well as the use of deep learning, have already shown that the underlying fiber structure can be identified from the original signals \cite{ginsburgerDis2019}.
This can potentially be applied to \ac{3D-PLI} in a similar way.
% 
% 
% 
% -----------------------------------------------------------------
% 
% 
\chapter{Conclusion}
\label{sec:conclusion}
% 
In this work, two algorithms were presented within the software package \ac{fastPLI}.
The first algorithm is capable of designing non-colliding nerve fiber models in a 3d volume.
These models can then be used in the second algorithm, which simulates the interaction of polarized light with the modeled nerve fibers in a virtual \ac{3D-PLI} microscopic setup.
% 
% 
% 
\paragraph{Nerve fiber modelling software}
% 
The algorithm for developing non-colliding nerve fiber models takes as input a list of 4d points for each nerve fiber, where the first three values are the $x,y,z$ coordinate in space and the fourth value is the radius of the nerve fiber at that point.
Therefore a nerve fiber is represented as a cylindrical conical segments that can change its radius along the path.
For each fiber segment containing two adjacent points, a test is performed to determine if there is a collision with another fiber segment.
If this is the case, both fiber segments are moved slightly away from each other.
This is done for all fiber segments in the volume as long as a collision is detected.
Both the length of a fiber segment and the bending radius of a nerve fiber are controlled by the user-defined software parameters.
These parameters change the discretization and stiffness of the fiber models.
\par
% 
To speed up model generation, an octree is used in the algorithm to divide the volume into subvolumes.
This octree can be executed in parallel on multiple \acp{CPU}.
A visualization is provided to display the volume, allowing the user to interact with the algorithm after each computation step.
\par
% 
An additional project called \ac{MEDUSA} was developed in collaboration with Neurospin at \ac{CEA}, which allows designing non-colliding nerve fiber models with cells such as astrocytes or olegodendrocytes that connect to myelin-enveloped axons.
% 
% 
% 
\paragraph{Simulation software}
% 
The simulation software for \ac{3D-PLI} takes a configuration of nerve fibers as input and simulates the light-matter interaction within a \ac{3D-PLI} setup.
The simulation is divided into two consistent parts.
The first part generates a discretized 3d volume.
This volume is used by the second part, which calculates the resulting light intensity using the M\"{u}ller-Stokes calculation and the simulation of several light vectors through the volume.
The final light intensity is stored in a 2d array, which can be modeled as an \ac{CCD} array with a specified resolution and noise model.
The simulation is capable of simulating multiple tilted light beam, allowing multiple views of the same subvolume that the light is traversing.
In addition, the analysis algorithms are implemented into the software package, allowing the nerve fiber orientation to be calculated using the tilting analysis \ac{ROFL}.
\par
% 
The simulation is capable of using multiple cores as well as a system with multiple nodes communicating via \ac{MPI}.
This allows the simulation of a large volume of nerve fibers, which takes up a large amount of memory due to the discretized volume required.
% 
% 
% 
\paragraph{Software package \acs{fastPLI}}
% 
All algorithms are written as modules in \python{} within the software package \ac{fastPLI}.
The software package is published as an open source software package so that users can share, ask, and further develop the software to ensure high interchangeability. 
The software has been tested by several users and published in \ac{JOSS}.
% 
% 
% 
\paragraph{Nerve fiber modelling results}
% 
The nerve fiber modeling algorithm controls the necessary movement of the fibers to solve the collisions via the length of the fiber segments and the bending radius of the fibers.
Both parameters were characterized in terms of the resulting orientations and computational speed.
To reduce the dimensionality of the possible configurations, a set of parameters was designed to describe the model.
This set consists of a relative angle between the two fiber populations, an inclination angle of the first fiber population, a rotation angle of the second fiber population around the first one and a population fraction parameter.
The set allows nerve fiber models with up to two nerve fiber populations with arbitrary crossing angles to be examined in 3d without describing each fiber separately.
\par
% 
Based on these parameters, the properties of the modeling parameters were first characterized.
A set of values suitable to generate non-colliding nerve fiber models without introducing significant distortion to the initial configuration was identified.
In addition, these parameter values are suitable to reduce the runtime to a reasonable extent without losing configuration characteristics or distorting the resulting models.
To reduce the runtime even further, it was found that about an order of magnitude of computation time can be reduced by not solving the models completely.
However, the impact of a non-collision-free model remains to be investigated in an additional study.
\par
% 
With the model software parameters found, a library of up to two nerve fiber models described by the four model parameters was created for the \ac{3D-PLI} simulation.
The orientation distribution was analyzed to be used as a comparison for the orientation analysis of the simulation.
% 
% 
% 
\paragraph{\acs{3D-PLI} simulation results}
% 
Analogous to the nerve fiber modeling software, the parameters of the simulation software were first characterized and measured.
The optical resolution of the microscope used was reproduced with previous results, as was the optical noise of the system.
Tissue properties were derived from tissue samples measured in \ac{3D-PLI} so that the simulation could reproduce the results with its limitations.
The most important characteristic of the simulation, the accuracy and runtime, was characterized by studien multiple \Voxelsize{}.
It was possible to identify a lower bound that must be used for a given nerve fiber radius.
\par
% 
Using the identified software parameters and the prepared nerve fiber models, simulations were performed and analiesed with the tilting analysis.
The models were divided into four groups: a single, a flat crossing, an oblique crossing, and a free crossing nerve fiber population.
This allowed to focus on a specific behavior.
\par
% 
In the case of a single nerve fiber population, only the inclination of the models was a variable.
With the simulation results, it was shown that in the case of a single fiber population, the orientation can be correctly identified, with an increased uncertainty for very steep fibers.
Statistically the mean can be measured correctly when measuring a homoginous volume with multiple image pixel.
For very steep nerve fibers, the relative effective tissue thickness becomes unstable in the tilting analysis.
This behavior can be used as an indication of how uncertain the results are.
\par
% 
The flat crossing nerve fiber population results in a single value for the tilting analysis that appears to follow the circular mean value of the individual orientations.
With this behavior, the predominant nerve fiber population is mostly visible, with a slight systematic error due to the orientation to the second nerve fiber direction.
With the nerve fiber radius distribution and image resolution chosen here, the underlying orientation of the individual nerve fiber population could not be resolved.
\par
% 
The population of inclined crossing nerve fibers has similar characteristics to the population of flat crossing nerve fibers.
The inclination also appears to follow the circular mean value of each population, but due to the fact that inclined nerve fibers result in less change in the polarization of the light, the resulting orientation is biased toward the less inclined fiber population.
Again, no individual orientation of any of the nerve fiber populations can be identified from the tilting analysis.
\par
% 
The last models examined were unrestricted nerve fiber populations, so any configuration describable by the four model parameters was possible.
As a comparison an \ac{ODF} metric was used to determine the orientations of the models for which the tilting analysis of the \ac{3D-PLI} signal was error-prone and which were highly reliable.
In addition, the individual parameters were discussed and the underlying behavior leading to the erroneous orientations was determined.
\par
% 
An analysis of the simulation speed has shown that the simulation software can be parallelized very well with the capabilities of \ac{MPI} and can use a multi-node system that allows large volumes to be simulated without loss of computation time.
However, the tissue generation process is slightly less efficient.
Nevertheless, since multiple \ac{3D-PLI} simulations with tilted light beams are run on the same discretized tissue, this is not a critical issue.
\par
% 
Many possible further optimizations of the algorithm could be identified.
The general advice is to increase the computational power by using parallel computing architectures like a \ac{GPU}.
Existing algorithms suitable for this type of computation for both the nerve fiber modeling and \ac{3D-PLI} simulations could be identified and should be implemented in the future.
\par
% 
Overall this thesis \dummy{}.