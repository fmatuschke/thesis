\newpage\null\thispagestyle{empty}\newpage
\clearpage{\thispagestyle{empty}\cleardoublepage}
\part{Conclusion}
% 
% 
% 
\setcounter{chapter}{8}
\chapter{Outlook}
\label{sec:outlook}
% 
\paragraph{Nerve fiber modelling software}
% 
Currently the collision free nerve fiber modelling alrotihm is capable of generating suitable nerve fiber models for \ac{3D-PLI} simulations.
However it has two limiting factors.
The first is it takes depending on the parameters a certain amount of computational effort to generate such collision free models.
The second is, due to he fact the amount of individual nerve fiber segments raises with the volume, \ie{} the third potenz, one cannot generate hughe volumes relativ to the nerve fiber radiius in a reasonable amount of time.
\par
% 
Many optimizations exists, to enhanve this behavier.
First one can already try to design a better initial nerve fiber model, which does not need as much time to solve.
This however limits the users capabilities to produce any kind of initial configuration. 
Additionally it has been shown, that the last $\SI{90}{\percent}$ of the steps are used to solve the quite few remaining colissions.
\par
% 
Therefore a more suitable stategy is to try to design the algorithm in such a way, that the number of needed steps is reduced.
This however is a very complicated tasks.
For example one could increase the movement a fiber segemnt is allowed to move.
However this can lead to less densly packed models since without an external force or attractive force between the nerve fibers the nerve fibers can only move apart.
Such forces however would also lead to oscilations witch would increase the solving time since collision occour over and over again.
\par
% 
Another way is to speed up the runtime per step.
This can for example be done by choosing a more easier calculations.
This for example has already be utalized by the \ac{MEDUSA} agorithm which uses spheres instead of fiber segments.
This however increases the number of objects, since far more spheres are needed instead of conical fiber segments.
\par
% 
A currently limitation of the algorithm is also the parallelization on a multi core (or even multi node) system.
The problem with such an algorithm is, that the system has to communicate a lot, which takes relativ long times comparing to instructions on a \ac{CPU}.
Therefore one has to probably redesign the algorithm and finding an algorithm wich is capable of computing without the need of many communications.
For example one could already split the volume in sub volumes at the beginning of the algorithm and distributing the tasks onto multiple \acp{CPU}.
On the boundry between sub volumes one still has to exchange the positions of the nerve fiber segments, however one could do this for example only after a certain amount of steps, by that reducing the number of communications.
This however can lead to artifacts at the boundries.
\par
% 
At this point most promissing optimization would be to utalize the architecture of the \ac{GPU} like in the \ac{MEDUSA} algorithm.
a collision checking algorithm.
One type of these algorithms that is particularly suitable here is based on the \ac{AABB} used here and a z-ordered tree instead of an octree.
Not only is the collision checking computed in parallel, but the preparation of the data into a z-tree is also done in parallel on the \ac{GPU}.
There is no longer a need to use the \ac{CPU} between steps or to copy data between the \ac{CPU} and \ac{GPU}.
This drastically increases the speed of the collision check \cite{Karras2012}. 
% 
% 
% 
\paragraph{\ac{3D-PLI} software}
% 
\TODO{check Inahlt}
% 
As part of this thesis, two proof of concept projects were accomplished to develop a parallel GPU implementation for 3D-PLI simulations.
The first was a Seminararbeit aimed at implementing a parallel algorithm to calculat the discrete tissue volume for the 3D-PLI simulation \cite{Kobusch:Seminar}.
It was shown that the speedup of the discrete tissue computation was very fast, but the large memory requirements of the discretized volume negated this speedup overall, as current GPUs are relatively limited there.
Also, the need to transfer the data back to the CPU was a too high overhead.
Therefore the second project was to implement a raytracing algoithm, which would calculate a light matter collision and performing the matrix calculus on the fly without the need of precalculating the discretised volume \cite{Kobusch:887783}.
As a proof of concept a simple collision detection algorithm was implemented on the GPU, a uniaxially aligned search algorithm \cite{Karras2012}.
It could be shown that the acceleration possibilities on 125the GPU are enormous, However, due to the rather simple algorithm, the runtime was slightly longer, but still in the same order of magnitude as the CPU version.
The next logical step is therefore to use a more sophisticated algorithm.
As in the case of the fiber collision detection algorithm, the same collision detection algorithm and z-ordered tree can be used to detect collisions between light particles and fiber segments [Kar12].
This has several advantages.
The most important one is that the discretized volume is no longer necessary, which saves a lot of memory.
With such an implementation, a normal computing system should already be fast enough for reasonable simulation sizes.
The parallel algorithm developed above has already been implemented in such a way that the light rays can change their orientation.
Thus, simulations of light scattering will also be possible in the future.
% 
% 
% 
\paragraph{Nerve fiber modelling simulation}
% 
To improve the statistik of these models one can enhance the number of models.
Another improvment would also be, if a simpler, \ie{} less dimensionality, parameterset can be identified to classify the fiber populations or individual fiber configurations.
Also this these has its focus on rather small nerve fibers, one should also focus in the future on larger nerve fibers.
Not only to investigate anatomical larger nerve fibers but also because one could model a nerve fiber bundle, constitingc out of small fibers by a large single nerve fiber bundle object.
If this is possible one could reduce the number of objects quite significant which means, either one has very smal runtimes or one can model large volumes.
% 
% 
% 
\paragraph{\ac{3D-PLI} simulation}
% 
In the future, more complex models, probably up to three fiber populations, and different fiber radii need to be investigated.
It is expected that larger fiber radii will have a stronger influence on the inclination measurements.
How this can be properly analyzed is an open question at this time.
Since these large fiber radii exist in brains, this is an important consideration for future studies.
The use of machine learning can likely be used to find a new characterization of 3D PLI signals to better identify the underlying fiber structure.
In \ac{dMRI}, nerve fiber models and their simulations and the use of deep learning have already shown that the underlying fiber structure can be identified from the original signals \cite{ginsburgerDis2019}.
\ac{dMRI} is of course based on other physical models, but the use of deep learning can exploit effects on signals that are not easily analytically visible.
% 
% 
% 
% -----------------------------------------------------------------
% 
% 
\chapter{Conclusion}
\label{sec:conclusion}
% 
In this thesis two algorithm inside the software package \ac{fastPLI} were presented.
The first algorithm is capable of designing non colliding nerve fiber models in a 3d volume.
These models can then be used inside the second algorithm, which simulated the polarized light interaction with the modelled nerve fibers inside a virtual \ac{3D-PLI} setup.
Each algorithm was implemented as a independent \python{} module capable of utalizing multicore systems.
% 
\paragraph{Nerve fiber modelling software}
% 
The algorithm to desining non colliding nerve fiber models takes as input a list of 4d points for each nerve fiber, where the first three values indicate the $x,y,z$ coordinate in space and the forth value the radii of the nerve fiber at this point.
A nerve fiber is represented with this configuration as a cylindrical tube, capable of changin its radiius along the path.
Each fiber segments containing two adjecent points is checked if a collision occours with another fiber segment.
If so, both fiber segments are pushed slighty abort from each other.
This happends for all fiber segments in the volume as long as any collision is detected.
The length of a fiber segment as well as the bending radius of a nerve fiber is controlled by parameters.
By decrising the mean fiber sebment length the calculation can be significantly enhanced. 
By controlling the fiber bending radius the models are more stiff or \dummy{}.
\par
Multiple functions axists for users to fast generate commenly used nerve fiber  .. such as curves or fiber crossings.
% 
\par
% 
To speed up the process an octree is used to divide the volume intu subvolumes.
This octree can be parallel build and treversed on the \ac{CPU}.
A visualization is available to render the volume allowing a user to interact with the algorithm after each calculation step.
\par
% 
An additional project called \ac{MEDUSA} was desinged in cooperation with \ac{CEA} Neurospin allowing to desing non colliding nerve fiber modles with cells like astrocytes or olegodendrocytes which connect to myelin sorounding axons.
% 
% 
% 
\paragraph{Simulation software}
% 
The simulation software for \ac{3D-PLI} takes a configuration of nerve fibers as input and simulate the light matter interaction inside a \ac{3D-PLI} setup.
The simulation follows the M\"{u}ller-Stokes calculus.
The simulation is split into two conesequitiv parts.
The first parts generateds a discretised 3d tissue.
This tissue is used by the second part, which calculates the light matter iteranction from the stored tissue properties inside the volume by stepping multiple light vectors through the tissue.
The intensity is stored inside an 2d array which can be modelled as a \ac{CCD} array with resolution and noise.
The simulation is capable of simulating a tilted light beam wich allows to measure multiple views from the same sub volume the light is traversing.
Additionally the analysis algorithms are integrated in the software package allowing to calculate the nerve fiber orientation with the inclination analysis tool \ac{ROFL}.
\par
% 
The simulation is caplable of using multiple cores as well as multiple node system, which communicate via \ac{MPI}.
This allows to simulate large volume of nerve fibers which due to the necesarry discretised volume takes up a large .. of memory.
% 
% 
% 
\paragraph{Software package \acs{fastPLI}}
% 
All algorithms are written as modules in \python{} inside the software package \ac{fastPLI}.
The Software package is published as open source software package allowing users to share, ask and further develop the software for a high exchangeability. 
The software was tested by ultiple users and published inside the revied \ac{JOSS}.
% 
% 
% 
\paragraph{Nerve fiber modelling results}
% 
The nerve fiber modelling algorithm controlls the necesarry fiber movement to solve the collisions by the fiber segment length and the fiber bending radius.
Both parameters were characterized against the resulting orientations and computational speed.
To reduce the dimensionality of the possible configurations a set of parameters to describe the model was designed.
This set consit out of 2 relativ angles between the two fiber populations, an inclination angle of the entire model and a population fraction parameter.
This set allows to investigate nerve fiber models up to two nerve fiber populations with any crossing angle in 3d without describing every fiber by itself.
\par
% 
Using this parameters first the characteristics of the modelling parameters was charakterized.
A set of values could be identified which is suitable to generate non colliding nerve fiber models without distoring the initial configuration too much.
Additionally this parameter values are suitable to reduce the runtime in a reasanable amount without loosing any configuration charachteristics or impliing a bias on the resulting models.
To reduce the runtime even further it could be identified that about 1 order of magnitude the processing time can be reduced by not solving the models completly.
The influance of a non collision free model however has to be investigated in an aditionall study.
\par
% 
With the found model software parameters a library of up to two nerve fiber models described by the four model parameters was generated fo the later \ac{3D-PLI} simulation.
The orientation distribution was analysed to be able to use as a comparison for the simulation orientation analysis.
% 
% 
% 
\paragraph{\acs{3D-PLI} simulation results}
% 
Analog to the nerve fiber modelling software first the simulation software parameters were charackerized and measured.
The optical resolution of the used micrsoscope was reproduced with former results as well as the optical noise of the system.
From in \ac{3D-PLI} measured tissue samples the tissues properties were derived so that the simulation with means of its limitations can reproduce the results.
The most important characteristica of the simulations accuracy and speed, the \Voxelsize{} was characterized by using the upper prepaired models.
A lower limit could be identified which has to be used for a given nerve fiber radius.
\par
% 
Using the found software parameters and the prepaird nerve fiber models simulations were performed with a orientation analysis.
The models were spit into four groups, a single, a flat crossing, an inclined crossing and freely crossing nerve fiber populations.
This allowed to concentraid on specific behavier.
\par
% 
In the case of a single nerve fiber population only the inclination of the models was a variable.
With the simulation results it coud be shown, that in the case of a single fiber population the orientation can be correctly identified with an increased uncertanty for very stepp fibers.
Given multiple image pixel statistically the mean value can be correctly measured.
For very steep nerve fibers the relative effective tissue thickness becomes unstable in the inclination analysis.
This behavier can be used to take as an hint how uncertain the results are.
\par
% 
The flat crossing nerve fiber population results for the inclination analysis in a single value, which seems to follows the circular mean value of the individual orientations.
With this behavier mostly the dominant present nerve fiber population is visible, with a slighly systematic error prown to the orientation to the second nerve fiber direction.
With the here choosen nerve fiber radiius distribution and image resolution the involed nerve fiber population orientation could not be resolved.
\par
% 
The inclined crossing nerve fiber population has similar characteristics to the flat crossing case.
The inclination seems also to follor the circular mean value of each population, however due to the fact that inclined nerv fibers result in a smaller change of the polarization of the light, the resulting orientation is distorted to the less inclined fiber population.
Here also no individual orientation of one of the nerve fiber population can be identified from the inclination analysis.
\par
% 
The last studied models were a freely ... nerve fiber populations, so any configuration describable by the four model parameters were possible.
From a \ac{ODF} comparison metric model angles could be identified where the incination analysis of the \ac{3D-PLI} signal is tendet to an error prown and were it is very reliable.
The individual parameters were discussed as well and the underliing behavier could be identified leading to the error prown orientations.
\par
% 
A speedup analysis of the simulation showed that the simulation software with the capabilities of the \ac{MPI} is very parallizable and can utalize a multi node system which makes it possible to simulate large sections without the loss of computation time.
However the tissue generation process is sighly less performend.
However since multiple \ac{3D-PLI} tilted light beam simulation are run on the same discterized tissue this is not an issue.
\par
% 
Many possible further optimizations of the algorithm could be identified.
The overall ... is to increase the computation by utalizing parallel architecuters like an \ac{GPU}.
Already existing algorithms suitable for this kind of calculations, the nerve fiber modelling as well as the \ac{3D-PLI} simulations, could be identified.
\par
% 
Overall this thesis \dummy{}.