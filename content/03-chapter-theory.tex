\setcounter{chapter}{2}
\chapter{Theory}
\label{sec:theory}
% 
\cleanchapterquote{The first principle is that you must not fool yourself and you are the easiest person to fool.}{Richard P. Feynman}{}%\newline
%
\section{Introduction}
The following section lists the physical theories needed to describe the mathematics behind \ac{3D-PLI}. This includes the basic properties of light with the description of their polarization state, the optical models of the tissue, i.e. the nerve fibers and the mathematical description of the experimental setup of \ac{3D-PLI} and the analysis of its signal. The first part of this chapter was written with the help of \cite{demtroeder2, Fliebach2012}.
% 
% 
% 
\section{Electromagnetic Wave}
% 
Light is an electromagnetic wave. Electromagnetism is first fully described by James Clerk Maxwell. He formulated the Maxwell-Equations \cref{eq::maxwell_general}, generalized from the previous work of Johann Carl Friedrich Gau{\ss}, Michael Faraday and Andr\'{e}-Marie Amp\`{e}re and others:
% 
% \begin{align} 
% \begin{split} \label{eq::maxwell_macro}
%     \nabla \cdot \vec{D} &= 4\pi\rho_\text{f}\\
%     \nabla \cdot \vec{B} &= 0\\
%     \nabla \times \vec{E} &= -\frac{1}{c} \frac{\partial \vec{B}} {\partial t}\\
%     \nabla \times \vec{H} &= \frac{1}{c} \left(4\pi\vec{J}_\text{f} + \frac{\partial \vec{D}} {\partial t} \right)
% \end{split}
% \end{align}
% 
\begin{align} 
\begin{split} \label{eq::maxwell_general}
    \nabla \cdot \vec{E} &= \frac {\rho} {\varepsilon_0}\\
    \nabla \cdot \vec{B} &= 0\\
    \nabla \times \vec{E} &= -\frac{\partial \vec{B}} {\partial t}\\
    \nabla \times \vec{B} &= \mu_0\left(\vec{j} + \varepsilon_0 \frac{\partial \vec{E}} {\partial t} \right)
\end{split}
\end{align}
% 
with $\nabla \coloneqq \left({\frac{\partial}{\partial x}}, {\frac{\partial}{\partial y}}, {\frac{\partial}{\partial z}} \right)$ as vector differential operator, $\vec{E}$ as eletric field, $\rho$ as electric charge density, $\varepsilon_0$ as permittivity of free space, $\vec{B}$ as magnetic field, $\mu_0$ as permeability of free space and $\vec{j}$ as electric current density.
% 
The first equation states no electric field can be generated without an electric charge (charge conservation).
The second equation states, magnetic monopoles does not exists. The basic entity of an magnetic field is a dipole.
The third and forth equation show the interconnection of the electric and magnetic fields in space and time.
A change in the electric field yields to a magnetic field and vice versa.
Equation forth additional indicates the creation of a magnetic field from a electric current $\vec{j}$.
The maxwell equation fullfill the continuity equation $\divergence j + \frac{\partial \rho}{\partial t} = 0$.
This means that neither an electric of magnetic field can be created without either an electric current or a change of electric potential.
%
% \par
% 
\subsection{EM in a vacuum}
% 
In a vacuum \cref{eq::maxwell_general} simplifies with $\rho = 0$ and $\vec{j} = \vec{0}$ to:
% 
\begin{align}
\begin{split} \label{eq::maxwell_vacuum}
  \nabla \cdot \vec{E} &= 0 \quad\\
  \nabla \cdot \vec{B} &= 0 \quad\\
  \nabla \times \vec{E} &= -\frac{\partial\vec B}{\partial t}\\
  \nabla \times \vec{B} &= \mu_0\varepsilon_0 \frac{\partial\vec E}{\partial t}
  \end{split}
\end{align}
% 
With 
\begin{align}
\begin{split} 
    \nabla \times \nabla \times \vec{E} = -\nabla \times \frac{\partial \vec{B}} {\partial t} &= -\frac{\partial} {\partial t} \left( \nabla \times  \vec{B} \right)\\
    &= -\varepsilon_0 \cdot \mu_0 \frac{\partial^2 \vec{E}}{\partial t^2} \, ,
\end{split}
\end{align}
% 
the identity $\nabla \times \left( \nabla \times \vec{A} \right) = \nabla(\nabla \cdot \vec{A}) - \nabla^{2}\vec{A}$, $\mu_0\varepsilon_0 = \frac{1}{c^2}$ and $c$ as the speed of light\footnote{can be derived from Maxwells equation and Lorentz force in a vacuum}, this further simplifies to:
% 
\begin{align}
\begin{split} \label[pluralequation]{eq::maxwell_wave_equations}
  \frac{1}{c^2} \frac{\partial^2 \vec{E}}{\partial t^2} - \nabla^2 \vec{E} &= 0 \\
  \frac{1}{c^2} \frac{\partial^2 \vec{B}}{\partial t^2} - \nabla^2 \vec{B} &= 0
\end{split}
\end{align}
% 
% Another common representation is
% % 
% \begin{align}
% \begin{split} \label{eq::maxwell_wave_equations_box}
%   \Box \ \vec{E} &= 0 \\
%   \Box \ \vec{B} &= 0
% \end{split}
% \end{align}
% 
% with the help of the d'Alembert operator $\Box \coloneqq \partial^a\partial_a = \nabla^2 - \frac{1}{c^2} \frac{\partial^2}{\partial t^2}$.
% 
This also shows that $c^2  \frac{\partial B} {\partial z} = \frac{\partial E}{\partial t} \Rightarrow \vec{E} \cdot \vec{B} = 0$ and therefore the electric and magnetic field components are perpendicular to each other.
Additionally it also means, that space and time are interconnected and that light travels in vacuum with the speed of light $c$.
% 
\subsection{Solving MW Equ in vacuum}
% 
\Cref{eq::maxwell_wave_equations} have the shape of a wave equation and can therefore as known be solved by
% 
% Using the Maxwell equation in vacuum \ref{eq::maxwell_vacuum}, one can find the most simple solution to the differential equation is a wave:
% 
\begin{align}
\begin{split} \label{eq::dgl_solution}
  \vec{E}( \vec{r}, t ) &= g(\phi( \vec{r}, t )) = g( \omega t - \vec{k} \cdot \vec{r} + \varphi)\\
  \vec{B}( \vec{r}, t ) &= g(\phi( \vec{r}, t )) = g( \omega t - \vec{k} \cdot \vec{r} + \varphi )
\end{split}
\end{align}
% 
where $g$ is any well behaved function (continuous and differentiable) and therefore also their superposition.
% 
With the help of
% 
\begin{align}
k = \mathopen| \vec{k} \mathclose| = \frac{\omega}{c} =  \frac{2 \pi}{\lambda}
\end{align}
% 
a planar wave can be descriped by
% 
\begin{align}
\begin{split} \label{eq::plane_wave}
\vec{E}(\vec{r}) &= \vec{E}_0 \cdot e^{ -i \, \vec{k} \cdot \vec{r} }\\
 \vec{B}(\vec{r}) &= \vec{B}_0 \cdot e^{ -i \, \vec{k} \cdot \vec{r} }
\end{split}
\end{align}
% 
with $\vec{k}$ as Wave vector pointing into the travel direction of the light wave.
% 
% 
\subsection{Polarization}
% 
\begin{figure}[!t]
\centering
\setlength{\tikzwidth}{\textwidth}
\inputtikz[true]{gfx/pli/polarization_state}
\label{fig:polarization_state}
\caption{Illustration of light polarization state. Unpolarized light goes throug a linear polarizer which polarizes the light in one direction (w.l.o.g. $E_x$ and $E_y$ in phase). Afterwords it travels though a quarter wave retarder, which turn linear polarized light (of a specific wave length) into circular polarized light, where w.l.o.g. $E_x$ and $E_y$ are $\pi/2$ phase}
\end{figure}
% 
Since the light wave travels into one direction, and the electric field and magnetic field are perpendicular to $\vec{k}$ as well to each other, there orientation is a fundamental property of light called polarization.
Without any additional info, the polarization orientation is conventional in the direction of the electric field component.
% 
A superposition of x and y wave with z equal to direction of propagation yield w.l.o.g. to the general form
\begin{align}
\vec{E}(z,t) &= \begin{pmatrix} E_{0x} \cdot e^{ -i \phi_x } \\ E_{0y} \cdot e^{ -i \phi_y } \\ 0 \end{pmatrix}
e^{ -i (kz - \omega t)}
\end{align}
%
\begin{figure}[!t]
\centering
\tikzset{external/export=false}
%
% \captionsetup[sub]{textfont=normalsize}
% \captionsetup[sub]{position=top, skip=-14pt}
\tikzset{cross/.pic={
\draw[arrows={-Latex[scale=1]}] (-{sqrt(2)},0) -- ({sqrt(2)},0) node[anchor=north]{\small $E_x$};
\draw[arrows={-Latex[scale=1]}] (0,-{sqrt(2)}) -- (0,{sqrt(2)}) node[anchor=east]{\small $E_y$};
}}
%
% \subcaptionbox{}[.3\textwidth]{
\begin{tikzpicture}
\begin{scope}[shift={(0,0)}, local bounding box=A]
\pic at (0,0) {cross};
% \draw[] (0, 2) node[font=\small] () {linear, horizontal};
\draw[very thick, arrows={Latex[scale=1]-Latex[scale=1]}] (-1, 0) -- (1, 0);
\draw[] (0, -2) node () {$\begin{pmatrix} 1&0 \end{pmatrix}$};
\draw[] (0, -2.75) node () {$\begin{pmatrix} 1&1&0&0 \end{pmatrix}$};
\end{scope}
\node[anchor=south west, xshift=-1em] at (A.north west) {\small \textcolor{magenta}{\textbf{(a)}} linear, horizontal};
% }\hfill
% \subcaptionbox{}[.3\textwidth]{
\begin{scope}[shift={(4,0)}, local bounding box=B]
\pic at (0,0) {cross};
% \draw[] (0, 2) node[font=\small] () {linear, vertical};
\draw[very thick, arrows={Latex[scale=1]-Latex[scale=1]}] (0, -1) -- (0, 1);
\draw[] (0, -2) node () {$\begin{pmatrix} 0&1 \end{pmatrix}$};
\draw[] (0, -2.75) node () {$\begin{pmatrix} 1&-1&0&0 \end{pmatrix}$};
\end{scope}
\node[anchor=south west, xshift=-1em] at (B.north west) {\small \textcolor{magenta}{\textbf{(b)}} linear, vertical};
% }\hfill
% \subcaptionbox{}[.3\textwidth]{
\begin{scope}[shift={(8,0)}, local bounding box=C]
\pic at (0,0) {cross};
% \draw[] (0, 2) node[font=\small] () {linear, $\pi/4$};
\draw[very thick, arrows={Latex[scale=1]-Latex[scale=1]}] (-1, -1) -- (1, 1);
\draw[] (0, -2) node () {$\begin{pmatrix} 1&1 \end{pmatrix}$};
\draw[] (0, -2.75) node () {$\begin{pmatrix} 1&0&1&0 \end{pmatrix}$};
\begin{scope}[overlay]
\draw[] (2, -2) node () {Jones};
\draw[] (2, -2.75) node () {Stokes};
\end{scope}
\end{scope}
\node[anchor=south west, xshift=-1em] at (C.north west) {\small \textcolor{magenta}{\textbf{(c)}} linear, $\pi/4$};
% }
% \\[2em]
% \subcaptionbox{}[.32\textwidth]{
\begin{scope}[shift={(0,-5.75)}, local bounding box=D]
\pic at (0,0) {cross};
% \draw[] (0, 2) node[font=\small] () {left circular};
\draw[very thick] plot[domain=0:360,samples=90,smooth] ({cos(\x)},{sin(\x)});
\draw[very thick, arrows={-Latex[scale=1]}] plot[domain=44:45,samples=1] ({cos(\x)},{sin(\x)});
\draw[very thick, arrows={-Latex[scale=1]}] plot[domain=224:225,samples=1] ({cos(\x)},{sin(\x)});
\draw[] (0, -2) node () {$\begin{pmatrix} 1&i \end{pmatrix}$};
\draw[] (0, -2.75) node () {$\begin{pmatrix} 1&0&0&-1 \end{pmatrix}$};
\end{scope}
\node[anchor=south west, xshift=-1em] at (D.north west) {\small \textcolor{magenta}{\textbf{(d)}} left circular};
% }\hfill
% \subcaptionbox{}[.32\textwidth]{
\begin{scope}[shift={(4,-5.75)}, local bounding box=E]
\pic at (0,0) {cross};
% \draw[] (0, 2) node[font=\small] () {right circular};
\draw[very thick] plot[domain=0:360,samples=90,smooth] ({cos(\x)},{sin(\x)});
\draw[very thick, arrows={-Latex[scale=1]}] plot[domain=46:45,samples=1] ({cos(\x)},{sin(\x)});
\draw[very thick, arrows={-Latex[scale=1]}] plot[domain=226:225,samples=1] ({cos(\x)},{sin(\x)});
\draw[] (0, -2) node () {$\begin{pmatrix} 1&-i \end{pmatrix}$};
\draw[] (0, -2.75) node () {$\begin{pmatrix} 1&0&0&1 \end{pmatrix}$};
\end{scope}
\node[anchor=south west, xshift=-1em] at (E.north west) {\small \textcolor{magenta}{\textbf{(e)}} right circular};
% }\hfill
% \subcaptionbox{}[.32\textwidth]{
\begin{scope}[shift={(8,-5.75)}, local bounding box=F]
\pic at (0,0) {cross};
% \draw[] (0, 2) node[font=\small] () {unpolarized};
\draw[] (0, -2.75) node () {$\begin{pmatrix} 1&0&0&0 \end{pmatrix}$};
\begin{scope}[overlay]
\draw[] (2, -2) node () {Jones};
\draw[] (2, -2.75) node () {Stokes};
\end{scope}
\end{scope}
\node[anchor=south west, xshift=-1em] at (F.north west) {\small \textcolor{magenta}{\textbf{(f)}} unpolarized};
%
\path[] ($(A.west)!-0.075!(C.east)$) -- ($(A.west)!1.075!(C.east)$);
\end{tikzpicture}
% }
%
\caption{polarization states, check vector length,\itodo{test speed}} 
\label{fig:polarization_state_vectors}
\end{figure}
%
\Cref{fig:polarization_state_vectors} shows another representation of the polarization sate of a light wave.
It show the component perpendicular to the travelling direction.
Therefore the time variation can be shown on the xy-plane.
Additionally the states can be describe by the Jones or Stokes calculus, later described in \cref{sec:jones,sec:mueller_stokes}.
% 
% 
% 
\subsection{Light in medium}
% 
\todo{herleiten}
The Maxwell equations in \dummy{} are described above. They can be solved analog to \dummy{}. This yields to some special behavior, \eg{} the magnetic and electric field component get out of phase.
However here only the two properties of absorption and diffraction are described.
They derivation can be found \eg{} \cite{demtroeder2, Fliebach2012}.
% 
\subsection{Absorption}
% 
\begin{align}
    I = I_0 \exp(-\mu x)
\end{align}
\todo{herleiten}
% 
Beersche law of absorption with $\mu = \frac{4\pi \kappa}{\lambda_0}$ where $\kappa$ is the imaginary part of the complex refractive index.
% 
\subsection{Refraction}
% 
\begin{figure}[!t]
\centering
\setlength{\tikzwidth}{\textwidth}
\inputtikz{gfx/pli/refraction}
\label{fig:optic_refraction}
\caption{Illustration of refraction.}
\end{figure}
% 
Refraction is the change of direction of light as it passes from one medium to another.
This can be shown by using the full Maxwell equations  for non-conductive material, \ie{} $\vec{j} = 0, \rho = 0$, that the differential equating consist out of a primary wave with from atomic medium induced secondary waves, which leads to a reduction of the velocity of the resulting wave.
Mathematically this can be described by a complex number $n = c' / c$.
Using this relationship at a boundary surface between two media, one can show that the incident light beam splits into a reflecting and transmitting light wave.
The reflecting light wave has the same angle as the incident light beam relative the the surface normal.
The transmitting light beam however due to the reduction of the velocity, changes its direction described by the Snellius law
\begin{align}
    n_0 \sin \alpha = n_1 \sin \beta
\end{align}

% 
\begin{align}
\underline{n} = n + i\kappa
\end{align}
% 
\begin{align}
\begin{split}
\vec{E}(z, t) &= \operatorname{Re}\! \left[\vec{E}_0 \cdot e^{i\, (kz - \omega t)}\right] \\
&= \operatorname{Re}\! \left[\vec{E}_0 \cdot e^{i\, (2\pi(n + i\kappa)z/\lambda_0 - \omega t)}\right] \\
&= e^{-2\pi \kappa z/\lambda_0} \operatorname{Re}\! \left[\vec{E}_0 \cdot e^{i\, (kz - \omega t)}\right]
\end{split}
\end{align}\todo{?, was macht das hier}
% 
% 
% 
\subsection{Birefingence}
%
\begin{figure}[!t]
\centering
\setlength{\tikzwidth}{\textwidth}
\inputtikz[true]{gfx/pli/retardation}
\caption{Illustration of retardation. The linear $\SI{45}{\degree}$ polarized light wave is decomposed into the $x$ and $y$ component in the birefringend medium. The y compoinent travels faster than the x component. Therefore a pahse shift $\phi$ betwen both components occure. This leads in the case of a $\lambda/4$ retarder to a circulalar polarized light wave. }
\label{fig:optic_retardation}
\end{figure}
% 
% \begin{figure}[!t]
% \centering
% \subcaptionbox{}[.49\textwidth]{
% \setlength{\tikzwidth}{0.49\textwidth}
% \inputtikz{gfx/pli/ellipsoid_a}}
% \subcaptionbox{}[.49\textwidth]{
% \setlength{\tikzwidth}{0.49\textwidth}
% \inputtikz{gfx/pli/ellipsoid_b}}
% \caption{birefringence elipsoid}
% \label{fig:index_elipsoid}
% \end{figure}
% 
Material can have a different refractive index depending on the relative orientation and polarization of the light beam.
This property is known as birefringence.
The refractive index can be described as the refractive index $n_o$ and extraordinary  $n_e$.
Since these two are perpendicular to each other, one can split the light beam into the same perpendicular parts and describe each by its own.
These two light beams have a different direction, except for the trivial case of a light beam is perpendicular to the surface.
For small relative length (hängt von n ab, bzw der phasenänderung) the two light beams leave the material at the same point and recombined at the end.
The change of phase is called birefringence and is described by the difference between the .. and ...
% 
\begin{align}
    \Delta n = n_e - n_o \> .
\end{align}
% 
% This .. can be visualized by the index ellipsoid \cref{fig:index_elipsoid}.
% The change of the amplitude and phase can be described by the following two vector matrices descriptions.
% 
% 
% 
\subsection{Jones calculus}
\label{sec:jones}
% 
The Jones calculus, introduced by Robert Clark Jones in 1941, describes the polarization state of a light beam by a complex vector $J$:
% 
\begin{align}
    \vec{J} = \begin{pmatrix} E_x \exp(i \phi_x) \\ E_y \exp(i \phi_y) \end{pmatrix}
\end{align}
% 
The amplitude of the perpendicular components are $E_x$ and $E_y$ with their phase $\phi_x$ and $\phi_y$.
Optical elements, which changes the polarization state, such as a polarization filter and retarder, can be desribed by a matrix:
% 
\begin{align}
\begin{split}
\mat{P}_x = 
\begin{pmatrix}
1 & 0 \\ 0 & 0
\end{pmatrix}
, \enspace
\mat{P}_y = 
\begin{pmatrix}
0 & 0 \\ 0 & 1
\end{pmatrix}
, \enspace
\mat{M} =
\begin{pmatrix}
e^{i \varphi_x} & 0 \\ 0 & e^{i \varphi_y}
\end{pmatrix}
, \enspace
\Lambda_{1/4}=
e^{\frac{i \pi}{4}}
\begin{pmatrix}
1 & 0 \\ 0 & -i
\end{pmatrix}
\end{split}
\end{align}
% 
A rotation of an optical element can be achieved by a 2d rotation matrix $\mat{R}$:
% 
\begin{align}
\begin{split}
\mat{M}(\theta )=\mat{R}(\theta )\cdot\mat{M}\cdot\mat{R}(-\theta )
, \enspace
\mat{R}(\theta ) = 
\begin{pmatrix}
\cos \theta & -\sin \theta \\
\sin \theta & \cos \theta
\end{pmatrix}
\end{split}
\end{align}
% 
% 
% 
\subsection{M\"uller-Stokes}\label{sec:Mueller-Stokes}
\label{sec:mueller_stokes}
% 
Analog to \cref{sec:jones} the M\"uller-Stokes formalism, described by George Gabriel Stokes in 1852, also describes the polarization state of a light beam.
However it does not use the absolute .. but the relative polarization between both components:
% 
\paragraph{Stokes vector}
\begin{align}
\begin{split}
S_0 &= I \\
S_1 &= I p \cos 2\psi \cos 2\chi \\
S_2 &= I p \sin 2\psi \cos 2\chi \\
S_3 &= I p \sin 2\chi
\end{split} \hspace{-6em}
\begin{split}
& \equiv \langle E_x^{2} \rangle + \langle E_y^{2} \rangle \\
%  & = \langle E_a^{2} \rangle + \langle E_b^{2} \rangle \\
%  & =  \langle E_l^{2} \rangle + \langle E_r^{2} \rangle \\
& \equiv \langle E_x^{2} \rangle - \langle E_y^{2} \rangle \\
& \equiv \langle E_a^{2} \rangle - \langle E_b^{2} \rangle \\
& \equiv  \langle E_l^{2} \rangle - \langle E_r^{2} \rangle
\end{split}
, \enspace
\vec{S} =
\begin{pmatrix} S_0 \\ S_1 \\ S_2 \\ S_3\end{pmatrix}
\end{align}
% 
% \begin{align}
% \begin{split}
% S_0 & \equiv \langle E_x^{2} \rangle + \langle E_y^{2} \rangle \\
%  & = \langle E_a^{2} \rangle + \langle E_b^{2} \rangle \\
%  & =  \langle E_l^{2} \rangle + \langle E_r^{2} \rangle \\
% S_1 & \equiv \langle E_x^{2} \rangle - \langle E_y^{2} \rangle \\
% S_2 & \equiv \langle E_a^{2} \rangle - \langle E_b^{2} \rangle \\
% S_3 & \equiv  \langle E_l^{2} \rangle - \langle E_r^{2} \rangle
% \end{split}
% \end{align}
% 
Therefore the phase can not be described anymore. \todo{variablen definieren}
However the relative phase information is stored and can be used to describe also polarization states of polarization filters, which can not be described by Jones.
Analog to Jones one can formulate the matrices for the optical components:
% 
\paragraph{Linear Polarizer}
\begin{align}
\mat{P}_x=\frac{1}{2}
\begin{pmatrix}
    1 & 1 & 0 & 0 \\
    1 & 1 & 0 & 0 \\
    0 & 0 & 0 & 0 \\
    0 & 0 & 0 & 0
  \end{pmatrix}
, \;
\mat{P}_y=\frac{1}{2}
\begin{pmatrix}
     1 & -1 & 0 & 0 \\
    -1 &  1 & 0 & 0 \\
     0 &  0 & 0 & 0 \\
     0 &  0 & 0 & 0
\end{pmatrix}
\end{align}
% 
\paragraph{retarder}
\begin{align}
\mat{M}=\
\begin{pmatrix}
    1 & 0 & 0 &  0 \\
    0 & 1 & 0 &  0 \\
    0 & 0 & \cos \delta & \sin \delta \\
    0 & 0 & -\sin \delta &  \cos \delta
\end{pmatrix}
\end{align}
% 
\paragraph{Quarter-wave plate (fast-axis vertical)}
\begin{align}
\Lambda_{1/4}=\
\begin{pmatrix}
    1 & 0 & 0 &  0 \\
    0 & 1 & 0 &  0 \\
    0 & 0 & 0 & -1 \\
    0 & 0 & 1 &  0
\end{pmatrix}
\end{align}
% 
\paragraph{Rotation Matrix}
\begin{align}
\mat{R}(\theta)=
\begin{split}
\begin{pmatrix}
    1 &                0 &               0 & 0 \\
    0 & \cos{(2\theta)} & -\sin{(2\theta)} & 0 \\
    0 & \sin{(2\theta)} & \cos{(2\theta)} & 0 \\
    0 &                0 &               0 & 1
  \end{pmatrix}
\end{split}
\end{align}
% 
Analog to \dummy{} rotations are applied by
\begin{align}
\mat{M}(\theta )=\mat{R}(\theta )\cdot\mat{M}\cdot\mat{R}(-\theta )
\end{align}
% 
% \section{Tissue Discretization}
% % 
% By deviding the volume into small diskreticed subvolumes, one can multiply the .. all together and ... (analog Riemann sum)
% \begin{align}
%     \int F \, dt \approx \sum_n F \, \Delta t
% \end{align}
% see simulation?
% 
%  see simulation
% \paragraph{Micro vs Macro:}
% % 
% % \begin{align}
% %   \int_{y_\textit{min}}^{y_\textit{max}} \vec{v}(y) \,dy\\
% %   x = const
% %   y = y(\alpha,x) = tan(\alpha) \cdot x\\
% %   \vec{v}_r = \begin{pmatrix} \cos(\alpha)\\ \sin(\alpha)\\0\end{pmatrix}, \, \vec{v}_p = \begin{pmatrix} 0\\ 0\\1\end{pmatrix}\\
% %   \vec{v}_r = \begin{pmatrix} \cos(\arctan(y/x))\\ \sin(\arctan(y/x))\\0\end{pmatrix}\\
% %   \int_{y_\textit{min}}^{y_\textit{max}} \vec{v}_p(y) \,dy = (y_\textit{max} - y_\textit{min}) \cdot e_z\\
% %   \int_{y_\textit{min}}^{y_\textit{max}} \vec{v}_r(y) \,dy = \int_{y_\textit{min}}^{y_\textit{max}} \cos(\arctan(y/x)) dy \cdot e_x \\
% %   = x\left(\sinh^{-1}(y_\textit{max}/x) - \sinh^{-1}(y_\textit{min}/x)\right) \\
% %   = 2x \left(\sinh^{-1}\left(\frac{2\sqrt{R-x^2}}{x}\right)\right)
% % \end{align}
% % 
% % \begin{align}
% %     % \frac{\oint \vec{v}_z \mathrm{d}A}{\oint \vec{v}_x \mathrm{d}A} = 
% %     % \frac{\int_{-1}^{1}\abs{\vec{v}} \mathrm{d}x}{\int_{-\frac{\pi}{2}}^{\frac{\pi}{2}} \abs{\vec{v}} \cos(\varphi) \mathrm{d}\varphi} =
% %     % \frac{2}{1}
% %     \frac{A_{\Box}}{A_{\circ}} = \frac{4}{\pi}
% % \end{align}
% % 
% \todo{why 2*dn?}
% 
\section{3D-PLI setup}\label{sec:expSetup}
%
\begin{figure}[!t]
    \captionsetup[sub]{position=top}
    \setlength{\tikzwidth}{\textwidth}
	\centering
% 	\subcaptionbox{\label{setup-lap} \ac{LAP}}[\textwidth]{
% % 		\resizebox{0.95\textwidth}{!}{
% 		\inputtikz{gfx/pli/pli_setup}
% % 	}
% 	}\\
% 	\subcaptionbox{\label{setup-pm} \ac{PM}}[\textwidth]{
% % 		\resizebox{0.95\textwidth}{!}{
		\inputtikz{gfx/pli/pli_setup_pm}
% % 	}
% 	}
	\caption{Illustration of PLI setup.}
	\label{fig:pli_setup}
\end{figure}
%
% \begin{figure}[!t]
%     % \captionsetup[sub]{position=top}
%     \setlength{\tikzwidth}{0.3\textwidth}
% 	\centering
% 	\subcaptionbox{\ac{LAP}}[.45\textwidth]{\hfill
% 			\inputtikz{gfx/pli/pli_detail}\hfill}\hfill
% 	\subcaptionbox{\ac{PM}}[.45\textwidth]{\hfill
% 			\inputtikz{gfx/pli/pli_detail_pm}\hfill}
% 	\caption{Illustration of \acs{3D-PLI} setup.}
% 	\label{fig:pli_detail}
% \end{figure}
%
Three microscopic setup exists, which are based on the same physical principle \cite{Axer2011} (see \cref{fig:pli_setup}).
Polarized light of a wavelength of \SI{525}{\nano\meter} is shine through a tissue section. A circular polarizer (a \SI{45}{\degree} oriented quarter wave retarder and polarizer) is placed behind the tissue.
By rotating the polarizer, the change of intensity is messured by a \ac{CCD} sensor.
\par
% 
The first optical microscope called \ac{LAP} is able to measure an entire human brain section in a single image with a pixel size of \SI{60}{\micro\meter}\footnote{By changing the optic a pixel size of $\SI{40}{\micro\meter}$ and $\SI{20}{\micro\meter}$ is also available.}.
Also allows the tissue specimen to be tilted.
This leads to a change of the nerve fibers orientation, which leads to an change of the measured signal.
This change is later used to analyse the 3d orientation of the nerve fibers inside the section (see \dummy{}).
The optical setup of the \ac{LAP} is a little bit different in the matter that the quarter wave retarder is place between the first polarizer and the tissue section and all three optical elements are rotated at the same time.
This yields to the same results, however dust particels on one of the optical elements can easier be identified since they rotate as well.
The other microscopes are in a closed container which protects the optical elements for pollution.
\\
A higher resolution microscope is the \ac{LMP} which allows to measure a tile of 2048x2048 with a pixel size of \SI{1.3}{\micro\meter}.
By measuring the tissue with multiple overlapping images, the overlap can be used to stitch them together to a complete image.
This setup is not able to change the light path.
The 3d information can be estimated by analysing the distribution of the retardation and transmittance.
However the sign of the inclination can not be identified.
\\
The third setup \ac{LMP3D} is able to change the light path \cite{Wiese:887678}.
It allows this by using a conical light path.
When using a slit, only light of a certain angle is allowed to pass through the tissue.
By changing the position of the slit, the different light paths can be measured.
\\
% 
The tilted light beam (or tissue section) is usually measured in 4 perpendicular orientations: north, east, south and west.
\par
% 
The final image is taken by a camera, which uses a \ac{CCD} sensor.
Generally a \ac{CCD} sensor consist out of an array of metal oxide semiconductor (MOS) capacitors.
In each capacitor an electric charge will be stored, which is set free by an incoming photons in means of the photo electric effect.
After a read out process, which also contains a electric gain, the resulting values can be stored as an image.
Its value is, as long as the capacitors are not saturated or gain does not exceed its limits, linear correlated with the number of photons.
The noise of the signal comes mainly from two parts.
The first one is thermal noise, which can lead to electric charges in the MOS capacitors.
Secondly the gain of the signal underlays noise.
These noise sources combined yield to an poison like distribution, due too the nature of digital positive values.
For intensity values $>> 0$ it can be modelled by a normal distribution.
% 
% 
% 
\subsection{Signal}
% 
From the M\"{u}ller Matrices (\cref{sec:Mueller-Stokes}) as shown in \cite{MenzelMaster,MenzelDissertation}, the intensity signal, which is the first component of the Stokes vector, follows a sinusiodal curve:
% 
% \begin{align}
% I = \underbrace{\frac{I_0}{2}}_{\mathclap{\mathit{transmittance}}}
%     \bigl[ \sin\bigl(2\rho - 2{\underbrace{\vphantom{\frac{I_0}{2}} \varphi}_{\mathclap{\mathit{direction}}}} \bigr)
%     \cdot \sin \bigl( {\underbrace{\vphantom{\frac{I_0}{2}} 2\pi\frac{d \dn}{\lambda} \cos^2\left( \alpha \right)}_{\mathclap{\delta \, \coloneqq \, \mathit{retardation}}}} \bigr) \bigr]
% \end{align}
\begin{align}
\label{eq:pli_signal}
\begin{split}
I(\rho, \varphi, \alpha, d) =\frac{I_0}{2}\bigl[ \sin\bigl(2\rho - 2\varphi \bigr)\cdot \sin \bigl( 2\pi\frac{d \dn}{\lambda} \cos^2\left( \alpha \right) \bigr) \bigr] \\
\text{with} \enspace \delta \coloneqq 2\pi\frac{d \dn}{\lambda} \cos^2\left( \alpha \right) \enspace 
\text{and} \enspace \trel \coloneqq \frac{d \dn}{\lambda}
\end{split}
\end{align}
% 
Since \cref{eq:pli_signal} describes a sinosoidal signal, the fourier analysis is a obvious choise.
For a discrete, aquidistant measurement of the rotation angles $\rho$, one can use the fourier series with the first three coefficients to describe the signal:
% 
% \begin{align}
% \begin{split}
% A \sin(\omega t + \alpha) + B \sin(\omega t + \beta) &= \sqrt{C^2 + D^2} \cdot \sin \, \left( \omega t + \tan^{-1} \left( \frac{D}{C} \right) \right)
% \end{split}
% \\
% \begin{split}
% C &= A \cos(\alpha)+ B \cos(\beta)\\
% D &= A \sin(\alpha)+ B \sin(\beta)
% \end{split} \nonumber 
% \end{align}

\begin{align}
\begin{split}
\rho &= [\SI{0}{\degree}, \frac{1\cdot180}{N+1}\si{\degree}, \frac{2\cdot180}{N+1}\si{\degree}, ..., \frac{N\cdot180}{N+1}\si{\degree}]\\
a_0 &= \frac{1}{N} \sum_i^N I_i\\
a_1 &= \frac{2}{N} \sum_i^N I_i \cdot \sin(2 \cdot \rho_i)\\
b_1 &= \frac{2}{N} \sum_i^N I_i \cdot \cos(2 \cdot \rho_i)
\end{split}
\end{align}

With these the final \ac{3D-PLI} modalities are calculated:

\begin{align}
\begin{split}
\mathit{transmittance} &= 2 \cdot a_0 -> I_0/2\\
\mathit{direction} &= 0.5 \cdot \atantwo(-b_1 / a_1) -> \varphi\\
\mathit{retardation} &= \frac{\sqrt{a_1^2 + b_1^2}}{a_0} -> \sin(...)
\end{split}
\end{align}
% 
\itodo{Section Images}
% 
% 
% 
\subsection{Inclination analysis}
% 
To be able to analyse the inclination $\alpha$ one has to distinguish the relative strength of the birefringence from the $\cos^2(\alpha)$.
For this purpose a tiltable specimen was developed, which allows measuring the light signal through the tissue with a different angle of incidence.
This mean, that tissue and therefore the nerve fibers also change their orientation by the tilting angles $\theta, \varphi$.
Additionally the distance the light travels through the tissue, elongates by $1/\cos(\theta)$ \cref{fig:tilted_side_view}.
% 
Depending on the \pixelsize{}, this light travels though the same volume, but with another orientation.
Therefore a measurement of multiple light paths can be ... and the resulting signals can be used to analyse the inclination and relative birefrigente tissue thickness \trel{}.
The angle of incidence on the glass ... and the tissue also changes the angle of the light by Snellius law.
All angles mention here are if not specified always the change of angle of the light inside the tissue (see \cref{fig:tilting_camera_view}). 
This also leads to a perspective shift, which has to be corrected by a registration process.
In this case an affine transformation.
This effect is however neglected in the simulation, since it only adds a parallel offset.
% 
In \cite{Schmitz2018} this analysis published under the name of \ac{ROFL}, which builds up on the work of \cite{Wiese:887678}.
The idea is to fit the measured signals of all light paths simultaniously.
Because the change of signal is proportional to the $\cos(\alpha)$ this however means, that for steep fibers, the changes are not only small, but also the amplitude of the signal is very small.
This leads to the problem of increasing uncertanty with increasing inclination angle.
\\
A further problem is that for a smaller \pixelsize{} the light path cannot be neglected anymore.
For the \ac{LMP3D} system this means, that with a maximal tilting angle of about $\SI{3.9}{\degree}$ and a usually tissue thickness of $\SI{60}{\micro\meter}$ the light path is measured in n \dummy{} pixels away from the non tilting case, if the rotation point is in the center of the tissue.
\\
For homogeneous tissue regions like parts in the dense \ac{WM}, this can be ignored.
For single fiber paths in the \ac{GM} however this is currently an unsolved problem.
% 
% 
% 
\section{Orientation visualization}
% 
\begin{figure}[!t]
\centering
\setlength{\tikzwidth}{0.4\textwidth}
\begin{center}
\begin{tabular}{m{6cm}m{6cm}}
\includegraphics[width=\tikzwidth]{gfx/pli/color_sphere.png}
&
\inputtikz{gfx/pli/hsv_sphere}
\\
\begin{minipage}[t]{0.42\textwidth}
\leavevmode\subcaption{2d hsv sphere}
\end{minipage}
&
\begin{minipage}[t]{0.42\textwidth}
\leavevmode\subcaption{\label{fig:sphere}3d hsv sphere}
\end{minipage}
\end{tabular}
\end{center}
% 
\vspace{-1em} % because of tabular?
\caption{collor spheres}
\label{fig:spheres}
\end{figure}
% 
% 
The orientation of the birefringence axis is descibed by the direction angle $\varphi$ and inclination angle $\alpha$ (see \cref{fig:sphere}).
To be able to show the 3d information, the \textit{hsv-black} is usually shown in \ac{3D-PLI}.
It color code the orientation by:
\begin{align}
\begin{split}
    H &= \varphi/\pi\\
    S &= 1\\
    V &= 1-\alpha / \pi/2
\end{split}
\end{align}
This way the color corresponds to the direction, and the color value to the inclination.
Since the orientation, unlike a vector, covers only a half sphere, the colors are point symmetric.
% 
\itodo{Section Images}
% fig:spheres
% 
% 
\section{optical resolution}
\label{sec:opticalResolution}
% 
The optical resolution of an imaging system describes the minimal size of on object which can still be resolved.
This property is limited by aberration and diffraction.
Aberration causes a blurring of the image while diffraction can lead to superimposed diffraction patterns.
If the diffraction is caused by many small objects regarding to the resolution this also looks like a blurring.\\
Ernst Abbe described as on of the first that the resolution correlates with the light wave $\lambda$: 
\begin{align}
d=\frac{ \lambda}{2 n \sin \theta} = \frac{\lambda}{2\mathrm{NA}} \> .
\end{align}
$d$ is the minimal resolvable distance, $n$ the refractive index, $\theta$ the angle of a spot light, which can be combined to the better known numerical aperture $\mathrm{NA}$.
This results in an absolute threshold for a light microscope above which the resolution can no longer be improved.
For the wavelength used in \ac{3D-PLI} with $\lambda = \SI{525}{\nano\meter}$ and a numerical apperatur of $\mathrm{NA} = \SI{}{}$ this yields to a limit of \dummy{}.
% 
\begin{figure}[!t]
\setlength{\tikzwidth}{0.45\textwidth}
\centering
% \subcaptionbox{}{
\inputtikz{gfx/pli/rayleigh}
\caption[Raylay criterium]{rayleigh criteria. The minima of the one function is in the maxima of the other.}
\label{fig:rayleigh}
\end{figure}
% 
\itodo{check} 
% 
To account for the optical setup, three things have to be applied to a simulated measurement.
% 
\paragraph{Blurring}
The optical resolution the light rays have to be blurred.
This is classically done via a 2d Gaussian convolution:
\begin{align}
    (f * g)(x) = \int f(\tau)g(x-\tau)d\tau
\end{align}
% 
\paragraph{Sampling}
Since the number and final position of the light rays is according to the voxels, all intensities of an image pixel have to be combined.
Here it is done via a mean value sampling:
\begin{align}
    \hat{I}(n,m) = \sum_{i=n \cdot dx}^{(n+1) \cdot dx-1}\sum_{j=m \cdot dy}^{(m+1) \cdot dy-1} I(i,j)
\end{align}
This in contrast to resizing, does not interpolates the image.
% 
\paragraph{Noise}
The last step is to replicate the noise of the image setup.
To account for this a noise model has to be applied to each image pixel.
\cite{Wiese:887678} showed that this can be done via a normal distribution.
% 
\begin{align}
    I = I + normal(\sigma, \mu)
\end{align}
% 
\par
% 
All three effects must be characterized for the system being simulated.
% 
% 
% 
\section{Computational speedup techniques}
% 
Among other specific techniques described in the algorithm chapters \cref{chap:sof:modelling,cha:sof:simulation}, one major technique was utilized to speed up the calculations.
\par
% 
The computational heavy code is written in \cpp{}.
There the \code{std::vector} has the advantage that the data is linear in memory.
Modern \acp{CPU} have a built-in method called \say{ache prefetching}.
Data must be prepared and sent from the \ac{RAM} to the cache of the \acp{CPU}.
This takes time.
The main advantage of the cache is that it is very fast.
Even the distance is so small, that the speed of light is no longer limiting the  \acp{CPU} connection to the cache.
It is constructed inside the \ac{CPU}.
Therefore its size is very limited , usually around $\si{\mega\byte}$.
The prefetcher is an ingenious directive, which not only obtains the item at address $i$ in memory, but also the item next to it ($i+1$ or $i-1$ depending on the algorithm).
Since many algorithms pass through arrays, the next item to be calculated is usually the next (or previous) item.
Therefore the time needed to copy the data into the cache and prepare it is reduced.
It can be shown that for linear operations on the memory the cache prefetcher reduces the time so much that it behaves as if the \ac{CPU} has an infinite cache.
\par
% 
Another technique is the use of modern compiler open source compiler like \name{clang} or \name{gcc}.
They have optimization algorithm build in, which optimizes the code to the architecture of the machine.
For example if the number of iterations is known at compile time a for loop can be \name{unrolled} to speed up the calculations, because it does not have to check anymore, if the conditions are fulfilled.
To check this optimizations the bottlenecking code was checked with tool like \name{Compiler Explorer}\footnote{https://godbolt.org/} and \name{C++ Insight}\footnote{https://cppinsights.io/}.