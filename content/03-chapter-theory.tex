\setcounter{chapter}{2}
\chapter{Theory}
\label{sec:theory}
%
The following section lists the physical theories needed to describe the mathematics behind \ac{3D-PLI}. This includes the basic properties of light with the description of their polarization state, the optical models of the tissue, i.e. the nerve fibers and the mathematical description of the experimental setup of \ac{3D-PLI}. The first part of this chapter was written with the help of \cite{demtroeder2, Fliebach2012}.
% 
% 
% 
\section{Electromagnetic Wave}
% 
Light is an electromagnetic wave. Electromagnetism is first fully described by James Clerk Maxwell. He formulated the Maxwell-Equations \cref{eq::maxwell_general}, generalized from the previous work of Johann Carl Friedrich Gau{\ss}, Michael Faraday and Andr\'{e}-Marie Amp\`{e}re and others. 
% 
% \begin{align} 
% \begin{split} \label{eq::maxwell_macro}
%     \nabla \cdot \vec{D} &= 4\pi\rho_\text{f}\\
%     \nabla \cdot \vec{B} &= 0\\
%     \nabla \times \vec{E} &= -\frac{1}{c} \frac{\partial \vec{B}} {\partial t}\\
%     \nabla \times \vec{H} &= \frac{1}{c} \left(4\pi\vec{J}_\text{f} + \frac{\partial \vec{D}} {\partial t} \right)
% \end{split}
% \end{align}
% 
\begin{align} 
\begin{split} \label{eq::maxwell_general}
    \nabla \cdot \vec{E} &= \frac {\rho} {\varepsilon_0}\\
    \nabla \cdot \vec{B} &= 0\\
    \nabla \times \vec{E} &= -\frac{\partial \vec{B}} {\partial t}\\
    \nabla \times \vec{B} &= \mu_0\left(\vec{j} + \varepsilon_0 \frac{\partial \vec{E}} {\partial t} \right)
\end{split}
\end{align}
% 
with $\nabla \coloneqq \left({\frac{\partial}{\partial x}}, {\frac{\partial}{\partial y}}, {\frac{\partial}{\partial z}} \right)$ as vector differential operator, $\vec{E}$ as eletric field, $\rho$ as electric charge density, $\varepsilon_0$ as permittivity of free space, $\vec{B}$ as magnetic field, $\mu_0$ as permeability of free space and $\vec{j}$ as electric current density.
% 
The first equation states no electric field can be generated without an electric charge (charge conservation).
The second equation states, magnetic monopoles does not exists. The basic entity of an magnetic field is a dipole.
The third and forth equation show the interconnection of the electric and magnetic fields in space and time. A change in the electric field yields to a magnetic field and vice versa. Equation forth additional indicates the creation of a magnetic field from a electric current. The maxwell equation fullfill the continoiut ... $\divergence j + \frac{\partial \rho}{\partial t} = 0$
% 
In a vacuum this simplifies with $\rho = 0$ and $\vec{J} = \vec{0}$ to:
% 
\begin{align}
\begin{split} \label{eq::maxwell_vacuum}
  \nabla \cdot \vec{E} &= 0 \quad\\
  \nabla \cdot \vec{B} &= 0 \quad\\
  \nabla \times \vec{E} &= -\frac{\partial\vec B}{\partial t}\\
  \nabla \times \vec{B} &= \mu_0\varepsilon_0 \frac{\partial\vec E}{\partial t}
  \end{split}
\end{align}
% 
With 
\begin{align}
\begin{split} 
    \nabla \times \nabla \times \vec{E} = -\nabla \times \frac{\partial \vec{B}} {\partial t} &= -\frac{\partial} {\partial t} \left( \nabla \times  \vec{B} \right)\\
    &= -\varepsilon_0 \cdot \mu_0 \frac{\partial^2 \vec{E}}{\partial t^2}
\end{split}
\end{align}
% 
the identity $\nabla \times \left( \nabla \times \vec{A} \right) = \nabla(\nabla \cdot \vec{A}) - \nabla^{2}\vec{A}$ and $\mu_0\varepsilon_0 = \frac{1}{c^2}$, with $c$ as the speed of light\footnote{can be derived from Maxwells equation and Lorentz force in a vacuum}, this further simplifies to:
% 
\begin{align}
\begin{split} \label[pluralequation]{eq::maxwell_wave_equations}
  \frac{1}{c^2} \frac{\partial^2 \vec{E}}{\partial t^2} - \nabla^2 \vec{E} &= 0 \\
  \frac{1}{c^2} \frac{\partial^2 \vec{B}}{\partial t^2} - \nabla^2 \vec{B} &= 0
\end{split}
\end{align}
% 
Another common representation is
% 
\begin{align}
\begin{split} \label{eq::maxwell_wave_equations_box}
  \Box \ \vec{E} &= 0 \\
  \Box \ \vec{B} &= 0
\end{split}
\end{align}
% 
with the help of the d'Alembert operator $\Box \coloneqq \partial^a\partial_a = \nabla^2 - \frac{1}{c^2} \frac{\partial^2}{\partial t^2}$.
% 
This also shows $c^2  \frac{\partial B} {\partial z} = \frac{\partial E}{\partial t} \Rightarrow \vec{E} \cdot \vec{B} = 0$ that the electric and magnetic field components are perpendicular to each other.
% 
\subsection{Solving MW Equ in vacuum}
% 
\Cref{eq::maxwell_wave_equations} have the shape of a wave equation and can therefore as known be solved by
% 
% Using the Maxwell equation in vacuum \ref{eq::maxwell_vacuum}, one can find the most simple solution to the differential equation is a wave:
% 
\begin{align}
\begin{split} \label{eq::dgl_solution}
  \vec{E}( \vec{r}, t ) &= g(\phi( \vec{r}, t )) = g( \omega t - \vec{k} \cdot \vec{r} + \varphi)\\
  \vec{B}( \vec{r}, t ) &= g(\phi( \vec{r}, t )) = g( \omega t - \vec{k} \cdot \vec{r} + \varphi )
\end{split}
\end{align}
% 
where $g$ is any well behaved function and therefore also their superposition.
% 
With the help of
% 
\begin{align}
k = \mathopen| \vec{k} \mathclose| = \frac{\omega}{c} =  \frac{2 \pi}{\lambda}
\end{align}
% 
a planar wave can be descriped by
% 
\begin{align}
\begin{split} \label{eq::plane_wave}
\vec{E}(\vec{r}) &= \vec{E}_0 \cdot e^{ -i \, \vec{k} \cdot \vec{r} }\\
 \vec{B}(\vec{r}) &= \vec{B}_0 \cdot e^{ -i \, \vec{k} \cdot \vec{r} }
\end{split}
\end{align}
% 
% 
% 
\subsection{Polarization}
% 
\begin{figure}[!t]
\centering
\def\tikzwidth{\textwidth}
\inputtikz{gfx/pli/polarization_state}
\label{fig:polarization_state}
\caption{Illustration of light polarization state.}
\end{figure}
% 
This shows, that the light wave travels in direction of $\vec{k}$ (no it does not, you have to show, that k is perpendicular to E and B).
Therefore the light has a property called polarization.
Polarization is the direction of the, usually, electric field component.
% 
A superposition of x and y wave with z equal to direction of propagation yield to the general form
\begin{align}
\vec{E}(z,t) &= \begin{pmatrix} E_{0x} \cdot e^{ -i \phi_x } \\ E_{0y} \cdot e^{ -i \phi_y } \\ 0 \end{pmatrix}
e^{ -i (kz - \omega t)}
\end{align}
%
\begin{figure}[!t]
\centering
\tikzset{external/export=false}
%
% \captionsetup[sub]{textfont=normalsize}
% \captionsetup[sub]{position=top, skip=-14pt}
\tikzset{cross/.pic={
\draw[arrows={-Latex[scale=1]}] (-{sqrt(2)},0) -- ({sqrt(2)},0) node[anchor=north]{\small $E_x$};
\draw[arrows={-Latex[scale=1]}] (0,-{sqrt(2)}) -- (0,{sqrt(2)}) node[anchor=east]{\small $E_y$};
}}
%
% \subcaptionbox{}[.3\textwidth]{
\begin{tikzpicture}
\begin{scope}[shift={(0,0)}, local bounding box=A]
\pic at (0,0) {cross};
% \draw[] (0, 2) node[font=\small] () {linear, horizontal};
\draw[very thick, arrows={Latex[scale=1]-Latex[scale=1]}] (-1, 0) -- (1, 0);
\draw[] (0, -2) node () {$\begin{pmatrix} 1&0 \end{pmatrix}$};
\draw[] (0, -2.75) node () {$\begin{pmatrix} 1&1&0&0 \end{pmatrix}$};
\end{scope}
\node[anchor=south west, xshift=-1em] at (A.north west) {\small \textcolor{magenta}{\textbf{(a)}} linear, horizontal};
% }\hfill
% \subcaptionbox{}[.3\textwidth]{
\begin{scope}[shift={(4,0)}, local bounding box=B]
\pic at (0,0) {cross};
% \draw[] (0, 2) node[font=\small] () {linear, vertical};
\draw[very thick, arrows={Latex[scale=1]-Latex[scale=1]}] (0, -1) -- (0, 1);
\draw[] (0, -2) node () {$\begin{pmatrix} 0&1 \end{pmatrix}$};
\draw[] (0, -2.75) node () {$\begin{pmatrix} 1&-1&0&0 \end{pmatrix}$};
\end{scope}
\node[anchor=south west, xshift=-1em] at (B.north west) {\small \textcolor{magenta}{\textbf{(b)}} linear, vertical};
% }\hfill
% \subcaptionbox{}[.3\textwidth]{
\begin{scope}[shift={(8,0)}, local bounding box=C]
\pic at (0,0) {cross};
% \draw[] (0, 2) node[font=\small] () {linear, $\pi/4$};
\draw[very thick, arrows={Latex[scale=1]-Latex[scale=1]}] (-1, -1) -- (1, 1);
\draw[] (0, -2) node () {$\begin{pmatrix} 1&1 \end{pmatrix}$};
\draw[] (0, -2.75) node () {$\begin{pmatrix} 1&0&1&0 \end{pmatrix}$};
\begin{scope}[overlay]
\draw[] (2, -2) node () {Jones};
\draw[] (2, -2.75) node () {Stokes};
\end{scope}
\end{scope}
\node[anchor=south west, xshift=-1em] at (C.north west) {\small \textcolor{magenta}{\textbf{(c)}} linear, $\pi/4$};
% }
% \\[2em]
% \subcaptionbox{}[.32\textwidth]{
\begin{scope}[shift={(0,-5.75)}, local bounding box=D]
\pic at (0,0) {cross};
% \draw[] (0, 2) node[font=\small] () {left circular};
\draw[very thick] plot[domain=0:360,samples=90,smooth] ({cos(\x)},{sin(\x)});
\draw[very thick, arrows={-Latex[scale=1]}] plot[domain=44:45,samples=1] ({cos(\x)},{sin(\x)});
\draw[very thick, arrows={-Latex[scale=1]}] plot[domain=224:225,samples=1] ({cos(\x)},{sin(\x)});
\draw[] (0, -2) node () {$\begin{pmatrix} 1&i \end{pmatrix}$};
\draw[] (0, -2.75) node () {$\begin{pmatrix} 1&0&0&-1 \end{pmatrix}$};
\end{scope}
\node[anchor=south west, xshift=-1em] at (D.north west) {\small \textcolor{magenta}{\textbf{(d)}} left circular};
% }\hfill
% \subcaptionbox{}[.32\textwidth]{
\begin{scope}[shift={(4,-5.75)}, local bounding box=E]
\pic at (0,0) {cross};
% \draw[] (0, 2) node[font=\small] () {right circular};
\draw[very thick] plot[domain=0:360,samples=90,smooth] ({cos(\x)},{sin(\x)});
\draw[very thick, arrows={-Latex[scale=1]}] plot[domain=46:45,samples=1] ({cos(\x)},{sin(\x)});
\draw[very thick, arrows={-Latex[scale=1]}] plot[domain=226:225,samples=1] ({cos(\x)},{sin(\x)});
\draw[] (0, -2) node () {$\begin{pmatrix} 1&-i \end{pmatrix}$};
\draw[] (0, -2.75) node () {$\begin{pmatrix} 1&0&0&1 \end{pmatrix}$};
\end{scope}
\node[anchor=south west, xshift=-1em] at (E.north west) {\small \textcolor{magenta}{\textbf{(e)}} right circular};
% }\hfill
% \subcaptionbox{}[.32\textwidth]{
\begin{scope}[shift={(8,-5.75)}, local bounding box=F]
\pic at (0,0) {cross};
% \draw[] (0, 2) node[font=\small] () {unpolarized};
\draw[] (0, -2.75) node () {$\begin{pmatrix} 1&0&0&0 \end{pmatrix}$};
\begin{scope}[overlay]
\draw[] (2, -2) node () {Jones};
\draw[] (2, -2.75) node () {Stokes};
\end{scope}
\end{scope}
\node[anchor=south west, xshift=-1em] at (F.north west) {\small \textcolor{magenta}{\textbf{(f)}} unpolarized};
%
\path[] ($(A.west)!-0.075!(C.east)$) -- ($(A.west)!1.075!(C.east)$);
\end{tikzpicture}
% }
%
\caption{polarization states, check vector length,\itodo{test speed}} 
\label{fig:polarization_state_vectors}
\end{figure}
%
\Cref{fig:polarization_state_vectors} shows another representation of the polarization sate of a light wave.
It show the component perpendicular to the travelling direction.
Therefore the time variation can be shown on the xy-plane.
Additionally the states can be describe by the Jones or Stokes calculus, later described in \cref{sec:jones,,sec:mueller_stokes}.
% 
% 
% 
\subsection{Light in medium}
% 
The maxwell equations in ... are described above. They can be solved analog to ... . This yields to some special behavier, \eg{} the magnetic and electic field component get out of phase.
However here only the two properties of absorptiona and difraction are described.
They derivation can be found \eg{} \cite{demtroeder2, Fliebach2012}.
% 
\subsection{Absorption}
% 
\begin{align}
    I = I_0 \exp(-\mu x)
\end{align}
% 
Beersche law of absorption with $\mu = \frac{4\pi \kappa}{\lambda_0}$ where $\kappa$ is the imaginary part of the complex refractive index.
% 
\subsection{Refraction}
% 
\begin{figure}[!t]
\centering
\def\tikzwidth{\textwidth}
\inputtikz{gfx/pli/refraction}
\label{fig:optic_refraction}
\caption{Illustration of refraction.}
\end{figure}
% 
Refraction is the property of light to change direction as it passes from one medium to another. This can be shown by using the full Maxwell equations  for non-conductive material, \ie{} $\vec{j} = 0, \rho = 0$, that the dgl consit out of a primary wave with from atomic medim induced secundary waves, which leads to a reduction of the velocity of the resulting wave. Mathematically this can be described by a complex number $n = c' / c$.
Using this relationship at a boundary surface between two media, one can show that the incident light beam splits into a reflecting and transmittang light wave. The reflecting light wave has the same angle as the incident light beam relativ the the surface normal. The transmitting light beam however due to the reduction of the velocity, changes its direction described by the snellius ...
\begin{align}
    n_0 \sin \alpha = n_1 \sin \beta
\end{align}

% 
\begin{align}
\underline{n} = n + i\kappa
\end{align}
% 
\begin{align}
\begin{split}
\vec{E}(z, t) &= \operatorname{Re}\! \left[\vec{E}_0 \cdot e^{i\, (kz - \omega t)}\right] \\
&= \operatorname{Re}\! \left[\vec{E}_0 \cdot e^{i\, (2\pi(n + i\kappa)z/\lambda_0 - \omega t)}\right] \\
&= e^{-2\pi \kappa z/\lambda_0} \operatorname{Re}\! \left[\vec{E}_0 \cdot e^{i\, (kz - \omega t)}\right]
\end{split}
\end{align}
% 
% 
% 
\subsection{Birefingence}
%
\begin{figure}[!t]
\centering
\def\tikzwidth{\textwidth}
\inputtikz{gfx/pli/retardation}
\caption{Illustration of retardation.}
\label{fig:optic_retardation}
\end{figure}
% 
\begin{figure}[!t]
\centering
\subcaptionbox{}[.49\textwidth]{
\def\tikzwidth{0.49*\textwidth}
\inputtikz{gfx/pli/ellipsoid_a}}
\subcaptionbox{}[.49\textwidth]{
\def\tikzwidth{0.49*\textwidth}
\inputtikz{gfx/pli/ellipsoid_b}}
\caption{birefringence elipsoid}
\label{fig:index_elipsoid}
\end{figure}
% 
Material can have a different refractive index depending on the relative orientation and polarization of the light beam.
This property is known as birefringence.
The refractive index can be described as the refractive index $n_o$ and extraordinary  $n_e$.
Since these two are perpendicular to each other, one can split the light beam into the same perpendicular parts and describe each by its own.
These two light beams, except for the trivial case of a light beam is perpendicular to the surfave, have a different direction.
However for small relative length (hängt von n ab, bzw der phasenänderung) the two light beams leave the material at the same point and recombined at the end.
The change of phase is called birefringence and is described by the difference between the .. and ...
% 
\begin{align}
    \Delta n = n_e - n_o \> .
\end{align}
% 
This .. can be visualized by the index ellipsoid \cref{fig:index_elipsoid}.
The change of the amplitude and phase can be described by the following techniques.
% 
% 
% 
\subsection{Jones}
\label{sec:jones}
% 
The Jones calculus, introduced by Robert Clark Jones in 1941, describes the polarization state of a light beam by a complex vector $J$:
% 
\begin{align}
    \vec{J} = \begin{pmatrix} E_x \exp(i \phi_x) \\ E_y \exp(i \phi_y) \end{pmatrix}
\end{align}
% 
The amplitude of the perpendicular components are $E_x$ and $E_y$ with their phase $\phi_x$ and $\phi_y$.
Optical elements, which changes the polarization state, such as a polarization filter and retarder, can be desribed by a matrix:
% 
\begin{align}
\begin{split}
\mat{P}_x = 
\begin{pmatrix}
1 & 0 \\ 0 & 0
\end{pmatrix}
, \,
\mat{P}_y = 
\begin{pmatrix}
0 & 0 \\ 0 & 1
\end{pmatrix}
, \,
\mat{M} =
\begin{pmatrix}
e^{i \varphi_x} & 0 \\ 0 & e^{i \varphi_y}
\end{pmatrix}
, \,
\Lambda_{1/4}=
e^{\frac{i \pi}{4}}
\begin{pmatrix}
1 & 0 \\ 0 & -i
\end{pmatrix}
\end{split}
\end{align}
% 
A rotation of the axis can be achieved by a 2d rotation matrix $\mat{R}$:
% 
\begin{align}
\begin{split}
\mat{M}(\theta )=\mat{R}(\theta )\cdot\mat{M}\cdot\mat{R}(-\theta )
, \>
\mat{R}(\theta ) = 
\begin{pmatrix}
\cos \theta & -\sin \theta \\
\sin \theta & \cos \theta
\end{pmatrix}
\end{split}
\end{align}
% 
% 
% 
\subsection{M\"uller-Stokes}\label{sec:Mueller-Stokes}
\label{sec:mueller_stokes}
% 
Analog to \cref{sec:jones} the M\"uller-Stokes formalism, described by George Gabriel Stokes in 1852, also describes the polarication state of a light beam.
However it does not use the absolute .. but the relative polarication between both components:
% 
\paragraph{Stokes vector}
\begin{align}
\begin{split}
S_0 &= I \\
S_1 &= I p \cos 2\psi \cos 2\chi \\
S_2 &= I p \sin 2\psi \cos 2\chi \\
S_3 &= I p \sin 2\chi
\end{split} \hspace{-6em}
\begin{split}
& \equiv \langle E_x^{2} \rangle + \langle E_y^{2} \rangle \\
%  & = \langle E_a^{2} \rangle + \langle E_b^{2} \rangle \\
%  & =  \langle E_l^{2} \rangle + \langle E_r^{2} \rangle \\
& \equiv \langle E_x^{2} \rangle - \langle E_y^{2} \rangle \\
& \equiv \langle E_a^{2} \rangle - \langle E_b^{2} \rangle \\
& \equiv  \langle E_l^{2} \rangle - \langle E_r^{2} \rangle
\end{split}
, \,
\vec{S} =
\begin{pmatrix} S_0 \\ S_1 \\ S_2 \\ S_3\end{pmatrix}
\end{align}
% 
% \begin{align}
% \begin{split}
% S_0 & \equiv \langle E_x^{2} \rangle + \langle E_y^{2} \rangle \\
%  & = \langle E_a^{2} \rangle + \langle E_b^{2} \rangle \\
%  & =  \langle E_l^{2} \rangle + \langle E_r^{2} \rangle \\
% S_1 & \equiv \langle E_x^{2} \rangle - \langle E_y^{2} \rangle \\
% S_2 & \equiv \langle E_a^{2} \rangle - \langle E_b^{2} \rangle \\
% S_3 & \equiv  \langle E_l^{2} \rangle - \langle E_r^{2} \rangle
% \end{split}
% \end{align}
% 
Therefore the phase can not be described anymore.
However the relative phase information is stored and can be used to describe also polarization states of polarization filters, which can not be described by Jones.
Aanlog to Jones one can formulate the matrices for the optical components:
% 
\paragraph{Linear Polarizer}
\begin{align}
\mat{P}_x=\frac{1}{2}
\begin{pmatrix}
    1 & 1 & 0 & 0 \\
    1 & 1 & 0 & 0 \\
    0 & 0 & 0 & 0 \\
    0 & 0 & 0 & 0
  \end{pmatrix}
, \,
\mat{P}_y=\frac{1}{2}
\begin{pmatrix}
     1 & -1 & 0 & 0 \\
    -1 &  1 & 0 & 0 \\
     0 &  0 & 0 & 0 \\
     0 &  0 & 0 & 0
\end{pmatrix}
\end{align}
% 
\todo{allgemeiner retarder}
% 
\paragraph{Quarter-wave plate (fast-axis vertical)}
\begin{align}
\Lambda_{1/4}=\
\begin{pmatrix}
    1 & 0 & 0 &  0 \\
    0 & 1 & 0 &  0 \\
    0 & 0 & 0 & -1 \\
    0 & 0 & 1 &  0
\end{pmatrix}
\end{align}
% 
\paragraph{Rotation Matrix}
\begin{align}
\mat{R}(\theta)=
\begin{split}
\begin{pmatrix}
    1 &                0 &               0 & 0 \\
    0 & \cos{(2\theta)} & -\sin{(2\theta)} & 0 \\
    0 & \sin{(2\theta)} & \cos{(2\theta)} & 0 \\
    0 &                0 &               0 & 1
  \end{pmatrix}
\end{split}
\end{align}
% 
% \paragraph{Rotation Matrix}
% \begin{align}
% \frac{1}{2}
% \begin{pmatrix}
%     1 &                0 &               0 & 0 \\
%     0 &  \cos{(2\theta)} & \sin{(2\theta)} & 0 \\
%     0 & -\sin{(2\theta)} & \cos{(2\theta)} & 0 \\
%     0 &                0 &               0 & 1
% \end{pmatrix}
% \end{align}
\todo{matrix multi}
% 
% \section{Tissue Discretization}
% % 
% By deviding the volume into small diskreticed subvolumes, one can multiply the .. all together and ... (analog Riemann sum)
% \begin{align}
%     \int F \, dt \approx \sum_n F \, \Delta t
% \end{align}
% see simulation?
% 
%  see simulation
% \paragraph{Micro vs Macro:}
% % 
% % \begin{align}
% %   \int_{y_\textit{min}}^{y_\textit{max}} \vec{v}(y) \,dy\\
% %   x = const
% %   y = y(\alpha,x) = tan(\alpha) \cdot x\\
% %   \vec{v}_r = \begin{pmatrix} \cos(\alpha)\\ \sin(\alpha)\\0\end{pmatrix}, \, \vec{v}_p = \begin{pmatrix} 0\\ 0\\1\end{pmatrix}\\
% %   \vec{v}_r = \begin{pmatrix} \cos(\arctan(y/x))\\ \sin(\arctan(y/x))\\0\end{pmatrix}\\
% %   \int_{y_\textit{min}}^{y_\textit{max}} \vec{v}_p(y) \,dy = (y_\textit{max} - y_\textit{min}) \cdot e_z\\
% %   \int_{y_\textit{min}}^{y_\textit{max}} \vec{v}_r(y) \,dy = \int_{y_\textit{min}}^{y_\textit{max}} \cos(\arctan(y/x)) dy \cdot e_x \\
% %   = x\left(\sinh^{-1}(y_\textit{max}/x) - \sinh^{-1}(y_\textit{min}/x)\right) \\
% %   = 2x \left(\sinh^{-1}\left(\frac{2\sqrt{R-x^2}}{x}\right)\right)
% % \end{align}
% % 
% % \begin{align}
% %     % \frac{\oint \vec{v}_z \mathrm{d}A}{\oint \vec{v}_x \mathrm{d}A} = 
% %     % \frac{\int_{-1}^{1}\abs{\vec{v}} \mathrm{d}x}{\int_{-\frac{\pi}{2}}^{\frac{\pi}{2}} \abs{\vec{v}} \cos(\varphi) \mathrm{d}\varphi} =
% %     % \frac{2}{1}
% %     \frac{A_{\Box}}{A_{\circ}} = \frac{4}{\pi}
% % \end{align}
% % 
% \todo{why 2*dn?}
% 
\section{Experimental Setup}\label{sec:expSetup}
%
\begin{figure}[!t]
    \captionsetup[sub]{position=top}
    \def\tikzwidth{\textwidth}
	\centering
	\subcaptionbox{\label{setup-lap} \ac{LAP}}[\textwidth]{
% 		\resizebox{0.95\textwidth}{!}{
		\inputtikz{gfx/pli/pli_setup}
% 	}
	}\\
	\subcaptionbox{\label{setup-pm} \ac{PM}}[\textwidth]{
% 		\resizebox{0.95\textwidth}{!}{
		\inputtikz{gfx/pli/pli_setup_pm}
% 	}
	}
	\caption{Illustration of PLI setup.}
	\label{fig:pli_setup}
\end{figure}
%
\begin{figure}[!t]
    % \captionsetup[sub]{position=top}
    \def\tikzwidth{0.42*\textwidth}
	\centering
	\subcaptionbox{\ac{LAP}}[.49\textwidth]{
			\inputtikz{gfx/pli/pli_detail}}\hfill
	\subcaptionbox{\ac{PM}}[.49\textwidth]{
			\inputtikz{gfx/pli/pli_detail_pm}}
	\caption{Illustration of detail PLI setup.}
	\label{fig:pli_detail}
\end{figure}
%
The \ac{3D-PLI} setup is described in \cite{Axer2011} with the tiltable light beam microscope (LMP) in \cite{Wiese:887678}.
For the routine measurements two (three) microscope systems are currently in use.
The first use of a high throughput microscope is the \ac{LAP} with a pixel size of about \SI{60}{\micro\meter}. \footnote{can be change by the optic also to $\SI{40}{\micro\meter}$ and $\SI{20}{\micro\meter}$, but remains for the routine measurements with $\SI{60}{\micro\meter}$.}
It consist out of a led light source (\SI{525}{\nano\meter}), a rotatable polarization filter, a rotatable quarter wave retarder, the specimen stage, which can be tilted for oblique views, a rotatable polarizer.
Both polarizer and quarter wave retarder are rotated at the same time.
The both polarizers have a phase offset of $\SI{90}{\degree}$, the first polarizer and the quarter wave retarder of $\SI{45}{\degree}$ (see (see \cref{setup-lap}).
The second system \ac{LMP3D} microscope fullfills the same measurments, however the retarder is placed after the tissue specimen and fixed with the polarizer without any rotations (see \cref{setup-pm}).
Mathematically it measures the same signal.
\footnote{The rotatable filters of the \ac{LAP} has the advantage, that noise like dust particles can be easy identified and removed. Since the microscop is in a more confined "box" this is not necesarry.}
A difference is that the optical path is in reverse to the \ac{LAP} system.
Since there is a mirror in the path, the image is flipped again so that the coordinate systems stays the same (see \cref{fig:pli_detail}).
% 
\subsection{Signal}
% 
From the M\"{u}ller Matrices (\cref{sec:Mueller-Stokes}) as shown in \cite{MenzelMaster,MenzelDissertation}, that the intensity signal, measured by the \ac{CCD}-sensor, follows a sinusiodal curve:
% 
% \begin{align}
% I = \underbrace{\frac{I_0}{2}}_{\mathclap{\mathit{transmittance}}}
%     \bigl[ \sin\bigl(2\rho - 2{\underbrace{\vphantom{\frac{I_0}{2}} \varphi}_{\mathclap{\mathit{direction}}}} \bigr)
%     \cdot \sin \bigl( {\underbrace{\vphantom{\frac{I_0}{2}} 2\pi\frac{d \dn}{\lambda} \cos^2\left( \alpha \right)}_{\mathclap{\delta \, \coloneqq \, \mathit{retardation}}}} \bigr) \bigr]
% \end{align}
\begin{align}
\label{eq:pli_signal}
\begin{split}
I =\frac{I_0}{2}\bigl[ \sin\bigl(2\rho - 2\varphi \bigr)\cdot \sin \bigl( 2\pi\frac{d \dn}{\lambda} \cos^2\left( \alpha \right) \bigr) \bigr] \\
\text{with} \enspace \delta \coloneqq 2\pi\frac{d \dn}{\lambda} \cos^2\left( \alpha \right) \enspace 
\text{and} \enspace \trel \coloneqq \frac{d \dn}{\lambda}
\end{split}
\end{align}
% 
Since \cref{eq:pli_signal} describes a sinosoidal signal, the fourier analysis is a obvious choise.
For a discrete, aquidistant measurement of the rotation angles $\rho$, one can use the fourier series with the three coefficients to describe the signal:
% 
% \begin{align}
% \begin{split}
% A \sin(\omega t + \alpha) + B \sin(\omega t + \beta) &= \sqrt{C^2 + D^2} \cdot \sin \, \left( \omega t + \tan^{-1} \left( \frac{D}{C} \right) \right)
% \end{split}
% \\
% \begin{split}
% C &= A \cos(\alpha)+ B \cos(\beta)\\
% D &= A \sin(\alpha)+ B \sin(\beta)
% \end{split} \nonumber 
% \end{align}

\begin{align}
\begin{split}
\rho &= [\SI{0}{\degree}, \frac{1\cdot180}{N+1}\si{\degree}, \frac{2\cdot180}{N+1}\si{\degree}, ..., \frac{N\cdot180}{N+1}\si{\degree}]\\
a_0 &= \frac{1}{N} \sum(\mathit{data})\\
a_1 &= \frac{2}{N} \sum(\mathit{data} \cdot \sin(2 \cdot \rho)\\
b_1 &= \frac{2}{N} \sum(\mathit{data} \cdot \cos(2 \cdot \rho)
\end{split}
\end{align}

With these the final \ac{3D-PLI} modalities are calculated:

\begin{align}
\begin{split}
\mathit{transmittance} &= 2 \cdot a_0\\
\mathit{direction} &= 0.5 \cdot \atantwo(-b_1 / a_1)\\
\mathit{retardation} &= \frac{\sqrt{a_1^2 + b_1^2}}{a_0}
\end{split}
\end{align}



% 
\subsection{Tilted signal}
% 
To be able to analyse the inclination $\alpha$ one has to distiguish the relative strength of the birefringence from the $\cos^2(\alpha)$.
For this purpose a tiltable specimen was developed, which allows measuring the light signal through the tissue with a different angle of incidenc.
This mean, that tissue and therefore the nerve fibers also change their orientation by the tilting angles $\theta, \varphi$.
Additionally the distance the light travels through the tissue, elongigates by $1/\cos(\theta)$ \cref{fig:tilted_side_view}.
% 
Depending on the \pixelsize{}, in the case of the \ac{LAP} system, this light travels though the same volume, but with another orientation.
Therefore a measurement of multiple light paths can be ... and the resulting signals can be used to analyse the inclination and relative birefrigente tissue thickness \trel{}.
In the case of a tiltible specimen, the angle of incidence on the glass ... and the tissue also changes the angle of the light by snellius law.
All angles mentiont here are if not specified always the change of angle of the light inside the tissue (see \cref{fig:tilting_camera_view}). 
This also leads to a perspective shift, which has to be corrected by a registration process.
In this case an affine transformation.
% 
In \cite{Schmitz2018} this analys published under the name of \ac{ROFL}, which builds up on the work of \cite{Wiese:887678}.
The idea is to fit the measured signals of all light paths simultaniously.
Because the change of signal is proportional to the $\cos(\alpha)$ this however means, that for steep fibers, the changes are not only small, but also the amplitude of the signal is very small.
This leads to the problem of increasing uncertanty with increasing inclination angle.
\\
A further problem is that for a smaller \pixelsize{} the light path cannot be neglected anymore.
For the \ac{LMP3D} system this means, that with a maximal tilting angle of about $\SI{3}{\degree}$ and a usually tissue thickness of $\SI{60}{\micro\meter}$ the light path is measured in n pixels away from the non tilting case, if the rotation point is in the center of the tissue.
\\
However for homoginoius tissue, like the \ac{WM} may be some regions, the analysis technique can still be applied.
A change in density, orientation or dispersion however will lead to a false result.
% 
% 
% 
\section{Orientation}
% 
\begin{figure}[!t]
\centering
\setlength{\tikzwidth}{0.4\textwidth}
\begin{center}
\begin{tabular}{m{6cm}m{6cm}}
\includegraphics[width=\tikzwidth]{gfx/pli/color_sphere.png}
&
\inputtikz{gfx/pli/hsv_sphere}
\\
\begin{minipage}[t]{0.42\textwidth}
\leavevmode\subcaption{2d hsv sphere}
\end{minipage}
&
\begin{minipage}[t]{0.42\textwidth}
\leavevmode\subcaption{\label{fig:sphere}3d hsv sphere}
\end{minipage}
\end{tabular}
\end{center}
% 
\vspace{-1em} % because of tabular?
\caption{collor spheres}
\label{fig:spheres}
\end{figure}
% 
% 
The orientation of the birefringence axis is descibed by the direction angle $\varphi$ and inclination angle $\alpha$ (see \cref{fig:sphere}).
To be able to show a 3d information, the \textit{hsv-black} is usually shown in \ac{3D-PLI}.
It color code the orientation by:
\begin{align}
\begin{split}
    H &= \varphi/\pi\\
    S &= 1\\
    V &= 1-\alpha / \pi/2
\end{split}
\end{align}
This way the color corresponds to the direction, and the color value to the inclination.
Since the orientation, unlike a vector, covers only a half sphere, the colors are point symmetric.
% 
% fig:spheres
% 
% 
\section{optical resolution}
\label{sec:opticalResolution}
% 
The optical resolution of an imaging system describes the minimal size of on object which can still be resolved.
This property is limited by aberration and diffraction which both cause a blurring of the resulting image.
Ernst Abbe described as on of the first that the resulition corrlates with the lightwave $\lambda$: 
\begin{align}
d=\frac{ \lambda}{2 n \sin \theta} = \frac{\lambda}{2\mathrm{NA}} \> .
\end{align}
$d$ is the minimal resolvable distance, $n$ the refractive index, $\theta$ the angle of a spot light, which can be combined to the better known numerical aperture $\mathrm{NA}$.
This results in an absolute threshold for a light microscope above which the resolution can no longer be improved.
For the wavelength used in \ac{3D-PLI} with $\lambda = \SI{525}{\nano\meter}$ and a numerical apperatur of $\mathrm{NA} = \SI{}{}$ this yields to a limit of \dummy{}.
% 
\begin{figure}[!t]
\setlength{\tikzwidth}{0.45\textwidth}
\centering
% \subcaptionbox{}{
\inputtikz{gfx/pli/rayleigh}
\caption[Raylay criterium]{rayleigh criteria. The minima of the one function is in the maxima of the other.}
\label{fig:rayleigh}
\end{figure}
% 