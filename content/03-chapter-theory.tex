\setcounter{chapter}{2}
\chapter{3D-PLI}
\label{sec:theory}
%
\comment{\paragraph{Ziele:} 
\begin{itemize}
    \item physik
    \item analysen
\end{itemize}}
%
\section{Electromagnetic Wave}
% 
\begin{align} 
\begin{split} \label{eq::maxwell_macro}
    \nabla \cdot \mathbf{D} &= 4\pi\rho_\text{f}\\
    \nabla \cdot \mathbf{B} &= 0\\
    \nabla \times \mathbf{E} &= -\frac{1}{c} \frac{\partial \mathbf{B}} {\partial t}\\
    \nabla \times \mathbf{H} &= \frac{1}{c} \left(4\pi\mathbf{J}_\text{f} + \frac{\partial \mathbf{D}} {\partial t} \right)
\end{split}
\end{align}
% 
\begin{align} 
\begin{split} \label{eq::maxwell_general}
    \nabla \cdot \mathbf{E} &= \frac {\rho} {\varepsilon_0}\\
    \nabla \cdot \mathbf{B} &= 0\\
    \nabla \times \mathbf{E} &= -\frac{\partial \mathbf{B}} {\partial t}\\
    \nabla \times \mathbf{B} &= \mu_0\left(\mathbf{J} + \varepsilon_0 \frac{\partial \mathbf{E}} {\partial t} \right)
\end{split}
\end{align}
% 
In a vacuum $\rho = 0$ and $\mathbf{J} = \mathbf{0}$:
% 
\begin{align}
\begin{split} \label{eq::maxwell_vacuum}
  \nabla \cdot \mathbf{E} &= 0 \quad\\
  \nabla \cdot \mathbf{B} &= 0 \quad\\
  \nabla \times \mathbf{E} &= -\frac{\partial\mathbf B}{\partial t}\\
  \nabla \times \mathbf{B} &= \mu_0\varepsilon_0 \frac{\partial\mathbf E}{\partial t}
  \end{split}
\end{align}
% 
with the identity $\nabla \times \left( \nabla \times \mathbf{A} \right) = \nabla(\nabla \cdot \mathbf{A}) - \nabla^{2}\mathbf{A}$ and $\mu_0\varepsilon_0 = \frac{1}{c^2}$
% 
\begin{align}
\begin{split} \label{eq::maxwell_wave_equations}
  \frac{1}{c^2} \frac{\partial^2 \mathbf{E}}{\partial t^2} - \nabla^2 \mathbf{E} =& 0 \\
  \frac{1}{c^2} \frac{\partial^2 \mathbf{B}}{\partial t^2} - \nabla^2 \mathbf{B} =& 0
\end{split}
\end{align}
% 
\begin{align}
\begin{split} \label{eq::maxwell_wave_equations_box}
  \Box \ \mathbf{E} &= 0 \\
  \Box \ \mathbf{B} &= 0
\end{split}
\end{align}
% 
with $\Box \coloneqq \partial^a\partial_a = \nabla^2 - \frac{1}{c^2} \frac{\partial^2}{\partial t^2}$
%
\begin{figure}[!t]
\centering
\def\tikzwidth{\textwidth}
% \resizebox{0.95\textwidth}{!}{
\inputtikz{gfx/pli/polarization_state.tikz}
% }
\label{fig:polarization_state}
\caption{Illustration of light polarization state.}
\end{figure}
%
\begin{figure}[!t]
\centering
\def\tikzwidth{\textwidth}
% \resizebox{0.95\textwidth}{!}{
\inputtikz{gfx/pli/retardation.tikz}
% }
\label{fig:optic_retardation}
\caption{Illustration of retardation.}
\end{figure}
% 
\subsection{Polarization}
% 
solve DGL \ref{eq::maxwell_vacuum}
% 
\begin{align}
\begin{split} \label{eq::dgl_solution}
   \mathbf{E}( \mathbf{r}, t ) &= g(\phi( \mathbf{r}, t )) = g( \omega t - \mathbf{k} \cdot \mathbf{r} + \varphi)\\
   \mathbf{B}( \mathbf{r}, t ) &= g(\phi( \mathbf{r}, t )) = g( \omega t - \mathbf{k} \cdot \mathbf{r} + \varphi )
\end{split}
\end{align}
% 
\begin{align}
k = \mathopen| \mathbf{k} \mathclose| = \frac{\omega}{c} =  \frac{2 \pi}{\lambda}
\end{align}
% 
Plane wave:
\begin{align}
\begin{split} \label{eq::plane_wave}
\mathbf{E}(\mathbf{r}) &= \mathbf{E}_0 \cdot e^{ -i \, \mathbf{k} \cdot \mathbf{r} }\\
 \mathbf{B}(\mathbf{r}) &= \mathbf{B}_0 \cdot e^{ -i \, \mathbf{k} \cdot \mathbf{r} }
\end{split}
\end{align}
% 
With $c^2  \frac{\partial B} {\partial z} = \frac{\partial E}{\partial t} \Rightarrow \mathbf{E} \cdot \mathbf{B} = 0$.
% \begin{align}
% \begin{split} \label{eq::sin_wave}
% \mathbf{E} ( \mathbf{r}, t ) &= \mathbf{E}_0 \cos( \omega t - \mathbf{k} \cdot \mathbf{r} + \phi_0 )\\
% \mathbf{B} ( \mathbf{r}, t ) &= \mathbf{B}_0 \cos( \omega t - \mathbf{k} \cdot \mathbf{r} + \phi_0 ) 
% \end{split}
% \end{align}
% 
\subsubsection{Polarization}
% 
\begin{align}
\mathbf{E}(\mathbf{r}) &= \mathbf{E}_0 \cdot e^{ -i \, \mathbf{k} \cdot \mathbf{r}}
\end{align}
% 
combination (superposition) of x and y wave with z equal to direction of propagation.
\begin{align}
\mathbf{E}(z,t) &= \begin{pmatrix} E_{0x} \cdot e^{ -i \phi_x } \\ E_{0y} \cdot e^{ -i \phi_y } \\ 0 \end{pmatrix}
e^{ -i (kz - \omega t)}
\end{align}
%
\begin{figure}[!t]
\centering
%
% \captionsetup[sub]{textfont=normalsize}
% \captionsetup[sub]{position=top, skip=-14pt}
\tikzset{cross/.pic={
\draw[arrows={-Latex[scale=1]}] (-{sqrt(2)},0) -- ({sqrt(2)},0) node[anchor=north]{\small $E_x$};
\draw[arrows={-Latex[scale=1]}] (0,-{sqrt(2)}) -- (0,{sqrt(2)}) node[anchor=east]{\small $E_y$};
}}
%
% \subcaptionbox{}[.3\textwidth]{
\begin{tikzpicture}
\begin{scope}[shift={(0,0)}, local bounding box=A]
\pic at (0,0) {cross};
% \draw[] (0, 2) node[font=\small] () {linear, horizontal};
\draw[very thick, arrows={Latex[scale=1]-Latex[scale=1]}] (-1, 0) -- (1, 0);
\draw[] (0, -2) node () {$\begin{pmatrix} 1&0 \end{pmatrix}$};
\draw[] (0, -2.75) node () {$\begin{pmatrix} 1&1&0&0 \end{pmatrix}$};
\end{scope}
\node[anchor=south west, xshift=-1em] at (A.north west) {\small \textcolor{magenta}{\textbf{(a)}} linear, horizontal};
% }\hfill
% \subcaptionbox{}[.3\textwidth]{
\begin{scope}[shift={(4,0)}, local bounding box=B]
\pic at (0,0) {cross};
% \draw[] (0, 2) node[font=\small] () {linear, vertical};
\draw[very thick, arrows={Latex[scale=1]-Latex[scale=1]}] (0, -1) -- (0, 1);
\draw[] (0, -2) node () {$\begin{pmatrix} 0&1 \end{pmatrix}$};
\draw[] (0, -2.75) node () {$\begin{pmatrix} 1&-1&0&0 \end{pmatrix}$};
\end{scope}
\node[anchor=south west, xshift=-1em] at (B.north west) {\small \textcolor{magenta}{\textbf{(b)}} linear, vertical};
% }\hfill
% \subcaptionbox{}[.3\textwidth]{
\begin{scope}[shift={(8,0)}, local bounding box=C]
\pic at (0,0) {cross};
% \draw[] (0, 2) node[font=\small] () {linear, $\pi/4$};
\draw[very thick, arrows={Latex[scale=1]-Latex[scale=1]}] (-1, -1) -- (1, 1);
\draw[] (0, -2) node () {$\begin{pmatrix} 1&1 \end{pmatrix}$};
\draw[] (0, -2.75) node () {$\begin{pmatrix} 1&0&1&0 \end{pmatrix}$};
\begin{scope}[overlay]
\draw[] (2, -2) node () {Jones};
\draw[] (2, -2.75) node () {Stokes};
\end{scope}
\end{scope}
\node[anchor=south west, xshift=-1em] at (C.north west) {\small \textcolor{magenta}{\textbf{(c)}} linear, $\pi/4$};
% }
% \\[2em]
% \subcaptionbox{}[.32\textwidth]{
\begin{scope}[shift={(0,-5.75)}, local bounding box=D]
\pic at (0,0) {cross};
% \draw[] (0, 2) node[font=\small] () {left circular};
\draw[very thick] plot[domain=0:360,samples=90,smooth] ({cos(\x)},{sin(\x)});
\draw[very thick, arrows={-Latex[scale=1]}] plot[domain=44:45,samples=1] ({cos(\x)},{sin(\x)});
\draw[very thick, arrows={-Latex[scale=1]}] plot[domain=224:225,samples=1] ({cos(\x)},{sin(\x)});
\draw[] (0, -2) node () {$\begin{pmatrix} 1&i \end{pmatrix}$};
\draw[] (0, -2.75) node () {$\begin{pmatrix} 1&0&0&-1 \end{pmatrix}$};
\end{scope}
\node[anchor=south west, xshift=-1em] at (D.north west) {\small \textcolor{magenta}{\textbf{(d)}} left circular};
% }\hfill
% \subcaptionbox{}[.32\textwidth]{
\begin{scope}[shift={(4,-5.75)}, local bounding box=E]
\pic at (0,0) {cross};
% \draw[] (0, 2) node[font=\small] () {right circular};
\draw[very thick] plot[domain=0:360,samples=90,smooth] ({cos(\x)},{sin(\x)});
\draw[very thick, arrows={-Latex[scale=1]}] plot[domain=46:45,samples=1] ({cos(\x)},{sin(\x)});
\draw[very thick, arrows={-Latex[scale=1]}] plot[domain=226:225,samples=1] ({cos(\x)},{sin(\x)});
\draw[] (0, -2) node () {$\begin{pmatrix} 1&-i \end{pmatrix}$};
\draw[] (0, -2.75) node () {$\begin{pmatrix} 1&0&0&1 \end{pmatrix}$};
\end{scope}
\node[anchor=south west, xshift=-1em] at (E.north west) {\small \textcolor{magenta}{\textbf{(e)}} right circular};
% }\hfill
% \subcaptionbox{}[.32\textwidth]{
\begin{scope}[shift={(8,-5.75)}, local bounding box=F]
\pic at (0,0) {cross};
% \draw[] (0, 2) node[font=\small] () {unpolarized};
\draw[] (0, -2.75) node () {$\begin{pmatrix} 1&0&0&0 \end{pmatrix}$};
\begin{scope}[overlay]
\draw[] (2, -2) node () {Jones};
\draw[] (2, -2.75) node () {Stokes};
\end{scope}
\end{scope}
\node[anchor=south west, xshift=-1em] at (F.north west) {\small \textcolor{magenta}{\textbf{(f)}} unpolarized};
%
\path[] ($(A.west)!-0.075!(C.east)$) -- ($(A.west)!1.075!(C.east)$);
\end{tikzpicture}
% }
%
\caption{polarization states, check vector length} 
\end{figure}
%
\subsection{Birefingence}
\subsubsection{Refraction}
\begin{align}
\underline{n} = n + i\kappa
\end{align}
% 
\begin{align}
\begin{split}
\mathbf{E}(z, t) &= \operatorname{Re}\! \left[\mathbf{E}_0 \cdot e^{i\, (kz - \omega t)}\right] \\
&= \operatorname{Re}\! \left[\mathbf{E}_0 \cdot e^{i\, (2\pi(n + i\kappa)z/\lambda_0 - \omega t)}\right] \\
&= e^{-2\pi \kappa z/\lambda_0} \operatorname{Re}\! \left[\mathbf{E}_0 \cdot e^{i\, (kz - \omega t)}\right]
\end{split}
\end{align}
%
\begin{figure}[!t]
\centering
\subcaptionbox{}[.49\textwidth]{
\def\rot{false} 
\def\tikzwidth{0.49*\textwidth}
\inputtikz{gfx/pli/ellipsoid.tikz}}
%
\subcaptionbox{}[.49\textwidth]{
\def\rot{true} 
\def\tikzwidth{0.49*\textwidth}
\inputtikz{gfx/pli/ellipsoid.tikz}}
\caption{birefringence elipsoid}
\end{figure}
% 
% 
\subsection{Jones}
% 
\begin{align}
\begin{split}
\begin{pmatrix}
0 & 0 \\ 0 & 1
\end{pmatrix}
\begin{pmatrix}
1 & 0 \\ 0 & 0
\end{pmatrix}
\begin{pmatrix}
e^{i\, \varphi_x} & 0 \\ 0 & e^{i\, \varphi_y}
\end{pmatrix}
e^{\frac{i \pi}{4}}
\begin{pmatrix}
1 & 0 \\ 0 & -i
\end{pmatrix}
\end{split}
\end{align}
% 
\begin{align}
\begin{split}
M(\theta )=R(\theta )\,M\,R(-\theta )\\
R(\theta ) = 
\begin{pmatrix}
\cos \theta & -\sin \theta \\
\sin \theta & \cos \theta
\end{pmatrix}
\end{split}
\end{align}
% 
\subsection{Mueller-Stokes}\label{sec:Mueller-Stokes}
% 
\paragraph{Stokes vector}
\begin{align}
\begin{split}
S_0 &= I \\
S_1 &= I p \cos 2\psi \cos 2\chi \\
S_2 &= I p \sin 2\psi \cos 2\chi \\
S_3 &= I p \sin 2\chi
\end{split} \hspace{-6em}
\begin{split}
& \equiv \langle E_x^{2} \rangle + \langle E_y^{2} \rangle \\
%  & = \langle E_a^{2} \rangle + \langle E_b^{2} \rangle \\
%  & =  \langle E_l^{2} \rangle + \langle E_r^{2} \rangle \\
& \equiv \langle E_x^{2} \rangle - \langle E_y^{2} \rangle \\
& \equiv \langle E_a^{2} \rangle - \langle E_b^{2} \rangle \\
& \equiv  \langle E_l^{2} \rangle - \langle E_r^{2} \rangle
\end{split}
\hspace{2em}
\mathbf{S} =
\begin{pmatrix} S_0 \\ S_1 \\ S_2 \\ S_3\end{pmatrix}
\end{align}
% 
% \begin{align}
% \begin{split}
% S_0 & \equiv \langle E_x^{2} \rangle + \langle E_y^{2} \rangle \\
%  & = \langle E_a^{2} \rangle + \langle E_b^{2} \rangle \\
%  & =  \langle E_l^{2} \rangle + \langle E_r^{2} \rangle \\
% S_1 & \equiv \langle E_x^{2} \rangle - \langle E_y^{2} \rangle \\
% S_2 & \equiv \langle E_a^{2} \rangle - \langle E_b^{2} \rangle \\
% S_3 & \equiv  \langle E_l^{2} \rangle - \langle E_r^{2} \rangle
% \end{split}
% \end{align}
% 
\paragraph{Rotation Matrix}
\begin{align}
\begin{split}
\begin{pmatrix}
    1 &                0 &               0 & 0 \\
    0 & \cos{(2\theta)} & -\sin{(2\theta)} & 0 \\
    0 & \sin{(2\theta)} & \cos{(2\theta)} & 0 \\
    0 &                0 &               0 & 1
  \end{pmatrix}
\end{split}
\end{align}
% 
\paragraph{Rotation Matrix}
\begin{align}
\frac{1}{2}
\begin{pmatrix}
    1 &                0 &               0 & 0 \\
    0 &  \cos{(2\theta)} & \sin{(2\theta)} & 0 \\
    0 & -\sin{(2\theta)} & \cos{(2\theta)} & 0 \\
    0 &                0 &               0 & 1
\end{pmatrix}
\end{align}
% 
\paragraph{Linear Polarizer}
\begin{align}
\frac{1}{2}
\begin{pmatrix}
    1 & 1 & 0 & 0 \\
    1 & 1 & 0 & 0 \\
    0 & 0 & 0 & 0 \\
    0 & 0 & 0 & 0
  \end{pmatrix}
\end{align}
\begin{align}
\frac{1}{2}
\begin{pmatrix}
     1 & -1 & 0 & 0 \\
    -1 &  1 & 0 & 0 \\
     0 &  0 & 0 & 0 \\
     0 &  0 & 0 & 0
\end{pmatrix}
\end{align}
% 
\paragraph{Quarter-wave plate (fast-axis vertical)}
\begin{align}
\begin{pmatrix}
    1 & 0 & 0 &  0 \\
    0 & 1 & 0 &  0 \\
    0 & 0 & 0 & -1 \\
    0 & 0 & 1 &  0
\end{pmatrix}
\end{align}
% 
\section{Discretization}
\paragraph{Micro vs Macro:}
% 
\begin{align}
    % \frac{\oint \vec{v}_z \mathrm{d}A}{\oint \vec{v}_x \mathrm{d}A} = 
    % \frac{\int_{-1}^{1}\abs{\vec{v}} \mathrm{d}x}{\int_{-\frac{\pi}{2}}^{\frac{\pi}{2}} \abs{\vec{v}} \cos(\varphi) \mathrm{d}\varphi} =
    % \frac{2}{1}
    \frac{A_{\Box}}{A_{\circ}} = \frac{4}{\pi}
\end{align}
% 
\section{Experimental Setup}
%
\begin{figure}[!t]
    \captionsetup[sub]{position=top}
    \def\tikzwidth{\textwidth}
	\centering
	\subcaptionbox{}[\textwidth]{
% 		\resizebox{0.95\textwidth}{!}{
		\inputtikz{gfx/pli/pli_setup.tikz}
% 	}
	}\\
	\subcaptionbox{}[\textwidth]{
% 		\resizebox{0.95\textwidth}{!}{
		\inputtikz{gfx/pli/pli_setup_pm.tikz}
% 	}
	}
	\label{fig:pli_setup}
	\caption{Illustration of PLI setup. (a) LAP, (b) PM.}
\end{figure}
%
\begin{figure}[!t]
\def\tikzwidth{0.49*\textwidth}
	\centering
	\subcaptionbox{}[.49\textwidth]{
			\inputtikz{gfx/pli/pli_detail.tikz}}
	\subcaptionbox{}[.49\textwidth]{
			\inputtikz{gfx/pli/pli_detail_pm.tikz}}
	\label{fig:pli_detail}
	\caption{Illustration of detail PLI setup. (a) LAP, (b) PM.}
\end{figure}
%
% 
\section{Signal Analysis}
% 
\begin{quote}
Definition 1 (Streamline).
A streamline is a finite sequence of points that is used to represent a connection in the brain obtained using dMRI based tractography.
Following the sequence notation, we denote a streamline as $\Gamma = (p_0, p_1, p_2, \cdots )$ where $p_i \in \mathbb{R}^3$. % https://www.ncbi.nlm.nih.gov/pmc/articles/PMC6139055/
\end{quote}

\begin{align}
\begin{split}
A \sin(\omega t + \alpha) + B \sin(\omega t + \beta) &= \sqrt{C^2 + D^2} \cdot \sin \, \left( \omega t + \tan^-1 \left( \frac{D}{C} \right) \right)
\end{split}
\\
\begin{split}
C &= A \cos(\alpha)+ B \cos(\beta)\\
D &= A \sin(\alpha)+ B \sin(\beta)
\end{split} \nonumber 
\end{align}
% 
\begin{figure}[!t]
\centering
% \resizebox{0.95\textwidth}{!}{
\def\tikzwidth{0.5*\textwidth}
\inputtikz{gfx/simpli/vector_interpolation.tikz}
% }
\caption{vector interpolation}
\label{fig:vectorfield_disc}
\end{figure}
% 
\begin{figure}[!t]
\centering
\def\tikzwidth{0.45*\textwidth}
\subcaptionbox{rgb}[.49\textwidth]{
\inputtikz{gfx/pli/rgb_sphere.tikz}}
\subcaptionbox{hsv}[.49\textwidth]{
\inputtikz{gfx/pli/hsv_sphere.tikz}}
\caption{collr spheres}
\label{fig:spheres}
\end{figure}
