% \captionsetup[sub]{position=top, skip=-14pt}
%
\tikzset{
    pics/cone/.style n args={4}{
      code = {
        \begin{scope}[local bounding box=bb]
        \def\pa{#1}
        \def\pb{#3}
        \pgfmathsetmacro{\ra}{#2}
        \pgfmathsetmacro{\rb}{#4}
        \pgfmathsetmacro{\out}{\ra/(\ra - \rb)}
        \node[circle,minimum size=2 * \ra cm,inner sep=0pt] (l1) at \pa {};
        \node[circle,minimum size=2 * \rb cm,inner sep=0pt] (l2) at \pb {};
        \path (l1.center) -- node[coordinate,pos=\out] (out) {}  (l2.center);
        % 
        \foreach \i in {1,2}{
        \foreach \j in {1,2}{
            \coordinate (t\i\j) at (tangent cs:node=l\i,point={(out)},solution=\j);
        }}
        % 
        \foreach \i in {1,2}{
            \draw[] (t1\i) --  (t2\i);
        }
        % 
        % \node () at (t11) {t11};
        % \node () at (t12) {t12};
        % \node () at (t21) {t21};
        % \node () at (t22) {t22};
        % \node () at \pa {pa};
        % \node () at \pb {pb};
        % 
        \begin{scope}[overlay]
            \coordinate (t1) at ($(t11)!-\ra!(t21)$);
            \coordinate (t2) at ($(t12)!-\ra!(t22)$);
            % \node () at (t1) {t1};
            % \node () at (t2) {t2};
            % \draw[green] (t11)--(t12)--(t2)--(t1)--cycle;
            \begin{pgfinterruptboundingbox}
            \clip (t11)--(t12)--(t2)--(t1)--cycle;
            \end{pgfinterruptboundingbox}
            \draw \pa circle (\ra);
        \end{scope}
        \begin{scope}[overlay]
            \coordinate (t1) at ($(t21)!-\rb!(t11)$);
            \coordinate (t2) at ($(t22)!-\rb!(t12)$);
            % \node () at (t1) {t1};
            % \node () at (t2) {t2};
            % \draw[green] (t21)--(t22)--(t2)--(t1)--cycle;
            \begin{pgfinterruptboundingbox}
            \clip (t21)--(t22)--(t2)--(t1)--cycle;
            \end{pgfinterruptboundingbox}
            \draw \pb circle (\rb);
        \end{scope}
        \draw[red] (bb.north west) rectangle (bb.south east);
        \end{scope}
    }}
}
% 
% \centering
\subcaptionbox{}[0.5\textwidth]{
\resizebox{.42\textwidth}{!}{
\begin{tikzpicture}
% 
\path [] (-13,-9) grid (13,9);
% 
\begin{scope}[shift={(-7,0)}]
\foreach \pa/\pb/\ra/\rb [count=\ii] in {
(-1,5)/{(3,1)}/3/2,
(3,1)/{(-3,-5)}/2/2.5
} {
    \def\l{c\ii}
    % 
    \pgfmathsetmacro{\rone}{\ra}
    \pgfmathsetmacro{\rtwo}{\rb}
    \pgfmathsetmacro{\out}{\rone/(\rone - \rtwo)}
    \node[draw, circle,minimum size=2 * \rone cm,inner sep=0pt] (\l1) at \pa {};
    \node[circle,minimum size=2 * \rtwo cm,inner sep=0pt] (\l2) at \pb {};
    \path (\l1.center) -- node[coordinate,pos=\out] (out) {}  (\l2.center);
    % 
    \foreach \i in {1,2}
    \foreach \j in {1,2}
    \foreach \k in {out}
    \coordinate (t\i\j\k) at (tangent cs:node=\l\i,point={(\k)},solution=\j);
    % 
    \foreach \i in {1,2}
    \foreach \k in {out}
    \draw[red] ($(t1\i\k)!-0cm!(t2\i\k)$) --  ($(t2\i\k)!-0cm!(t1\i\k)$);
    % 
    % last circle
    \ifnum\ii=2
        \node[draw,circle,minimum size=2 * \rtwo cm,inner sep=0pt] (last) at \pb {};
    \fi
}
\end{scope}
% 
\draw[very thick, ->, >=latex, line width=5pt] (-1,0) -- (1,0);
% 
\begin{scope}[shift={(9,0)}]
\foreach \pa/\pb/\ra/\rb [count=\ii] in {
(-1,5)/{(-3,-5)}/3/2.5
} {
    \def\l{c\ii}
    % 
    \pgfmathsetmacro{\rone}{\ra}
    \pgfmathsetmacro{\rtwo}{\rb}
    \pgfmathsetmacro{\out}{\rone/(\rone - \rtwo)}
    \node[draw, circle,minimum size=2 * \rone cm,inner sep=0pt] (\l1) at \pa {};
    \node[circle,minimum size=2 * \rtwo cm,inner sep=0pt] (\l2) at \pb {};
    \path (\l1.center) -- node[coordinate,pos=\out] (out) {}  (\l2.center);
    % 
    \foreach \i in {1,2}
    \foreach \j in {1,2}
    \foreach \k in {out}
    \coordinate (t\i\j\k) at (tangent cs:node=\l\i,point={(\k)},solution=\j);
    % 
    \foreach \i in {1,2}
    \foreach \k in {out}
    \draw[red] ($(t1\i\k)!-0cm!(t2\i\k)$) --  ($(t2\i\k)!-0cm!(t1\i\k)$);
    % 
    % last circle
    \ifnum\ii=1
        \node[draw,circle,minimum size=2 * \rtwo cm,inner sep=0pt] (last) at \pb {};
    \fi
}
\end{scope}
\end{tikzpicture}
}
}
% 
% \newline
%
% \centering
\subcaptionbox{}[0.5\textwidth]{
\resizebox{.42\textwidth}{!}{
\begin{tikzpicture}
% 
\path [] (-13,-9) grid (13,9);
% 
\begin{scope}[shift={(-7,0)}]
\foreach \pa/\pb/\ra/\rb [count=\ii] in {
(1,5)/{(-1,-5)}/2.5/2
} {
    \def\l{c\ii}
    % 
    \pgfmathsetmacro{\rone}{\ra}
    \pgfmathsetmacro{\rtwo}{\rb}
    \pgfmathsetmacro{\out}{\rone/(\rone - \rtwo)}
    \node[draw, circle,minimum size=2 * \rone cm,inner sep=0pt] (\l1) at \pa {};
    \node[circle,minimum size=2 * \rtwo cm,inner sep=0pt] (\l2) at \pb {};
    \path (\l1.center) -- node[coordinate,pos=\out] (out) {}  (\l2.center);
    % 
    \foreach \i in {1,2}
    \foreach \j in {1,2}
    \foreach \k in {out}
    \coordinate (t\i\j\k) at (tangent cs:node=\l\i,point={(\k)},solution=\j);
    % 
    \foreach \i in {1,2}
    \foreach \k in {out}
    \draw[red] ($(t1\i\k)!-0cm!(t2\i\k)$) --  ($(t2\i\k)!-0cm!(t1\i\k)$);
    % 
    % last circle
    \ifnum\ii=1
        \node[draw,circle,minimum size=2 * \rtwo cm,inner sep=0pt] (last) at \pb {};
    \fi
}
\end{scope}
% 
\draw[very thick, ->, >=latex, line width=5pt] (-1,0) -- (1,0);
% 
\begin{scope}[shift={(7,0)}]
\foreach \pa/\pb/\ra/\rb [count=\ii] in {
(1,5)/{(0,0)}/2.5/2.25,
(0,0)/{(-1,-5)}/2.25/2
} {
    \def\l{c\ii}
    % 
    \pgfmathsetmacro{\rone}{\ra}
    \pgfmathsetmacro{\rtwo}{\rb}
    \pgfmathsetmacro{\out}{\rone/(\rone - \rtwo)}
    \node[draw, circle,minimum size=2 * \rone cm,inner sep=0pt] (\l1) at \pa {};
    \node[circle,minimum size=2 * \rtwo cm,inner sep=0pt] (\l2) at \pb {};
    \path (\l1.center) -- node[coordinate,pos=\out] (out) {}  (\l2.center);
    % 
    \foreach \i in {1,2}
    \foreach \j in {1,2}
    \foreach \k in {out}
    \coordinate (t\i\j\k) at (tangent cs:node=\l\i,point={(\k)},solution=\j);
    % 
    \foreach \i in {1,2}
    \foreach \k in {out}
    \draw[red] ($(t1\i\k)!-0cm!(t2\i\k)$) --  ($(t2\i\k)!-0cm!(t1\i\k)$);
    % 
    % last circle
    \ifnum\ii=2
        \node[draw,circle,minimum size=2 * \rtwo cm,inner sep=0pt] (last) at \pb {};
    \fi
}
\end{scope}
\end{tikzpicture}
}
}